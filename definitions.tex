%\section{Conjunctive Queries, RDF and SPARQL Prerequisites}
\section{Definitions}

\begin{definition}[Conjunctive Queries]
	$U$ is an infinite set of constants and $X$ is an infinite set of variables.
	Assuming $\sigma$ is a relational schema we have that a relational atom over
	$\sigma$ is an expression of the form $R(\vec{v})$, where $R \in \sigma$ and 
	$R$ has arity $n>0$. $\vec{v}$ is an $n-tuple$ over $X \cup U$.
	A database $D$ over $\sigma$ is a set of relational atoms without variables
	where each relational atom is in $\sigma$.

	A conjunctive query (CQ) $q$ over $\sigma$ is a rule of the form:
	\begin{align*}
		X \leftarrow R_1(\vec{v_1}), \dots, R_m(\vec{v_m}).
	\end{align*}
	Each $R_i(\vec{v_i}) (1\leq i \leq m)$ is a relational atom in $\sigma$ and 
	$X$ is a set of distinct variables among the ones that appear in the
	$\vec{v_i}$'s.

	In order to evaluate a CQ given a database $D$ over a relational schema $\sigma$
	we need to define the semantics which is given in terms of homomorphisms.
	Let $D$ be a database over $\sigma$.  A homomorphism from a CQ $Q$ to $D$ is a
	partial mapping $h: X \rightarrow U$ such that $R_i(h(\vec{v_i})) \in D,$ for 
	$1 \leq i \leq m$. $h_{X}$ is the restriction of $h$ to the variables in
	$X$. The evaluation $Q(D)$ is the set of all mappings of the form
	$h_{X}$, such that $h$ is a homomorphism from $q$ to $D$.
\end{definition}


\begin{definition}[RDF]
	The focus is on ground RDF graphs and we assume them to be composed of URIs only.
	$\U$ is the infinite set of URIs. An RDF triple is a tuple in $\U \times \U \times \U$, whereas an
	RDF graph is a finite set of RDF triples. $dom(G) \subseteq \U$ of an RDF graph $G$ is the set of URIs actually appearing in $G$.
\end{definition}

\bigskip
\begin{definition}[SPARQL SYNTAX]
	$\V$ is an infinite set of variables with $\U \cap V = \emptyset$. 
	Variables in $\V$ are denoted with by the letters $x,y,z,x',y',z'$. 
	A sparql triple pattern is a tuple in 
	$(\U \cup \V) \times (\U \cup \V) \times (\U \cup \V)$. 
	SPARQL graph patterns are recursively defined as follows:
	\begin{enumerate}
		\item A triple pattern is a graph pattern.
		\item If $P_1$ and $P_2$ are graph patterns, then $(P_1  \ \circ \ P_2)$ for
			$\circ \in \{ AND, OPT, UNION\}$ is a graph pattern.
		\item If $P$ is a graph pattern and $a \in (I \cup V)$, then $(\mbox{GRAPH} \  a \ P)$ is a graph pattern.
		\item If $P$ is a graph pattern and $a \in (I \cup V)$, then $(\mbox{SERVICE} \  a \ P)$ is a graph pattern.
	\end{enumerate}
	\noindent Assuming $P$ is a graph pattern, $vars(P)$ denotes the set of variables occuring in $P$.
\end{definition}

\begin{definition}[SPARQL Semantics]
	A mapping is a function $\mu: A \rightarrow  \U$ for some $A \subset \V$. 
	For a triple pattern $t$ with $vars(t) \subseteq dom(\mu)$, we write $\mu(t)$ to 
	denote the triple after replacing the variables in $t$ by the corresponding 
	URIs according to $\mu$. 

	\noindent Two mappings $\mu_1$ and $\mu_2$ are called compatible (denoted $\mu_1 \sim \mu_2$) 
	if $\mu_1(?X) = \mu_2(?X)$ for all $?X \in dom(\mu_1) \cap dom(\mu_2)$.

	\noindent A mapping $\mu_1$ is subsumed by $\mu_2$ (written $\mu_1 \sqsubseteq \mu_2$) 
	if $\mu_1 \sim \mu_2$ and $dom(\mu_1) \subseteq dom(\mu_2)$. $\mu_2$ is then called ``extension'' of $\mu_1$.

	A dataset is a set $DS = \{(def, G), (g_1,G_1), \dots, (g_k, G_k) \}$, with
	$k\geq 0$ is a set of pairs of URIs and graphs,
	where the default graph $G$ is identified by the special symbol $def \notin \U$
	and the remaining so-called ``named'' graphs $(G_i)$ are identified by URIs
	$(g_i \in \U)$. We assume that any query is evaluated over a fixed dataset $DS$
	and that any SPARQL endpoint that is identified by an URI $c \in \U$ evaluates
	its queries against its own fixed dataset 
	$DS_c = \{ (def, G_c),(g_{c,1},G_{c,1}), \dots, (g_{c,k},G_{c,k})\}$.
	We assume a function $graph(g,DS)$ which, given a Dataset $DS$ and a graph name $g$ as
	input returns the graph corresponding to the symbol $g$. The function
	$names(DS)$ returns the set of names of the given dataset $DS$:
	$\{g_1,g_2,\dots,g_k\}$.
	We also assume a partial function $ep: \U \rightarrow DS$, such that for every $c \in
	\U$, if $ep(c)$ is defined, then $ep(c) = DS_c$, i.e., the dataset associated with
	the endpoint accessible via the URI c.
	$\mu_\emptyset$ denotes the mapping with empty domain. This means it is
	compatible with any other mapping.
	The evaluation of graph patterns over an RDF Graph $G$ is formalized as a
	function  $\llbracket \cdot \rrbracket_G^{DS}$, which, given a SPARQL graph pattern
	returns a set of mappings.
	For a graph pattern $P$, it is again defined recursively:
	\scalebox{0.64}{
		\vbox{
			\begin{enumerate}
				\item $\ll t \rr_G^{DS} = \{ \mu \ | \ dom(\mu) = vars(t) \mbox{ and } \mu(t)
					\in G \}$ for a triple pattern $t$.
				\item $\ll P_1 \ \mbox{AND} \ P_2 \rr_G^{DS} = \{ \mu_1 \cup \mu_2  \ | \ \mu_1 \in \ll P_1
					\rr_G^{DS},  \ \mu_2 \in \ll P_2 \rr_G^{DS} \mbox{ and } \mu_1 \sim \mu_2 \}$.
				\item $\ll P_1 \ \mbox{OPT} \ P_2 \rr_G^{DS} = \ll P_1 \ \mbox{AND} \ P_2 \rr_G^{DS}
					\cup \{ \mu_1 \in \ll P_1
					\rr_G^{DS} \ | \ \forall \mu_2 \in \ll P_2 \rr_G^{DS}: \mu_1 \not\sim \mu_2\}$.
				\item $\ll P_1 \ \mbox{UNION} \ P_2 \rr_G^{DS} = \ll P_1 \rr_G^{DS} \cup \ll P_2
					\rr_G^{DS}$.
				\item  $\ll \mbox{GRAPH } c \ P_1 \rr_G^{DS}  = \begin{cases} 
						\ll P_1\rr_{graph(c,DS)}^{DS} &\mbox{if } c \in names(DS)\\	
						\{ \} & \mbox{if } c \in I \backslash names(DS)\\
						\bigg\{\mu \cup [c \rightarrow s ] \mid \exists s \in names(DS), \mu
							\in \ll P_1 \rr^{DS}_{graph(s,DS)} \land [ c \rightarrow s ] \sim \mu
						\bigg\} & \mbox{if } c \in \V
						\end{cases}$\\ for $c \in \U \cup \V$.

					\item  $\ll \mbox{SERVICE } c \ P_1 \rr_G^{DS}  = \begin{cases} 
							\ll P_1\rr_{graph(def,ep(c))}^{ep(c)} &\mbox{if } c \in dom(ep)\\	
							\{ \mu_\emptyset\} & \mbox{if } c \in I \backslash dom(ep)\\
							\bigg\{\mu \cup [c \rightarrow s ] \mid  \exists s \in dom(ep), \mu
								\in \ll P_1 \rr^{ep(s)}_{graph(def,ep(s))} \land [ c \rightarrow s ] \sim \mu
							\bigg\} & \mbox{if } c \in \V
							\end{cases}$\\ for $c \in \U \cup \V$.
					\end{enumerate}
				}
			}
		\end{definition}

			\begin{definition}[Well-designed SPARQL]
				A graph pattern $P$ built only from AND and OPT is well-designed if there does
				not exist a subpattern $P' = (P_1 \  \mbox{OPT} \ P_2)$ of $P$ and a variable $?X \in
				vars(P_2)$ that occurs in $P$ outside $P'$, but not in $P_1$. A graph pattern $P
				= P_1 \mbox{ UNION } \dots \mbox{ UNION } P_n$ is well-designed if each subpattern $P_i$ is UNION
				free and well-designed.
			\end{definition}

			\begin{definition}[OPT normal form]
				A graph pattern containing only the operators AND and OPT is in OPT normal form
				if the OPT operator never occurs in the scope of an AND operator. In
				\cite{letelier2013static} it was shown that every well-designed graph pattern
				can be transformed into OPT normal form in polynomial time. Moreover the graph
				patterns allow for a natural tree representation, formalized by so-called
				pattern trees.
			\end{definition}


			\begin{definition}[WDPTs~\cite{barcelo2015efficient}]\label{def:wdpt}
				A WDPT over a relational schema $\sigma$ is a tuple $(T, \lambda, \overline{x})$
				such that the following holds:
				\begin{enumerate}
					\item $T$ is tree rooted in a distinguished node $r$, the root and $\lambda$
						maps each node $t$ in $T$ to a set of relational atoms over $\sigma$.
					\item For every variable $y$ that appears in $T$, the set of nodes of $T$ where
						$y$ is mentioned is connected.
					\item We have that $\overline{x}$ is a tuple of distinct variables occurring in
						$T$. They are the free variables of the WDPT.
				\end{enumerate}
			\end{definition}

			\begin{definition}
				A WDPT $(T,\lambda, \overline{x})$ is called projection-free if $\overline{x}$
				contains all variables mentioned in $T$.
			\end{definition}

			\begin{definition}\label{wdptq}
				Assume $p = (T,\lambda,\overline{x})$ is a WDPT over $\sigma$. We write $r$ to
				denote the root of $T$. Given a subtree $T'$ of $T$ rooted in $r$ we define
				$q_{T'}$ to be the CQ $Y \leftarrow R_1(\overline{v_1}), \dots,
				R_m(\overline{v}_m)$, where the $R_i(\overline{v_i})$'s are the reational atoms
				that label the nodes of $T'$, i.e., 
				\begin{align*}
					\{ R_1(\overline{v}_1), \dots, R_m(\overline{v_m}) \} = \bigcup_{t \in T'} \lambda(t) 
				\end{align*} and $\overline{y}$ are all the variables that are mentioned in
				$T'$. 
			\end{definition}

			The main idea of the semantics of a WDPT is to look at each subtree $T'$ of
			$T$ rooted in $r$. As mentioned above, each of them describes a pattern, i.e.,
			the conjunctive query CQ $q_T'$. A mapping $h$ satisfies $(T,\lambda)$ over
			a database $D$ if $h$ satisfies the pattern defined by a subtree $T'$ and there
			is no subtree $T''$ which is bigger than $T'$ and $h$ can be extended to satisfy
			$T''$.
			\begin{definition}[Semantics of WDPTs~\cite{barcelo2015efficient}]
				Let $p=(T,\lambda,\overline{x})$ be a WDPT and $D$ a database over $\sigma$.

				\begin{enumerate}
					\item A homomorphism from $p$ to $D$ is a partial mapping $h: X \rightarrow U$,
						where $X$ is an infinite set of a variables and $U$ an infinite set of
						constants, for which it is the case that there is a subtree $T'$ of $T$ rooted
						in $r$ such that $h \in q_{T'}(D)$.
					\item The homomorphism $h$ is maximal if there is no homomorphism $h'$ from $p$
						to $D$ such that $h \sqsubset h'$.
				\end{enumerate}
				The evaluation of WDPT $p = (T,\lambda,\overline{x})$ over $D$ denoted $p(D)$,
				corresponds to all mappings of the form $h_{\overline{x}}$, such that $h$ is a
				maximal homomorphism from $p$ to $D$.
			\end{definition}

			It is important to notice that WDPTs properly extend CQs.
			Given a CQ $q(x)$ of the form $X \leftarrow R_1(v_1),\dots,R_m(v_m)$
			it is easy to see that $q(x)$ is equivalent to the WDPT $p = (T,\lambda,x)$,
			where $T$ consists of a single node $r$ and $\lambda(r) =
			\{R_1(v_1),\dots,R_m(v_m)\}$. In other words, $q(D) = p(D)$ for each database
			$D$. Further on we will not distinguish between a CQ and the single node WDPT
			that represents it. WDPTs can on the other hand represent interesting properties
			that cannot be expressed as CQs, which we will discuss in the following example:
			\begin{example}
				$Q =  \bigg(((x,recorded\_by,y) \ AND \ (x,published, ``after\_2010'')) \ \\ OPT \ (x,
				NME\_rating,z)\bigg) \ OPT \ (y, formed\_in, z') $

				\noindent Consider now the following RDF database $D$ consisting of the triples:

				\begin{verbatim}
				(``title_1'' recorded_by, ``studio_1''),
				(``title_1'' published, ``after_2010''),
				(``title_2'' recorded_by, ``studio_1''),
				(``title_2'' published, ``after_2010''),
				(``title_2'' NME_rating, ``2'')
				(``studio_1'' formed_in, ``miami''),
				\end{verbatim}
				Evaluating the query $Q$ over $D$ results in two mappings $\mu_1$ and $\mu_2$:
				$\mu_1 = \{ x \rightarrow ``title\_1'', y \rightarrow ``studio_1'',z'
				\rightarrow ``miami'' \}$ 
				$\mu_2 = \{ x \rightarrow ``title\_2'', y \rightarrow ``studio_1'',z\rightarrow
				``2'', z' \rightarrow ``miami'' \}$ 
			\end{example}

			\bigskip \noindent
			\textbf{Well designed pattern forests~\cite{barcelo2015efficient}}
			To capture the UNION operator the definition of well designed pattern trees
			need to be modified.  
			\begin{definition}[Unions of WDPTs]
				A Union of WDPTs is an expression $\phi$ of the form $\bigcup_{1\leq i \leq n} p_i$, 
				where each $p_i$ is a WDPT over $\sigma$.
				We denote $\varphi(D)$ as the evaluation of $\phi$ over database $D$.
				It corresponds to the set $\bigcup_{1\leq i \leq n}p_i(D)$.

			\end{definition}
			\begin{definition}[Destination of a SERVICE-operator]
				Given a pattern $P$ of the form $P = \mbox{SERVICE } u \ P_1$ we call $u$ the
				destination of the pattern $P$.

			\end{definition}

			\begin{definition}
				polynomial hierarchy $\Sigma^P_2$ etc..
			\end{definition}

			\begin{definition}[$P_{wdsg}$]
				A pattern $Q \in P_{wdsg}$ if it adheres to the following grammar:
				\begin{align*}
					Q::= &  (Y \OPT R)  \mid B \\
					Y::= & (Y \AND Y) \mid (\mbox{SERVICE } u \ Y) \mid (\mbox{GRAPH } u \
					Y) \mid  B\\
					R::= &(R \OPT R) \mid (\mbox{GRAPH } u \ R) \mid (\mbox{SERVICE } u \ R) \mid B  \\
					B::= &(u,v,w)
				\end{align*}
				where	$u,v,w \in \U \cup \V$. In the first layer we seperate the pattern
				into an AND-part and an OPT-part. In the AND- and OPT-part we can use
				SERVICE and GRAPH freely.

			\end{definition}

			\subsection*{Pattern trees and SPARQL graph patterns
			Section 2 from Pichler Skritek}

			\section{Thoughts on the Semantics of the SERVICE and GRAPH Operator}

			We tried to propose an algorithm for evaluating CONTAINMENT[$S_1$,$S_2$] in the
			fragment $P_{wdgs}$ and
			along the way we recognized some unintuitive features regarding the semantics of the
			SERVICE and GRAPH operators. Consider the following example:

			\begin{example}
				$DS=\{(G_0,def), (G_1,a) \}$ where $G_0 = \{
				(c,k,g), (a,k,g) \}$, $G_1$ is arbitrary and  $Q = (GRAPH \ ?X  \ (GRAPH  \ c \  P_1)) \ AND \ (?X,k,g)$.
			\end{example}

			By evaluating the first part of the query, namely $\ll (GRAPH \ ?X  \ (GRAPH  \
			c \  P_1)) \rr$ where $P_1$ could be any arbitrary pattern, because $c \notin
			names(DS)$ and thus $\ll (GRAPH \ c  \ P_1) \rr = \mu_\emptyset$  we obtain $?X
			\rightarrow a$ ($\mu_\emptyset$ is compatible with every mapping). This is in
			our opinion unintuitive because it affects the rest of the query.
			If we would evaluate the right side of the $AND$ we would obtain the following: $\ll(?X,k,g)\rr = \{ (X \rightarrow a) , (X \rightarrow c) \}$ considering the
			left part of the $AND$-Operator we get the result of the query: $\{ (X \rightarrow a )
			\}$. Note that even though the graph with the URI $c$ does not exist and we are not able
			to evaluate pattern $P_1$ over it, we are able to somehow receive results using
			the $AND$ operator.
			\bigskip

			\noindent Considering containment/equivalence of two queries which just use the GRAPH
			Operator, there are specific instances where it is not obvious if one query is contained in the other query even though it should be intuitively.

			\begin{example}
				$Q_1$ = (GRAPH ?X ( GRAPH c ( GRAPH ?Y) $P_1$)),\\ 
				$Q_2$ = (GRAPH ?Y ( GRAPH c ( GRAPH ?X) $ P_1$))\\
				$?X,?Y \notin P_1$.
			\end{example}

			Approaching this example intuitively, one could say that $\ll Q_1 \rr = \ll Q_2
			\rr$ because $?X$ and $?Y$ get bound to every variable in the dataset because neither
			$?X$ nor $?Y$ occur in $P_1$, the triple pattern at the end
			of the query is the same and also the graph queried second is the same. 
			To show the opposite, just  consider the following dataset:
			$DS=\{(def,G_0), (a,G_1)\}$. Note that the URI $c$ does not occur in the dataset.
			But this means that $\ll Q_1 \rr^{DS}_G = \{?X \rightarrow a\}$ and  $\ll Q_2
			\rr^{DS}_G = \{?Y \rightarrow a\}$ rendering neither equivalence nor containment
			in any possible way.

			\noindent\textbf{The history of the formal definition of the GRAPH and the SERVICE
			operator}\\
			We were introduced to the GRAPH and
			SERVICE operators by \cite{BuilAranda20131} where Buil-Aranda et al.
			describe the syntax and semantics of the SERVICE operator and how an efficient query
			evaluation system can be implemented. In \cite{BuilAranda20131} the authors
			state that the GRAPH-Operator was introduced formally
			in \cite{perez2009semantics} but we were not able to find the definition in \cite{perez2009semantics}.
			Instead we tried to look into the w3c-standard\cite{w3standard} 
			where we found the following definition:

			\begin{lstlisting}
			if IRI is a graph name in D
			eval(D(G), Graph(IRI,P)) = eval(D(D[IRI]), P)
			if IRI is not a graph name in D
			eval(D(G), Graph(IRI,P)) = the empty multiset
			eval(D(G), Graph(var,P)) =
			Let R be the empty multiset
			foreach IRI i in D
			R := Union(R, Join( eval(D(D[i]), P) , $\Omega$(?var->i) )
			the result is R
			\end{lstlisting}

			If we look at the second case, we can see that the empty multiset (because the
			definition goes with bag semantics)
			is returned if the IRI is not a graph name in the Dataset. 
			To eliminate the unintuitive semantics of the SERVICE and GRAPH Operators we
			would thus like to define the following semantics:\\
			\scalebox{0.68}{
				\vbox{
					\begin{enumerate}
						\item  $\ll \mbox{GRAPH } c \ P_1 \rr_G^{DS}  = \begin{cases} 
								\ll P_1\rr_{graph(c,DS)}^{DS} &\mbox{if } c \in names(DS)\\	
								\{ \} & \mbox{if } c \in I \backslash names(DS)\\
								\bigg\{\mu \cup [c \rightarrow s ] \mid \exists s \in names(DS), \mu
									\in \ll P_1 \rr^{DS}_{graph(s,DS)} \land [ c \rightarrow s ] \sim \mu
								\bigg\} & \mbox{if } c \in \V
								\end{cases}$\\ for $c \in \U \cup \V$.
						\end{enumerate}
					}
				}

				Looking at GRAPH again, it's a bit unintuitive that if we have
				the third case, namely $c \in \V$, the query $P_1$ will not be considered over the
				default graph. The reason for this can also be found in the definition of the
				dataset in the standard \cite{w3standard}:
				``A SPARQL query is executed against an RDF Dataset which represents a collection
				of graphs. An RDF Dataset comprises one graph, the default graph, which does not
				have a name, and zero or more named graphs, where each named graph is identified
				by an IRI. A SPARQL query can match different parts of the query pattern against
				different graphs as described in section 13.3 Querying the Dataset.''
				In~\cite{pichler2014containment} the complexity of the \textbf{CONTAINMENT},
				\textbf{EQUIVALENCE} and \textbf{SUBSUMPTION}
				problems of well-designed SPARQL regarding the operators AND, OPT, UNION and 
				projection were examined.  We extend the results with the GRAPH- and the SERVICE
				operator. 

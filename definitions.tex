We will introduce conjunctive queries first and then proceed with RDF and
SPARQL. After defining these three concepts we will discuss the semantics of
the GRAPH and SERVICE operator in SPARQL.

\section{Conjunctive Queries, RDF, SPARQL}
We will start out with the definition of a 
relational schema, database and conjunctive query. 

A relational name describes the name and arity of a relation.
\begin{definition}[Relational Schema]
	A relational schema is a nonempty finite set of relational names.
\end{definition}
Following up we have the definition of a database and relational atom over a
schema..

\begin{definition}[Database and relational atom]
	Let $\sigma$ be a relational schema, 
	$\U$ be an infinite set of constants and $\V$ an infinite set of variables.
	A \emph{relational} atom over $\sigma$ is an expression of the form $R(v)$ where $R$
	is a relational name with arity $n$ over $\sigma$ and $v$ is an $n$-tuple over
	$\U \cup \V$.

	A \emph{database} $D$ over $\sigma$ is a set of relational atoms over $\sigma$ and
	each $n$-tuple is in $\U$.
\end{definition}

After defining database and schema we are ready to define
conjunctive queries.

\begin{definition}[Conjunctive Queries]
	Let $\sigma$ be a relational schema.
	A conjunctive query (CQ) $q$ over $\sigma$ is a rule of the form:
	\begin{align*}
		X \leftarrow R_1(\vec{v_1}), \dots, R_m(\vec{v_m}).
	\end{align*}
	Each $R_i(\vec{v_i}) (1\leq i \leq m)$ is a relational atom over $\sigma$ and 
	$X$ is a subset of the set of variables that appear in the
	$\vec{v_i}$s.
\end{definition}

In order to evaluate a CQ given a database over a relational schema, 
we need to define the semantics which is given in terms of homomorphisms.
\begin{definition}[Semantics of Conjunctive Queries]
	Let $D$ be a database over schema $\sigma$.  A homomorphism from a CQ $Q$ to $D$ is a
	partial mapping $h: X \rightarrow U$ such that $R_i(h(\vec{v_i})) \in D,$ for 
	$1 \leq i \leq m$. $h_{|X}$ is the restriction of $h$ to the variables in
	$X$. The evaluation $Q(D)$ is the set of all mappings of the form
	$h_{|X}$, such that $h$ is a homomorphism from $q$ to $D$.
\end{definition}

We will now define a simplified but absolutely (for our purposes) sufficient version of the
original formalisation of the RDF W3C-recommendation~\cite{rdf}:
URI stands for Uniform Resource Identifier and is a string which is used to identify
abstract and physical resources in the web. 
We will focus on ground RDF graphs which means that we assume them to be composed of URIs only.
Usually RDF graphs would also support the usage of blank nodes, which work like
existential variables (see~\cite{hogan2014everything} for details).

\begin{definition}[RDF]
	Let $\U$ is the infinite set of URIs. An RDF triple is a tuple in $\U \times \U \times \U$, whereas an
	RDF graph is a finite set of RDF triples. Let $G$ be an RDF graph. Then
	$dom(G) \subseteq \U$ is the set of URIs actually appearing in $G$.
\end{definition}

RDF is the data format and SPARQL will be our query language. To use SPARQL we need
to define the SPARQL syntax and semantics.

\begin{definition}[SPARQL Syntax~\cite{BuilAranda20131,perez2006semantics}]
	Again $\U$ is an infinite set of URIs. 
	$\V$ is an infinite set of variables with $\U \cap \V = \emptyset$. 
	We will denote Variables in $\V$ with the letters $x,y,z,x',y',z',\dots$ and
	symbols which are in $\V \cup \U$ with $u,v,w,u',v',w',\dots$ and
	symbols which are in $\U$ with $a,b,c,d,e,f,a',b',c',d',e',f',\dots$.
	
	Filter constraints are conditions of the following form:
	\begin{enumerate}
		\item $\top, x=a, x=y$, or $bound(x)$ (atomic constraints),
		\item If $R_1, R_2$ are filter constraints, then $\neg R_1$, $R_1 \land R_2$,
			or $R_1 \lor R_2$ are filter constraints.
	\end{enumerate}

	A SPARQL triple pattern is a tuple in  $(\U \cup \V) \times (\U \cup \V) \times (\U \cup \V)$. 
	SPARQL graph patterns are recursively defined as follows:
	\begin{enumerate}
		\item A triple pattern is a graph pattern.
		\item If $P_1$ and $P_2$ are graph patterns, then $(P_1  \circ P_2)$ for
			$\circ \in \{ \mbox{AND}, \mbox{OPT}, \mbox{UNION}\}$ is a graph pattern.
		\item If $P$ is a graph pattern and $R$ a filter constraint, then $(P \ FILTER \ R)$ is a graph pattern.
		\item If $P$ is a graph pattern and $u \in (\U \cup \V)$, then $(\mbox{GRAPH} \  u \ P)$ is a graph pattern.
		\item If $P$ is a graph pattern and $u \in (\U \cup \V)$, then $(\mbox{SERVICE} \  u \ P)$ is a graph pattern.
	\end{enumerate}
If $P$ is a graph pattern, $\vars(P)$ denotes the set of variables occurring in $P$.
\end{definition}

We proceed by defining the semantics of SPARQL. To begin with, we need to define
what mappings are:
\begin{definition}[Mapping~\cite{perez2006semantics}]
	A mapping is a function $\mu: A \rightarrow  \U$ for some $A \subset \V$. 
\end{definition}

For a triple pattern $t$ with $\vars(t) \subseteq dom(\mu)$, we write $\mu(t)$ to 
denote the triple after replacing the variables in $t$ by the corresponding 
URIs according to $\mu$. 

We can then proceed to define when two mappings are compatible.
\begin{definition}[Compatible Mappings~\cite{perez2006semantics}]
	Two mappings $\mu_1$ and $\mu_2$ are called compatible (denoted $\mu_1 \sim \mu_2$) 
	if $\mu_1(x) = \mu_2(x)$ for all $x \in dom(\mu_1) \cap dom(\mu_2)$.
\end{definition}

We denote $\mu_\emptyset$ as the mapping with empty domain. This means it is
compatible with any other mapping.

A given pattern is subsumed by another pattern if we can ``extend'' the original
pattern to the other pattern.
\begin{definition}[\cite{perez2006semantics}]
	A mapping $\mu_1$ is subsumed by $\mu_2$ (written $\mu_1 \sqsubseteq \mu_2$) 
	if $\mu_1 \sim \mu_2$ and $dom(\mu_1) \subseteq dom(\mu_2)$. $\mu_2$ is then called ``extension'' of $\mu_1$.
\end{definition}

We proceed with the definition of a dataset.
\begin{definition}[\cite{BuilAranda20131,perez2006semantics}]
	A dataset is a set $DS = \{(def, G), (g_1,G_1), \dots, (g_k, G_k) \}$, with
	$k\geq 0$. It contains pairs of URIs and graphs,
	where the default graph $G$ is identified by the special symbol $def \notin \U$
	and the remaining so-called ``named'' graphs $(G_i)$ are identified by URIs
	$(g_i \in \U)$.
\end{definition}
We assume that any query is evaluated over a fixed dataset $DS$
and that any SPARQL endpoint that is identified by an URI $c \in \U$ evaluates
its queries against its own fixed dataset 
$DS_c = \{ (def, G_c),(g_{c,1},G_{c,1}), \dots, (g_{c,k},G_{c,k})\}$.

The functions $graph$, $names$ and $ep$ are further assumed to define the
semantics of SPARQL.
\begin{definition}[The functions graph, names and ep~\cite{
	BuilAranda20131,perez2006semantics}]
	We assume a function $graph(g,DS)$ which, given a Dataset $DS=\{(def, G), (g_1,G_1), \dots, (g_k, G_k) \}$
	and a graph name $g$ as
	input returns the graph corresponding to the symbol $g$. The function
	$names(DS)$ returns the set of names of the input dataset $DS$:
	For example $names(DS) = \{g_1,g_2,\dots,g_k\}$.
	We also assume a partial function $ep: \U \rightarrow DS$, such that for every $c \in
	\U$, if $ep(c)$ is defined, then $ep(c) = DS_c$, i.e., the dataset associated with
	the endpoint accessible via the URI $c$.
\end{definition}
Finally we proceed to define the evaluation given a graph pattern, a dataset and
a graph.
\begin{definition}[SPARQL Semantics~\cite{
	BuilAranda20131,perez2006semantics}]\label{def:sparqlsem}
	The evaluation of graph patterns over an RDF Graph $G$ and Dataset $\DS$, where $(g_1,G) \in
	\DS$ for $g_1 \in \U\cup\{def\}$ is formalized as a
	function  $\llbracket \cdot \rrbracket_G^{DS}$, which, given a SPARQL graph pattern
	returns a set of mappings.
	For a graph pattern $P$, it is defined recursively:\\
	\begin{enumerate}
		\item $\ll t \rr_G^{DS} = \{ \mu \ | \ dom(\mu) = \vars(t) \mbox{ and } \mu(t)
			\in G \}$ for a triple pattern $t$.
		\item $\ll P_1 \ \mbox{AND} \ P_2 \rr_G^{DS} = \{ \mu_1 \cup \mu_2  \ | \ \mu_1 \in \ll P_1
			\rr_G^{DS},  \ \mu_2 \in \ll P_2 \rr_G^{DS} \mbox{ and } \mu_1 \sim \mu_2 \}$.
		\item $\ll P_1 \ \mbox{OPT} \ P_2 \rr_G^{DS} = \ll P_1 \ \mbox{AND} \ P_2 \rr_G^{DS}
			\cup \{ \mu_1 \in \ll P_1
			\rr_G^{DS} \ | \ \forall \mu_2 \in \ll P_2 \rr_G^{DS}: \mu_1 \not\sim \mu_2\}$.
		\item $\ll P_1 \ \mbox{UNION} \ P_2 \rr_G^{DS} = \ll P_1 \rr_G^{DS} \cup \ll P_2
			\rr_G^{DS}$.
		\item  $\ll \mbox{GRAPH } u \ P_1 \rr_G^{DS}  = 
			\begin{cases} 
				\ll P_1\rr_{graph(u,DS)}^{DS} &\mbox{if } u \in names(DS)\\	
				\{ \}						  & \mbox{if } u \in \U \backslash names(DS)\\
				*					  & \mbox{if } u \in \V
				\end{cases}\\
				*=\{\mu \cup [u \rightarrow s] \mid s \in names(DS), \mu
					\in \ll P_1 \rr^{DS}_{graph(s,DS)} \land [u \rightarrow s] \sim \mu
				\}	$\\ 
				for $u \in \U \cup \V$.

			\item $\ll \mbox{SERVICE } u \ P_1 \rr_G^{DS}  = 
				\begin{cases} 
					\ll P_1\rr_{graph(def,ep(u))}^{ep(u)} &\mbox{if } u \in dom(ep)\\	
					\{ \mu_\emptyset\} & \mbox{if } u \in \U \backslash dom(ep)\\
					*'=	   & \mbox{if } u \in \V
					\end{cases}$\\
					$*'=\{\mu \cup [u \rightarrow s ] \mid s \in dom(ep), \mu
						\in \ll P_1 \rr^{ep(s)}_{graph(def,ep(s))} \land [u \rightarrow s ] \sim \mu
					\}$\\ 
					for $u \in \U \cup \V$.
				\item $\ll (P' \FILTER R) \rr_G = \{ \mu \mid \mu \in \ll P'
					\rr_G \mbox{ and } \mu \models R \}$,
			\end{enumerate}
		where $\mu$ satisfies a filter constraint $R$, denoted $\mu \models R$,
		if one of the following holds:
		\begin{enumerate}
			\item $R$ is $\top$;
			\item $R$ is $x = a$, $x \in dom(\mu)$ and $\mu(x) = a$;
			\item $R$ is $x = y$, $\{x,y\} \subseteq dom(\mu)$ and $\mu(x) =
				\mu(y)$;
			\item $R$ is $bound(x)$ and $x \in dom(\mu)$;
			\item $R$ is a boolean combination of filter constraints evaluating
				to true under the usual interpretation of $\neg,\land$ and
				$\lor$.
		\end{enumerate}
		\end{definition}

		When we don't use the SERVICE or GRAPH operators and assume a graph $G$
		we will implicitly assume a dataset $DS = \{(def,G)\}$ and
		we will denote the evaluation function with $\ll \cdot \rr_G$ instead
		of $\ll \cdot \rr^{DS}_G$.

		We often use the term ``destination'' of a subquery containing the SERVICE or
		GRAPH operator in the top level.
		The destination refers to the URI or variable in between the graph
		pattern and the GRAPH or SERVICE operator.

		\begin{definition}[Destination of a SERVICE- or GRAPH-operator]
			Given a pattern $P$ of the form $P = (\mbox{SERVICE } u \ P_1)$ or
			$P	= (\mbox{GRAPH } u \ P_1)$ we call $u$ the
			destination of the pattern $P$.
		\end{definition}

		\section{The Semantics of the SERVICE and GRAPH Operator}
		We were introduced to the GRAPH and
		SERVICE operators by \cite{w3standard}
		but in the first formalization of it by Buil-Aranda et
		al.~\cite{BuilAranda20131} the definition varied from the standard.
		We tweaked their definitions so it reflected the standard.
		Conversely to our semantic definition of the GRAPH operator in 
		Definition~\ref{def:sparqlsem}, they provided the
		following definition for GRAPH (note the difference in the second case):	
				\begin{enumerate}
					\item  $\ll \mbox{GRAPH } u \ P_1 \rr_G^{DS}  = \begin{cases} 
							\ll P_1\rr_{graph(i,DS)}^{DS} &\mbox{if } u \in names(DS)\\	
							\{ \mu_{\emptyset}\} & \mbox{if } u \in \U \backslash names(DS)\\
							^*
							 & \mbox{if } u \in \V
							\end{cases}$\\
							$^*\big\{\mu \cup [u \rightarrow s ] \mid \exists s \in names(DS), \mu
								\in \ll P_1 \rr^{DS}_{graph(s,DS)} \land [ u \rightarrow s ] \sim \mu
							\big\}$\\ for $u \in \U \cup \V$.
					\end{enumerate}

			%Trying to propose an algorithm for evaluating CONTAINMENT[$S_1$,$S_2$] in the
			%fragment $P_{wdgs}$ and unveils some unintuitive features regarding the semantics of the
			%GRAPH operator. 
			Consider the following example:
			\begin{example}
				$DS=\{(def,G_0), (a,G_1) \}$ where $G_0 = \{
				(c,k,g), (a,k,g) \}$, $G_1$ is arbitrary and  $Q = (GRAPH \ x  \
				(GRAPH  \ c \  P_1)) \ AND \ (x,k,g)$.
			\end{example}

			Evaluating the first part of the query, namely $\ll(GRAPH \ x  \ (GRAPH  \
			c \  P_1))\rr^{DS}_{G_0} $ where $P_1$ could be any arbitrary
			pattern yields an interesting result. Because $c \notin	names(DS)$ 
			and thus $\ll (GRAPH \ c  \ P_1) \rr^{DS}_{G_1} = \mu_\emptyset$  we
			obtain $x \rightarrow a$ ($\mu_\emptyset$ is compatible with every
			mapping). This seems unintuitive because it affects the rest of the query.
			If we would evaluate the right side of the $AND$ we would obtain the
			following: $\ll(x,k,g)\rr^{DS}_{G_0} = \{ (x \rightarrow a) , (x \rightarrow c) \}$ considering the
			left part of the $AND$-Operator we get the result of the query: $\{ (x \rightarrow a )
			\}$. Note that even though the graph with the URI $c$ does not exist and we are not able
			to evaluate pattern $P_1$ over it, we are able to somehow receive results using
			the $AND$ operator. This syntax might be fitting for the SERVICE
			operator where we query SPARQL endpoints in the web which might be
			currently unavailable, but for the GRAPH operator where we query
			different graphs within a reachable endpoint
			this semantics is not needed.
			\bigskip

			\noindent Considering containment/equivalence of two queries which just use the GRAPH
			Operator, there are specific instances where it is not obvious if one query is contained in the other query even though it should be intuitively.

			\begin{example}
				$Q_1$ = (GRAPH x ( GRAPH c ( GRAPH y) $P_1$)),\\ 
				$Q_2$ = (GRAPH y ( GRAPH c ( GRAPH x) $ P_1$))\\
				$?X,?Y \notin P_1$.
			\end{example}

			Approaching this example intuitively, one could say that $\ll Q_1
			\rr^{DS}_{G} = \ll Q_2 \rr^{DS}_G$ for all datasets $DS$ and graphs
			$G$ in $DS$ because $x$ and $y$ get bound to every variable in the dataset because neither
			$x$ nor $y$ occur in $P_1$, the triple pattern at the end
			of the query is the same and also the graph queried second is the same. 
			To show the opposite, just  consider the following dataset:
			$DS=\{(def,G), (a,G_1)\}$. Note that the URI $c$ does not occur in the dataset.
			But this means that $\ll Q_1 \rr^{DS}_G = \{x \rightarrow a\}$ and  $\ll Q_2
			\rr^{DS}_G = \{y \rightarrow a\}$.			
			In the W3C-recommendation\cite{w3standard} 
			the following definition can be found:

			\begin{algorithm}
				\caption{W3C-recommendation on evaluating the GRAPH operator}
			%\begin{lstlisting}
			if IRI is a graph name in D\\
				\quad eval(D(G), Graph(IRI,P)) = eval(D(D[IRI]), P)\\
			if IRI is not a graph name in D\\
				\quad eval(D(G), Graph(IRI,P)) = the empty multiset\\
				\quad eval(D(G), Graph(var,P)) =\\
			Let R be the empty multiset\\
				\quad foreach IRI i in D\\
					\qquad R := Union(R, Join( eval(D(D[i]), P) , $\Omega$(?var->i) )\\
			the result is R
			%\end{lstlisting}
		\end{algorithm}

			If we look at the second case, we can see that the empty multiset (because the
			definition goes with bag semantics)
			is returned if the URI is not a graph name in the Dataset resulting
			in our original definition of SPARQL semantics. 
			
			However remembering the definition of the SERVICE
			operator in Definition~\ref{def:sparqlsem}, we emphasize
			that the second case of the SERVICE-operator (if $c \in \U \backslash
			dom(ep)$ return $\mu_\emptyset$) is intended and correct. It makes
			sure that if a SPARQL endpoint is temporarily unavailable the query
			does not fail.
			
			%Finally I wish to make the remark that bit unintuitive at first that if we have
			%the third case (of both GRAPH and SERVIE), 
			%namely $c \in \V$, the query $P_1$ will not be considered over the
			%default graph. The reason for this can also be found in the definition of the
			%dataset in the standard \cite{w3standard}:
			%``A SPARQL query is executed against an RDF Dataset which represents a collection
			%of graphs. An RDF Dataset comprises one graph, the default graph, which does not
			%have a name, and zero or more named graphs, where each named graph is identified
			%by an IRI. A SPARQL query can match different parts of the query pattern against
			%different graphs as described in section 13.3 Querying the Dataset.''


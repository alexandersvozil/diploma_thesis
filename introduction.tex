\section{Motivation and Problem Statement}
Within the last years more and more Resource Description Framework
(RDF)~\cite{rdf} data is
published on the Web. RDF is an integral part of the semantic web concept
because it allows to express relationships between resources similar to class
diagrams. Like basic grammar, RDF allows to make statements in the form of subject-predicate-object
expressions. These expressions are called ``triples'' in RDF terminology and are
stored in so called RDF graphs, the RDF database.
%%Who formalized this concept first
An important way to access RDF data, i.e., triples, in the internet is the SPARQL Protocal And
RDF Query Language (SPARQL). SPARQL is standardized by the RDF Data Access
Working Group of the World Wide Web Consortium (WWC)~\cite{w3standard}.
%%formalizing this standard
SPARQL is similar to conjunctive queries as it also allows the use of conjunction but SPARQL is even
more powerful: It also allows the use of disjunction and optional patterns and one
can even switch between several SPARQL endpoints and graphs which would mean using several
databases in one query. %with the use of the SERVICE and GRAPH queries.
Optional patterns allow to construct queries where we know that some part of the
queried data is in the RDF database for sure and some may be not. 
An example would be a phone book where some people decide to
only to put their name and telephone number but not all of them put their
address as well. If we want to query all names, telephone numbers and
addresses (if provided) in the database we could query the whole database and put the adress in the
optional pattern. When only the name and telephone number of a person are
specified our query will not fail to give an answer but just return the provided (incomplete)
information. Summing it up we can say that SPARQL offers an interesting set of tools which need to be
formalized~\cite{perez2006semantics} and analyzed thoroughly. For this reason a complexity analysis of
SPARQL~\cite{perez2009semantics} is needed. It is mandatory to extract fragments of SPARQL which are easy to evaluate and thus feasible for practical use~\cite{perez2009semantics}. We also need algorithms which can
analyze if two SPARQL queries always return the same
result~\cite{pichler2014containment}.
If we had two queries and one is slow to evaluate and the other one fast we could
analyze them and if they always return the same results swap out the slow query
for the fast query.
%evtl pichler hier zitieren

%Optional patterns also play a huge role in the complexity of SPARQL as can
%potentially make them very difficult to evaluate. 

\section{Aim of the Work}
The aim of this work is to provide a well-rounded overview of the complexity of
SPARQL. A fragment of SPARQL which is easy to evaluate should be the foundation
of our analysis and well-designed SPARQL 
would lend itself perfectly for this job. The two ``database changing'' tools of SPARQL, i.e., the GRAPH and
SERVICE operator are not yet thoroughly analyzed complexity wise and will be looked into. 
We want to do a statical analysis of the GRAPH and SERVICE operator (analyze if queries using the two
operators always return the same result) and thus it would be very useful to understand
the previous work on statical analysis of well-designed SPARQL. We also want to
examine a way to use the SERVICE operator more generally in practice. Finally we
want to catch a glimpse beyond well-designed SPARQL where some constraints of
well-designedness are relaxed to obtain a more powerful fragment called  weakly well-designed SPARQL.

\section{Methodological Approach}
The methodological approach consists of the following steps:
\begin{enumerate}
	\item \textbf{Literature Research.}\\
		Background information must be gathered which will serve as the
		theoretical basis of our mathematical work.
	\item \textbf{Presentation of Results}\\
		After the literature research the obtained results are discussed and
		presented in detail to give a good understanding of the matter.
	\item \textbf{The mathematical proof}\\
		Obviously to prove any statement we must be able to argue
		logically correct. 
\end{enumerate}
\section{Structure of the Work}
\begin{enumerate}
	\item Introduction
	\item Preliminaries
		\begin{enumerate}
			\item Definitions
			\item Thoughts on the Semantics of the SERVICE and GRAPH operators
		\end{enumerate}
	\item Well-Designed SPARQL
		\begin{enumerate}
			\item Decidable Containment
			\item Undecidable Containment
			\item Equivalence
		\end{enumerate}
	\item Complexity of well-designed SPARQL with GRAPH and SERVICE
		\begin{enumerate}
			\item Translations to well-designed pattern forests
			\item The complexity of EVAL($\mathcal{L}$) where $\mathcal{L} \in \{
				P, \mathcal{U}, \mathcal{S} \}$
			\item Static Analysis
		\end{enumerate}
	\item The SERVICE-operator in Practice
		\begin{enumerate}
			\item Binding the SERVICE-operator
			\item The difference between boundedness and strong boundedness
		\end{enumerate}
	\item Beyond well-designed SPARQL
		\begin{enumerate}
			\item OPT-FILTER-Normal Form and Constraint Pattern Trees
			\item Evaluation of wwd-Patterns
			\item Epressivity of wwd-Patterns and their Extensions
			\item Static Analysis of wwd-Patterns
		\end{enumerate}
\end{enumerate}

\section{State-of-the Art}
The W3C released a recommendation to standardize SPARQL~\cite{w3standardold}. 
In 2006 the authors of~\cite{perez2006semantics} provided a formalization of the SPARQL
language recommendation which is absolutely necessary to conclude any complexity results. 
In the 2009 version of the paper~\cite{perez2009semantics} the authors
provide several complexity results of SPARQL and introduce the
well-designed fragment. It was also found out in~\cite{perez2009semantics} that
the SPARQL evaluation problem using all operators is PSPACE-complete whereas well-designed patterns are
coNP-complete. In~\cite{pichler2014containment}, Pichler and Skritek made a fine
grained analysis of the Containment and Equivalence Problems in the
well-designed SPARQL fragment. Lately well-designed SPARQL was augmented to
weakly-well designed SPARQL~\cite{kaminski_bwd}. Weakly well-designed
SPARQL enables the use of the FILTER operator in the well-designed fragment and
relaxes some constraints of the well-designed fragment. The weakly well-designed
fragment comprises $99\%$ of the queries using the OPT operator, over DBPedia.
DBpedia is a large-scale, multilingual knowledge base extracted from
Wikipedia~\cite{lehmann2015dbpedia} with a SPARQL endpoint which stores data in
RDF format.
The W3C standard was extended in 2011~\cite{w3standard} with the SERVICE
operator. The SERVICE operator was formalized in~\cite{builaranda20131} where
also a practical approach for evaluation was proposed. 

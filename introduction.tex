Within the last years an increasing number of Resource Description Framework
(RDF) \cite{rdf} data has been published on the web. RDF is an integral part of
the semantic web concept~\cite{berners2001semantic}
because one can express relationships between resources similar to class
diagrams. Like basic grammar, RDF allows to make statements in the form of subject-predicate-object
expressions. These expressions are called ``triples'' in RDF terminology and are
stored in so called RDF graphs, the RDF databases.
One of the most cited and important publisher of
RDF data is DBPedia, a large-scale, multilingual knowledge base extracted from
Wikipedia~\cite{lehmann2015dbpedia}. Other important initiatives which use RDF
as a core technology are Open Linked Data~\cite{bizer2009linked,lee2006design}
and Open Government Data~\cite{datagov,datagovuk}. Several research efforts have been pursued
towards understanding RDF and developing specific techniques to efficiently use
it in practice~\cite{abadi2007scalable,weiss2008hexastore,
sidirourgos2008column,schmidt2008experimental,neumann2010rdf,gutierrez2004foundations,hogan2014everything}.

The standardized way to query RDF data on the internet is the SPARQL Protocol And
RDF Query Language (SPARQL).
SPARQL is standardized by the RDF Data Access Working Group of the World Wide
Web Consortium (W3C)~\cite{w3standard} and is a key technology for the
semantic web which can be witnessed by the active development of SPARQL query
engines. Some examples for these systems are AllegroGraph~\cite{allegro}, Apache
Jena~\cite{jena}, Sesame\cite{sesame}
and OpenLink Virtuoso~\cite{virtuoso}. 

A peculiarity of SPARQL is the optional 
matching feature which allows to construct queries where can specify some part
of the queried data which is mandatory to be returned and some optional data
which is optionally returned if it is available. An example would be a phone book where some people decide to
only to put their name and telephone number but not all of them put their
address. If we want to query for maximum information in this database, 
we could ask for the name, the telephone number and put the address into the
optional part. When only the name and telephone number of a person are
specified our query will not fail to give an answer but just return the provided (incomplete)
information. This feature is sought-after when querying data in the semantic web
because everybody has their own policy on how to publish their data.
A lot of work has already been done studying SPARQL and especially regarding 
the complexity of the optional matching
feature~\cite{angles2008expressive,perez2006semantics,perez2009semantics,
schmidt2010foundations,arenas2011querying,kaminski_bwd,kostylev2014semantics}.
Two of the main results of this analysis are, that evaluating general SPARQL is
PSPACE-complete~\cite{perez2006semantics} and equally expressive as relational
calculus~\cite{angles2008expressive,polleres2007sparql}.
Thus a new fragment where the evaluation problem is only coNP-complete called well-designed
SPARQL~\cite{perez2009semantics} was established. It was also found out that the
SPARQL query containment and query equivalence problems are undecidable in the
general SPARQL fragment. Static query analysis of SPARQL, i.e., query containment
and query equivalence is used to optimize queries. Optimizing a query can be
done by replacing a given
query by an equivalent one which possesses superior computational properties.
Obviously this requires a procedure to decide whether two queries are equivalent.
Several papers have been published regarding query containment and query equivalence for intrinsic fragments of SPARQL and optimization~\cite{serfiotis2005containment,wudage2012sparql,letelier2012static,letelier2013static,
schmidt2010foundations,chekol2011psparql,stocker2008sparql,bischof2014schema}. 
The most thorough analysis of the containment and equivalence problems over well-designed SPARQL was done in~\cite{pichler2014containment}.
%%formalizing this standard

The GRAPH operator, which is a feature of SPARQL 1.0, can be used to query
multiple local RDF graphs in a single query. Lately the number of SPARQL endpoints
on the internet grew and with this growth the need for federated
SPARQL queries also rose. This led to the introduction of the SPARQL 1.1 Federated Query
extension~\cite{w3fed}. The Federated Query Extension extension brought up the SERVICE operator. 
Similar to the GRAPH operator, the SERVICE operator is
used to query multiple SPARQL endpoints on the internet in a single query. 
A big effort has already been put into analyzing the operators, benchmarking
implementations of federated query processing
and querying distributed RDF data sources with SPARQL in
general~\cite{BuilAranda20131,schwarte2011fedx,quilitz2008querying,
buil2014towards,montoya2015federated,
saleem2013daw,acosta2011anapsid,gorlitz2011splendid,
hose2012towards,buil2014strategies} but the complexity 
analysis of the GRAPH and SERVICE operators remains an open question.

Using the SERVICE operator in practice elicits a lot of new difficulties.
As described the SERVICE operator allows to evaluate a query over multiple
endpoints in the web in just a single query. An example for a SPARQL graph
pattern using the SERVICE operator would be $Q = (\mbox{SERVICE } u \ P_1)$.
The symbol $u$ could be a variable or a Uniform Resource Identifier (URI), which is a
string used to identify a resource. The symbol $P_1$ stands for another SPARQL
query. The SERVICE operator is supposed to evaluate the query $P_1$ over the
SPARQL endpoint defined by $u$. If $u$ is a URI, the endpoint the query $P_1$ gets
evaluated over is the one identified by the URI $u$. 
If $u$ is a variable the semantic of the query $Q$ is not defined by the
W3C standard~\cite{w3standardservice}. 
Obviously the query cannot be evaluated over all the possible endpoints on the internet, so we need to bind the variable to finitely many URIs. 
In~\cite{BuilAranda20131} two notations called
boundedness and strong boundedness are used to cope with this problem.
Deciding boundedness of a variable in a query is undecidable whereas deciding strong
boundedness of a variable in a query is doable in polynomial time. Even though
these two notions have been brought up for practical use, the difference of the two remains
unknown.

We will analyze the evaluation problem of SPARQL with the SERVICE and GRAPH
operator but we will restrict ourselves to well-designed SPARQL. The reason for
this restriction is that we don't want the complexity of full SPARQL to cloud
the complexity of the SERVICE and GRAPH operators. We will thus extend the
well-designed SPARQL fragment to a new fragment which additionally allows the
free use of the SERVICE and GRAPH operator. Then we will analyze the complexity
of the evaluation problem in this fragment. We will also discuss the difference between
boundedness and strong boundedness and provided an example of a query where a
variable is bounded but not strongly bounded. We will also conjecture how a
procedure to decide boundedness could look like. Even though it was shown
in~\cite{BuilAranda20131} that deciding if a variable is bounded in a pattern is
undecidable for general SPARQL it might be decidable for fragments of SPARQL
where the query equivalence problem is not undecidable.

While writing the thesis another interesting fragment was published, namely
weakly well-designed SPARQL, which further extends well-designed SPARQL by
relaxing several restrictions without increasing the
complexity~\cite{kaminski_bwd}. We will also look into 
weakly well-designed SPARQL. 

After the introduction, Chapter 2 provides the most important basic definitions
(like conjunctive queries, RDF and SPARQL). In the second part of Chapter 2 
we will discuss the peculiarities of the semantics of the SERVICE operator. In
Chapter 3 well-designed SPARQL is introduced and
we look into the results of~\cite{pichler2014containment}. The proofs of the
results in~\cite{pichler2014containment} provide
useful methods for proving the complexity results with GRAPH and SERVICE. Following up
in Chapter 4 we analyse the evaluation problem of well-designed SPARQL which includes 
the GRAPH and SERVICE operator. In Chapter 5 we study how SERVICE queries could be evaluated in practice. 
The second part of Chapter
5 discusses the difference of boundedness and strong boundedness. In 
Chapter 6 we will look at weakly well-designed SPARQL which was introduced in~\cite{kaminski_bwd}.

%The W3C released a recommendation to standardize SPARQL~\cite{w3standardold}. 
%In 2006 the authors of~\cite{perez2006semantics} provided a formalization of the SPARQL
%language recommendation which is absolutely necessary to conclude any complexity results. 
%In the 2009 version of the paper~\cite{perez2009semantics} the authors
%provide several complexity results of SPARQL and introduce the
%well-designed fragment. It was also found out in~\cite{perez2009semantics} that
%the SPARQL evaluation problem using all operators is PSPACE-complete whereas well-designed patterns are
%coNP-complete. In~\cite{pichler2014containment}, Pichler and Skritek made a fine
%grained analysis of the Containment and Equivalence Problems in the
%well-designed SPARQL fragment. 
%The W3C standard was extended in 2011~\cite{w3standard} with the SERVICE
%operator. The SERVICE operator was formalized in~\cite{builaranda20131} where
%also a practical approach for evaluation was proposed. 

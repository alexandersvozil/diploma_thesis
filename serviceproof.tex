We will introduce an equivalent definition of wdPTs in this section. The
main structural difference to Definition~\ref{def:pt} is that the nodes in the
tree are labelled with a set of relational atoms over a schema instead of a set of triples. This different but
equivalent definition will be used to prove complexity results for the fragment
$P_{wdgs}$. 

\begin{definition}[wdPTs~\cite{barcelo2015efficient}]\label{def:wdpt}
	A wdPT over a relational schema $\sigma$ is a tuple $(T, \lambda, \overline{x})$
	such that the following holds:
	\begin{enumerate}
		\item $T$ is tree rooted in a distinguished node $r$, the root and $\lambda$
			maps each node $t$ in $T$ to a set of relational atoms over $\sigma$.
		\item For every variable $y$ that appears in $T$, the set of nodes of $T$ where
			$y$ is mentioned is connected.
		\item We have that $\overline{x}$ is a tuple of distinct variables occurring in
			$T$. They are the free variables of the wdPT.
	\end{enumerate}
\end{definition}

\begin{definition}
	A wdPT $(T,\lambda, \overline{x})$ is called projection-free if $\overline{x}$
	contains all variables mentioned in $T$.
\end{definition}

\begin{definition}\label{wdptq}
	Assume $p = (T,\lambda,\overline{x})$ is a wdPT over $\sigma$. We write $r$ to
	denote the root of $T$. Given a subtree $T'$ of $T$ rooted in $r$ we define
	$q_{T'}$ to be the CQ $Y \leftarrow R_1(\overline{v_1}), \dots,
	R_m(\overline{v}_m)$, where the $R_i(\overline{v_i})$'s are the reational atoms
	that label the nodes of $T'$, i.e., 
	\begin{align*}
		\{ R_1(\overline{v}_1), \dots, R_m(\overline{v_m}) \} = \bigcup_{t \in T'} \lambda(t) 
	\end{align*} and $\overline{y}$ are all the variables that are mentioned in
	$T'$. 
\end{definition}

The main idea of this equivalent but different semantics of a wdPT is to look at each subtree $T'$ of
$T$ rooted in $r$. As mentioned above, each of them describes a pattern, i.e.,
the conjunctive query CQ $q_T'$. A mapping $h$ satisfies $(T,\lambda)$ over
a database $D$ if $h$ satisfies the pattern defined by a subtree $T'$ and there
is no subtree $T''$ which is bigger than $T'$ and $h$ can be extended to satisfy
$T''$.
\begin{definition}[Semantics of wdPTs~\cite{barcelo2015efficient}]
	Let $p=(T,\lambda,\overline{x})$ be a wdPT and $D$ a database over $\sigma$.

	\begin{enumerate}
		\item A homomorphism from $p$ to $D$ is a partial mapping $h: X \rightarrow U$,
			where $X$ is an infinite set of a variables and $U$ an infinite set of
			constants, for which it is the case that there is a subtree $T'$ of $T$ rooted
			in $r$ such that $h \in q_{T'}(D)$.
		\item The homomorphism $h$ is maximal if there is no homomorphism $h'$ from $p$
			to $D$ such that $h \sqsubset h'$.
	\end{enumerate}
	The evaluation of wdPT $p = (T,\lambda,\overline{x})$ over $D$ denoted $p(D)$,
	corresponds to all mappings of the form $h_{\overline{x}}$, such that $h$ is a
	maximal homomorphism from $p$ to $D$.
\end{definition}

It is important to notice that wdPTs properly extend CQs.
Given a CQ $q(x)$ of the form $X \leftarrow R_1(v_1),\dots,R_m(v_m)$
it is easy to see that $q(x)$ is equivalent to the wdPT $p = (T,\lambda,x)$,
where $T$ consists of a single node $r$ and $\lambda(r) =
\{R_1(v_1),\dots,R_m(v_m)\}$. In other words, $q(D) = p(D)$ for each database
$D$. Further on we will not distinguish between a CQ and the single node wdPT
that represents it. wdPTs can on the other hand represent interesting properties
that cannot be expressed as CQs, which we will discuss in the following example:
\begin{example}
	$Q =  \bigg(((x,recorded\_by,y) \ AND \ (x,published, ``after\_2010'')) \ \\ OPT \ (x,
	NME\_rating,z)\bigg) \ OPT \ (y, formed\_in, z') $

	\noindent Consider now the following RDF database $D$ consisting of the triples:
	\begin{verbatim}
	(``title_1'' recorded_by, ``studio_1''),
	(``title_1'' published, ``after_2010''),
	(``title_2'' recorded_by, ``studio_1''),
	(``title_2'' published, ``after_2010''),
	(``title_2'' NME_rating, ``2'')
	(``studio_1'' formed_in, ``miami''),
	\end{verbatim}
	Evaluating the query $Q$ over $D$ results in two mappings $\mu_1$ and $\mu_2$:
	$\mu_1 = \{ x \rightarrow ``title\_1'', y \rightarrow ``studio_1'',z'
	\rightarrow ``miami'' \}$ 
	$\mu_2 = \{ x \rightarrow ``title\_2'', y \rightarrow ``studio_1'',z\rightarrow
	``2'', z' \rightarrow ``miami'' \}$ 
\end{example}

To capture the UNION operator the definition of well-designed pattern trees
need to be modified.  
\begin{definition}[Unions of wdPTs or well-designed pattern Forests (wdPF)]
	A Union of wdPTs is an expression $\phi$ of the form $\bigcup_{1\leq i \leq n} p_i$, 
	where each $p_i$ is a wdPT over $\sigma$.
	We denote $\varphi(D)$ as the evaluation of $\phi$ over database $D$.
	It corresponds to the set $\bigcup_{1\leq i \leq n}p_i(D)$.
	Unions of wdPTs are also called well-designed pattern
	forests(WDPF).
\end{definition}



First we are going to show an easy example of how to translate a graph pattern
only using AND and triple patterns to a conjunctive query. Then we propose a
polynomial time translation from a graph pattern $P \in P_{wdgs}$ to $Q \in P_{wd}$. For this
translation we need to construct a special database and a wdPT depending on the
original query. After we established the translation we prove the equivalence of $P$ and $Q$ in
Theorem~\ref{biglemma}. The last section deals with the problem
\textbf{EVAL}($P_{wdgs}$.
%, \textbf{CONTAINMENT}($P_{wdgs}$),
%\textbf{EQUIVALENCE}($P_{wdgs}$) and \textbf{SUBSUMPTION}($P_{wdgs}$).

\section{Translations to well-designed pattern forests}
It is an easy observation that if we restrict SPARQL to the AND operator, we
simply get conjunctive queries without existentially quantified variables.
We will show this in Example~\ref{exconjq}. 
\begin{example}\label{exconjq}
	Consider the following SPARQL default Graph $G$ in a dataset $DS$ with the query $Q$ in $SPARQL[\land]$
	\begin{align*}
		G &=\{ (a,a,a), (b,c,c), (b,c,a)  \}\\
		Q &= (x,y,a) \ AND \ (z,c,c) \ AND \ (x,c,y)
	\end{align*}
	Let $T$ be a three-ary relation, $D = \{ T \}$ be the following database and
	$CQ$ be the following 
	conjunctive query: 
	\begin{align*}
		T &= \{ (a,a,a), (b,c,c), (b,c,a)\}\\
		CQ &= ans(x,y,z) \leftarrow T(x,y,a), T(z,c,c), T(x,c,y)\\
	\end{align*}
	It is easy to see that the mappings in $CQ(D)$ are the same as in $\ll Q
	\rr_G^{DS}$.
\end{example}
Similar to Example~\ref{exconjq} we are going to transform the dataset and query 
not into conjunctive queries but an extension of them: well-designed pattern
trees. We 
will define a polynomial time translation from a graph pattern $P \in P_{wdgs}$
to a well-designed pattern tree. This enables us to use well known algorithms 
for which the computational complexity is known. 

\subsection{Creating the database}
We first describe the function $data$ which transforms a dataset into the
database.

Consider the function $data: DS \mapsto D$, where $DS$ is a
dataset and $D$ is a database.
$data$ then is defined as follows:
Let $DS$ be an arbitrary dataset and $u_{DS}$ the URI such that $ep(u_{DS}) = DS$.
\begin{align*}
	DS=\{(def,G),(u_1,G_1),\dots,(u_n,G_n)\}
\end{align*}
The output of $data$ is the database $D = \{ T,LOC\}$ where $T$ is a 5-ary relation containing all the triples of a graph, the corresponding graph URI and the dataset URI. The binary relation $LOC$
captures all graph URIs and their corresponding dataset URI. 
For our dataset this would mean, assuming 
\begin{align*}
	G &= \{(x_1,y_1,z_1), \dots, (x_a,y_a,z_a)\},\\ 
	G_1 &= \{(x_{11}, y_{11},z_{11}), \dots, (x_{1b},y_{1b},z_{1b}) \},\dots,\\ 
	G_n &= \{(x_{n1},y_{n1},z_{n1}),\dots,(x_{nc},y_{nc},z_{nc})\}
\end{align*} that we construct our output database $D$ as follows:
\begin{align*}
	T &= \{ (u_{DS},def,x_1,y_1,z_1), (u_{DS},def,x_a,y_a,z_a), \dots, (u_{DS},u_1,x_{11}, y_{11},z_{11}),
	\dots,\\& (u_{DS},u_1,x_{1b},y_{1b},z_{1b} ), \dots,
(u_{DS},u_n,x_{n1},y_{n1},z_{n1}), \dots (u_{DS},u_n,x_{nc},y_{nc},z_{nc})\}&. \\
	LOC &= \{ (u_{DS},def),(u_{DS},u_1),\dots (u_{DS},u_n) \}&.
\end{align*}

\subsection{Transforming the pattern to a wdPT}
We proceed in defining the function \textit{trans} which will in polynomial time 
transform a graph pattern in $P_{wdgs}$ into a well-designed pattern tree with the same meaning without the
SERVICE and GRAPH operators. Lemma~\ref{smalllemma} concludes that the output pattern and the
input pattern are equivalent.

\bigskip\noindent
The transformation function $trans: P \times \U\cup\{\V\} \times \U \cup \{def \}\cup\{\V\}
\mapsto Q$ takes  three parameters as input: 
$P$ is a graph pattern in OPT normal form which allows the usage of AND, OPT, GRAPH, SERVICE and
UNION, $\U\cup\V$ is the infinite
set of URIs with the infinite set of variables and $\U\cup\{def\}\cup \V$ is the infinite set of
URIs containing the $def$ identifier conjoined with the infinite set of
variables. The output $Q$ is a well-designed pattern tree. 

\bigskip
\noindent
Assume the input $(P,ds,g)$ and let each of the parameters be
arbitrary. 
\begin{enumerate}
	\item If $P$ is a triple pattern $(u,v,w)$,  
		\begin{align*}
		trans(P,ds,g) = vars(ds,g,u,v,w) \leftarrow T(ds,g,u,v,w).\\	
		\end{align*}

	\item If $P$ is $(P_1  \AND  P_2)$, let
		\begin{align*}
			&trans(P_1,ds,g)	= O_1 \leftarrow q_1,\\
			&trans(P_2,ds,g)	= O_2 \leftarrow q_2\\
			&trans(P,ds,g)		= O_1\cup O_2 \leftarrow q_1, q_2.
		\end{align*}
	\item If $P$ is $(P_1  \OPT  P_2)$, let\\
		\begin{align*}
			&trans(P_1,ds,g) =  (T_1, \lambda_1,x_1) \mbox{ and }\\
			&trans(P_2,ds,g) = (T_2, \lambda_2, x_2).
		\end{align*}
		$trans(P,ds,g) = (T,\lambda,x)$ for which $T = T_1 \cup T_2 \cup (r_1,
		r_2)$ where $r_1,r_2$ are the roots of $T_1,T_2$ respectively,
		$\lambda = \lambda_1 \cup \lambda_2$ and $x = x_1 \cup x_2$.

	\item If $P$ is $(\mbox{GRAPH} \ u \ P_1)$, let\\
		$(T_1,\lambda,x_1) = trans(P_1,ds,u)$.	
		Assuming $r_1$ is the root of $T_1$,
		and $\lambda(r_1) = q_1$ we define \[ \lambda'(x) =\begin{dcases*} 
				q_1, LOC(u,ds),LOC(g,ds)& if $x = r_1$\\
				\lambda(x) & otherwise	\\
			\end{dcases*}
		\] and $trans(P,ds,g) = (T_1,\lambda',x_1)$.

	\item $P$ is of the form $(\mbox{SERVICE} \ u \ P_1)$. 
		Case distinction:
		\begin{enumerate}
			\item If $u \in \U$ and $u \notin dom(ep)$:
				$trans(P,ds,g) =  \{\}\leftarrow$.
			\item Otherwise let $(T_1,\lambda,x_1) = trans(P_1,u,g)$.
				Assuming $r_1$ is the root of $T_1$,
				and $\lambda(r_1) = q_1$ we define \[ \lambda'(x) =\begin{dcases*} 
				q_1, LOC(def,u),LOC(g,ds)& if $x = r_1$\\
				\lambda(x) & otherwise	\\
			\end{dcases*}
		\]  and $trans(P,ds,g) = (T_1,\lambda',x_1)$.
		\end{enumerate}

%	\item If $P_1$ and $P_2$ are graph patterns and $P$ is $(P_1 \UNION  P_2)$,
%		then $P_1$ must correspond to a well formed pattern forest $Q_1 =
%		trans(P_1,ds,g)$ and $P_2$ to a
%		well designed pattern forest $Q_2 = trans(P_2,g,ds)$. 
%		Define $Q = Q_1 \cup Q_2$.  
\end{enumerate}
Observe that the function is well-defined since only patterns $P_{wdgs}$ are
considered: This allows us to argue over conjunctive queries for the case $P$ is $P_1
\AND P_2$ as the AND-operator may not occur in the scope of an OPT-operator in the fragment
$P_{wdgs}$. \\
Towards our goal to show that for all datasets $DS$ identified by URI
$ds$ and graph patterns $P$ the following property holds: 
Let $Q = trans(P,def,ds)$. Then $Q(D) = \ll P
\rr^{DS}_{graph(def,DS)}$, where $D=\bigcup\limits_{c \in dom(ep)} data(ep(c))$, we prove Lemma~\ref{smalllemma} first.

\begin{lemma}\label{smalllemma}
	Let $P \in P_{wdgs}$, $DS$ a dataset so that $ep(ds) = DS$ and $G$ a graph in
	the dataset $DS$ so that $(g,G) \in DS$. 
	Let $D = \bigcup\limits_{c \in dom(ep)} data(ep(c))$.
	Let $Q = trans(P,g,ds)$.
	Then 
	\begin{align*}
		Q(D) = \begin{dcases*}
			\ll P \rr^{ep(ds)}_{graph(g,ep(ds))} 
			& if $g \in \U,ds \in \U$\\
			\bigcup\limits_{ds' \in dom(ep)} \{ \mu \cup [ds \mapsto ds'] \mid\\ \mu \in
			\ll P\rr^{ep(ds')}_{graph(g,ep(ds))}, \mu \sim 	[ds\mapsto ds'] \} 
			&if $g \in \U,ds \in \V$\\
			\bigcup\limits_{g' \in names(ep(ds))}\{ \mu \cup [g \mapsto g']
				\mid\\ \mu \in
			\ll P\rr ^{ep(ds)}_{graph(g',ep(ds))} , \mu \sim [g\mapsto g'] \} 
			& if $g \in \V,ds \in \U$ \\
			\bigcup\limits_{ds' \in dom(ep), g' \in names(ep(ds'))} \big\{ \mu \cup
				\{[ds\mapsto ds'],[g \mapsto g']\} \mid\\ \mu \in
				\ll P\rr^{ep(ds')}_{graph(g',ep(ds'))}, 
				\mu \sim
			\{[ds\mapsto ds'], [g \mapsto g']\}\big\} 
			& if $g \in \V,ds \in \V$\\
		\end{dcases*}
	\end{align*}
	holds.
\end{lemma}
\begin{proofidea}
	The four different cases in the lemma are needed to distinguish if we are currently inside a GRAPH 
	pattern, inside a SERVICE pattern or inside both a GRAPH and a SERVICE
	pattern where the destination was a variable. As the evaluation operator of
	SPARQL, asssuming $P$ is a graph pattern, $\ll P \rr^{DS}_G$ only allows $DS$ to be a dataset and $G$ to be
	a graph in the dataset $DS$, we utilize the union operator to simulate the
	evaluation of $P$ over all possible datasets and their respective graphs if we are
	inside both a GRAPH and a SERVICE operator (last case). If the pattern is only inside a GRAPH
	operator the evaluation of $P$ over all the graphs in the current dataset is
	simulated (third case). If the pattern is only inside a SERVICE operator the evaluation
	of $P$ over all datasets in $range(ep)$ is simulated (second case). After receiving each
	result mapping of $P$ over the corresponding dataset and graph we add
	and additional mapping to the resultmapping, depending again if we are inside a SERVICE operator, a
	GRAPH operator or inside both: This mapping has the variable(s) of
	either the GRAPH operator or SERVICE operator or both in the domain. The
	variable(s) map(s) to the current URI(s) of the dataset(second case) or
	graph(third case) or both(fourth case). If this mapping is compatible with
	our resultmapping the union of both mappings is added.

	The statement is shown with an induction over the
	structure of a graph pattern.
	Because of the construction of $P_{wdgs}$, which doesn't allow
	that an OPT operator occurs in the scope of an AND operator we are able to
	argue over conjunctive queries in the cases where $P$ is a triple pattern or
	$P$ is of the following form $P = (P_1 \AND P_2)$. In all the other cases we
	need to argue over well-designed pattern trees. In all cases we need to
	distinguish if we are inside a GRAPH operator, a SERVICE operator,in
	both or in none. The long proof is due to this distinction.
	The cases were $P$ is a triple pattern or a pattern of the form $P=(P_1 \AND
	P_2)$ is described in Example\ref{exconjq}. The only difference is that we
	have now have to consider the cases where we are inside a GRAPH operator, a SERVICE
	operator or both. For this reason we are using 5-tuples in our $T$ relation
	of our database, where the first two positions of the tuple describe the
	dataset we are currently in and the graph of the dataset we want to evaluate
	the tuple over. For the case that $P$ is $(P_1 \OPT P_2)$ we merge the wdPT
	of $P_1$ and the wdPT of $P_2$ so that the roots of the two wdpts are
	connected. We then use the semantics of wdpts to show that this ``joining''
	of trees models the semantics of OPT. 
	The GRAPH and the SERVICE operators are similary designed:
	The query $P_1$ inside the GRAPH or SERVICE operator is evrluated by
	$trans(P_1,ds,u)$ (for GRAPH, ds is the current dataset URI) and $trans(P_1,u,def)$ for SERVICE to a wdPT $Q_1$. To the root of $Q_1$ we add two relational
	atoms. For the GRAPH operator $LOC(u,ds)$ and $LOC(g,ds)$ are added and
	needed in the proof to justify that if $u$ and $ds$ are URIs there is a
	dataset $ds$ so that $ds \in dom(ep)$ and $u \in names(ds)$. $LOC(g,ds)$ is
	needed in the case that $g$ is a variable and we have a nested occurence of
	GRAPH (e.g. $(\mbox{GRAPH} g (\mbox{GRAPH} u P_1)$). The mapping of $\mu(x)$ would
	otherwise be lost in the query. The same arguments are there for SERVICE
	except by semantics of the SERVICE operator $LOC(def,u)$ is needed because
	SERVICE always changes to the default graph of a dataset.
	The only difference in GRAPH and SERVICE is that again by semantics of
	SERVICE we need to return the empty mapping instead of the empty set when an endpoint is not
	reachable or it doesn't exist, which is modelled by $u \notin dom(ep)$. We
	thus need to check the $ep$ function if the URI is in the domain of $ep$ and
	if it isn't return $\{\} \leftarrow$.
\end{proofidea}

The full proof of Lemma~\ref{smalllemma} can be found in the Appendix. 

The next step is to show that our well-designed pattern tree that we received
from our $trans$ function is indeed a pattern tree, as it was defined by Definition~\ref{def:wdpt}.

\begin{lemma}
	Let $ds \in dom(ep)$, $g \in names(ep(ds))$ and $P \in P_{wdgs}$ be a SPARQL pattern.
	Then $Q = trans(P,ds,g)$ is a well-designed pattern tree. 
\end{lemma}

\begin{proof}
	We prove each property of Definition~\ref{def:wdpt}seperately:
	\begin{enumerate}
		\item $T$ is rooted in a distinguished node $r$, the root and maps
			each node $t$ in  $T$ to a set of relational atoms over $\sigma$:
			This is easy to see as we only use $LOC$ and $T$ in our
			construction. Also $T$ always has a distinguished node $r$ as root
			because the only step of $trans$ that changes the structure of $T$
			is if a subppattern of the form $(P_1 \OPT P_2)$ occurs: 
			$Q_1 = (T_1,\lambda_1,x_1)  = trans(P_1,ds,g)$,
			$Q_2 = (T_2,\lambda_2,x_2)  = trans(P_2,ds,g)$. 
			It connects the roots of $T_1$ and $T_2$, call them $r_1,r_2$ with
			an edge $(r_1,r_2)$ making $r_1$ the new root.  

		\item For every variable $x$ that appears
			in $T$, the tree of the wdPT, the set of nodes of $T$ where $x$ is mentioned is connected.
			It is important to remember that we assumed that our inputpattern $P
			\in P_{wdgs}$ and is well-designed.
			The induction will be over the structure of the subpatterns $\hat{P}$ of $P$.
			\begin{enumerate}
			
				\item Basecase: If $\hat{P}$ is a triple pattern we only return one node, so there can't be any
					violation of the property.

				\item Induction Step: If $\hat{P}$ is $(P_1 \AND  P_2)$ there can 
					also never be a violation of the
					property because we assumed that $P$ is in $P_{wdgs}$ and thus OPT never
					occurs in the scope of AND. We thus also only return a single node making a
					violation impossible.

				\item Induction Step: If $\hat{P}$ is $(P_1 \OPT  P_2)$:
					By induction hypothesis $Q_1 = trans(P_1,ds,g)$ and $Q_2 =
					trans(P_2,ds,g)$ fulfill the property and $P \in P_{wdgs}$ 
					and is a well-designed SPARQL pattern by assumption. 

					Assume that making $T_2$ a child of the root of $T_1$ 
					results in a wdPT $Q$ which does not fulfill the well-designedness property.
					There must thus be a variable $x$ in $T$ (the roots of $T_1$
					and $T_2$ get connected) which appears in two different subgraphs of $T$. This
					two subgraphs can only be situated in $T_1$ and $T_2$ respectively by induction
					hypothesis.	Because these subgraphs are not connected 
					by our assumption we proceed by	case distinction and assume 
					\begin{enumerate}
						\item $T_1$ contains a subgraph containing $x$ but the root $r_1$ does not
							contain the variable $x$. There must be a subpattern of $P_1$ by our
							construction $(P' \OPT P'')$ where $P''$ contains $x$ but $P'$ doesn't. 
							As $(P' \OPT P'')$ is part of $P_1$ 
							and we have $(P_1 \OPT P_2)$ and $P_2$ contains $x$ we have a
							contradiction to the assumption that $P$ is a well-designed SPARQL
							patterns.
						\item $T_2$ contains a subgraph containing $x$ but the root $r_2$ does not
							contain the variable $x$. This can be shown
							analogously. %Analogously to the previous case.
					\end{enumerate}


					%If we then make the wdPT $Q_2$ a child of the wdPT $Q_1$ we know 
					%that $Q$ fulfills the property. 

				\item Induction Step: If $\hat{P}$ is $(\mbox{GRAPH }  u \ P_1)$:
					We know that $trans(P_1,ds,u) = (T, \lambda,x)$ 
					is part of the endresult $Q$ and by induction hypothesis,
					we know that $(T, \lambda,x)$ fulfills the property. Let
					$r_1$ be the root of $T$. $trans(\cdot)$ adds the conjunct
					$LOC(u,ds)$ and $LOC(g,ds)$ to $r_1$. W.l.o.g. assume $u$,$g$
					and $ds$ are variables. For every of those three variables it
					remains to prove that the property holds.
					Let $x \in \{u,g,ds\}$ and assume that our resulting wdPT $Q$ is
					not fulfilling the property. There must thus be a node $n_1$ in $Q$ which
					doesn't contain $x$ and a node $n_2$ for which $n_1$ is a parent 
					which again contains $x$ creating the conflict.
					By the definition of the function $trans(\cdot)$ we know that
					there must have been a subpattern $(P' OPT P'')$ and both $n_1,n_2$ must
					have been created by this subpattern. But this again means, that
					$P'$ did not contain $x$ and $P''$ did contain $x$. 
					%As the root
					%of $Q$, i.e. $r_1$, cannot be $n_1$, we have that there must be
					%another OPT subpattern making $r_1$ an ancestor of $n_1$.(not
					%needed)
					Depending on whether $x = u$, $x=g$ or $x=ds$ we have a
					contradiction:
					\begin{enumerate}
						\item Let $x=u$. Because we assumed $\hat{P}=
							(\mbox{GRAPH } u \ P_1)$ we know that
							$P$ is not well-designed because $P_1$ contains the
							subpattern $(P' \OPT P'')$.
						\item Let $x = g$. This means that we are inside a graph
							pattern $(\mbox{GRAPH } g \ P^\sim)$ and thus $P$ is
							not well-designed because $P_1$ contains the
							subpattern $(P' \OPT P'')$.
						\item Let $x = ds$. This means that we are inside a
							SERVICE	pattern\\ $(\mbox{SERVICE } g \ P^\sim)$ and thus $P$ is
							not well-designed because $P_1$ contains the
							subpattern $(P' \OPT P'')$.
					\end{enumerate}



				\item Induction Step: If $\hat{P}$ is $(\mbox{SERVICE } u \ P_1)$:
						The proof is analogously to case where $\hat{P}$ is
						$(\mbox{GRAPH} u \ P_1)$.
				%	We know that $Q = trans(P_1,ds,u)$ and by induction hypothesis,
				%	we know that $Q = (T, \lambda,x)$ fulfills the property. Let
				%	$r_1$ be the root of $T$. $trans(\cdot)$ adds the conjunct
				%	$LOC(def,u)$ and $LOC(g,ds)$ to $r_1$. W.l.o.g. assume $u$,$g$
				%	and $ds$ are variables. For every of those three variables it
				%	remains to prove that the property holds.
				%	Let $x \in \{u,g,ds\}$ and assume that our resulting wdPT $Q$ is
				%	not fulfilling the property. There must thus be a node $n_1$ in $Q$ which
				%	doesn't contain $x$ and a node $n_2$ for which $n_1$ is a parent 
				%	which again contains $x$ creating the conflict.
				%	By the definition of the function $trans(\cdot)$ we know that
				%	there must have been a subpattern $(P' \OPT P'')$ and both $n_1,n_2$ must
				%	have been created by this subpattern. But this again means, that
				%	$P'$ did not contain $x$ and $P''$ did contain $x$. 
				%	%As the root
				%	%of $Q$, i.e. $r_1$, cannot be $n_1$, we have that there must be
				%	%another OPT subpattern making $r_1$ an ancestor of $n_1$.(not
				%	%needed)
				%	Depending on whether $x = u$, $x=g$ or $x=ds$ we have a
				%	contradiction:
				%	\begin{enumerate}
				%		\item Let $x=u$. 
				%			Because we assumed $\hat{P}= (\mbox{SERVICE } u \ P_1)$ we know that
				%			$P$ is not well designed because $P_1$ contains the
				%			subpattern $(P' \OPT P'')$.
				%		\item Let $x = g$. This means that we are inside a graph
				%			pattern $(\mbox{GRAPH } g \ P^\sim)$ and thus $P$ is
				%			not well designed because $P_1$ contains the
				%			subpattern $(P' \OPT P'')$.
				%		\item Let $x = ds$. This means that we are inside a
				%			SERVICE
				%			pattern $(\mbox{SERVICE } g \ P^\sim)$ and thus $P$ is
				%			not well designed because $P_1$ contains the
				%			subpattern $(P' \OPT P'')$.
				%	\end{enumerate}

			%	\item Induction Step: If $P$ is $(P_1 \UNION  P_2)$:
			%		We get a forest and because the property held for $Q_1$ and $Q_2$ by i.h. we are
			%		done.
			\end{enumerate}
		\item The last property is that $x$ is a tuple of distinct variables
			occurring in $T$. As we don't use projection and use set operations
			to merge our free variables in the OPT case, this is an obivous
			observation.
	\end{enumerate}
\end{proof}


We will now write our final result: When we have a graph pattern $P$ in
$P_{wdgs}$ with a dataset $DS$ and a function $ep$ we want to transform it into a
wdPT $Q$ and database $D$ with our function $trans$ so that
$\ll P \rr^{DS}_def = Q(D)$.
\begin{theorem}\label{biglemma}
    Let $P$ be a graph pattern in $P_{wdgs}$, $DS$ a dataset and $G$ a graph in $DS$.
    Let $DS = ep(ds)$  and $G = graph(g,DS)$. Let $Q = trans(P,ds,g)$ 
    and $D = \bigcup\limits_{c \in dom(ep)} data(c)$. Then $\ll P \rr^{DS}_G = Q(D)$.
\end{theorem}
\begin{proof}
	The database $D$ is the same database that is created in
	Lemma~\ref{smalllemma}.
	Use Lemma~\ref{smalllemma}: 
	$DS = ep(ds)$  and $G = graph(g,DS)$ hold and thus $ds,g \in \U$:
	$\ll P \rr^{DS}_G = Q(D)$ follows.
\end{proof}

%For the static analysis we dont have $DS$ as input, so we want to
%generalize the theorem in the following way:
%\begin{theorem}\label{biglemma}
%	Let $P$ be a graph pattern in $P_{wdgs}$, 
%	$DS$ be an arbitrary dataset in $img(ep)$.
%	Let $x \in \V$. Then $Q = trans(P,x,def)$ be the wdPT and 
%	$D = \bigcup\limits_{c \in dom(ep)} data(c)$. 
%	Then $\ll P \rr^{DS}_G = Q(D)$.
%\end{theorem}

\section{The complexity of EVAL($\l$) where $\l \in \{P,\u,\s\}$ }

\begin{corollary}
	The problem $\mathit{\mathbf{EVAL(P_{wdgs})}}$ is coNP-complete.
%	$\mathit{\mathbf{EVAL(\u_{wdgs})}}$ are coNP-complete and
%	$\mathit{\mathbf{EVAL(\s_{wdgs})}}$ is $\Pi^P_2$-complete.
\end{corollary}
\begin{proof}\label{proof:p_wdgs}
	The problem \textbf{EVAL} is defined in the following way:
\begin{framed}\noindent \textbf{EVAL($\mathcal{L}$)}\\
	\textbf{INPUT:} Dataset $DS$, graph pattern $P \in  \mathcal{L}$ and a mapping $\mu$.\\
	\textbf{QUESTION:} Is $\mu$ in $\ll P \rr^{DS}_{def}$.
\end{framed}
Hardness of the problem follows immediately from the hardness of 
\textbf{EVAL($P_{wd}$)}.
Assume $ds = ep^{-1}(DS)$. %and $G = graph(def,DS)$.
For the membership we propose the following procedure:
Use the transformation function on the input graph pattern $P$ to obtain a wdPT
$Q$, and the data function to obtain the database $D$. More formally:
$Q = trans(P,ds,def)$ and $D = \bigcup\limits_{c\in dom(ep)} data(ep(c))$. This transformation is obviously
possible in polynomial time if we assume $dom(ep)$ to be finite. We know by Theorem~\ref{biglemma} that 
$\ll P \rr^{DS}_{def} = Q(D)$.
For $\mathbf{\mathit{EVAL(P_{wdgs}}}$ we can use the results in 
\cite{perez2009semantics}, i.e., that the evaluation problem for well-designed
pattern trees without projection is coNP-complete and conclude a coNP runtime to
check if $\mu \in Q(D)$.  \end{proof}

\begin{corollary}
The problem $\mathit{\mathbf{EVAL(\u_{wdgs})}}$ is coNP-complete. 
\end{corollary}
\begin{proof}
Hardness of the problem follows immediately from the hardness of 
\textbf{EVAL($\u_{wd}$)}.
Assume $ds = ep^{-1}(DS)$. %and $G = graph(def,DS)$.
For the membership we propose the following procedure:
If we have top-level union in the input graph pattern $P = P_{1} \UNION \cdots
\UNION P_{n}$ use the transformation function on all the graph patterns of $P$, i.e., $P_1,\dots,P_n$
to obtain a wdPF. More formally:
$\bigcup\limits_{i=1,\dots,n}Q_i = trans(P_i,ds,def)$ and $D = \bigcup\limits_{c\in dom(ep)} data(ep(c))$. This transformation is obviously
possible in polynomial time if we assume $dom(ep)$ to be finite. We know by Theorem~\ref{biglemma} that 
$\ll P \rr^{DS}_{def} = Q(D)$.
For $\mathbf{\mathit{EVAL(P_{wdgs}}}$ we can use the results in 
\cite{perez2009semantics}, i.e., that the evaluation problem for well-designed
pattern trees without projection is coNP-complete and conclude a coNP runtime to
check if $\mu \in Q(D)$. 
\end{proof}

\begin{corollary}
The problem	$\mathit{\mathbf{EVAL(\s_{wdgs})}}$ is $\Pi^P_2$-complete.
\end{corollary}
\begin{proof}
Hardness of the problem follows immediately from the hardness of 
\textbf{EVAL($\s_{wd}$)}.
Assume $ds = ep^{-1}(DS)$. %and $G = graph(def,DS)$.
For the membership we propose the following procedure:
If we have top-level projection of the input graph pattern $P$ we  use the
transformation function on the wdPT. More formally:
$\bigcup\limits_{i=1,\dots,n}Q_i = trans(P_i,ds,def)$ and $D = \bigcup\limits_{c\in dom(ep)} data(ep(c))$. This transformation is obviously
possible in polynomial time if we assume $dom(ep)$ to be finite. We know by Theorem~\ref{biglemma} that 
$\ll P \rr^{DS}_{def} = Q(D)$.
For $\mathbf{\mathit{EVAL(P_{wdgs}}}$ we can use the results in 
\cite{letelier2013static}, i.e., that the evaluation problem for well-designed
pattern trees with projection is $\Pi^P_2$-complete and conclude a $Pi^Pi_2$ runtime to
check if $\mu \in Q(D)$. 
\end{proof}

%\section{Static Analysis}
%\begin{theorem}
%	The problems \textbf{EQUIVALENCE}($\l$), \textbf{CONTAINMENT}($\l$)  
%	are $NP$-complete for $\l = P_{wdgs}$ and $\Pi^p_2$-complete $\l =
%	\u_{wdgs}$ when we assume the function $ep(\cdot)$ as additional input.
%\end{theorem}
%\begin{proof}
%	Hardness for each of these problems follows immediately from the fact that
%	that they are already NP-complete for the language
%	$P_{wd}$~\cite{letelier2012static}.
%	%and $\Pi^{P}_2$ for the language $\U_{wd}$~\cite{pichler2014containment}.\\
%	For the membership, we propose the following algorithm:
%	Let $P_1,P_2$ be two arbitrary patterns where $P_1,P_2 \in P_{wdgs}$ or
%	$P_1,P_2 \in \u_{wdgs}$ holds.
%	Let $x \in \V$ and use our polynomial time transformation function twice to obtain two wdpts: 
%	$Q_i = trans(P_i,x,def)$ for $i\in \{1,2\}$. 
%	Using the NP-algorithm proposed in~\cite{letelier2012static} for $\l =
%	P_{wdgs}$ we can decide whether $Q_1 \subseteq Q_2$:
%%	or the
%%	$\Pi^P_2$ algorithm proposed in~\cite{pichler2014containment} when $\l =
%%	\u_{wdgs}$.
%	\begin{enumerate}
%		\item $Q_1 \subseteq Q_2$:
%			Let $ds \in dom(ep)$ be arbitrary, i.e., 
%			an arbitrary URI of a dataset and $DS = ep(ds)$. 
%			Let $D =\bigcup\limits_{c\in dom(ep)} data(ep(c))$. The function data is also
%			obviously in polynomial time if $dom(ep)$ is assumed to be finite.
%			Let $\mu$ be an arbitrary mapping in $\ll P_1 \rr^{DS}_{def}$. Because
%			of Theorem~\ref{biglemma} we can conclude that $\mu \in Q_1(D)$.
%			Because of our assumption we get that $\mu \in Q_2(D)$. Because
%			again $Q_2(D) = \ll P_2 \rr^{DS}_{def}$ we get that $\mu \in \ll P_2
%			\rr^{DS}_{def}$.
%
%		\item $Q_1 \not\subseteq Q_2$: 
%			Let $ds \in dom(ep)$ be arbitrary, i.e., an arbitrary URI of a dataset 
%			and $DS = ep(ds)$.
%			Let $D =\bigcup\limits_{c\in dom(ep)} data(ep(c))$. The function data is also
%			obviously in polynomial time if $dom(ep)$ is assumed to be finite.
%			By assumption there is a $\mu \in
%			Q_1(D)$ such that $\mu \notin Q_2(D)$. Because
%			of Theorem~\ref{biglemma} and $\mu \in Q_1(D)$ we can conclude that $\mu \in \ll P_1
%			\rr^{DS}_{def}$.
%			Because of our assumption we get that $\mu \notin Q_2(D)$. Because
%			again $Q_2(D) = \ll P_2 \rr^{DS}_{def}$ we get that $\mu \notin \ll P_2
%			\rr^{DS}_{def}$.
%	\end{enumerate}
%\end{proof}
%%der proof geht nicht weil wenn man oben die variable reinhaut kriegt man kein
%%nicht \ll P \rr^DS_G = Q(D) sondern was anderes.
%To generalize this result which means removing the $ep(\cdot)$ function from the
%input, we observed that a problem occurred in the step where we transformed the
%SERVICE function into a wdpt. 
%In detail, this is the case when we have a query of the form $(\mbox{SERVICE }
%a \
%P_1)$, where $a \in \U \backslash dom(ep)$. In this case our transformation
%function $trans(\cdot)$ directly accesses the function $ep(\cdot)$ and checks if $a \in
%ep(\cdot)$ and produces a result directly depending on the outcome of this
%function. When transforming the GRAPH operator into a wdPT we don't run into the
%same problem. Assume we want to transform $(\mbox{GRAPH} \ a \ P_1)$ with $trans(\cdot)$
%and our current dataset is $ds$.
%Then $trans(\cdot)$ produces the same output regardless of the outcome of $a \in
%dom(names(ds))$.
%\begin{theorem}
%	The problem \textbf{CONTAINMENT}($P_{wdgs}$)  
%	is NP-hard and in $\Sigma^P_2$.
%\end{theorem}
%\begin{proof}
%	NP-Hardness for the problem follows immediately from the fact that
%	that they are already NP-complete for the language
%	$P_{wd}$~\cite{letelier2012static}.
%	%and $\Pi^{P}_2$ for the language $\U_{wd}$~\cite{pichler2014containment}.\\
%
%	For the $\Sigma^P_2$ membership  of \textbf{CONTAINMENT}($P_{wdgs}$), we propose the following algorithm:
%	Let $P_1,P_2$ be two arbitrary patterns where $P_1,P_2 \in P_{wdgs}$ holds.
%	Scan the queries $P_1,P_2$ for every occurrence of $(\mbox{SERVICE} \ a \ P_1)$ where $a \in
%	\U$. Put each destination URI of a SERVICE subpattern in $P_i$ into the set $O_i$ for $i\in
%	\{1,2\}$.
%	Let $O = O_1 \cup O_2$ so we have all the distinct occurences of destination
%	URIs in both queries. Assume $|O| = n$. 
%	Notice that we have exactly $2^n$ possibilities to construct $dom(ep)$ that would
%	influence our translation function $trans(\cdot)$ because this is the
%	number of subsets of the set $O$.
%	Let this family of domains be described as $Doms$.
%	Let $x \in \V$ and guess a domain $D_i\in Doms$. Then let 
%	$Q_i = trans(P_i,x,def,D_i)$ for $i\in \{1,2\}$. 
%	Using the NP-algorithm proposed in~\cite{letelier2012static} 
%	we can decide whether $Q_1 \subseteq Q_2$:
%	\begin{enumerate}
%		\item $Q_1 \subseteq Q_2$:
%			Let $ep$ have the domain $D_i$ and assume it else to be arbitrary.
%			Let $ds \in dom(ep)$ be arbitrary, i.e., an arbitrary
%			URI of a dataset and $DS = ep(ds)$. 
%			Let $D =\bigcup\limits_{c\in dom(ep)} data(ep(c))$. The function data is also
%			obviously in polynomial time if $dom(ep)$ is assumed to be finite. 
%			Let $\mu$ be an arbitrary mapping in $\ll P_1 \rr^{DS}_{def}$. Because
%			of Theorem~\ref{biglemma} we can conclude that $\mu \in Q_1(D)$.
%			Because of our assumption we get that $\mu \in Q_2(D)$. Because
%			again $Q_2(D) = \ll P_2 \rr^{DS}_{def}$ we get that $\mu \in \ll P_2
%			\rr^{DS}_{def}$.
%		\item $Q_1 \not\subseteq Q_2$: 
%			Let $ep$ have the domain $D_i$ and assume it else to be arbitrary.
%			Let $ds \in dom(ep)$ be arbitrary, i.e., an arbitrary URI of a dataset 
%			and $DS = ep(ds)$.
%			Let $D =\bigcup\limits_{c\in dom(ep)} data(ep(c))$. The function data is also
%			obviously in polynomial time if $dom(ep)$ is assumed to be finite.
%			By assumption there is a $\mu \in
%			Q_1(D)$ such that $\mu \notin Q_2(D)$. Because
%			of Theorem~\ref{biglemma} and $\mu \in Q_1(D)$ we can conclude that $\mu \in \ll P_1
%			\rr^{DS}_{def}$.
%			Because of our assumption we get that $\mu \notin Q_2(D)$. Because
%			again $Q_2(D) = \ll P_2 \rr^{DS}_{def}$ we get that $\mu \notin \ll P_2
%			\rr^{DS}_{def}$.
%	\end{enumerate}
%\end{proof}
%Through Lemma~\ref{biglemma} we get that SPARQL evaluation with the GRAPH and
%the SERVICE operator are the
%same as without the $GRAPH$ and $SERVICE$ patterns.
%Thus the complexity for the problems CONTAINMENT, EQUIVALENCE and SUBSUMPTION do not 
%differ from the results in \cite{pichler2014containment} using $GRAPH$ and
%$SERVICE$.

We need a different but equivalent definitions of WDPTs in this section. The
main structural difference to Definition~\ref{def:pt} is that the nodes in the
tree are labelled with a set
of relational atoms over a schema instead of a set of triples. Also the semantic
evaluation of WDPTs use the notion of conjunctive queries.

\begin{definition}[WDPTs~\cite{barcelo2015efficient}]\label{def:wdpt}
	A WDPT over a relational schema $\sigma$ is a tuple $(T, \lambda, \overline{x})$
	such that the following holds:
	\begin{enumerate}
		\item $T$ is tree rooted in a distinguished node $r$, the root and $\lambda$
			maps each node $t$ in $T$ to a set of relational atoms over $\sigma$.
		\item For every variable $y$ that appears in $T$, the set of nodes of $T$ where
			$y$ is mentioned is connected.
		\item We have that $\overline{x}$ is a tuple of distinct variables occurring in
			$T$. They are the free variables of the WDPT.
	\end{enumerate}
\end{definition}

\begin{definition}
	A WDPT $(T,\lambda, \overline{x})$ is called projection-free if $\overline{x}$
	contains all variables mentioned in $T$.
\end{definition}

\begin{definition}\label{wdptq}
	Assume $p = (T,\lambda,\overline{x})$ is a WDPT over $\sigma$. We write $r$ to
	denote the root of $T$. Given a subtree $T'$ of $T$ rooted in $r$ we define
	$q_{T'}$ to be the CQ $Y \leftarrow R_1(\overline{v_1}), \dots,
	R_m(\overline{v}_m)$, where the $R_i(\overline{v_i})$'s are the reational atoms
	that label the nodes of $T'$, i.e., 
	\begin{align*}
		\{ R_1(\overline{v}_1), \dots, R_m(\overline{v_m}) \} = \bigcup_{t \in T'} \lambda(t) 
	\end{align*} and $\overline{y}$ are all the variables that are mentioned in
	$T'$. 
\end{definition}

The main idea of the semantics of a WDPT is to look at each subtree $T'$ of
$T$ rooted in $r$. As mentioned above, each of them describes a pattern, i.e.,
the conjunctive query CQ $q_T'$. A mapping $h$ satisfies $(T,\lambda)$ over
a database $D$ if $h$ satisfies the pattern defined by a subtree $T'$ and there
is no subtree $T''$ which is bigger than $T'$ and $h$ can be extended to satisfy
$T''$.
\begin{definition}[Semantics of WDPTs~\cite{barcelo2015efficient}]
	Let $p=(T,\lambda,\overline{x})$ be a WDPT and $D$ a database over $\sigma$.

	\begin{enumerate}
		\item A homomorphism from $p$ to $D$ is a partial mapping $h: X \rightarrow U$,
			where $X$ is an infinite set of a variables and $U$ an infinite set of
			constants, for which it is the case that there is a subtree $T'$ of $T$ rooted
			in $r$ such that $h \in q_{T'}(D)$.
		\item The homomorphism $h$ is maximal if there is no homomorphism $h'$ from $p$
			to $D$ such that $h \sqsubset h'$.
	\end{enumerate}
	The evaluation of WDPT $p = (T,\lambda,\overline{x})$ over $D$ denoted $p(D)$,
	corresponds to all mappings of the form $h_{\overline{x}}$, such that $h$ is a
	maximal homomorphism from $p$ to $D$.
\end{definition}

It is important to notice that WDPTs properly extend CQs.
Given a CQ $q(x)$ of the form $X \leftarrow R_1(v_1),\dots,R_m(v_m)$
it is easy to see that $q(x)$ is equivalent to the WDPT $p = (T,\lambda,x)$,
where $T$ consists of a single node $r$ and $\lambda(r) =
\{R_1(v_1),\dots,R_m(v_m)\}$. In other words, $q(D) = p(D)$ for each database
$D$. Further on we will not distinguish between a CQ and the single node WDPT
that represents it. WDPTs can on the other hand represent interesting properties
that cannot be expressed as CQs, which we will discuss in the following example:
\begin{example}
	$Q =  \bigg(((x,recorded\_by,y) \ AND \ (x,published, ``after\_2010'')) \ \\ OPT \ (x,
	NME\_rating,z)\bigg) \ OPT \ (y, formed\_in, z') $

	\noindent Consider now the following RDF database $D$ consisting of the triples:
	\begin{verbatim}
	(``title_1'' recorded_by, ``studio_1''),
	(``title_1'' published, ``after_2010''),
	(``title_2'' recorded_by, ``studio_1''),
	(``title_2'' published, ``after_2010''),
	(``title_2'' NME_rating, ``2'')
	(``studio_1'' formed_in, ``miami''),
	\end{verbatim}
	Evaluating the query $Q$ over $D$ results in two mappings $\mu_1$ and $\mu_2$:
	$\mu_1 = \{ x \rightarrow ``title\_1'', y \rightarrow ``studio_1'',z'
	\rightarrow ``miami'' \}$ 
	$\mu_2 = \{ x \rightarrow ``title\_2'', y \rightarrow ``studio_1'',z\rightarrow
	``2'', z' \rightarrow ``miami'' \}$ 
\end{example}

To capture the UNION operator the definition of well designed pattern trees
need to be modified.  
\begin{definition}[Unions of WDPTs]
	A Union of WDPTs is an expression $\phi$ of the form $\bigcup_{1\leq i \leq n} p_i$, 
	where each $p_i$ is a WDPT over $\sigma$.
	We denote $\varphi(D)$ as the evaluation of $\phi$ over database $D$.
	It corresponds to the set $\bigcup_{1\leq i \leq n}p_i(D)$.
	Unions of WDPTs are also called Well-designed pattern
	forests(WDPF).
\end{definition}



First we are going to show an easy example of how to translate a graph pattern
only using AND and triple patterns to a conjunctive query. Then we propose a
polynomial time translation from a graph pattern $P \in P_{wdgs}$ to $Q \in P_{wd}$. For this
translation we need to construct a special database and a wdpt depending on the
original query. After we established the translation we prove the equivalence of $P$ and $Q$ in
Theorem~\ref{biglemma}. The last two sections deal with the problems
\textbf{EVAL}($P_{wdgs}$), \textbf{CONTAINMENT}($P_{wdgs}$),
\textbf{EQUIVALENCE}($P_{wdgs}$) and \textbf{SUBSUMPTION}($P_{wdgs}$).

\section{Translations to well designed pattern forests}
It is an easy observation that if we restrict SPARQL to the AND operator, we
simply get conjunctive queries without existentially quantified variables.
We will show this in Example~\ref{exconjq}. 
\begin{example}\label{exconjq}
	Consider the following SPARQL default Graph $G$ in a dataset $DS$ with the query $Q$ in $SPARQL[\land]$
	\begin{align*}
		G &=\{ (a,a,a), (b,c,c), (b,c,a)  \}\\
		Q &= (x,y,a) \ AND \ (z,c,c) \ AND \ (x,c,y)
	\end{align*}
	Let $T$ be a three-ary relation, $D = \{ T \}$ be the following database and
	$CQ$ be the following 
	conjunctive query: 
	\begin{align*}
		T &= \{ (a,a,a), (b,c,c), (b,c,a)\}\\
		CQ &= ans(x,y,z) \leftarrow T(x,y,a), T(z,c,c), T(x,c,y)\\
	\end{align*}
	It is easy to see that the mappings in $CQ(D)$ are the same as in $\ll Q
	\rr_G^{DS}$.
\end{example}
Similar to Example~\ref{exconjq} we are going to transform the dataset and query 
not into conjunctive queries but an extension of them: well designed pattern forests. We 
will define a polynomial time translation from a graph pattern $P \in P_{wdgs}$
to a well designed pattern tree. This enables us to use well known algorithms 
for which the computational complexity is known. 

\subsection{Creating the database}
We first describe the function $data$ which transforms a dataset into the
database.

Consider the function $data: DS \mapsto D$, where $DS$ is a
dataset and $D$ is a database.
$data$ then is defined as follows:
Let $DS$ be an arbitrary dataset and $u_{DS}$ the URI such that $ep(u_{DS}) = DS$.
\begin{align*}
	DS=\{(def,G),(u_1,G_1),\dots,(u_n,G_n)\}
\end{align*}
The output of $data$ is the database $D = \{ T,LOC\}$ where $T$ is a 5-ary relation containing all the triples of a graph, the corresponding graph URI and the dataset URI. The binary relation $LOC$
captures all graph URIs and their corresponding dataset URI. 
For our dataset this would mean, assuming 
\begin{align*}
	G &= \{(x_1,y_1,z_1), \dots, (x_a,y_a,z_a)\},\\ 
	G_1 &= \{(x_{11}, y_{11},z_{11}), \dots, (x_{1b},y_{1b},z_{1b}) \},\dots,\\ 
	G_n &= \{(x_{n1},y_{n1},z_{n1}),\dots,(x_{nc},y_{nc},z_{nc})\}
\end{align*} that we construct our output database $D$ as follows:
\begin{align*}
	T &= \{ (u_{DS},def,x_1,y_1,z_1), (u_{DS},def,x_a,y_a,z_a), \dots, (u_{DS},u_1,x_{11}, y_{11},z_{11}),
	\dots,\\& (u_{DS},u_1,x_{1b},y_{1b},z_{1b} ), \dots,
(u_{DS},u_n,x_{n1},y_{n1},z_{n1}), \dots (u_{DS},u_n,x_{nc},y_{nc},z_{nc})\}&. \\
	LOC &= \{ (u_{DS},def),(u_{DS},u_1),\dots (u_{DS},u_n) \}&.
\end{align*}

\subsection{Transforming the pattern to a wdpt}
We proceed in defining the function \textit{trans} which will in polynomial time 
transform a graph pattern in $P_{wdgs}$ to a well designed pattern tree with the same meaning without the
SERVICE and GRAPH operators. Lemma~\ref{smalllemma} concludes that the outputpattern and the
inputpattern are equivalent.

\bigskip\noindent
The transformation function $trans: P \times \U\cup\{\V\} \times \U \cup \{def \}\cup\{\V\}
\mapsto Q$ takes  three parameters as input: 
$P$ is a graph pattern in OPT normal form which allows the usage of AND, OPT, GRAPH, SERVICE and
UNION, $\U\cup\V$ is the infinite
set of uris with the infinite set of variables and $\U\cup\{def\}\cup \V$ is the infinite set of
uris containing the $def$ identifier conjoined with the infinite set of
variables. The output $Q$ is a well designed pattern tree. 

\bigskip
\noindent
Assume the input $(P,ds,g)$ and let each of the parameters be
arbitrary. 
\begin{enumerate}
	\item If $P$ is a triple pattern $(u,v,w)$,  
		\begin{align*}
		trans(P,ds,g) = vars(ds,g,u,v,w) \leftarrow T(ds,g,u,v,w).\\	
		\end{align*}

	\item If $P$ is $(P_1  \AND  P_2)$, let
		\begin{align*}
			&trans(P_1,ds,g)	= O_1 \leftarrow q_1,\\
			&trans(P_2,ds,g)	= O_2 \leftarrow q_2\\
			&trans(P,ds,g)		= O_1\cup O_2 \leftarrow q_1, q_2.
		\end{align*}
	\item If $P$ is $(P_1  \OPT  P_2)$, let\\
		\begin{align*}
			&trans(P_1,ds,g) =  (T_1, \lambda_1,x_1) \mbox{ and }\\
			&trans(P_2,ds,g) = (T_2, \lambda_2, x_2).
		\end{align*}
		$trans(P,ds,g) = (T,\lambda,x)$ for which $T = T_1 \cup T_2 \cup (r_1,
		r_2)$ where $r_1,r_2$ are the roots of $T_1,T_2$ respectively,
		$\lambda = \lambda_1 \cup \lambda_2$ and $x = x_1 \cup x_2$.

	\item If $P$ is $(\mbox{GRAPH} \ u \ P_1)$, let\\
		$(T_1,\lambda,x_1) = trans(P_1,ds,u)$.	
		Assuming $r_1$ is the root of $T_1$,
		and $\lambda(r_1) = q_1$ we define \[ \lambda'(x) =\begin{dcases*} 
				q_1, LOC(u,ds),LOC(g,ds)& if $x = r_1$\\
				\lambda(x) & otherwise	\\
			\end{dcases*}
		\] and $trans(P,ds,g) = (T_1,\lambda',x_1)$.

	\item $P$ is of the form $(\mbox{SERVICE} \ u \ P_1)$. 
		Case distinction:
		\begin{enumerate}
			\item If $u \in \U$ and $u \notin dom(ep)$:
				$trans(P,ds,g) =  \{\}\leftarrow$.
			\item Otherwise let $(T_1,\lambda,x_1) = trans(P_1,u,g)$.
				Assuming $r_1$ is the root of $T_1$,
				and $\lambda(r_1) = q_1$ we define \[ \lambda'(x) =\begin{dcases*} 
				q_1, LOC(def,u),LOC(g,ds)& if $x = r_1$\\
				\lambda(x) & otherwise	\\
			\end{dcases*}
		\]  and $trans(P,ds,g) = (T_1,\lambda',x_1)$.
		\end{enumerate}

%	\item If $P_1$ and $P_2$ are graph patterns and $P$ is $(P_1 \UNION  P_2)$,
%		then $P_1$ must correspond to a well formed pattern forest $Q_1 =
%		trans(P_1,ds,g)$ and $P_2$ to a
%		well designed pattern forest $Q_2 = trans(P_2,g,ds)$. 
%		Define $Q = Q_1 \cup Q_2$.  
\end{enumerate}
Observe that the function is well-defined since only patterns $P_{wdgs}$ are
considered: This allows us to argue over conjunctive queries for the case $P$ is $P_1
\AND P_2$ as the AND-operator may not occur in a scope of an OPT-operator in the fragment
$P_{wdgs}$. \\
Towards our goal to show that for all datasets $DS$ identified by URI
$ds$ and graph patterns $P$ the following property holds: 
Let $Q = trans(P,def,ds)$. Then $Q(D) = \ll P
\rr^{DS}_{graph(def,DS)}$, where $D=\bigcup\limits_{c \in dom(ep)} data(ep(c))$, we prove Lemma~\ref{smalllemma} first.

\begin{lemma}\label{smalllemma}
	Let $P$ be a graph pattern in OPT normal form containing the SERVICE and GRAPH operator
	and $D = \bigcup\limits_{c \in dom(ep)} data(ep(c))$.
	\begin{align*}
		Q(D) = \begin{dcases*}
			\ll P \rr^{ep(ds)}_{graph(g,ep(ds))} 
			& if $g \in \U,ds \in \U$\\
			\bigcup\limits_{ds' \in dom(ep)} \{ \mu \cup [ds \mapsto ds'] \mid\\ \mu \in
			\ll P\rr^{ep(ds')}_{graph(g,ep(ds))}, \mu \sim 	[ds\mapsto ds'] \} 
			&if $g \in \U,ds \in \V$\\
			\bigcup\limits_{g' \in names(ep(ds))}\{ \mu \cup [g \mapsto g']
				\mid\\ \mu \in
			\ll P\rr ^{ep(ds)}_{graph(g',ep(ds))} , \mu \sim [g\mapsto g'] \} 
			& if $g \in \V,ds \in \U$ \\
			\bigcup\limits_{ds' \in dom(ep), g' \in names(ep(ds'))} \big\{ \mu \cup
				\{[ds\mapsto ds'],[g \mapsto g']\} \mid\\ \mu \in
				\ll P\rr^{ep(ds')}_{graph(g',ep(ds'))}, 
				\mu \sim
			\{[ds\mapsto ds'], [g \mapsto g']\}\big\} 
			& if $g \in \V,ds \in \V$\\
		\end{dcases*}
	\end{align*}
%	\begin{eumerate}
%		\ite Let $ds \in \U$, $g \in \V \cup \U \cup \{def\}$ and $Q = trans(P,ds,g).$ Then 	
%		\ite if $g \in \U:$
%			begin{align*}
%			\ll P \rr^{ep(ds)}_{graph(g,ep(ds))} = Q(D)  
%			end{align*}

%		\item if $g \in \V$:\\
%		\begin{align*}
%				\bigcup\limits_{g' \in names(ep(ds))}\{ \mu \cup [g \mapsto g'] \mid \mu \in
%					\ll P\rr ^{ep(ds)}_{graph(g',ep(ds))} , \mu \sim
%				[g\mapsto g'] \}\\ = Q(D)
%			\end{align*}
%		\item Let $ds\in \V$, $g\in \V\cup \U\cup \{def\}$ and $Q = trans(P,ds,g).$ Then 	
%		\item if $g \in \U$:\\
%			\begin{align*}
%				\bigcup\limits_{ds' \in dom(ep)} \{ \mu \cup [ds \mapsto ds'] \mid \mu \in
%					\ll P\rr^{ep(ds')}_{graph(g,ep(ds))}, \mu \sim
%				[ds\mapsto ds'] \}\\  = Q(D)
%			\end{align*}
%
%		\item if $g \in \V$: \\
%			\begin{align*}
%				\bigcup\limits_{ds' \in dom(ep), g' \in names(ep(ds'))} \big\{ \mu \cup
%					\{[ds\mapsto ds'],[g \mapsto g']\} \mid\\ \mu \in
%					\ll P\rr^{ep(ds')}_{graph(g',ep(ds'))}, 
%					\mu \sim
%				\{[ds\mapsto ds'], [g \mapsto g']\}\big\} = Q(D)  
%			\end{align*}
%	\end{enumerate}
\end{lemma}


\begin{proof}
	We proceed to prove the statement using structural induction on $P$.
	\noindent 
	\begin{enumerate}
		\item  For the base case we assume that $P=(u,v,w)$ is a triple pattern:\\
			By construction we have that $Q: vars(ds,g,u,v,w) \leftarrow
			T(ds,g,u,v,w)$.\\
			$\subseteq:$\\

			Let $ds,g \in \U$. Let $DS = ep(ds)$ and $G = graph(g,DS)$.
			Let $\mu \in \ll P \rr^{DS}_{G}$ be arbitrary. 
			Then $dom(\mu) = vars(P)$ and $\mu(P) \in G$ by SPARQL semantics. 
			Because we have $\mu(P) \in G$ and our database $D$ contains by
			construction $T(ds,g,\mu(x),\mu(y),\mu(z))$ we have that $\mu \in Q(D)$.

			\bigskip\noindent
			Let $ds \in \V$ and $g \in \U$. 
			Because $ds$ is a variable we have to show that 
			$\bigcup\limits_{ds' \in dom(ep)} \{ \mu \cup [ds \mapsto ds'] \mid \mu \in \ll P
			\rr^{ep(ds')}_{graph(g,ep(ds'))},\mu \sim [ds \mapsto ds']\} \subseteq Q(D)$. 
			Let $ds' \in dom(ep)$ be arbitrary, let $\mu \in \ll P
			\rr^{ep(ds')}_{graph(g,ep(ds'))}$ where $\mu \sim [ds \mapsto ds']$.
			Because we have $\mu(P) \in graph(g,ep(ds'))$ by SPARQL semantics and our database
			$D$ contains by construction\\ $T(ds',g,\mu(x),\mu(y),\mu(z))$ we have that
			$\mu \cup [ds \rightarrow ds'] \in Q(D)$. 

			\bigskip\noindent
			Let $ds \in \U$ and $g \in \V$. Let $DS = ep(ds)$. 
			Because $g$ is a variable we have to show that 
			$\bigcup\limits_{g' \in names(ep(ds))} \{ \mu \cup [g \mapsto g'] \mid \mu \in \ll P
			\rr^{DS}_{graph(g',DS)},\mu \sim [g \mapsto g']\} \subseteq Q(D)$.
			Let $g' \in names(DS)$ be arbitrary, let $\mu \in \ll P
			\rr^{DS}_{graph(g',DS)}$ where $\mu \sim [g \mapsto g']$.
			Because we have $\mu(P) \in graph(g',DS)$ by SPARQL semantics and our database
			$D$ contains by construction\\ 
			$T(ds,g',\mu(x),\mu(y),\mu(z))$ we have that $\mu \cup [g
			\rightarrow g'] \in Q(D)$. 

			\bigskip\noindent
			Let $ds,g \in \V$.
			Because $ds$ and $g$ are variables we have to show that 
			$\bigcup\limits_{g' \in names(ds'), ds'\in dom(ep)} \{ \mu \cup \{[ds\mapsto ds'] [g
				\mapsto g']\} \mid \mu \in \ll P
				\rr^{ep(ds')}_{graph(g',ep(ds'))},\mu 
			\sim \{[ds \mapsto ds'][g \mapsto g']\}\} \subseteq Q(D)$. 

			Let $ds'\in dom(ep), g' \in names(ep(ds'))$ be arbitrary, let $\mu \in \ll P
			\rr^{ep(ds')}_{graph(g',ep(ds'))}$ where $\mu \sim \{[ds\mapsto ds'][g \mapsto
			g'] \}$.
			Because we have $\mu(P) \in graph(g',ep(ds'))$ by SPARQL semantics and our database
			$D$ contains by construction\\ $T(ds',g',\mu(x),\mu(y),\mu(z))$ we have that
			$\mu \cup \{[g \rightarrow g'][ds \rightarrow ds'] \}\in Q(D)$. 
			This can be done because $\mu 
			\sim \{[ds \rightarrow ds'][g \rightarrow g']\}$ was assumed. 

			\bigskip\noindent$\supseteq:$\\
			Let $ds,g \in \U$. Let $DS = ep(ds)$ and $G = graph(g,DS)$.
			Let $\mu \in Q(D)$ be arbitrary.
			We thus have a mapping $\mu$ from the variables in $vars(P)$ to constants s.t.
			$T(ds,g,\mu(u),\mu(v),\mu(w)) \in D$. 
			But by construction of $D$ this means that
			$(\mu(u),\mu(v),\mu(w)) \in G$ 
			and obviously $dom(\mu) = vars(P)$ thus $\mu \in \ll P
			\rr^{DS}_{G}$. 

			\bigskip\noindent
			Let $ds \in \V$, $g \in \U$.
			Because $ds$ is a variable we have to show that 
			$\bigcup\limits_{ds' \in dom(ep)} \{ \mu \cup [ds \mapsto ds'] \mid \mu \in \ll P
			\rr^{ep(ds')}_{graph(g,ep(ds'))},\mu \sim [ds \mapsto ds']\} \supseteq Q(D)$.
			Let $\sigma \in Q(D)$ be arbitrary.
			We thus have a mapping $\sigma$ from the 
			variables in $vars(ds,g,u,v,w)$ to constants s.t.\\
			$T(\sigma(ds),g,\sigma(u),\sigma(v),\sigma(w)))
			\in D$. Because $\sigma \in Q(D)$ and $ds \in \V$, we have
			$\sigma(ds) = ds'$, s.t. by construction $ds' \in dom(ep)$. 
			Let $\mu = \sigma\backslash[ds\mapsto ds']$. 
			By construction this means that
			$(\mu(u),\mu(v),\mu(w)) \in graph(g,ep(ds'))$ and obviously
			$dom(\mu) = vars(P)$. Thus $\mu \in \ll
			P\rr^{ep(ds')}_{graph(g,ep(ds'))}$ holds. 
			It remains to show that
			$\mu \sim [ds \mapsto ds']$
			but this is obvious because $\sigma$ contains $[ds \mapsto ds']$ and $\mu = \sigma
			\backslash [ds \mapsto ds']$. 

			\bigskip\noindent
			Let $ds \in \U$, $g \in \V$. Let $DS = ep(ds)$.
			Because $g$ is a variable we have to show that 
			$\bigcup\limits_{g' \in names(ep(a))} \{ \mu \cup [g \mapsto g'] \mid \mu \in \ll P
			\rr^{DS}_{graph(g,DS)},\mu \sim [g \mapsto g']\} \supseteq Q(D)$.
			Let $\sigma \in Q(D)$ be arbitrary.
			We thus have a mapping $\sigma$ from the variables in
			$vars(ds,g,u,v,w)$ to constants s.t.\\
			$T(ds,\sigma(g),\sigma(u),\sigma(v),\sigma(w))) \in D$.
			Because $\sigma \in Q(D)$ and $g \in \V$, we have
			$\sigma(g) = g'$, s.t. by construction $g' \in names(ds)$. Let $\mu =
			\sigma\backslash[g\mapsto g']$ be a mapping. By construction 
			this means that	$(\mu(u),\mu(v),\mu(w) \in graph(g',DS)$ 
			and obviously $dom(\mu) = vars(P)$. Thus $\mu \in \ll P
			\rr^{DS}_{graph(g',DS)}$ holds. It remains to show that $\mu \sim [g \mapsto g']$
			but this is obvious because $\sigma$ contains $[g \mapsto g']$ and $\mu = \sigma
			\backslash [g \mapsto g']$.

			\bigskip\noindent
			Let $ds \in \V$, $g \in \V$.
			Because $ds,g$ are variables we have to show that 
			$\bigcup\limits_{ds' \in dom(ep),g' \in names(ep(ds'))} 
			\{ \mu \cup \{[g\mapsto g'][ds \mapsto ds'] \} \\ 
				\mid \mu \in \ll P
			\rr^{ep(ds')}_{graph(g',ep(ds'))},
			\mu \sim \{[ds \mapsto ds'], [g \mapsto g']\}\} \supseteq Q(D)$.
			Let $\sigma \in Q(D)$ be arbitrary.
			We thus have a mapping $\sigma$ from the variables in $vars(ds,g,u,v,w)$ to constants s.t.
			$T(\sigma(ds),\sigma(g),\sigma(u),\sigma(v),\sigma(w)) \in D$.
			Because $\sigma \in Q(D)$ and $g \in \V, ds \in \V$, we have
			$\sigma(g) = g', \sigma(ds) = ds'$, s.t. by construction $g' \in
			names(ds)$ and $ds \in dom(ep)$. Let $\mu =
			\sigma\backslash\{[g\mapsto g'],
			[ds\mapsto ds']\}$. By construction
			this means that 	$\mu(u),\mu(v),\mu(w) \in graph(g',ep(ds'))$ 
			and obviously $dom(\mu) = vars(P)$. Thus $\mu \in \ll P
			\rr^{ep(ds')}_{graph(g',ep(ds'))}$ holds. It remains to show that $\mu \sim \{[ds \mapsto
			ds'],[g \mapsto g']\}$
			but this is obvious because $\sigma \supseteq \{[ds \mapsto d'],[g \mapsto
			g']\}$  and $\mu = \sigma \backslash \{[ds \mapsto ds'],[g \mapsto
			g']\}$.


			%%%%%AND%%%%
		\item Consider the case where $P$ is a graph pattern $(P_1 \AND P_2)$. \\
			By construction we have $Q = O_1 \cup O_2 \leftarrow q_1, q_2$. \\
			$\subseteq:$\\
			$ds,g \in \U$: Let $DS = ep(ds)$ and $G = graph(g,DS)$.
			Let $\mu \in \ll P \rr^{DS}_{G}$ be arbitrary. 
			By induction hypothesis we know that
			$\ll P_1 \rr^{DS}_{G} = Q_1(D)$  and
			$\ll P_2 \rr^{DS}_{G} = Q_2(D)$ hold.
			By semantics of SPARQL and $\mu \in \ll P \rr^{DS}_{G}$ we know that 
			$\mu \in \{\mu_1 \cup \mu_2 \mid \mu_1 \in \ll P_1 \rr^{DS}_{G}, \mu_2 \in \ll
			P_2 \rr^{DS}_{G} \mbox{ and }  \mu_1 \sim \mu_2\}$. 
			That means that there is a $\mu_1 \in \ll P_1 \rr_{G}^{DS}$ 
			and a $\mu_2 \in \ll P_2 \rr^{DS}_{G}$ such that $\mu = \mu_1 \cup \mu_2$.
			By induction hypothesis $\mu_1 \in Q_1(D)$ and $\mu_2 \in Q_2(D)$ hold.
			Because $\mu_1 \sim \mu_2$ holds we know that $\mu \in Q(D)$ holds. 

			\bigskip\noindent
			$ds \in \V$, $g \in \U$:
			Let $ds'$ be an arbitrary URI in $dom(ep)$.
			Let $\mu \in \ll P \rr^{ep(ds)}_{graph(g,ep(ds))}$ be arbitrary. 
			By induction hypothesis we know that
			$\bigcup\limits_{ds' \in dom(ep)} \{\mu_i \cup [ds \mapsto ds' ] \mid 
				\mu_i \in \ll P_i \rr^{ep(ds')}_{graph(g,ep(ds'))}, \mu \sim [ds
			\mapsto ds']\}  = Q_i(D)$
			for $i = 1,2$.
			By semantics of SPARQL and 
			$\mu \in \ll P \rr^{ep(ds')}_{graph(g,ep(ds'))}$ we know that 
			$\mu \in \{\mu_1 \cup \mu_2 \mid \mu_1 \in \ll P_1 \rr^{ep(ds')}_{graph(g,ep(ds')}, 
				\mu_2 \in \ll P_2 \rr^{ep(ds')}_{graph(g,ds')}\\ 
			\mbox{ and }  \mu_1 \sim \mu_2\}$. 
			That means that there is a $\mu_1 \in \ll P_1 \rr_{graph(g,ep(ds'))}^{ep(ds')}$ 
			and a $\mu_2 \in \ll P_2 \rr^{ep(ds')}_{graph(g,ep(ds'))}$ such that $\mu = \mu_1 \cup \mu_2$.
			Let  $\mu_1'= \mu_1\cup[ds\mapsto ds'] $ and $\mu_2' = \mu_2\cup[ds\mapsto
			d']$.By induction hypothesis we have a $\mu_1' \in Q_1(D)$ and a $\mu_2' \in
			Q_2(D)$.
			Because $\mu_1 \sim \mu_2$ and $\mu_1 \sim [ds\mapsto ds']$ and $\mu_2 \sim [ds
			\mapsto ds']$ we have $\mu \cup [ds\mapsto ds'] \in Q(D)$.

			\bigskip\noindent
			$ds \in \U$, $g \in \V$: Let $DS = ep(ds)$.
			Let $g'$ be an arbitrary URI in $names(ep(a))$.
			Let $\mu \in \ll P \rr^{ep(ds)}_{graph(g',ds)}$ be arbitrary. 
			By induction hypothesis we know that
			$\bigcup\limits_{g' \in names(DS)}\{\mu_i \cup [g \mapsto g' ] \mid 
				\mu_i \in \ll P_i \rr^{DS}_{graph(g',DS)}, \mu \sim [g
			\mapsto g']\} = Q_i(D)$ for  $i = 1,2$.
			By semantics of SPARQL and $\mu \in \ll P \rr^{DS}_{graph(g',DS)}$ we know that 
			$\mu \in \{\mu_1 \cup \mu_2 \mid \mu_1 \in \ll P_1 \rr^{DS}_{graph(g',DS)}, \mu_2 \in \ll
			P_2 \rr^{DS}_{graph(g',DS)}\\ \mbox{ and }  \mu_1 \sim \mu_2\}$. 
			That means that there is a $\mu_1 \in \ll P_1 \rr_{graph(g',DS)}^{DS}$ 
			and a $\mu_2 \in \ll P_2 \rr^{DS}_{graph(g',DS)}$ such that $\mu = \mu_1 \cup \mu_2$.
			 Let $\mu_1' =  \mu_1\cup[g\mapsto g']$ and $\mu_2' = \mu_2\cup[g\mapsto g']$.
			 By induction hypothesis we have a $\mu_1' \in Q_1(D)$ and a $\mu_2' \in
			Q_2(D)$.
			Because $\mu_1 \sim \mu_2$ and $\mu_1 \sim [g\mapsto g']$ and $\mu_2 \sim [g
			\mapsto g']$ we have $\mu \cup [g\mapsto g'] \in Q(D)$.

			\bigskip\noindent
			$ds,g \in \V$:
			Let $ds'$ be an arbitrary URI in $dom(ep)$.
			Let $g'$ be an arbitrary URI in $names(ep(d))$.
			Let $\mu \in \ll P \rr^{ep(ds')}_{graph(g',ep(ds'))}$ be arbitrary. 
			By induction hypothesis we know that
			$\bigcup\limits\limits_{ds'\in dom(ep),g' \in names(ep(ds'))}\{\mu_i \cup \{[g \mapsto g'
				],[ds \mapsto ds']\} \mid 
				\mu_i \in \ll P_i \rr^{ep(ds')}_{graph(g',ep(ds'))}, \mu
			\sim \{[g \mapsto g'], [ds\mapsto ds']\}\} \subseteq Q_i(D)$
			for $i = 1,2$.
			By semantics of SPARQL and $\mu \in \ll P \rr^{ep(ds')}_{graph(g',ep(ds'))}$ we know that 
			$\mu \in \{\mu_1 \cup \mu_2 \mid \mu_1 \in \ll P_1
				\rr^{ep(ds')}_{graph(g',ep(ds'))}, \mu_2 \in \ll
			P_2 \rr^{ep(ds')}_{graph(g',ep(ds'))}\\ \mbox{ and }  \mu_1 \sim \mu_2\}$. 
			That means that there is a $\mu_1 \in \ll P_1 \rr_{graph(g',ep(ds'))}^{ep(ds')}$ 
			and a $\mu_2 \in \ll P_2 \rr^{ep(ds')}_{graph(g',ep(ds')}$ such that 
			$\mu = \mu_1 \cup \mu_2$.
			 Let $\mu_1' = \mu_1\cup\{[ds\mapsto ds'],[g\mapsto g']\}$ and
			$\mu_2' = \mu_2\cup\{[g\mapsto g'],[ds\mapsto ds']\}$.
			By induction hypothesis we have a $\mu_1' \in Q_1(D)$ and a $\mu_2' \in
			Q_2(D)$.
			Because $\mu_1 \sim \mu_2$ and $\mu_1 \sim \{[ds \mapsto ds'],[g\mapsto g']\}$ and
			$\mu_2 \sim\{[ds \mapsto ds'], [g
			\mapsto g']\}$ we have $\mu \cup \{[g\mapsto g'], [ds \mapsto ds']\} \in Q(D)$.

			\bigskip
			\noindent$\supseteq:$\\
			%$trans(P_1,ds,g)=Q_1= O_1 \leftarrow q_1$ and $trans(P_2,ds,g)=Q_2=
			%O_2 \leftarrow q_2$. By construction: $Q: O_1 \cup O_2 \leftarrow
			%q_1,q_2$.
			$ds,g \in \U$.
			Let $\mu \in Q(D)$ be arbitrary. Let $DS = ep(ds)$ and $G = graph(g,ep(ds))$.
			By induction hypothesis we know that $\ll P_1 \rr^{DS}_{G} = Q_1(D)$ and
			$\ll P_2 \rr^{DS}_{G} = Q_2(D)$ hold. 
			Thus there must be some $\mu_1 \in Q_1(D)$, $\mu_2 \in Q_2(D)$ 
			where $\mu= \mu_1 \cup \mu_2$ holds by construction of $Q$.
			But thus $\mu_1 \in \ll P_1 \rr^{DS}_{G}$ and $\mu_2 \in \ll
			P_2\rr^{DS}_{G}$ by induction hypothesis. 
			By semantics of AND we know that $\mu \in \ll P \rr^{DS}_{G}$.

			\bigskip\noindent
			$ds \in V, g \in \U$.
			We want to show 
			$\bigcup\limits_{ds' \in dom(ep)} \{ \mu \cup
				[ds \mapsto ds'] \mid \mu \in \ll P\rr^{ep(ds')}_{graph(g,ep(ds'))}, \mu \sim
			[ds\mapsto ds'] \}  \supseteq Q(D)$.
			Let $\mu \in Q(D)$ be arbitrary.
			By induction hypothesis we know that  
			$\bigcup\limits_{ds' \in dom(ep)} \{ \mu \cup [ds \mapsto ds'] \mid \mu \in
				\ll P_i\rr^{ep(ds)}_{graph(g',ep(ds))}, \mu \sim
			[ds\mapsto ds'] \}  = Q_i(D) $ for $i\in \{1,2\}$.
			Thus there must be some $\mu_1 \in Q_1(D)$, $\mu_2 \in Q_2(D)$ 
			where $\mu= \mu_1 \cup \mu_2$ holds by construction of $Q$.
			By induction hypothesis we know that $\mu_i(ds) = ds'$ for some $ds'
			\in dom(ep)$ for $i \in \{1,2\}$.
			Again by i.h. $\mu_i\backslash\{[ds\mapsto ds'] \} \in \ll P_i
			\rr^{ep(ds)}_{graph(g,ep(ds'))}$ for $i \in \{1,2\}$.
			By semantics of AND we know that $\mu \backslash \{[ds\mapsto ds'] \} \in \ll P
			\rr^{ep(ds')}_{graph(g,ep(ds'))}$ and by induction hypothesis we know that we have 
			$\mu \sim \{[ds\mapsto ds'] \}$.

			\bigskip\noindent
			$ds \in \U, g \in \V$. Let $DS = ep(ds)$.
			We want to show 
			$\bigcup\limits_{g' \in names(DS)} \{ \mu \cup \{[g
				\mapsto g']\} \mid \mu \in
				\ll P\rr^{DS}_{graph(g',DS)}, \mu \sim
			\{[g \mapsto g']\} \}  \supseteq Q(D) $.
			Let $\mu \in Q(D)$ be arbitrary.
			By induction hypothesis we know that  
			$\bigcup\limits_{g' \in names(DS)} \{ \mu \cup \{[g
				\mapsto g']\} \mid \mu \in
				\ll P_i\rr^{DS}_{graph(g',DS)}, \mu \sim
			\{[g \mapsto g']\} \}  = Q_i(D) $ for $i=1,2$.
			Thus there must be some $\mu_1 \in Q_1(D)$, $\mu_2 \in Q_2(D)$ 
			where $\mu= \mu_1 \cup \mu_2$ construction of $Q$.
			By induction hypothesis we know that $\mu_i(g) = g'$ for some $g'
			\in names(ds)$ for $i \in \{1,2\}$.
			Again by i.h. $\mu_i\backslash\{ [g \mapsto g'] \} \in \ll P_i \rr^{DS}_{graph(g',DS)}$
			$i \in \{1,2\}$ by induction hypothesis. 
			By semantics of AND  we know that $\mu \backslash \{[g \mapsto g'] \} \in \ll P
			\rr^{DS}_{graph(g',DS)}$ and by induction hypothesis we know that we have 
			$\mu \sim \{[g \mapsto g'] \}$.

			\bigskip\noindent
			$ds,g \in \V$.
			We want to show 
			$\bigcup\limits_{ds \in dom(ep),g' \in names(ep(ds'))} \{ \mu \cup \{[ds \mapsto ds'],[g
				\mapsto g']\} \mid \mu \in
				\ll P \rr^{ep(ds')}_{graph(g', ep(ds'))}, \mu \sim
			\{[ds \mapsto ds'], [g \mapsto g']\} \}  \supseteq Q(D) $.
			Let $\mu \in Q(D)$ be arbitrary.
			By induction hypothesis we know that  
			$\bigcup\limits_{ds' \in dom(ep), g' \in names(ep(ds'))} 
			\{ \mu \cup \{[ds \mapsto ds'], [g	\mapsto g']\} \mid\\
				\mu \in	\ll P_i\rr^{ep(ds')}_{graph(g', ep(ds))}, \mu \sim
			\{[ds\mapsto ds'],[g \mapsto g']\} \}  = Q_i(D) $ for $i=1,2$.
			Thus there must be some $\mu_1 \in Q_1(D)$, $\mu_2 \in Q_2(D)$ 
			where $\mu= \mu_1 \cup \mu_2$ holds by semantics of $\land$ and construction of $Q$.
			By induction hypothesis we know that $\mu_i(ds) = ds'$ for some
			$ds' \in dom(ep)$ and $\mu_i(g) = g'$ for some $g' \in names(ds')$ for $i \in \{1,2\}$.
			By induction hypothesis $\mu_i\backslash\{[ds\mapsto ds'], [g \mapsto g'] \} \in \ll P_i
			\rr^{ep(ds')}_{graph(g',ep(ds'))}$ for $i \in \{ 1,2 \}$.
			By semantics of $AND$  we know that $\mu \backslash \{[ds\mapsto ds'], [g
			\mapsto g'] \} \in \ll P
			\rr^{ep(ds')}_{graph(g,ep(ds'))}$ and by induction hypothesis we know that we have 
			$\mu \sim \{[ds\mapsto ds'], [g \mapsto g'] \}$.


			%%%OPT

		\item Consider the case where $P$ is a graph pattern $(P_1 \OPT	P_2)$. \\
			By construction we have that $ Q_1 = trans(P_1,ds,g) =
			(T_1,\lambda_1,x_1)$,\\ $Q_2 = trans(P_2,ds,g) = (T_2,\lambda_2,x_2)$ 
			and $Q = trans(P,ds,g) = (T,\lambda,x)$ for which $T = T_1 \cup T_2
			\cup (r_1,r_2)$ where $r_1,r_2$ are the roots of $T_1,T_2$
			respectively, $\lambda = \lambda_1 \cup \lambda_2$ and $x = x_1 \cup
			x_2$.

			\bigskip\noindent
			$\subseteq:$\\
			Let $ds,g \in \U$:
			Let $DS = ep(ds)$ and $G = graph(g,DS)$.
			Let $\mu \in \ll P \rr^{DS}_{G}$ be arbitrary. 
			By semantics of OPT we have that $\mu \in \ll P_1 \ AND \ P_2
			\rr^{DS}_{G}$ or $\mu \in \big\{ \mu_1 \in \ll P_1 \rr^{DS}_{G} \mid \forall \mu_2 \in \ll P_2
				\rr^{DS}_{G}:
			\mu_1 \not\sim \mu_2 \big\}$. 
			We thus proceed by case distinction:
			\begin{enumerate}
				\item Assume $\mu \in \ll P_1 \ AND \ P_2 \rr^{DS}_{G}$. By induction
					hypothesis and semantics of AND, we thus have some $\mu_1
					\in Q_1(D)$ and some $\mu_2 \in Q_2(D)$ so that $\mu = \mu_1
					\cup \mu_2$. Because $\mu_i \in Q_i(D)$ we have by semantics
					of wdpts (recall Definition~\ref{wdptq}) that $\mu_i \in Q_{i,T'_i}$ so that $T'_i \subseteq
					T_i$ for $i\in\{1,2\}$.
					Let $T' = T'_1 \cup T'_2$.
					Because we have $\mu = \mu_1 \cup \mu_2$ we know that $\mu
					\in Q_T'$.
					It remains to show that this homomorphism is maximal: assume there was a
					bigger subtree $\hat{T}$ which would allow a mapping $\mu'
					\sqsupset \mu$. Then the
					node which was put additionally to $T$ must either be in
					$T_1$ or $T_2$.
					But then either $\mu_1$ or $\mu_2$ were not maximal defying the assumption
					that $\mu_1 \in Q_1$ and $\mu_2 \in Q_2$.
					Thus $\mu \in Q(D)$. 

				\item Assume $\mu \in \big\{\mu_1 \in \ll P_1 \rr^{DS}_{G} \mid 
					\forall \mu_2 \in \ll P_2 \rr^{DS}_{G}: \mu_1 \not\sim \mu_2
				\big\}$. \\
					Because of our assumption $\mu = \mu_1$ for some $\mu_1 \in \ll P_1
					\rr^{DS}_{G}$. We know that $\mu_1 \in Q_1(D)$ by induction
					hypothesis. Thus $\mu_1 \in Q_{1,T'_1}$ for some $T'_1
					\subset T_1$ by
					semantics of wdpts. $\mu \in Q_{T'_1}(D)$ follows.
					It remains to show that there is no $\mu' \sqsupset \mu$.
					Because of our assumption we know there is no mapping
					$\mu_2 \in Q_2$ which would be compatible with $\mu$. Thus $\mu$ could
					only be enlarged by a bigger mapping in $Q_1(D)$ but this defies the
					assumption that $\mu_1 \in Q_1(D)$.
			\end{enumerate}

			\bigskip\noindent
			Let $ds \in \V$ and $g \in \U$:
			We want to show  
			$\bigcup\limits_{ds' \in dom(ep)} \{ \mu \cup [ds \mapsto ds'] \mid \mu \in
				\ll P\rr^{ep(ds')}_{graph(g,ep(ds'))}, \mu \sim
			[ds\mapsto ds'] \}  \subseteq Q(D) $.
			Let $ds'$ be an arbitrary URI in $dom(ep)$. Let $\mu \in \ll P
			\rr^{ep(ds')}_{graph(g,ds')}$ be arbitrary.
			By semantics of OPT we have that $\mu \in \ll P_1 \mbox{ AND }  P_2
			\rr^{ep(ds')}_{graph(g,ds')}$ or 
			$\mu \in \big\{ \mu_1 \in \ll P_1 \rr^{ep(ds')}_{graph(g,ds')} \mid \forall \mu_2 \in \ll P_2
			\rr^{ep(ds')}_{graph(g,ds')}: \mu_1 \not\sim \mu_2 \big\}$. 
			We thus proceed by case distinction:
			\begin{enumerate}
				\item Assume $\mu \in \ll P_1 \AND  P_2 \rr^{ep(ds')}_{graph(g,ds')}$. 
					By induction hypothesis we have 
					$\bigcup\limits_{ds' \in dom(ep)} \{ \mu \cup [ds \mapsto ds'] \mid \mu \in
						\ll P_i\rr^{ep(ds')}_{graph(g,ep(ds'))}, \mu_i \sim
					[ds\mapsto ds'] \}  \subseteq Q_i(D) $ for $i \in \{1,2\}$.
					By semantics of SPARQL and $\mu \in \ll P \rr^{ep(ds')}_{graph(g,ds')}$ we
					know that $\mu \in \{\mu_1 \cup \mu_2 \mid \mu_1 \in \ll P_1
						\rr^{ep(ds')}_{graph(g,ds')}, \mu_2 \in \ll P_2
						\rr^{ep(ds')}_{graph(g,ds')}
					\mbox{ and } \mu_1 \sim \mu_2 \}$. Thus there is a $\mu_1 \in \ll P_1
					\rr^{ep(ds')}_{graph(g,ds')}$ and a $\mu_2 \in \ll P_2
					\rr^{ep(ds')}_{graph(g,ds')}$ such that $\mu = \mu_1 \cup \mu_2$.
					Let $\mu'_i = \mu_i \cup [ds \mapsto ds']$ for $i \in
					\{1,2\}$.  Because $\mu_i \in Q_i(D)$  we have a
					$T'_i \subseteq T_i$ such that $\mu'_i \in Q_{i,T'_i}(D)$ for $i\in \{1,2\}$.
					Let $T' = T'_1 \cup T'_2$.
					Because $\mu_1 \sim \mu_2$ and $\mu_1 \sim [ds \mapsto ds']$ 
					and $\mu_2 \sim [ds \mapsto ds']$ we have $\mu \cup [ds
					\mapsto ds'] \in Q_T'(D)$.
					It remains to show that this homomorphism $\mu\cup [ds \mapsto ds']$ is maximal: 
					assume there was a bigger subtree $\hat{T}$ which would allow a mapping 
					$\mu' \sqsupset (\mu\cup [ds \mapsto ds'])$. Then the
					node which was put additionally to $T$ must either be in $T_1$ or $T_2$.
					But then either $\mu_1'$ or $\mu_2'$ were not maximal defying the assumption
					that $\mu_1' \in Q_1(D)$ and $\mu_2' \in Q_2(D)$.
					Thus $\mu \cup [ds \mapsto ds']\in Q(D)$. 

				\item Assume $\mu \in \big\{ \mu_1 \in  \ll P_1 \rr^{ep(ds')}_{graph(g,ds')} \mid 
						\forall \mu_2 \in \ll P_2 \rr^{ep(ds')}_{graph(g,ds')}: 
					\mu_1 \not\sim \mu_2 \big\}$. 
					By induction hypothesis we have 
					$\bigcup\limits_{ds' \in dom(ep)} \{ \mu_1 \cup	[ds	\mapsto ds'] 
						\mid \mu_1 \in \ll	P_1\rr^{ep(ds')}_{graph(g,ep(ds'))}, \mu \sim
					[ds\mapsto ds'] \}  = Q_1(D)$.
					Because $\mu_1 \in \ll P_1
					\rr^{ep(ds')}_{graph(g,ds')}$ 
					we have $\mu\cup [ds \mapsto ds'] \in Q_{T'_1}(D)$, where
					$T'_1 \subseteq T_1$ by wdpt semantics.
					It remains to show that there
					is no $\mu' \sqsupset \mu$. But because we have $\{\mu_1 \not\sim \mu_2 \mid
					\forall \mu_2 \in \ll P_2 \rr^{ep(ds')}_{graph(g,ds')} \}$, 
					we know there is no mapping
					of $\mu_2$ of $Q_2$ which would be compatible with $\mu$. Thus $\mu$ could
					only be a bigger mapping $\mu' \in Q_1(D)$ but this defies the
					assumption that $\mu_1 \in Q_1(D)$ and thus we are done.
			\end{enumerate}

			\bigskip\noindent
			Let $ds\in \U$ and $g \in \V$: Let $DS = ep(ds)$.
			We want to show 
			$\bigcup\limits_{g' \in names(DS)}\{ \mu \cup [g \mapsto g'] \mid \mu \in
				\ll P\rr^{DS}_{graph(g',DS)} , \mu \sim
			[g\mapsto g'] \} \subseteq Q(D) $
			Let $g'$ be an arbitrary graph in $names(DS)$. Let $\mu \in \ll P
			\rr^{DS}_{graph(g',DS)}$ be arbitrary.
			By semantics of OPT we have that $\mu \in \ll P_1 \AND  P_2
			\rr^{DS}_{graph(g',DS)}$ or 
			$\mu \in \big\{ \mu_1 \in  \ll P_1 \rr^{DS}_{graph(g',DS)} \mid \forall \mu_2 \in \ll P_2
				\rr^{DS}_{graph(g',DS)}:
			\mu_1 \not\sim \mu_2 \big\}$. 
			We proceed by case distinction:
			\begin{enumerate}
				\item Assume $\mu \in \ll P_1 \ AND \ P_2 \rr^{DS}_{graph(g',DS)}$. 
					By induction hypothesis we have 
					$\bigcup\limits_{g' \in names(DS)} \{ \mu \cup [g \mapsto g'] \mid \mu \in
						\ll P_i\rr^{DS}_{graph(g',DS)}, \mu_i \sim
					[g\mapsto g'] \}  = Q_i(D) $ and $i = 1,2$.
					By semantics of SPARQL and $\mu \in \ll P
					\rr^{DS}_{graph(g',DS)}$ we
					know that $\mu \in \{\mu_1 \cup \mu_2 \mid \mu_1 \in \ll P_1
						\rr^{DS}_{graph(g',DS)}, \mu_2 \in \ll P_2
						\rr^{DS}_{graph(g',DS)}
					\mbox{ and } \mu_1 \sim \mu_2 \}$.
					Thus there is a $\mu_1 \in \ll P_1
					\rr^{ep(ds)}_{graph(g',ds)}$ and a $\mu_2 \in \ll P_2
					\rr^{ep(ds)}_{graph(g',ds)}$ such that $\mu = \mu_1 \cup \mu_2$.
					Let $\mu'_i = \mu_i \cup [g \mapsto g']$ for $i \in
					\{1,2\}$.  Because $\mu_i \in Q_i(D)$  we have a
					$T'_i \subseteq T_i$ such that $\mu'_i \in Q_{i,T'_i}(D)$ for $i\in \{1,2\}$.
					Let $T' = T'_1 \cup T'_2$.
					Because $\mu_1 \sim \mu_2$ and $\mu_1 \sim [g \mapsto g']$ 
					and $\mu_2 \sim [g \mapsto g']$ we have $\mu \cup [g
					\mapsto g'] \in Q_T'(D)$.
					It remains to show that this homomorphism $\mu\cup [g
					\mapsto g']$ is maximal: 
					assume there was a bigger subtree $\hat{T}$ which would allow a mapping 
					$\mu' \sqsupset (\mu\cup [g \mapsto g'])$. Then the
					node which was put additionally to $T$ must either be in $T_1$ or $T_2$.
					But then either $\mu_1'$ or $\mu_2'$ were not maximal defying the assumption
					that $\mu_1' \in Q_1(D)$ and $\mu_2' \in Q_2(D)$.
					Thus $\mu \cup [g \mapsto g']\in Q(D)$. 

				\item Assume $\mu \in \big\{ \mu_1 \in  \ll P_1 \rr^{DS}_{graph(g',DS)} \mid 
					\forall \mu_2 \in \ll P_2 \rr^{DS}_{graph(g',DS)}: \mu_1 \not\sim \mu_2 \big\}$. 
					By induction hypothesis we have 
					$\bigcup\limits_{g' \in names(DS)} 
					\{ \mu_1 \cup [g \mapsto g'] \mid \mu_1 \in
						\ll P_1\rr^{DS}_{graph(g',DS)}, \mu \sim
					[g\mapsto g'] \}  = Q_1(D)$.
					Because $\mu_1 \in \ll P_1
					\rr^{ep(ds)}_{graph(g',ds)}$ 
					we have $\mu\cup [g \mapsto g'] \in Q_{T'_1}(D)$, where
					$T'_1 \subseteq T_1$ by wdpt semantics.
					It remains to show that there
					is no $\mu' \sqsupset \mu$. But because we have $\{\mu_1 \not\sim \mu_2 \mid
					\forall \mu_2 \in \ll P_2 \rr^{ep(ds)}_{graph(g',ds)} \}$, 
					we know there is no mapping
					of $\mu_2$ of $Q_2$ which would be compatible with $\mu$. Thus $\mu$ could
					only be a bigger mapping $\mu' \in Q_1(D)$ but this defies the
					assumption that $\mu_1 \in Q_1(D)$ and thus we are done.
			\end{enumerate}

			\bigskip\noindent
			Let $ds,g \in \V$:
			We want to show  $\bigcup\limits_{ds' \in dom(ep), g' \in names(ep(d))} \{ \mu \cup
			\{[ds \mapsto ds'],[g \mapsto g']\} \mid \mu \in
			\ll P\rr^{ep(ds')}_{graph(g',ep(ds'))}, 
			\mu \sim \{[ds\mapsto ds'], [g \mapsto g']\}\} \supseteq Q(D) $. 
			Let $ds'$ be an arbitrary URI in dom(ep), and $g'$ be an arbitrary
			URI in $names(ep(d))$.
			Let $\mu \in \ll P \rr^{ep(ds')}_{graph(g',ep(ds'))}$ be arbitrary.
			By semantics of OPT we have that $\mu \in \ll P_1 \AND P_2 \rr^{ep(ds')}_{graph(g',ep(ds'))}$ or 
			$\mu \in \big\{ \mu_1 \in  \ll P_1 \rr^{ep(ds')}_{graph(g',ep(ds'))} \mid \forall \mu_2 \in \ll P_2
				\rr^{ep(ds')}_{graph(g',ep(ds'))}:
			\mu_1 \not\sim \mu_2 \big\}$. 
			We thus proceed by case distinction:
			\begin{enumerate}
				\item Assume $\mu \in \ll P_1 \AND P_2 \rr^{ep(ds')}_{graph(g',ep(ds'))}$. 
					By induction hypothesis we thus have 
					$\bigcup\limits_{ds'\in dom(ep),g' \in names(ep(d))} \{ \mu \cup \{[ds \mapsto
						ds'][g \mapsto g']\} \mid \mu \in
						\ll P_i\rr^{ep(ds')}_{graph(g',ep(ds'))}, \mu_i \sim
					\{[g\mapsto g'], [ds \mapsto ds']\} \}  = Q_i(D) $ and $i = 1,2$.
					By semantics of SPARQL and $\mu \in \ll P
					\rr^{ep(ds')}_{graph(g',ep(ds'))}$ we
					know that $\mu \in \{\mu_1 \cup \mu_2 \mid \mu_1 \in \ll P_1
						\rr^{ep(ds')}_{graph(g',ep(ds'))}, \mu_2 \in \ll P_2
						\rr^{ep(ds')}_{graph(g',ep(ds'))}
					\mbox{ and } \mu_1 \sim \mu_2 \}$. Thus there is a $\mu_1 \in \ll P_1
					\rr^{ep(ds')}_{graph(g',ep(ds'))}$ and a $\mu_2 \in \ll P_2
					\rr^{ep(ds')}_{graph(g',ep(ds'))}$ such that $\mu = \mu_1 \cup \mu_2$.
					Let $\mu'_i = \mu_i \cup \{ [ds \mapsto ds'][g \mapsto g']\}$ for $i \in
					\{1,2\}$.  Because $\mu_i \in Q_i(D)$  we have a
					$T'_i \subseteq T_i$ such that $\mu'_i \in Q_{i,T'_i}(D)$ for $i\in \{1,2\}$.
					Let $T' = T'_1 \cup T'_2$.
					Because $\mu_1 \sim \mu_2$ and $\mu_i \sim \{[g \mapsto
					g'],[ds \mapsto ds'] \}$ for
					$i\in\{1,2\}$, we have $\mu \cup [g	\mapsto g'] \in Q_T'(D)$.
					It remains to show that the mapping $\mu\cup \{[ds\mapsto
					ds'], [g\mapsto g'] \}$ is maximal: 
					assume there was a bigger subtree $\hat{T}$ which would allow a mapping 
					$\mu' \sqsupset (\mu\cup \{[ds \mapsto ds'],[g \mapsto g']\})$. Then the
					node which was put additionally to $T$ must either be in $T_1$ or $T_2$.
					But then either $\mu_1'$ or $\mu_2'$ were not maximal defying the assumption
					that $\mu_1' \in Q_1(D)$ and $\mu_2' \in Q_2(D)$.
					Thus $\mu \cup \{[g \mapsto g'],[ds \mapsto ds']\} \in Q(D)$. 


	\item Assume $\mu \in \big\{ \mu_1 \in  \ll P_1 \rr^{ep(ds')}_{graph(g',ep(ds'))} \mid 
			\forall \mu_2 \in \ll P_2 \rr^{ep(ds')}_{graph(g',ep(ds'))}:
		\mu_1 \not\sim \mu_2 \big\}$. By induction hypothesis we have 
		$\bigcup\limits_{ds' \in dom(ep), g' \in names(ep(ds'))} \{ \mu_1 \cup \{[ds' \mapsto
			ds] [g \mapsto g'] \} \mid \mu_1 \in
			\ll P_1\rr^{ep(ds')}_{graph(g',ep(ds'))}, \mu \sim
			\{[ds \mapsto ds'],[g\mapsto g']\}  = Q_1(D)$.
			Because $\mu_1 \in \ll P_1
			\rr^{ep(ds)}_{graph(g',ds)}$ 
			we have $\mu\cup \{[ds \mapsto ds'],[g \mapsto g'] \}\in Q_{T'_1}(D)$, where
			$T'_1 \subseteq T_1$ by wdpt semantics.
			It remains to show that there
			is no $\mu' \sqsupset \mu$. But because we have $\{\mu_1 \not\sim \mu_2 \mid
			\forall \mu_2 \in \ll P_2 \rr^{ep(ds')}_{graph(g',ds')} \}$, 
			we know there is no mapping
			of $\mu_2$ of $Q_2$ which would be compatible with $\mu$. Thus $\mu$ could
			only be a bigger mapping $\mu' \in Q_1(D)$ but this defies the
			assumption that $\mu_1 \in Q_1(D)$ and thus we are done.
	\end{enumerate}

	\bigskip\noindent$\supseteq:$\\
	Because of the construction of $Q$, a solution, call it $\mu$ must either
	adhere $\mu \in Q_{T'}(D)$ for $T' \subseteq T$. We will further distinguish two
	cases:
	\begin{enumerate}
		\item $\mu \in Q_{T'}(D)$ for some $T' \subseteq T_1$ 
		\item  $\mu \in	Q_{T'}(D)$ for some $T' = T'_1 \cup T'_2$ where $T'_1 \subseteq T_1$ and
	$T'_2 \subseteq T_2$.
	\end{enumerate}

	\bigskip\noindent
	Let $ds,g \in \U$. Let $DS= ep(ds)$ and $G = graph(g,DS)$. 
	Let $\mu \in Q(D)$ be arbitrary.
	Case distinction:
	\begin{enumerate}
		\item $\mu \in Q_{T'_1}(D)$: Thus $\mu \in Q_1(D)$ and by i.h. $\mu \in \ll P_1 \rr^{DS}_{G}$ 
			and then also $\{\mu \not\sim \mu_2 \mid \forall \mu_2 \in
			\ll P_2 \rr^{DS}_{G} \}$ by assumption. Thus $\mu \in \ll P
			\rr_{G}^{DS}$.
		\item If $\mu \in Q_{T'}(D)$ we then have $\mu_{|vars(Q_1)} \in Q_1(D)$  (restricted to
			the variables in $Q_1$) and $\mu_{|vars(Q_2)} \in Q_2(D)$(restricted to the variables
			in $Q_2$). Thus by i.h. and semantics of AND we have $\mu \in \ll P_1 \ AND \ P_2 \rr^{DS}_{G}$
			and $\mu \in \ll P \rr^{DS}_{G}$.
	\end{enumerate}

	%The case where we \sigma \in Q_1(D) maximal only because of value of ds is
	%taken care of:
	% because $\mu_1$ and $\mu_2$ need to both be compatible with assignment of
	% the variable ds$
	\bigskip\noindent
	Let $ds \in \V, g \in \U$.
	We want to prove  
	$\bigcup\limits_{ds' \in dom(ep)} \{ \mu \cup [ds \mapsto ds'] \mid\\ \mu
	\in \ll P \rr^{ep(ds')}_{graph(g,ep(ds'))}, \mu \sim [ds\mapsto ds'] \}
	\supseteq Q(D) $.
	Let $\sigma \in Q(D)$ be arbitrary.
	Because $\sigma \in Q(D)$ and $ds \in V$ we have that $\sigma(ds) =
	ds'$ for some $ds' \in dom(ep)$.
	Case distinction:
	\begin{enumerate}
		\item $\sigma \in Q_{T'_1}(D)$: Because the assumption implies $\sigma
			\in Q_1(D)$ we can then use the induction hypothesis  
			$\bigcup\limits_{ds' \in dom(ep)} \{ \mu \cup [ds \mapsto ds'] \mid \mu
			\in \ll P_1 \rr^{ep(ds')}_{graph(g,ep(ds'))}, \mu \sim [ds\mapsto ds'] \}  =
			Q_1(D) $. Let $\mu = \sigma \backslash [ds \mapsto ds']$. 
			Thus $\mu \in \ll P_1 \rr^{ep(ds')}_{graph(g,ep(ds'))}$ 
			and we also have $\{\mu \not\sim \mu_2 \mid \forall \mu_2 \in
			\ll P_2 \rr^{ep(ds')}_{graph(g,ep(ds'))} \}$ because of our
			assumption. Thus $\mu \in \ll P
			\rr_{graph(g,ep(ds')}^{ep(ds')}$ and $\mu \sim [ds \mapsto ds']$ by
			induction hypothesis.
		\item If $\sigma \in Q_{T'}(D)$  we can use the induction hypothesis twice:\\
			$\bigcup\limits_{ds' \in dom(ep)} \{ \mu_i \cup [ds \mapsto ds'] \mid \mu_i
			\in \ll P_i \rr^{ep(ds')}_{graph(g,ep(ds'))}, \mu_i \sim [ds\mapsto ds'] \}  =
			Q_i(D)$ for $i=1,2$.\\
			Let $\mu = \sigma \backslash [ds \mapsto ds']$.
			Because $\sigma \in Q_{T'}(D)$ we have $\mu = \mu_1 \cup \mu_2$ for
			some $\mu_{1|vars(Q_1)}\cup[ds \mapsto ds'] \in Q_1(D)$ (restricted to
			the variables in $Q_1$) and $\mu_{2|vars(Q_2)}\cup [ds \mapsto ds'] \in Q_2(D)$ (restricted to
			the variables in $Q_2$).
			Thus $\mu \in \ll P_1 \AND P_2 \rr^{ep(ds')}_{graph(ds',g)}$
			and $\mu \in \ll P \rr^{ep(ds')}_{graph(g,ds')}$. Also $\mu \sim [ds \mapsto
			ds']$ holds by induction hypothesis.
	\end{enumerate}

	\bigskip\noindent
	Let $ds \in \U, g \in \V$. Let $DS = ep(ds)$.
	We want to prove  
	$\bigcup\limits_{g' \in names(DS)} \{ \mu \cup [g \mapsto g'] \mid \mu
	\in \ll P \rr^{DS}_{graph(g,DS)}, \mu \sim [b\mapsto g] \}  \supseteq Q(D) $.
	Let $\sigma \in Q(D)$ be arbitrary.
	Because $\sigma \in Q_{D}$ and $g \in V$ we have that $\sigma(g) =
	g'$ for some $g' \in names(DS)$.
	Case distinction:
	\begin{enumerate}
		\item $\sigma \in Q_{T_1}(D)$:  Because the assumption implies $\sigma
			\in Q_1(D)$	we can then use the induction hypothesis  
			$\bigcup\limits_{g \in names(DS)} \{ \mu \cup [g \mapsto g'] \mid \mu
			\in \ll P_1 \rr^{DS}_{graph(g',DS)}, \mu \sim [g\mapsto g'] \}  =
			Q_1(D) $. Let $\mu = \sigma \backslash [g \mapsto g']$. 
			Thus $\mu \in \ll P_1 \rr^{DS}_{graph(g',DS)}$ 
			and we also have $\{\mu \not\sim \mu_2 \mid \forall \mu_2 \in
			\ll P_2 \rr^{DS}_{graph(g',DS)} \}$ because of our assumption.
			Thus $\mu \in \ll P \rr_{graph(g',DS)}^{DS}$ and $\mu \sim [g
			\mapsto g']$ by induction hypothesis.
		\item If $\sigma \in Q_{T'}(D)$  we can use the induction hypothesis twice:\\
			$\bigcup\limits_{g \in names(DS)} \{ \mu_i \cup [g \mapsto g'] \mid \mu_i
			\in \ll P_i \rr^{ep(ds)}_{graph(g',ep(ds))}, \mu_i \sim [g\mapsto g'] \}  =
			Q_i(D)$ for $i=1,2$.\\
			Let $\mu = \sigma \backslash [g \mapsto g']$.
			Because $\sigma \in Q_{T'}(D)$ we have $\mu = \mu_1 \cup \mu_2$ for
			some $\mu_{1|vars(Q_1)}\cup[g \mapsto g'] \in Q_1(D)$ (restricted to
			the variables in $Q_1$) and $\mu_{2|vars(Q_2)}\cup [g \mapsto g'] \in Q_2(D)$ (restricted to
			the variables in $Q_2$).
			Thus $\mu \in \ll P_1 \AND P_2 \rr^{ep(ds)}_{graph(ds,g')}$
			and $\mu \in \ll P \rr^{ep(ds)}_{graph(g',ds)}$. Also $\mu \sim [g \mapsto
			g']$ holds by induction	hypothesis.
	\end{enumerate}

	\bigskip\noindent
	Let $ds,g \in \V$ We want to prove  
	$\bigcup\limits_{g' \in names(ep(ds')), ds' \in dom(ep)} 
	\{ \mu \cup \{[ds \mapsto ds'] [g \mapsto g']\} \mid \mu
		\in \ll P \rr^{ep(ds')}_{graph(g',ep(ds'))}, \mu \sim \{[ds \mapsto ds'] [g
	\mapsto g']\} \}  \supseteq Q(D) $.
	Let $\sigma \in Q(D)$ be arbitrary.
	Because $\sigma \in Q_{D}$ and $ds,g \in V$ we have that $\sigma(g) =
	g'$ and $\sigma(ds) = ds'$ for some $g' \in names(DS)$ and $ds' \in dom(ep)$.

	Case distinction:
	\begin{enumerate}
		\item $\sigma \in Q_{T_1}(D)$: 
			we can then use the induction hypothesis  
			$\bigcup\limits_{ds' \in dom(ep), g' \in names(ep(d))} \{ \mu \cup \{[ds
				\mapsto ds'] [g	\mapsto g']\} \mid \mu
				\in \ll P_1 \rr^{ep(ds')}_{graph(g',ep(ds'))}, \mu \sim \{[ds
			\mapsto ds'] [g \mapsto g']\} \}  =	Q_1(D) $. 

			Thus $\mu \in \ll P_1 \rr^{ep(ds')}_{graph(g',ep(ds'))}$ 
			and we also have $\{\mu \not\sim \mu_2 \mid \forall \mu_2 \in
			\ll P_2 \rr^{ep(ds')}_{graph(g',ep(ds'))} \}$ because of our assumption.
			Thus $\mu \in \ll P \rr_{graph(g',ep(ds'))}^{ep(ds')}$ and $\mu \sim
			\{[ds' \mapsto ds],[g'\mapsto g']\}$ by induction hypothesis.

		\item If $\sigma \in Q_{T}(D)$  we can use the induction hypothesis twice:\\
			$\bigcup\limits_{ds' \in dom(ep),g' \in names(ep(ds'))} \{ \mu_i \cup\{[ds
					\mapsto ds'] [g \mapsto g'\}  \mid \mu_i
					\in \ll P_i \rr^{ep(ds')}_{graph(g',ep(ds'))}, \mu_i \sim \{[ds
				\mapsto ds'] [g \mapsto g']\}\}  = Q_i(D)$ for $i=1,2$.\\
				Let $\mu = \sigma \backslash \{[ds \mapsto ds'] [g \mapsto g']\}$.
				Because $\sigma \in Q_{T'}(D)$ we have $\mu = \mu_1 \cup \mu_2$ for
				some $\mu_{1|vars(Q_1)}\cup\{[ds \mapsto ds'] [g \mapsto g']\}$(restricted to
			the variables in $Q_1$) and 
			$\mu_{1|vars(Q_1)}\{[ds \mapsto ds'] [g \mapsto g']\}$(restricted to
			the variables in $Q_2$).
				Thus $\mu \in \ll P_1 \ AND \ P_2 \rr^{ep(ds')}_{graph(g',ep(ds'))}$
				and $\mu \in \ll P \rr^{ep(ds')}_{graph(g,ep(ds'))}$. Also $\mu \sim
				\{[ds \mapsto ds'] [g \mapsto g']\}$ holds by induction
				hypothesis.
		\end{enumerate}

	%%% GRAPH %%%%%%
	\item Consider the case where $P$ is a graph pattern $(\mbox{GRAPH} \ u \ P_1)$. \\
		Our outputquery is constructed as follows: Let $Q_1 = trans(P_1,u,g)$. 
		Assuming $r_1$ is the root of $T_1$ and $\lambda(r_1) = q_1$ we define
		\[ \lambda'(x) =\begin{dcases*} 
				q_1, LOC(u,ds),LOC(g,ds)& if $x = r_1$\\
				\lambda(x) & otherwise	\\
			\end{dcases*}
		\] and $trans(P,ds,g) = (T_1,\lambda',x_1)$.
		
		$\subseteq:$\\
		$ds,g \in \U:$ \\
		Let $DS = ep(ds)$ and $G = graph(g,DS)$.
		Let $\mu \in \ll P \rr^{DS}_{G}$ be arbitrary.
		Proceed by case distinction:
		\begin{enumerate}
			\item $u \in names(DS)$: We have $\mu \in \ll P_1
				\rr^{DS}_{graph(u,DS)}$ by SPARQL semantics and assumption. By i.h. we have 
				$Q_1(D) =  \ll P_1\rr^{DS}_{graph(u,DS)}$ and thus $\mu \in
				Q_1(D)$. By
				construction of our query $Q$ we see that $\mu \in Q(D)$.

			\item $u \in \U \backslash names(DS)$:
				Then $\ll P_1 \rr^{DS}_{graph(u,DS)} = \{\}$ by SPARQL semantics. But then
				$Q(D) = \{\}$ because we added $LOC(u,ds)$ to the root of our
				query Q.
			\item $u \in V$:
				Then $\mu \in S_1$ where $S_1 =  \bigg\{\mu_1 \cup [ u \rightarrow s ] \mid
					s \in names(DS), \mu_1 \in \ll P_1
					\rr^{DS}_{graph(s,DS)}, [ u \rightarrow s ] \sim
				\mu_1 \bigg\}$. 
				By induction hypothesis we know that for $Q_1
				= trans(P_1,ds,u)$ we have
				$\bigcup\limits_{g'\in names(DS)} \{ \mu_1 \cup [u\mapsto g'] \mid
					\mu_1 \in \ll P_1 \rr^{DS}_{graph(g',DS)}, \mu_1 \sim [u
				\mapsto g']\} = Q_1(D)$. Because $\mu \in S_1$ there
				is an $s \in names(DS)$ such  that there is a $\mu_1 \in \ll P_1
				\rr^{DS}_{graph(s,DS)}$, $\mu = \mu_1 \cup [u \rightarrow
				s]$ and  $\mu_1 \sim [u \mapsto s]$. By 
				induction hypothesis we get that $\mu \in Q_1(D)$.
				Looking at the construction of $Q$ we get $\mu \in Q(D)$.
		\end{enumerate}

		\bigskip\noindent
		$ds \in \V, g \in \U:$ \\
		We need to show 
		$\bigcup\limits_{ds' \in dom(ep)} \{ \mu \cup [ds \mapsto ds'] \mid \mu \in
			\ll P\rr^{ep(ds')}_{graph(g,ep(ds'))}, \mu \sim
		[ds\mapsto ds'] \}  \subseteq Q(D)$.
		Let $ds' \in dom(ep)$ be arbitrary.
		Let $\mu \in \ll P \rr^{ep(ds')}_{graph(g,ds')}$ where 
		$\mu \sim [ds \mapsto ds']$ holds be arbitrary.
		Proceed by case distinction:
		\begin{enumerate}
			\item $u \in names(ep(ds')):$ We have $\mu \in \ll P_1
				\rr^{ep(ds')}_{graph(u,ep(ds'))}$ by SPARQL semantics. The
				induction hypothesis $\bigcup\limits_{ds' \in dom(ep)} \{ \mu \cup [ds \mapsto ds'] \mid \mu \in
					\ll P_1 \rr^{ep(ds')}_{graph(u,ep(ds'))}, \mu \sim
				[ds\mapsto ds'] \}  = Q_1(D)$ yields $\mu\cup [ds\mapsto ds'] \in Q_1(D)$. 
				By construction of our query $Q$ we see that $\mu \cup [ds\mapsto ds'] \in Q(D)$.
			\item $u \in \U \backslash names(ep(d))$:
				Then $\ll P_1 \rr^{ep(ds')}_{graph(u,ep(ds'))} = \{\}$. But then
				$Q(D) = \{\}$ because we added $LOC(u,ds)$ to the root of our
				query Q.
			\item $u \in V$:
				then $\mu \in S_1$ where $S_1 =  \bigg\{\mu_1 \cup [ u \rightarrow s ] \mid
					s \in names(ep(ds')),\\ \mu_1 \in \ll P_1
					\rr^{ep(ds')}_{graph(s,ep(ds'))} \land [ u \rightarrow s ] \sim
				\mu_1 \bigg\}$. 
				By induction hypothesis we know that we receive a wdpt $Q_1
				= trans(P_1,ds,u)$ for which \\
				$\bigcup\limits_{g'\in names(ep(ds')),ds' \in dom(ep)} \{ \mu_1 \cup
					\{[u\mapsto g'],[ds \mapsto ds']\} \mid\\
					\mu_1 \in \ll P_1 \rr^{ep(ds')}_{graph(g',ep(ds'))}, \mu_1 \sim
					\{[u \mapsto g'], [ds \mapsto ds']\} = Q_1(D)$ holds. 
					Because $\mu \in S_1$ there	is an $s \in names(ep(ds'))$ such
					that there is a $\mu_1 \in \ll P_1
					\rr^{ep(ds')}_{graph(s,ep(ds'))}$,
					$\mu = \mu_1 \cup \{[u \rightarrow
					s]\}$. We get that $\mu \in Q_1(D)$ by induction hypothesis.
					Also we have that $\mu \cup [ds \rightarrow ds'] \in Q(D)$
					because we conjunctively added $LOC(g,ds)$ to the root of $Q$ and we
					assumed $\mu \sim [ds \sim ds']$ .
			\end{enumerate}

			\bigskip\noindent
			$ds \in \U, g \in \V:$ \\ Let $DS = ep(ds)$.
			We need to show 
			$\bigcup\limits_{g' \in names(DS)} \{ \mu \cup [g\mapsto g'] \mid \mu \in
				\ll P \rr^{DS}_{graph(g',DS)}, \mu \sim
			[g\mapsto g'] \}  \subseteq Q(D) $.
			Let  $g' \in names(ep(ds))$ be arbitrary.
			Let $\mu \in \ll P \rr^{DS}_{graph(g',DS)}$ where $\mu \sim [g
			\mapsto g']$ holds be arbitrary.
			Proceed by case distinction:
			\begin{enumerate}
				\item $u \in names(DS)$:
					By SPARQL semantics we have that $\mu \in \ll P_1
					\rr^{DS}_{graph(u,DS)}$.
					We know that we receive a wdpt $Q_1	= trans(P_1,ds,u)$ 
					for which by i.h.
					$\ll P_1 \rr^{DS}_{graph(u,DS)} = Q_1(D)$ holds. Thus $\mu
					\in Q_1(D)$.
					Because of the construction of our query and especially the
					conjunct $LOC(g,ds)$ in the root of $Q$ we have that 
					$\mu \cup [g\mapsto g'] \in Q_1(D)$.

				\item $u \in \U \backslash names(ep(ds))$:
					then $\ll P_1 \rr^{DS}_{graph(u,DS)} = \{\}$. But then
					$Q(D) = \{\}$ because we added $LOC(u,ds)$ to the root of
					our query Q.
				\item $u \in V$:
					then $\mu \in S_1$ where $S_1 =  \bigg\{\mu_1 \cup [ u \rightarrow s ] \mid
						s \in names(DS), \mu_1 \in \ll P_1
						\rr^{DS}_{graph(s,DS)} \land [ u \rightarrow s ] \sim
					\mu_1 \bigg\}$. 
					By induction hypothesis we know that we receive a wdpt $Q_1
					= trans(P_1,ds,u)$ for which\\ $\bigcup\limits_{g'\in names(DS)} 
					\{ \mu_1 \cup \{[u\mapsto g']\} \mid 
						\mu_1 \in \ll P_1 \rr^{DS}_{graph(g',DS)}, 
					\mu_1 \sim \{[u \mapsto g']\}\} = Q_1(D)$ holds. Because $\mu \in S_1$ there
					is an $s \in names(DS)$ such  that there is a $\mu_1 \in \ll P_1
					\rr^{DS}_{graph(s,DS)}$, 
					$\mu = \mu_1 \cup \{[u \rightarrow s]\}$. 
					We get that $\mu \in Q_1(D)$ by induction hypothesis. 
					We thus get that $\mu\cup[g\mapsto g'] \in Q(D)$ because we
					conjunctively added $LOC(g,ds)$ to root of $Q$ and we
					asssumed $\mu \sim [g\mapsto g']$.
			\end{enumerate}

			\bigskip\noindent
			$ds,g \in \V:$ \\
			We need to show 
			$\bigcup\limits_{g' \in names(ep(ds')), ds' \in dom(ep)} \{ \mu \cup
				\{[g \mapsto g'],[ds \mapsto ds']\} \mid \mu \in
				\ll P\rr^{ep(ds')}_{graph(g',ep(ds'))}, \mu \sim
			\{[ds \mapsto ds'][g\mapsto g']\} \}  \subseteq Q(D) $.
			Let  $ds' \in dom(ep)$ and $g' \in names(ep(ds'))$ be arbitrary.
			Let $\mu \in \ll P \rr^{ep(ds')}_{graph(g',ep(ds'))}$ where $\mu
			\sim\{[ds\mapsto ds'], [g\mapsto g'] \}$ holds be arbitrary.
			Proceed by case distinction:
			\begin{enumerate}
				\item $u \in names(ep(ds))$:\\
					By SPARQL semantics we have $\mu \in \ll P_1
					\rr^{DS}_graph(u,DS)$.
					We know that we receive a wdpt $Q_1	= trans(P_1,ds',u)$ 
					for which by i.h.
					$\bigcup\limits_{ds' \in dom(ep)} \{ \mu \cup [ds \mapsto ds'] \mid \mu \in
						\ll P_1\rr^{ep(ds')}_{graph(u,ep(ds'))}, \mu \sim
					[ds\mapsto ds'] \}  = Q_1(D) $
					holds. Thus $\mu \cup [ds \mapsto ds'] \in Q_1(D)$.
					By construction of our query and especially the conjunct in the
					root of $Q$, i.e., $LOC(g,ds)$ we have that $\mu \cup
					\{[ds\mapsto ds'],[g \mapsto g']\} \in Q(D)$.

				\item $u \in \U \backslash names(ep(ds'))$
					then $\ll P_1 \rr^{ep(ds')}_{graph(u,ep(ds'))} = \{\}$. But then
					$Q(D) = \{\}$ because we added $LOC(u,ds)$ to the root of
					our query. 

				\item $u \in V$:
					then $\mu \in S_1$ where $S_1 =  \bigg\{\mu_1 \cup [ u \rightarrow s ]
						\mid s \in names(ep(ds')),\\ \mu_1 \in \ll P_1
						\rr^{ep(ds')}_{graph(s,ep(ds'))} \land [ u \rightarrow s ] \sim
					\mu_1 \bigg\}$. 
					By induction hypothesis we know that we receive a wdpt $Q_1
					= trans(P_1,ds',u)$ for which \\
					$\bigcup\limits_{ds' \in dom(ep), g' \in names(ep(ds'))} \{ \mu
						\cup \{[ds \mapsto ds'],[u \mapsto g']\} \mid \\ \mu \in
						\ll P\rr^{ep(ds')}_{graph(g',ep(ds'))}, 
					\mu \sim \{[ds\mapsto ds'], [g \mapsto g']\}\} = Q_1(D) $
					holds. Because $\mu \in S_1$ there
					is an $s \in names(ep(ds'))$ such  that there is a $\mu_1 \in \ll P_1
					\rr^{ep(ds')}_{graph(s,ep(ds'))}$, $\mu = \mu_1 \cup \{[u \rightarrow
					s]\}$ and  $\mu_1 \sim \{[u \mapsto s]\}$ by induction hypothesis. 
					We thus get that $\mu\cup \{[g\mapsto g'],[ds \mapsto ds']\} \in Q(D)$ because we
					conjunctively added $LOC(g,ds)$ to $Q$ and we assumed $\mu
					\sim \{[g\mapsto g'],[ds \mapsto ds']\}$.
			\end{enumerate}

			\bigskip\noindent$\supseteq:$\\
			$ds,g \in \U$\\
			Let $\mu \in Q(D)$ be arbitrary.
			\begin{enumerate}
				\item $u$ is a constant: Because $\mu \in Q(D)$ a part of $\mu$
					must also satisfy $Q_1(D)$ by construction of $Q$. %then $Q_1 = trans(P_1,ds,u)$ and 
					We have by induction hypothesis that $Q_1(D) = \ll P_1
					\rr^{ep(ds)}_{u}$. And thus by SPARQL semantics 
					$\mu \in \ll P\rr^{ep(ds)}_{g}$ holds.
				\item $u$ is a variable: By induction hypothesis
					$\bigcup\limits_{g' \in names(ep(ds))} \{ \mu_1 \cup [u\mapsto
						g'] \mid \mu_1 \in \ll P_1
						\rr^{ep(ds)}_{graph(g',ep(ds))}, \mu_1 \sim [u
					\mapsto g']\} = Q_1(D)$ holds. 
					By the fact that $\mu \in Q(D)$ and both $g$ and $ds$ are
					URIs we know that $\mu \in Q_1(D)$ by construction of $Q$. 
					By induction hypothesis we have $\mu = \mu_1 \cup [u \mapsto g']$ for some $g' \in
					names(ep(ds))$. We know that $\mu_1 \in \ll P_1
					\rr^{ep(ds)}_{graph(g',ep(ds))}$  and $\mu_1 \sim
					[u\mapsto g']$ by induction hypothesis.
					But this means by semantics of the GRAPH operator that $\mu \in \ll P
					\rr^{ep(ds)}_{graph(g,ep(ds))}$.
			\end{enumerate}

			\bigskip\noindent
			$ds \in \V,g \in \U$\\
			We want to show $\bigcup\limits_{ds' \in dom(ep)} \{ \mu \cup [ds
				\mapsto ds'] 
				\mid \mu \in \ll P\rr^{ep(ds')}_{graph(g,ep(ds'))}, \mu \sim
			[ds\mapsto ds'] \}  \supseteq Q(D)$.

			Let $\sigma \in Q(D)$ be arbitrary:
			\begin{enumerate}
				\item $u$ is a constant:
				   %Because $\sigma \in Q(D)$ a part of $\sigma$ must also satisfy
				   %$Q_1(D)$ by construction of $Q$.	
					We have by induction hypothesis that 
					$\bigcup\limits_{ds' \in dom(ep)} \{ \mu \cup [ds \mapsto ds'] \mid \mu \in
						\ll P_1\rr^{ep(ds')}_{graph(u,ep(ds'))}, \mu \sim
					[ds\mapsto ds'] \}  \supseteq Q_1(D)$. A part of
					$\sigma$, call it $\mu$ must fulfill $Q_1$ because of the
					construction of $Q$. Thus $\mu \in \ll P_1
					\rr^{ep(ds')}_{graph(u,ep(ds'))}$  and by SPARQL semantics $\mu \in \ll P_1
					\rr^{ep(ds')}_{graph(g,ep(ds'))}$ hold. 
					Looking at the construction of $Q$ we see that 
					$\sigma = \mu \cup [ds \mapsto ds']$ for some $ds' \in
					dom(ep)$ thus $\mu \sim [ds \mapsto ds']$ .

				\item $u$ is a variable: By induction hypothesis
					$\bigcup\limits_{ds' \in dom(ep), g' \in names(ep(ds'))} \{
						\mu_1 \cup \{[ds \mapsto ds'],[u \mapsto g']\} \mid \mu_1 \in
						\ll P_1\rr^{ep(ds)}_{graph(g',ep(ds))}, 
						\mu_1 \sim
					\{[ds \mapsto ds'], [u \mapsto g']\}\} = Q_1(D) $  
					holds.  Some submapping of $\sigma$ must fulfill $Q_1(D)$ by
					construction of $Q$ and $\sigma \in Q(D)$. Call it $\mu$.
					%Also $\sigma(ds) = ds'$ for some $ds' \in dom(ep)$ by
					%construction of $Q$ and $D$.
					By the fact that $\mu \in Q_1(D)$ we thus know by induction
					hypothesis that
					$\mu = \mu_1 \cup \{[u \mapsto g'],[ds \mapsto ds']\}$
					for some $g' \in names(ep(ds'))$ 
					and some $ds' \in dom(ep)$. Also, we know that $\mu_1 \in \ll P_1
					\rr^{ep(ds')}_{graph(g',ep(ds'))}$ and $\mu_1 \sim
					[u\mapsto g']$.
					But this means by semantics of the GRAPH operator that
					$(\mu_1\cup[u\mapsto ds]) \in \ll P
					\rr^{ep(ds')}_{graph(g,ep(ds'))}$.
					Because $\mu \in Q_1(D)$ and $\mu = \mu_1
					\cup \{ [ds \mapsto ds'],
					[u \mapsto g'] \}$ implies
					$(\mu_1\cup\{u \mapsto g'\}) \sim   [ds
					\mapsto ds']$ we are done.  
			\end{enumerate}

			\bigskip\noindent
			$ds \in \U,g \in \V$: Let $DS = ep(ds)$.
			Let $\sigma(g) = f$.  By construction of $Q$ and $D$ we
			get that $f \in names(ep(ds))$.

			We want to show 
			$\bigcup\limits_{f \in names(DS)}\{ \mu \cup [g \mapsto f] \mid \mu \in
				\ll P\rr^{DS}_{graph(f,DS)} , \mu \sim
			[g\mapsto f] \} \supseteq Q(D)$.
			Let $\sigma \in Q(D)$ be arbitrary.
			\begin{enumerate}
				\item 					By induction hypothesis we know that 
					$Q_1(D) = \ll P_1\rr^{DS}_{u}$. Looking at our mapping
					$\sigma$ we know that there is a part of $\sigma$, call it $\mu$
					for which $\mu \in Q_1(D)$ and thus $\mu \in \ll P_1
					\rr^{DS}_{u}$ and  by SPARQL semantics $\mu \in \ll P_1
					\rr^{DS}_{f}$ hold. 
					We have $\mu \sim [g \mapsto f]$ for some 
					$f \in names(DS)$ because of the
					conjunct $LOC(g,ds)$ in the root of $Q$ and $\sigma = \mu \cup [g \mapsto f]$.
				\item If $u$ is a variable then	consider $Q_1 = trans(P_1,ds,
					u)$. By induction hypothesis
					$\bigcup\limits_{g'\in names(DS)} \{ \mu_1 \cup [u\mapsto g'] \mid
						\mu_1 \in \ll P_1 \rr^{DS}_{graph(g',DS)}, \mu_1 \sim [u
					\mapsto g']\} = Q_1(D)$ holds. 
					Some part of $\sigma$ must satisfy $Q_1$ by the construction of
					$Q$ and $\sigma \in Q(D)$, call it $\mu$.
					We first show that $\mu \in \ll P\rr^{DS}_{graph(f,DS)}$:
					This means, we need to show that $\mu \in \ll P_1
					\rr^{DS}_{graph(s,DS}$ for $s \in names(DS)$ and $[u \mapsto s]
					\sim \mu$.
					By our i.h. and $\mu \in Q_1(D)$ we know that
					$\mu = \mu_1 \cup \{[u \mapsto s] \}$, for some $s \in
					names(DS)$. We know again by i.h. that $\mu_1 \in \ll P_1
					\rr^{DS}_{graph(s,DS)}$ and $\mu_1 \sim [u\mapsto s']$.
					But this means by semantics of the GRAPH operator that $\mu_1
					\cup [u \mapsto s] \in \ll P
					\rr^{DS}_{graph(f,DS)}$ It remains to show that $\mu \sim
					[g \mapsto f]$. Looking at the construction of the query we know
					that $LOC(g,ds)$ must be fulfilled by $\sigma$. 
					Thus $\mu \sim [g\mapsto f]$
					because $\mu$ is a part of $\sigma$.
			\end{enumerate}

			\bigskip\noindent
			$ds,g \in \V$:\\
			We want to show 
			$\bigcup\limits_{ds' \in dom(ep), f \in names(ep(ds'))} \{ \mu \cup
				\{[ds \mapsto ds'],[g \mapsto f]\} \mid \mu \in
				\ll P\rr^{ep(ds')}_{graph(g',ep(ds'))}, 
				\mu \sim
			\{[ds\mapsto ds'], [g \mapsto f]\}\} \supseteq Q(D) $
			Let $\sigma \in Q(D)$ be arbitrary. Let $\sigma(ds) = ds'$ and
			$\sigma(g) = f$. By the construction of $Q$ and $D$, $ds' \in
			dom(ep)$ and $f \in names(ep(ds'))$.
			\begin{enumerate}
				\item $u$ is a constant: \\
					By induction hypothesis we know that 
					$\bigcup\limits_{ds' \in dom(ep)} \{ \mu \cup [ds \mapsto ds'] \mid \mu \in
						\ll P_1 \rr^{ep(ds')}_{graph(g,ep(ds'))}, \mu \sim
					[ds\mapsto ds'] \}  = Q_1(D) $. Looking at our mapping
					$\sigma$ we know that there is a part of $\sigma$, call it $\mu$
					for which $\mu \in Q_1(D)$ and thus
					$\mu \backslash [ds\mapsto ds'] \in \ll P_1
					\rr^{ep(ds')}_{u}$. By semantics of graph we have 
					$\mu \backslash [ds\mapsto ds'] \in \ll P_1
					\rr^{ep(ds')}_{f}$.
					We have $\mu \backslash [ds \mapsto ds'] \sim [g \mapsto f]$ 
					and $\mu \sim [ds \mapsto ds']$ because  $\mu \backslash [ds \mapsto
					ds']\subseteq \sigma$, $\sigma \in Q(D)$ and 
					because of the conjunct $LOC(g,ds)$ in the root of $Q$.
				\item $u$ is a variable:
					By induction hypothesis
					$\bigcup\limits_{ds' \in dom(ep), g' \in names(ep(ds'))} \{ \mu
						\cup \{[ds \mapsto ds'],[u \mapsto g']\} \mid \mu \in
						\ll P_1\rr^{ep(ds')}_{graph(g',ep(ds'))}, 
						\mu \sim
					\{[ds\mapsto ds'], [u \mapsto g']\}\} = Q_1(D) $  
					holds. 
					Some part of $\sigma$ must satisfy $Q_1$ by the construction of
					$Q$ and $\sigma \in Q(D)$, call it $\mu$. Let $\mu(u) = g'$
					for some $g' \in names(ds')$. Because $\mu \in
					Q_1(D)$ we can use the induction hypothesis:
					This means $\mu\backslash\{[ds \mapsto ds'],[u
					\mapsto g']\} \in \ll
					P_1\rr^{ep(ds')}_{graph(g',ep(ds'))}$. Obviously $\mu
					\backslash\{[ds \mapsto ds']\} \sim [u \mapsto g']\}$.
					From the semantics of the GRAPH operator we get 
					$\mu\backslash\{[ds \mapsto ds']\} \in \ll
					P_1\rr^{ep(ds')}_{graph(f,ep(ds'))}$.
					It remains to show that $\mu \sim
					[g \mapsto f]$ and $\mu \sim [ds \mapsto ds']$. Looking at
					the construction of the query we know
					that $LOC(g,ds)$ must be fulfilled by $\sigma$. 
					Thus $\mu \sim [g\mapsto f]$
					because $\mu$ is a part of $\sigma$ the same arguments
					hold for $\mu \sim [ds \mapsto ds']$.
			\end{enumerate}

			%%% SERVICE %%%%%
		\item Consider the case where $P$ is a graph 
			pattern of the form $(\mbox{SERVICE} \ u \ P_1)$.
			$trans$ checks if $u \in \U$ and $u \notin dom(ep)$. If this is the
			case $Q: \{\} \rightarrow$. Otherwise we let $Q = trans(P_1,u,g)$
			and add $LOC(u,ds)$ and $LOC(g,ds)$ to the root of $Q$.\\
			$\subseteq:$\\
			Let $ds,g \in U$. Let $DS = ep(ds)$ and $G = ep(g,DS)$.
			Let $\mu \in \ll P \rr^{DS}_{G}$ be arbitrary.
			\begin{enumerate}
				\item  $u \in dom(ep)$:
					that means that $\mu \in \ll P_1
					\rr^{ep(u)}_{graph(def,ep(u))}$
					but by i.h. we know that 
					$\ll P_1 \rr^{ep(u)}_{graph(def,ep(u))}$ =
					$Q_1(D)$.
					By construction of $Q$ we have $\mu \in Q(D)$.
				\item $u \in \U\backslash dom(ep)$:
					Thus $\mu$ = $\mu_\emptyset$ which is the empty
					mapping. This is the same mapping our query
					$\{\} \leftarrow$ returns and we are
					done.
				\item $u \in \V$:
					Thus $\mu \in \{ \mu_1 \cup [u \rightarrow s ] \mid
						s \in dom(ep), \mu_1 \in \ll P_1
						\rr^{ep(s)}_{graph(def,ep(s))} \land
					[u \rightarrow s] \sim \mu_1 \}$ by semantics. 
					Assume $\mu(u) = s$.
					By induction hypothesis we know that 
					$\bigcup\limits_{ds' \in dom(ep)} \{ \mu_1 \cup [u
						\mapsto ds'] \mid \mu_1 \in
						\ll P_1\rr^{ep(ds')}_{graph(def,ep(ds'))},
					\mu_1 \sim [u\mapsto ds'] \}  =
					Q_1(D) $. By semantics of SPARQL $\mu = \mu_1 \cup
					[u \mapsto s]$ 
					for $\mu_1 \in \ll P_1 \rr^{ep(s)}_{graph(def,ep(s))}$. 
					By induction hypothesis we can conclude
					$\mu \in Q_1(D)$.
					By construction of $Q$ we have have $\mu = (\mu_1 \cup [u
					\mapsto s])	\in Q(D)$. 
			\end{enumerate}

			\bigskip\noindent
			Let $ds \in \V,g \in \U$.
			We want to show  
			$\bigcup\limits_{ds' \in dom(ep)} \{ \mu \cup [ds \mapsto ds'] \mid \mu \in
			\ll P\rr^{ep(ds')}_{graph(g,ep(ds'))}, \mu \sim  [ds\mapsto ds'] \}  
			\subseteq Q(D)$.
			Let $ds' \in dom(ep)$ be arbitrary.
			Let $\mu \in \ll P \rr^{ep(ds')}_{graph(g,ds')}$ so that $\mu \sim
			[ds\mapsto ds']$.
			\begin{enumerate}
				\item  $u \in dom(ep)$:
					$\mu \in \ll P_1\rr^{ep(u)}_{graph(def,ep(u))}$ by semantics.
					By i.h. we know that 
					$\ll P_1 \rr^{ep(u)}_{graph(def,ep(u))}$ =
					$Q_1(D)$. By construction of $q$ and especially the conjunct
					$LOC(ds,g)$ in the root, we get $\mu\cup[ds \mapsto ds'] \in Q$.
				\item $u \in I\backslash dom(ep)$:
					By semantics $\mu$ = $\mu_\emptyset$ which is the empty
					mapping. This is the same mapping our query
					$\{\} \leftarrow$ returns and we are
					done.
				\item $u \in \V$:\\
					By semantics $\mu \in \{ \mu_1 \cup [u \rightarrow s ] \mid
						s \in dom(ep), \mu_1 \in \ll P_1
						\rr^{ep(s)}_{graph(def,ep(s))} \land
					[u \rightarrow s] \sim \mu_1 \}$. Assume $\mu(u) = s$.
					By induction hypothesis we know that 
					$\bigcup\limits_{ds' \in dom(ep)} \{ \mu_1 \cup [u
						\mapsto ds'] \mid \mu_1 \in
						\ll P_1\rr^{ep(ds')}_{graph(def,ep(ds'))},
					\mu_1 \sim [u\mapsto ds'] \}  =
					Q_1(D) $. By semantics of SPARQL $\mu = \mu_1 \cup
					[u \mapsto s]$ for $\mu_1 \in \ll P_1 \rr^{ep(s)}_{graph(def,ep(s))}$.
					By induction hypothesis we can conclude
					$\mu \in Q_1(D)$.
					By construction of $Q$ and especially the conjunct
					$LOC(ds,g)$ in the root, we have 
					$\mu \cup [ds\mapsto ds'] = (\mu_1 \cup [u
					\mapsto s] \cup [ds \mapsto ds']) \in Q(D)$. 
			\end{enumerate}

			\bigskip\noindent
			Let $ds \in \U,g \in \V$.
			We want to show  $\bigcup\limits_{g' \in names(ep(ds))}\{ \mu \cup [g
				\mapsto g'] \mid \mu \in \ll P\rr^{ep(ds)}_{graph(g',ep(ds))},
			\mu \sim	[g\mapsto g'] \} \subseteq Q(D)$.
			Let $g \in names(ep(ds))$ be arbitrary.
			Let $\mu \in \ll P \rr^{ep(ds)}_{graph(g',ep(ds))}$ so that $\mu \sim
			[g\mapsto g']$.

			\begin{enumerate}
				\item  $u \in dom(ep)$:
					By semantics $\mu \in \ll P_1
					\rr^{ep(u)}_{graph(def,ep(u))}$
					and by i.h. we know that $\ll P_1 \rr^{ep(u)}_{graph(def,ep(u))}$ =	$Q_1(D)$.
					By construction of $Q$ and especially the conjunct
					$LOC(ds,g)$ in the root of $Q$ we have that $\mu\cup[g \mapsto g'] \in Q$.
				\item $u \in \U\backslash dom(ep)$:
					By semantics $\mu$ = $\mu_\emptyset$ which is the empty
					mapping. This is the same mapping our query
					$\{\} \leftarrow$ returns and we are
					done.
				\item $u \in \V$:
					By semantics $\mu \in \{ \mu_1 \cup [u \rightarrow s ] \mid
						s \in dom(ep), \mu_1 \in \ll P_1
						\rr^{ep(s)}_{graph(def,ep(s))} \land
					[u \rightarrow s] \sim \mu_1 \}$. Assume $\mu(u) = s$.
					By induction hypothesis we know that 
					$\bigcup\limits_{ds' \in dom(ep)} \{ \mu_1 \cup [u
						\mapsto ds'] \mid \mu_1 \in
						\ll P_1\rr^{ep(ds')}_{graph(def,ep(ds'))},
					\mu_1 \sim [u\mapsto ds'] \} = Q_1(D) $.
					By semantics of SPARQL $\mu = \mu_1 \cup
					[u \mapsto s]$ for  $\mu_1 \in \ll
					P_1\rr^{ep(s)}_{graph(def,ep(s))}$. 
					By induction hypothesis we can conclude
					$\mu \in Q_1(D)$.
					By construction of $Q$ and especially the conjunct
					$LOC(ds,g)$ we have $(\mu \cup [g \mapsto g']) = (\mu_1 \cup [u \mapsto s] 
					\cup [g	\mapsto g']) \in Q(D)$. 
			\end{enumerate}

			\bigskip\noindent
			Let $ds,g \in \V$.
			We want to show  	
			$\bigcup\limits_{ds' \in dom(ep), g' \in names(ep(ds'))} \{ \mu \cup
				\{[ds \mapsto ds'],[g \mapsto g']\} \mid \mu \in
				\ll P\rr^{ep(ds')}_{graph(g',ep(ds'))}, 
				\mu \sim
			\{[ds \mapsto ds'], [g \mapsto g']\}\} = Q(D)$.
			Let $ds' \in dom(ep)$ and $g' \in nameS(ep(ds'))$ be arbitrary.
			Let $\mu \in \ll P \rr^{ep(ds')}_{graph(g,g')}$ so that $\mu \sim
			[ds\mapsto ds']$.
			\begin{enumerate}
				\item  $u \in dom(ep)$:
					By semantics $\mu \in \ll P_1
					\rr^{ep(u)}_{graph(def,ep(u))}$
					but by i.h. we know that 
					$\ll P_1 \rr^{ep(u)}_{graph(def,ep(u))}$ =
					$Q_1(D)$.
					By construction of $Q$ and especially the conjunct
					$LOC(ds,g$ in the root of $Q$ we
					have that $\mu\cup \{[ds \mapsto ds'][g \mapsto g']\} \in Q$.
				\item $u \in \U\backslash dom(ep)$:
					By semantics $\mu$ = $\mu_\emptyset$ which is the empty
					mapping. This is the same mapping our query
					$\{\} \leftarrow$ returns and we are
					done.
				\item $u \in \V$:
					
					By semantics $\mu \in \{ \mu_1 \cup [u \rightarrow s ] \mid
						s \in dom(ep), \mu_1 \in \ll P_1
						\rr^{ep(s)}_{graph(def,ep(s))} \land
					[u \rightarrow s] \sim \mu_1 \}$. Assume $\mu(u) = s$. We know that $s \in dom(ep)$.
					By induction hypothesis we know that 
					$\bigcup\limits_{ds' \in dom(ep)} \{ \mu_1 \cup [u
						\mapsto ds'] \mid \mu_1 \in
						\ll P_1\rr^{ep(ds')}_{graph(def,ep(ds'))},
					\mu_1 \sim [u\mapsto ds'] \}  = Q_1(D) $. 
					By semantics of SPARQL $\mu = \mu_1 \cup
					[u \mapsto s]$ for $\mu_1 \in \ll P_1\rr^{ep(s)}_{graph(def,ep(s))}$. 
					By induction hypothesis we can conclude
					$\mu \in Q_1(D)$.
					By construction of $Q$ and especially its conjunct in the
					root $LOC(ds,g)$ we have 
					$(\mu \cup \{[ds \mapsto ds'][g \mapsto g']\}) =
					\mu_1 \cup \{[ds \mapsto ds'][g \mapsto g'], [u
					\mapsto s]\} \in Q(D)$. 
			\end{enumerate}



			\bigskip\noindent$\supseteq:$\\
			Let $ds,g \in \U$ and $\mu \in Q(D)$ be arbitrary.
			\begin{enumerate}
				\item Assume $u$ is an URI. 
					By i.h. we know that 
					$\ll P_1 \rr^{ep(u)}_{graph(def,ep(u))}$ =
					$Q_1(D)$ and thus by construction of the query $Q$ and the
					fact that $\mu \in Q(D)$, $\mu \in \ll P
					\rr^{ep(ds)}_{graph(g,ep(ds))}$
				\item Assume $u$ is a variable. By induction hypothesis we
					know that  $\bigcup\limits_{ds' \in dom(ep)} \{ \mu_1 \cup [u
						\mapsto ds'] \mid \mu_1 \in
						\ll P_1\rr^{ep(ds')}_{graph(def,ep(ds'))},
					\mu_1 \sim [u\mapsto ds'] \}  = Q_1(D) $.
					%As $ Q = Q_1(D)  \land LOC(ds,g) \land  LOC(u,g)$ and the fact that
					%$\mu \in Q(D)$ we can instantly see that $\mu \in \ll P
					%\rr^{ep(ds)}_{graph(g,ep(ds))}$:
					Assume w.l.o.g. $\mu(u) = s$.
					$\mu_1 = \mu \backslash [u \mapsto s]$. By induction hypothesis we know
					that $\mu_1 \in \ll P_1 \rr^{ep(s)}_{graph(def,ep(s))}$. By
					construction of $Q$ especially the conjunct $LOC(u,g)$, $s \in dom(ep)$ 
					and by our induction hypothesis $\mu_1 \sim	[u \mapsto s]$ holds.
					By semantics $\mu \in \ll P \rr^{ds}_{graph(g,ep(ds))}$ follows.
			\end{enumerate}

			\bigskip\noindent
			Let $ds \in \V, g \in \U$ and $\sigma \in Q(D)$ be arbitrary.
			We need to show  $\bigcup\limits_{ds' \in dom(ep)} \{ \mu \cup [ds
				\mapsto ds'] \mid \mu \in
				\ll P\rr^{ep(ds')}_{graph(g,ep(ds'))}, \mu \sim
			[ds \mapsto ds'] \}  \supseteq Q(D) $.
			Assume that $\sigma(ds) = ds'$, because of the conjunct $LOC(ds,g)$
			we have that $ds' \in dom(ep)$.
			\begin{enumerate}
				\item Assume $u$ is an URI. 
					By i.h. we know that 
					$\ll P_1 \rr^{ep(u)}_{graph(def,ep(u))}$ =
					$Q_1(D)$. By the fact that $\sigma \in Q(D)$ and the
					construction of $Q$ we can deduce that for a part of $\sigma$,
					call it $\mu$, $\mu \in Q_1(D)$ holds. By i.h. we get
					$\mu \in  \ll P_1 \rr^{ep(u)}_{graph(def,ep(u))}$.
					From this we can deduce $\mu \in  \ll P_1
					\rr^{ep(ds')}_{graph(g,ep(ds'))}$. Because $[ds \mapsto
					ds'] \in \sigma$ and $\mu \subseteq \sigma$ we have $\mu
					\sim [ ds \mapsto ds']$.

					%and  because $Q = Q_1 \land LOC(ds,g) \land
					%LOC(u,g)$ 
					%we have $\sigma = \mu \cup [ ds \mapsto ds']$ for some $\mu \in Q_1 (D)$.
					%It is by construction of our query thus obvious that
					%$\mu \sim [ds \mapsto ds']$ 
					%and $\mu \in \ll P \rr^{ep(ds')}_{graph(g,ep(ds')}$ 
					%for some $ds'\in dom(ep)$ hold.
				\item Assume $u$ is a variable. By induction hypothesis we
					know that  $\bigcup\limits_{ds' \in dom(ep)} \{ \mu_1 \cup [u
						\mapsto ds'] \mid \mu_1 \in
						\ll P_1\rr^{ep(ds')}_{graph(def,ep(ds'))},
					\mu_1 \sim [u\mapsto ds'] \}  = Q_1(D)$.
					Let $\mu = \sigma \backslash [ds \mapsto ds']$. 
					We can see that $\mu \in \ll P
					\rr^{ep(ds')}_{graph(g,ep(ds'))}$ for some $ds' \in dom(ep)$:
					Assume w.l.o.g. $\mu(u) = s$.
					$\mu_1 = \mu \backslash [u \mapsto s]$. By induction hypothesis we know
					that $\mu_1 \in \ll P_1 \rr^{ep(s)}_{graph(def,ep(s))}$. By our
					construction $s \in dom(ep)$ and $\mu_1 \sim
					[u \mapsto s]$ holds. It remains to show that $\mu \sim
					[ds \mapsto ds']$
					which follows from the construction of the query $Q$, i.e., the conjunct
					$LOC(ds,g)$ in the root of $Q$ and the fact that $\sigma \in Q(D)$.
			\end{enumerate}

			\bigskip\noindent
			Let $ds \in \U, g \in \V$ and $\sigma \in Q(D)$ be arbitrary.
			We need to show 
			$\bigcup\limits_{g' \in names(ep(ds))}\{ \mu \cup [g \mapsto g'] \mid \mu \in
				\ll P\rr^{ep(ds)}_{graph(g',ep(ds))} , \mu \sim
			[g\mapsto g'] \} \supseteq Q(D) $. 
			Assume that $\sigma(g) = g'$, because of the conjunct $LOC(ds,g)$
			we have that $g' \in names(ep(ds))$.
			\begin{enumerate}
				\item Assume $u$ is an URI. 
					By i.h. we know that 
					$\ll P_1 \rr^{ep(u)}_{graph(def,ep(u))}$ =
					$Q_1(D)$. By the fact that $\sigma \in Q(D)$ and the
					construction of $Q$ we can deduce that for a part of $\sigma$,
					call it $\mu$, $\mu \in Q_1(D)$ holds. By i.h. we get
					$\mu \in  \ll P_1 \rr^{ep(u)}_{graph(def,ep(u))}$.
					From this we can deduce $\mu \in  \ll P_1
					\rr^{ep(ds')}_{graph(g,ep(ds'))}$. Because $[g \mapsto
					g'] \in \sigma$ and $\mu \subseteq \sigma$ we have $\mu
					\sim [ g \mapsto g']$.
					%By i.h. we know that 
					%$\ll P_1 \rr^{ep(u)}_{graph(def,ep(u)}$ =
					%$Q_1(D)$ and  because $Q = Q_1 \land LOC(ds,g) \land
					%LOC(u,g)$ we have $\sigma = \mu \cup [ ds \mapsto ds']$ 
					%for some $\mu \in Q_1 (D)$.
					%It is by construction of our query obvious that $\mu
					%\sim [g\mapsto g']$  
					%and $\mu \in \ll P \rr^{ep(ds)}_{graph(g',ep(ds))}$ 
					%for some $g' \in names(ep(ds))$ hold.

				\item Assume $u$ is a variable. By induction hypothesis we
					know that  $\bigcup\limits_{ds' \in dom(ep)} \{ \mu_1 \cup [u
						\mapsto ds'] \mid \mu_1 \in
						\ll P_1\rr^{ep(ds')}_{graph(def,ep(ds'))},
					\mu_1 \sim [u\mapsto ds'] \}  = Q_1(D) $.
					%We have by construction $ Q = Q_1(D)  \land LOC(ds,g) \land  LOC(u,g)$
					%and by assumption 
					%$\sigma \in Q(D)$. 
					%Let $\sigma(g) = g'$ by the fact that
					%$\sigma \in Q(D)$ and $Q =Q_1 \land LOC(ds,g) \land
					%LOC(u,g)$ we have that $g \in names(ds))$. 
					Let $\mu = \sigma \backslash [g
					\mapsto g']$.
					We can see that $\mu \in \ll P
					\rr^{ep(ds)}_{graph(g',ep(ds))}$:
					Assume w.l.o.g. $\mu(u) = s$.
					$\mu_1 = \mu \backslash [u \mapsto s]$. By induction hypothesis we know
					that $\mu_1 \in \ll P_1 \rr^{ep(s)}_{graph(def,ep(s))}$. By our
					construction $s \in dom(ep)$ and by our induction hypothesis $\mu_1 \sim
					[u \mapsto s]$ holds. It remains to show that $\mu \sim
					[g \mapsto g']$
					which follows from the construction of the query $Q$, i.e., the conjunct
					$LOC(ds,g)$ and the fact that $\sigma \in Q(D)$.
			\end{enumerate}

			\bigskip\noindent
			Let $ds,g \in \V$ and $\sigma \in Q(D)$ be arbitrary.
			We need to show  \\
			$\bigcup\limits_{ds' \in dom(ep), g' \in names(ep(d))} \{ \mu \cup
				\{[ds \mapsto ds'],[g \mapsto g']\} \mid\\ \mu \in
				\ll P\rr^{ep(ds')}_{graph(g',ep(ds'))}, 
				\mu \sim
			\{[ds\mapsto ds'], [g \mapsto g']\}\} \supseteq Q(D) $. 	
			Assume that $\sigma(ds) = ds'$ and $\sigma(g) = g'$, because of the conjunct $LOC(ds,g)$
			we have that $ds' \in dom(ep)$ and $g' \in names(ep(ds'))$.
			\begin{enumerate}
				\item Assume $u$ is an URI. 
					By i.h. we know that 
					$\ll P_1 \rr^{ep(u)}_{graph(def,ep(u))}$ =
					$Q_1(D)$. By the fact that $\sigma \in Q(D)$ and the
					construction of $Q$ we can deduce that for a part of $\sigma$,
					call it $\mu$, $\mu \in Q_1(D)$ holds. By i.h. we get
					$\mu \in  \ll P_1 \rr^{ep(u)}_{graph(def,ep(u))}$.
					From this we can deduce $\mu \in  \ll P_1
					\rr^{ep(ds')}_{graph(g,ep(ds'))}$. Because $\{[ds \mapsto
					ds'],[g \mapsto g']\} \in \sigma$ and $\mu \subseteq \sigma$ we have $\mu
					\sim \{[ ds \mapsto ds'], [g \mapsto g']\}$.
					%By i.h. we know that 
					%$\ll P_1 \rr^{ep(u)}_{graph(def,ep(u)}$ =
					%$Q_1(D)$ and  because $Q = Q_1 \land LOC(ds,g) \land
					%LOC(u,g)$ 
					%we have $\sigma = \mu \cup [ ds \mapsto ds']$ for some $\mu \in Q_1 (D)$.
					%It is by construction of our query obvious that $\mu
					%\sim [g\mapsto g']$  
					%and $\mu \in \ll P \rr^{ep(ds')}_{graph(g',ep(ds'))}$ 
					%for some $ds' \in dom(ep)$ and $g' \in names(ep(ds'))$ hold.

				\item Assume $u$ is a variable. By induction hypothesis we
					know that  $\bigcup\limits_{ds' \in dom(ep)} \{ \mu_1 \cup [u
						\mapsto ds'] \mid \mu_1 \in
						\ll P_1\rr^{ep(ds')}_{graph(def,ep(ds'))},
					\mu_1 \sim [u\mapsto ds'] \}  = Q_1(D) $.
				%	We have by construction $ Q = Q_1(D)  \land LOC(ds,g) \land  LOC(u,g)$
				%	and by assumption 
				%	$\sigma \in Q(D)$.
				%	
				%	Let $\sigma(g) = g'$ and $\sigma(ds) = ds'$ by the fact that
				%	$\sigma \in Q(D)$ and $Q =Q_1 \land LOC(ds,g) \land
				%	LOC(u,g)$ we have that $ds' \in dom(ep)$ and $g \in names(ds'))$. 
					Let $\mu = \sigma \backslash \{ [g
				\mapsto g'][ds \mapsto ds']\}$.
					We can see that $\mu \in \ll P
					\rr^{ep(ds')}_{graph(g',ep(ds'))}$ for some $ds' \in
					dom(ep)$ and $g' \in names(ep(ds'))$:
					Assume w.l.o.g. $\mu(u) = s$.
					$\mu_1 = \mu \backslash [u \mapsto s]$. By induction hypothesis we know
					that $\mu_1 \in \ll P_1 \rr^{ep(s)}_{graph(def,ep(s))}$. By our
					construction $s \in dom(ep)$ and by our induction hypothesis $\mu_1 \sim
					[u \mapsto s]$ holds. 
					It remains to show that $\mu \sim \{[ds \mapsto ds'] [g
					\mapsto g'] \}$
					which follows from the construction of the query $Q$, i.e., the conjunct
					$LOC(ds,g)$ and the fact that $\sigma \in Q(D)$.
			\end{enumerate}

			%%% UNION%%%%%
	%	\item Consider the case where $P$ is a graph pattern $(P_1 \ UNION \ P_2)$. 
	%		Let $ds,g \in \U$ because UNION is top level.\\
	%		$\subseteq:$\\
	%		Let $\mu \in \ll P \rr^{ep(ds)}_{graph(g,ep(ds))}$ be arbitrary:
	%		By semantics of SPARQL $\mu \in \ll P_1
	%		\rr^{ep(ds)}_{graph(g,ep(ds))}$ or 
	%		$\mu \in \ll P_2 \rr^{ep(ds)}_{graph(g,ep(ds))}$. By i.h. 
	%		$\ll P_i \rr^{ep(ds)}_{graph(g,ep(ds))} = Q_i(D)$
	%		where $i = 1,2$. 
	%		Assume $\mu_1 = \mu$ then we have by induction hypothesis
	%		$\mu_1 = \mu \in Q(D)$.
	%		Analogously for $\mu_2 = \mu$.

	%		\noindent$\supseteq:$\\
	%		Let $\mu \in Q(D)$ be arbitrary:
	%		Because $Q = \{Q_1,Q_2\}$ is a union of conjunctive queries we have
	%		that $\mu \in Q_1(D)$ or $\mu \in Q_2(D)$.
	%		Assume $\mu \in Q_1(D)$.
	%		Because  $ Q_1(D) = \ll P_1 \rr_{ep(ds)}^{graph(g,ds)}$ 
	%		we have that $\mu \in \ll P \rr^{ep(ds)}_{graph(g,ds)}$.
	\end{enumerate}
\end{proof}

%\noindent Let $tr$ be the following function $tr: P \times DS \times \mu \mapsto Q \times D \times h$
%where $P$ is a SPARQL graph pattern, 
%$DS$ is a SPARQL dataset and $\mu$ is a mapping.  $Q$ is a
%well-designed pattern tree, $D$ is a database and $h$ is a mapping 
%(homomorphism from $Q$ to $D$). Our aim is to define $trans$ in 
%such way, that the following equivalence holds: 
%$\mu \in \ll P \rr^{DS}_{G} \Leftrightarrow h \in Q(D)$.
The next step is to show that our well designed pattern tree that we received
from our $trans$ function is a pattern tree, as it was defined by Definition~\ref{def:wdpt}.

\begin{lemma}
	Let $ds \in dom(ep)$, $g \in names(ep(ds))$ and $P$ be a SPARQL pattern.
	Then $F = trans(P,ds,g)$ is a well-designed pattern tree. 
\end{lemma}

\begin{proof}
	Let $Q=(T,\lambda,x) \in F$ be arbitrary. We prove each property seperately:
	\begin{enumerate}
		\item $T$ is rooted in a distinguished node $r$, the root and maps
			each node $t$ in  $T$ to a set of relational atoms over $\sigma$:
			This is easy to see as we only use $LOC$ and $T$ in our
			construction. Also $T$ always has a distinguished node $r$ as root
			because the only step of $trans$ that changes the structure of $T$
			is if a subppattern of the form $(P_1 \OPT P_2)$ occurs: 
			$Q_1 = (T_1,\lambda_1,x_1)  = trans(P_1,ds,g)$,
			$Q_2 = (T_2,\lambda_2,x_2)  = trans(P_2,ds,g)$. 
			It connects the roots of $T_1$ and $T_2$, call them $r_1,r_2$ with
			an edge $(r_1,r_2)$ making $r_1$ the new root.  

		\item For every variable $y$ that appears
			in $T$, the tree of the wdPT, the set of nodes of $T$ where $y$ is mentioned is connected.
			It is important to remember that we assumed that our inputpattern $P$ is 
			in OPT normal form and well-designed.
			The induction will be over the structure of the subpatterns $P'$ of $P$.
			\begin{enumerate}
			
				\item Basecase: If $\hat{P}$ is a triple pattern we only return one node, so there can't be any
					violation of the property.

				\item Induction Step: If $\hat{P}$ is $(P_1 \AND  P_2)$ there can 
					also never be a violation of the
					property because we assumed that $P$ is in OPT normal form and thus OPT never
					occurs in the scope of AND. We thus also only return a single node making a
					violation impossible.

				\item Induction Step: If $\hat{P}$ is $(P_1 \OPT  P_2)$:
					By induction hypothesis $Q_1 = trans(P_1,ds,g)$ and $Q_2 =
					trans(P_2,ds,g)$ fulfill the property and $P$ 
					is a well designed SPARQL pattern in OPT normal form by assumption. 

					Assume that making $T_2$ a child of the root of $T_1$ 
					results in a wdPT $Q$ which does not fulfill the well designedness property.
					There must thus be a variable $x$ in $T$ (the roots of $T_1$
					and $T_2$ get connected) which appears in two different subgraphs of $T$. This
					two subgraphs can only be situated in $T_1$ and $T_2$ respectively by induction
					hypothesis.	Because these subgraphs are not connected 
					by our assumption we proceed by	case distinction and assume 
					\begin{enumerate}
						\item $T_1$ contains a subgraph containing $x$ but the root $r_1$ does not
							contain the variable $x$. There must be a subpattern of $P_1$ by our
							construction $(P' \OPT P'')$ where $P''$ contains $x$ but $P'$ doesn't. 
							As $(P' \OPT P'')$ is part of $P_1$ 
							and we have $(P_1 \OPT P_2)$ and $P_2$ contains $x$ we have a
							contradiction to the assumption that $P$ is a well designed SPARQL
							patterns.
						\item $T_2$ contains a subgraph containing $x$ but the root $r_2$ does not
							contain the variable $x$. This can be shown
							analogously. %Analogously to the previous case.
					\end{enumerate}


					%If we then make the wdPT $Q_2$ a child of the wdPT $Q_1$ we know 
					%that $Q$ fulfills the property. 

				\item Induction Step: If $\hat{P}$ is $(\mbox{GRAPH }  u \ P_1)$:
					We know that $Q = trans(P_1,ds,u)$ and by induction hypothesis,
					we know that $Q = (T, \lambda,x)$ fulfills the property. Let
					$r_1$ be the root of $T$. $trans(\cdot)$ adds the conjunct
					$LOC(u,ds)$ and $LOC(g,ds)$ to $r_1$. W.l.o.g. assume $u$,$g$
					and $ds$ are variables. For every of those three variables it
					remains to prove that the property holds.
					Let $x \in \{u,g,ds\}$ and assume that our resulting wdPT $Q$ is
					not fulfilling the property. There must thus be a node $n_1$ in $Q$ which
					doesn't contain $x$ other than the root and a node $n_2$ for which $n_1$ is a parent 
					which again contains $x$ creating the conflict.
					By the definition of the function $trans(\cdot)$ we know that
					there must have been a subpattern $(P' OPT P'')$ and both $n_1,n_2$ must
					have been created by this subpattern. But this again means, that
					$P'$ did not contain $x$ and $P''$ did contain $x$. 
					%As the root
					%of $Q$, i.e. $r_1$, cannot be $n_1$, we have that there must be
					%another OPT subpattern making $r_1$ an ancestor of $n_1$.(not
					%needed)
					Depending on whether $x = u$, $x=g$ or $x=ds$ we have a
					contradiction:
					\begin{enumerate}
						\item Let $x=u$. Because we assumed $\hat{P}=
							(\mbox{GRAPH } u \ P_1)$ we know that
							$P$ is not well designed because $P_1$ contains the
							subpattern $(P' \OPT P'')$.
						\item Let $x = g$. This means that we are inside a graph
							pattern $(\mbox{GRAPH } g \ P^\sim)$ and thus $P$ is
							not well designed because $P_1$ contains the
							subpattern $(P' \OPT P'')$.
						\item Let $x = ds$. This means that we are inside a
							SERVICE
							pattern $(\mbox{SERVICE } g \ P^\sim)$ and thus $P$ is
							not well designed because $P_1$ contains the
							subpattern $(P' \OPT P'')$.
					\end{enumerate}



				\item Induction Step: If $\hat{P}$ is $(\mbox{SERVICE } u \ P_1)$:
						The proof is analogously to case where $\hat{P}$ is
						$(\mbox{GRAPH} u \ P_1)$.
				%	We know that $Q = trans(P_1,ds,u)$ and by induction hypothesis,
				%	we know that $Q = (T, \lambda,x)$ fulfills the property. Let
				%	$r_1$ be the root of $T$. $trans(\cdot)$ adds the conjunct
				%	$LOC(def,u)$ and $LOC(g,ds)$ to $r_1$. W.l.o.g. assume $u$,$g$
				%	and $ds$ are variables. For every of those three variables it
				%	remains to prove that the property holds.
				%	Let $x \in \{u,g,ds\}$ and assume that our resulting wdPT $Q$ is
				%	not fulfilling the property. There must thus be a node $n_1$ in $Q$ which
				%	doesn't contain $x$ and a node $n_2$ for which $n_1$ is a parent 
				%	which again contains $x$ creating the conflict.
				%	By the definition of the function $trans(\cdot)$ we know that
				%	there must have been a subpattern $(P' \OPT P'')$ and both $n_1,n_2$ must
				%	have been created by this subpattern. But this again means, that
				%	$P'$ did not contain $x$ and $P''$ did contain $x$. 
				%	%As the root
				%	%of $Q$, i.e. $r_1$, cannot be $n_1$, we have that there must be
				%	%another OPT subpattern making $r_1$ an ancestor of $n_1$.(not
				%	%needed)
				%	Depending on whether $x = u$, $x=g$ or $x=ds$ we have a
				%	contradiction:
				%	\begin{enumerate}
				%		\item Let $x=u$. 
				%			Because we assumed $\hat{P}= (\mbox{SERVICE } u \ P_1)$ we know that
				%			$P$ is not well designed because $P_1$ contains the
				%			subpattern $(P' \OPT P'')$.
				%		\item Let $x = g$. This means that we are inside a graph
				%			pattern $(\mbox{GRAPH } g \ P^\sim)$ and thus $P$ is
				%			not well designed because $P_1$ contains the
				%			subpattern $(P' \OPT P'')$.
				%		\item Let $x = ds$. This means that we are inside a
				%			SERVICE
				%			pattern $(\mbox{SERVICE } g \ P^\sim)$ and thus $P$ is
				%			not well designed because $P_1$ contains the
				%			subpattern $(P' \OPT P'')$.
				%	\end{enumerate}

			%	\item Induction Step: If $P$ is $(P_1 \UNION  P_2)$:
			%		We get a forest and because the property held for $Q_1$ and $Q_2$ by i.h. we are
			%		done.
			\end{enumerate}
		\item The last property is that $x$ is a tuple of distinct variables
			occurring in $T$. As we don't use projection and use set operations
			to merge our free variables in the OPT case, this is an obivous
			observation.
	\end{enumerate}
\end{proof}


\begin{theorem}\label{biglemma}
	Let $P$ be a SPARQL graph pattern, $DS$ a dataset and $G$ a graph in $DS$.
	Let $DS = ep(ds)$  and $G = graph(g,DS)$. Let $Q = trans(P,ds,g)$ and $D =
	(data(DS))$.	Then $\ll P \rr^{DS}_G = Q(D)$.
\end{theorem}
\begin{proof}
	The database $D$ is the same database that is created in
	Lemma~\ref{smalllemma}.
	Use Lemma~\ref{smalllemma}: 
	$DS = ep(ds)$  and $G = graph(g,DS)$ hold and thus $ds,g \in \U$:
	$\ll P \rr^{DS}_G = Q(D)$ follows.
\end{proof}

\section{The complexity of EVAL($\l$) where $\l \in \{P,\u,\s\}$ }

\begin{corollary}
	The problems $EVAL(P_{wdgs})$ and $EVAL(\u_{wdgs})$ are coNP-complete and
	$EVAL(\s_{wdgs})$ is $\Pi^P_2$-complete.
\end{corollary}
\begin{proof}
The problem EVAL is defined in the following way:
\begin{framed}\noindent \textbf{EVAL($\mathcal{L}$)}\\
	\textbf{INPUT:} Dataset $DS$, graph pattern $P \in  \mathcal{L}$ and a mapping $\mu$.\\
	\textbf{QUESTION:} Is $\mu$ in $\ll P \rr^{DS}_{def}$.
\end{framed}
Assume $ds = ep^{-1}(DS)$. %and $G = graph(def,DS)$.
For the membership we propose the following procedure:
Use the transformation function on the input graph pattern $P$ to obtain a wdPF
$Q$, and the data function to obtain the database $D$. More formally:
$Q = trans(P,ds,def)$ and $D = \bigcup\limits_{c\in dom(ep)} data(ep(c))$. This transformation is obviously
possible in polynomial time if we assume $dom(ep)$ to be finite. We know by Theorem~\ref{biglemma} that 
$\ll P \rr^{DS}_{def} = Q(D)$.
For the projection-free case we can use the results in 
\cite{perez2009semantics}, i.e., that the evaluation problem for well-designed
pattern trees without projection is coNP-complete and conclude a coNP runtime to
check if $\mu \in Q(D)$.
The hardness of the problem follows immediately from the hardness of
\textbf{EVAL($P_{wd}$)}.
In case of projection we use the results in \cite{letelier2013static}
that the evaluation of $\mu \in Q(D)$ is $\Pi^P_2$-complete.
Again hardness of the problem follows immediately from the hardness of 
\textbf{EVAL($\s_{wd}$)}.
\end{proof}

\section{Static Analysis}
\begin{theorem}
	The problems \textbf{EQUIVALENCE}($\l$), \textbf{CONTAINMENT}($\l$)  
	are $NP$-complete for $\l = P_{wdgs}$ and $\Pi^p_2$-complete $\l =
	\u_{wdgs}$ when we assume the function $ep(\cdot)$ as additional input.
\end{theorem}
\begin{proof}
	Hardness for each of these problems follows immediately from the fact that
	that they are already $NP$-complete for the language
	$P_{wd}$~\cite{letelier2012static}.
	%and $\Pi^{P}_2$ for the language $\U_{wd}$~\cite{pichler2014containment}.\\
	For the membership, we propose the following algorithm:
	Let $P_1,P_2$ be two arbitrary patterns where $P_1,P_2 \in P_{wdgs}$ or
	$P_1,P_2 \in \u_{wdgs}$ holds.
	Let $x \in \V$ and use our polynomial time transformation function twice to obtain two wdpts: 
	$Q_i = trans(P_i,x,def)$ for $i\in \{1,2\}$. 
	Using the NP-algorithm proposed in~\cite{letelier2012static} for $\l =
	P_{wdgs}$ we can decide whether $Q_1 \subseteq Q_2$:
%	or the
%	$\Pi^P_2$ algorithm proposed in~\cite{pichler2014containment} when $\l =
%	\u_{wdgs}$.
	\begin{enumerate}
		\item $Q_1 \subseteq Q_2$:
			Let $ds \in dom(ep)$ be arbitrary, i.e. an arbitrary URI of a dataset. 
			Let $D =\bigcup\limits_{c\in dom(ep)} data(ep(c))$. The function data is also
			obviously in polynomial time if $dom(ep)$ is assumed to be finite.
			Let $\mu$ be an arbitrary mapping in $\ll P_1 \rr^{DS}_{def}$. Because
			of Theorem~\ref{biglemma} we can conclude that $\mu \in Q_1(D)$.
			Because of our assumption we get that $\mu \in Q_2(D)$. Because
			again $Q_2(D) = \ll P_2 \rr^{DS}_{def}$ we get that $\mu \in \ll P_2
			\rr^{DS}_{def}$.
		\item $Q_1 \not\subseteq Q_2$: 
			Let $ds \in dom(ep)$ be arbitrary, i.e. an arbitrary URI of a dataset. 
			Let $D =\bigcup\limits_{c\in dom(ep)} data(ep(c))$. The function data is also
			obviously in polynomial time if $dom(ep)$ is assumed to be finite.
			By assumption there is a $\mu \in
			Q_1(D)$ such that $\mu \notin Q_2(D)$. Because
			of Theorem~\ref{biglemma} and $\mu \in Q_1(D)$ we can conclude that $\mu \in \ll P_1
			\rr^{DS}_{def}$.
			Because of our assumption we get that $\mu \notin Q_2(D)$. Because
			again $Q_2(D) = \ll P_2 \rr^{DS}_{def}$ we get that $\mu \notin \ll P_2
			\rr^{DS}_{def}$.
	\end{enumerate}
\end{proof}
%Through Lemma~\ref{biglemma} we get that SPARQL evaluation with the GRAPH and
%the SERVICE operator are the
%same as without the $GRAPH$ and $SERVICE$ patterns.
%Thus the complexity for the problems CONTAINMENT, EQUIVALENCE and SUBSUMPTION do not 
%differ from the results in \cite{pichler2014containment} using $GRAPH$ and
%$SERVICE$.

% Copyright (C) 2014-2015 by Thomas Auzinger <thomas@auzinger.name>

\documentclass[draft,final]{vutinfth} % Remove option 'final' to obtain debug information.

% Load packages to allow in- and output of non-ASCII characters.
\usepackage{lmodern}        % Use an extension of the original Computer Modern font to minimize the use of bitmapped letters.
\usepackage[T1]{fontenc}    % Determines font encoding of the output. Font packages have to be included before this line.
\usepackage[utf8]{inputenc} % Determines encoding of the input. All input files have to use UTF8 encoding.

% Extended LaTeX functionality is enables by including packages with \usepackage{...}.
\usepackage{listings}
\usepackage{fixltx2e}   % Provides fixes for several errors in LaTeX2e.
\usepackage{stmaryrd}
\usepackage{amsmath}    % Extended typesetting of mathematical expression.
\usepackage{amsthm}
\usepackage{amssymb}    % Provides a multitude of mathematical symbols.
\usepackage{mathtools}  % Further extensions of mathematical typesetting.
\usepackage{microtype}  % Small-scale typographic enhancements.
\usepackage{enumitem}   % User control over the layout of lists (itemize, enumerate, description).
\usepackage{multirow}   % Allows table elements to span several rows.
\usepackage{booktabs}   % Improves the typesettings of tables.
\usepackage{subcaption} % Allows the use of subfigures and enables their referencing.
\usepackage[ruled,linesnumbered,algochapter]{algorithm2e} % Enables the writing of pseudo code.
\usepackage[usenames,dvipsnames,table]{xcolor} % Allows the definition and use of colors. This package has to be included before tikz.
\usepackage{nag}       % Issues warnings when best practices in writing LaTeX documents are violated.
\usepackage{hyperref}  % Enables cross linking in the electronic document version. This package has to be included second to last.
\usepackage[acronym,toc]{glossaries} % Enables the generation of glossaries and lists fo acronyms. This package has to be included last.
\usepackage{tikz}

% Define convenience functions to use the author name and the thesis title in the PDF document properties.
\newcommand{\authorname}{Alexander Svozil} % The author name without titles.
\newcommand{\thesistitle}{Complexity of SPARQL} % The title of the thesis. The English version should be used, if it exists.

% Set PDF document properties
\hypersetup{
    pdfpagelayout   = TwoPageRight,           % How the document is shown in PDF viewers (optional).
    linkbordercolor = {Melon},                % The color of the borders of boxes around crosslinks (optional).
    pdfauthor       = {\authorname},          % The author's name in the document properties (optional).
    pdftitle        = {\thesistitle},         % The document's title in the document properties (optional).
    pdfsubject      = {Subject},              % The document's subject in the document properties (optional).
    pdfkeywords     = {a, list, of, keywords} % The document's keywords in the document properties (optional).
}
\lstset{
  basicstyle=\ttfamily,
  mathescape,
  numbers=left,                   % where to put the line-numbers
  numberstyle=\footnotesize,      % the size of the fonts that are used for the
  stepnumber=1,                   % the step between two line-numbers. If it is
  numbersep=10pt,                  % how far the line-numbers are from the code
  escapeinside={(*@}{@*)},
 %numbers=left,               % Ort der Zeilennummern
  %numberstyle=\tiny,          % Stil der Zeilennummern
         %stepnumber=2,               % Abstand zwischen den Zeilennummern
     %    numbersep=5pt,              % Abstand der Nummern zum Text
    %     tabsize=2,                  % Groesse von Tabs
    %     extendedchars=true,         %
         breaklines=true,            % Zeilen werden Umgebrochen
    %     keywordstyle=\color{red},
    %        frame=b,         
 %        keywordstyle=[1]\textbf,    % Stil der Keywords
 %        keywordstyle=[2]\textbf,    %
 %        keywordstyle=[3]\textbf,    %
 %        keywordstyle=[4]\textbf,   \sqrt{\sqrt{}} %
    %     stringstyle=\color{white}\ttfamily, % Farbe der String
    %     showspaces=false,           % Leerzeichen anzeigen ?
    %     showtabs=false,             % Tabs anzeigen ?
         xleftmargin=17pt,
         framexleftmargin=17pt,
         framexrightmargin=5pt,
         framexbottommargin=4pt
         %backgroundcolor=\color{lightgray},
    %     showstringspaces=false      % Leerzeichen in Strings anzeigen ?        
}

\setsecnumdepth{subsection} % Enumerate subsections.

\nonzeroparskip             % Create space between paragraphs (optional).
\setlength{\parindent}{0pt} % Remove paragraph identation (optional).

\makeindex      % Use an optional index.
\makeglossaries % Use an optional glossary.
%\glstocfalse   % Remove the glossaries from the table of contents.

% Set persons with 4 arguments:
%  {title before name}{name}{title after name}{gender}
%  where both titles are optional (i.e. can be given as empty brackets {}).
\setauthor{}{\authorname}{Bsc.}{male}
\setadvisor{Univ.Prof. Mag.rer.nat. Dr.techn.}{Reinhard Pichler}{}{male}

% For bachelor and master theses:
%\setfirstassistant{Professor Dr.}{Reinhard Pichler}{}{male}
\setfirstassistant{Univ.Ass. Dipl.-Ing. Dr.techn.}{Sebastian Skritek}{}{male}
\setsecondassistant{Univ.Ass. Dipl.-Ing.}{Markus Kröll}{}{male}

% For dissertations:
\setfirstreviewer{Pretitle}{Forename Surname}{Posttitle}{male}
\setsecondreviewer{Pretitle}{Forename Surname}{Posttitle}{male}

% For dissertations at the PhD School:
\setsecondadvisor{Pretitle}{Forename Surname}{Posttitle}{male}

% Required data.
\setaddress{Wasnergasse 13}
\setregnumber{1026213}
\setdate{27}{4}{2016}
\settitle{\thesistitle}{Complexity of SPARQL} % Sets English and German version of the title (both can be English or German).
%\setsubtitle{Optional Subtitle of the Thesis}{Optionaler Untertitel der Arbeit} % Sets English and German version of the subtitle (both can be English or German).

% Select the thesis type: bachelor / master / doctor / phd-school.
% Bachelor:
\setthesis{master}
%
% Master:
%\setthesis{master}
\setmasterdegree{dipl.} % dipl. / rer.nat. / rer.soc.oec. / master
%
% Doctor:
%\setthesis{doctor}
%\setdoctordegree{rer.soc.oec.}% rer.nat. / techn. / rer.soc.oec.
%
% Doctor at the PhD School
%\setthesis{phd-school} % Deactivate non-English title pages (see below)

% For bachelor and master:
\setcurriculum{Computational Intelligence}{Computational Intelligence} % Sets the English and German name of the curriculum.

% For dissertations at the PhD School:
\setfirstreviewerdata{Affiliation, Country}
\setsecondreviewerdata{Affiliation, Country}

% Define convenience macros.
\newcommand{\todo}[1]{{\color{red}\textbf{TODO: {#1}}}} % Comment for the final version, to raise errors.
\newcommand{\U}{\mathbf{U}}
\renewcommand{\u}{\mathcal{U}}
\renewcommand{\l}{\mathcal{L}}
\newcommand{\s}{\mathcal{S}}
\newcommand{\V}{\mathbf{V}}
\newcommand{\OPT}{\mbox{ OPT } }
\newcommand{\UNION}{\mbox{ UNION } }
\newcommand{\GRAPH}{\mbox{ GRAPH } }
\newcommand{\SERVICE}{\mbox{ SERVICE } }
\newcommand{\AND}{\mbox{ AND }}
\newcommand{\FILTER}{\mbox{ FILTER }}
\newcommand{\SELECT}{\mbox{ SELECT }}
\newcommand{\WHERE}{\mbox{ WHERE }}
\renewcommand{\ll}{\llbracket}
\newcommand{\rr}{\rrbracket}
\newcommand{\T}{\mathcal{T}}
\newcommand{\Qo}{\llbracket Q_1 \rrbracket^{DS}_G}
\newcommand{\Qt}{\llbracket Q_2 \rrbracket^{DS}_G}
\renewcommand{\c} {\overrightarrow{c}}
\newtheorem{example}{Example}
\newtheorem{corollary}{Corollary}
\newtheorem{definition}{Definition}
\newtheorem{proposition}{Proposition}
\newtheorem{theorem}{Theorem}
\newenvironment{proofidea}{\noindent\textit{Proof Idea.}}{\hfill$\square$}
\newtheorem{lemma}{Lemma}

\begin{document}
\frontmatter % Switches to roman numbering.
% The structure of the thesis has to conform to
%  http://www.informatik.tuwien.ac.at/dekanat

\addtitlepage{naustrian} % German title page (not for dissertations at the PhD School).
\addtitlepage{english} % English title page.
\addstatementpage

\begin{danksagung*}
\todo{Ihr Text hier.}
\end{danksagung*}

\begin{acknowledgements*}
\todo{Enter your text here.}
\end{acknowledgements*}

\begin{kurzfassung}
\todo{Ihr Text hier.}
\end{kurzfassung}

\begin{abstract}
	abstract hier
\end{abstract}

% Select the language of the thesis, e.g., english or naustrian.
\selectlanguage{english}

% Add a table of contents (toc).
\tableofcontents % Starred version, i.e., \tableofcontents*, removes the self-entry.

% Switch to arabic numbering and start the enumeration of chapters in the table of content.
\mainmatter

\chapter{Introduction}

\chapter{Preliminaries}
Before providing any content it is important that several definitions that
establish a firm basis are introduced. We will begin with conjunctive
queries then proceed with RDF and SPARQL.

\section{Definitions}

We will start out with the definition of a 
relational schema, database and conjunctive query. 

A relational name describes the name and arity of a relation.
\begin{definition}[Relational Schema]
	A relational schema is a nonempty finite set of relational names.
\end{definition}
Following up we have the definition of a database and relational atom over a
schema..

\begin{definition}[Database and relational atom]
	Let $\sigma$ be a relational schema, 
	$\U$ be an infinite set of constants and $\V$ an infinite set of variables.
	A relational atom over $\sigma$ is an expression of the form $R(v)$ where $R$
	is a relational name in $\sigma$ and $v$ is an n-tuple over $\U \cup \V$

	A database $D$ over $\sigma$ is a set of relational atoms over $\sigma$ and
	each n-tuple is in $\U$.
\end{definition}

After defining the meaning of a database and a schema we are ready to define
conjunctive queries.

\begin{definition}[Conjunctive Queries]
	Let $\sigma$ be a relational schema.
	A conjunctive query (CQ) $q$ over $\sigma$ is a rule of the form:
	\begin{align*}
		X \leftarrow R_1(\vec{v_1}), \dots, R_m(\vec{v_m}).
	\end{align*}
	Each $R_i(\vec{v_i}) (1\leq i \leq m)$ is a relational atom in $\sigma$ and 
	$X$ is a subset of the set of variables that appear in the
	$\vec{v_i}$'s.
\end{definition}

It remains to define the semantics of a conjunctive query, i.e., how to evaluate
a query over a given database and schema.
\begin{definition}[Semantics of Conjunctive Queries]
	In order to evaluate a CQ given a database $D$ over a relational schema $\sigma$
	we need to define the semantics which is given in terms of homomorphisms.
	Let $D$ be a database over $\sigma$.  A homomorphism from a CQ $Q$ to $D$ is a
	partial mapping $h: X \rightarrow U$ such that $R_i(h(\vec{v_i})) \in D,$ for 
	$1 \leq i \leq m$. $h_{|X}$ is the restriction of $h$ to the variables in
	$X$. The evaluation $Q(D)$ is the set of all mappings of the form
	$h_{|X}$, such that $h$ is a homomorphism from $q$ to $D$.
\end{definition}

We will now define a simplified but absolutely (for our purposes) sufficient version of the
original formalisation of the RDF W3C-recommendation~\cite{rdf}:
URI stands for Uniform Resource Identifier and is a string which is used to identify
abstract and physical resources in the web.
We will focus on ground RDF graphs which means that we assume them to be composed of URIs only.

\begin{definition}[RDF]
	Let $\U$ is the infinite set of URIs. An RDF triple is a tuple in $\U \times \U \times \U$, whereas an
	RDF graph is a finite set of RDF triples. $dom(G) \subseteq \U$ of an RDF graph $G$ is the set of URIs actually appearing in $G$.
\end{definition}

RDF is the database and SPARQL will be our query language. To use SPARQL we need
to define the SPARQL syntax and semantics.

\begin{definition}[SPARQL SYNTAX~\cite{pichler2014containment,
	BuilAranda20131,perez2006semantics}]
	Again $\U$ is an infinite set of URIs. 
	$\V$ is an infinite set of variables with $\U \cap \V = \emptyset$. 
	We will denote Variables in $\V$ with the letters $x,y,z,x',y',z',\dots$ and
	symbols which are in $\V \cup \U$ with $u,v,w,u',v',w',\dots$ and
	symbols which are in $\U$ with $a,b,c,d,e,f,g,a',b',c',d',e',f',g',\dots$.
	
	Filter constraints are conditions of the following form:
	\begin{enumerate}
		\item $\top, x=a, x=y$, or $bound(x)$ (atomic constraints),
		\item If $R_1, R_2$ are filter constraints, then $\neg R_1$, $R_1 \land R_2$,
			or $R_1 \lor R_2$ are filter constraints.
	\end{enumerate}

	A SPARQL triple pattern is a tuple in  $(\U \cup \V) \times (\U \cup \V) \times (\U \cup \V)$. 
	SPARQL graph patterns are recursively defined as follows:
	\begin{enumerate}
		\item A triple pattern is a graph pattern.
		\item If $P_1$ and $P_2$ are graph patterns, then $(P_1  \ \circ \ P_2)$ for
			$\circ \in \{ AND, OPT, UNION\}$ is a graph pattern.
		\item If $P$ is a graph pattern and $R$ a filter constraint, then $(P \ FILTER \ R)$ is a graph pattern.
		\item If $P$ is a graph pattern and $u \in (I \cup V)$, then $(\mbox{GRAPH} \  u \ P)$ is a graph pattern.
		\item If $P$ is a graph pattern and $u \in (I \cup V)$, then $(\mbox{SERVICE} \  u \ P)$ is a graph pattern.
	\end{enumerate}
If $P$ is a graph pattern, $vars(P)$ denotes the set of variables occuring in $P$.
\end{definition}

We proceed in defining the semantics of SPARQL. 

\begin{definition}[SPARQL Semantics~\cite{pichler2014containment,
	BuilAranda20131,perez2006semantics}]\label{def:sparqlsem}
	A mapping is a function $\mu: A \rightarrow  \U$ for some $A \subset \V$. 
	For a triple pattern $t$ with $vars(t) \subseteq dom(\mu)$, we write $\mu(t)$ to 
	denote the triple after replacing the variables in $t$ by the corresponding 
	URIs according to $\mu$. 

	\noindent Two mappings $\mu_1$ and $\mu_2$ are called compatible (denoted $\mu_1 \sim \mu_2$) 
	if $\mu_1(x) = \mu_2(x)$ for all $x \in dom(\mu_1) \cap dom(\mu_2)$.

	\noindent A mapping $\mu_1$ is subsumed by $\mu_2$ (written $\mu_1 \sqsubseteq \mu_2$) 
	if $\mu_1 \sim \mu_2$ and $dom(\mu_1) \subseteq dom(\mu_2)$. $\mu_2$ is then called ``extension'' of $\mu_1$.

	A dataset is a set $DS = \{(def, G), (g_1,G_1), \dots, (g_k, G_k) \}$, with
	$k\geq 0$. It contains pairs of URIs and graphs,
	where the default graph $G$ is identified by the special symbol $def \notin \U$
	and the remaining so-called ``named'' graphs $(G_i)$ are identified by URIs
	$(g_i \in \U)$. We assume that any query is evaluated over a fixed dataset $DS$
	and that any SPARQL endpoint that is identified by an URI $c \in \U$ evaluates
	its queries against its own fixed dataset 
	$DS_c = \{ (def, G_c),(g_{c,1},G_{c,1}), \dots, (g_{c,k},G_{c,k})\}$.
	We assume a function $graph(g,DS)$ which, given a Dataset $DS$ and a graph name $g$ as
	input returns the graph corresponding to the symbol $g$. The function
	$names(DS)$ returns the set of names of the input dataset $DS$.
	$names(DS) = \{g_1,g_2,\dots,g_k\}$.
	We also assume a partial function $ep: \U \rightarrow DS$, such that for every $c \in
	\U$, if $ep(c)$ is defined, then $ep(c) = DS_c$, i.e., the dataset associated with
	the endpoint accessible via the URI c.
	$\mu_\emptyset$ denotes the mapping with empty domain. This means it is
	compatible with any other mapping.
	The evaluation of graph patterns over an RDF Graph $G$, where $(g_1,G) \in
	DS$ for $g_1 \in \U\cup\{def\}$ is formalized as a
	function  $\llbracket \cdot \rrbracket_G^{DS}$, which, given a SPARQL graph pattern
	returns a set of mappings.
	For a graph pattern $P$, it is defined recursively:\\
	\scalebox{0.64}{
		\vbox{
			\begin{enumerate}
				\item $\ll t \rr_G^{DS} = \{ \mu \ | \ dom(\mu) = vars(t) \mbox{ and } \mu(t)
					\in G \}$ for a triple pattern $t$.
				\item $\ll P_1 \ \mbox{AND} \ P_2 \rr_G^{DS} = \{ \mu_1 \cup \mu_2  \ | \ \mu_1 \in \ll P_1
					\rr_G^{DS},  \ \mu_2 \in \ll P_2 \rr_G^{DS} \mbox{ and } \mu_1 \sim \mu_2 \}$.
				\item $\ll P_1 \ \mbox{OPT} \ P_2 \rr_G^{DS} = \ll P_1 \ \mbox{AND} \ P_2 \rr_G^{DS}
					\cup \{ \mu_1 \in \ll P_1
					\rr_G^{DS} \ | \ \forall \mu_2 \in \ll P_2 \rr_G^{DS}: \mu_1 \not\sim \mu_2\}$.
				\item $\ll P_1 \ \mbox{UNION} \ P_2 \rr_G^{DS} = \ll P_1 \rr_G^{DS} \cup \ll P_2
					\rr_G^{DS}$.
				\item  $\ll \mbox{GRAPH } u \ P_1 \rr_G^{DS}  = 
					\begin{cases} 
						\ll P_1\rr_{graph(u,DS)}^{DS} &\mbox{if } u \in names(DS)\\	
						\{ \}						  & \mbox{if } u \in \U \backslash names(DS)\\
						\bigg\{\mu \cup [u \rightarrow s] \mid s \in names(DS), \mu
							\in \ll P_1 \rr^{DS}_{graph(s,DS)} \land [u \rightarrow s] \sim \mu
						\bigg\}						  & \mbox{if } u \in \V
						\end{cases}
						$\\ 
						for $u \in \U \cup \V$.

					\item $\ll \mbox{SERVICE } u \ P_1 \rr_G^{DS}  = 
						\begin{cases} 
							\ll P_1\rr_{graph(def,ep(u))}^{ep(u)} &\mbox{if } u \in dom(ep)\\	
							\{ \mu_\emptyset\} & \mbox{if } u \in \U \backslash dom(ep)\\
							\bigg\{\mu \cup [u \rightarrow s ] \mid s \in dom(ep), \mu
								\in \ll P_1 \rr^{ep(s)}_{graph(def,ep(s))} \land [u \rightarrow s ] \sim \mu
							\bigg\}			   & \mbox{if } u \in \V
							\end{cases}$\\ 
							for $u \in \U \cup \V$.
				\item $\ll (P' \FILTER R) \rr_G = \{ \mu \mid \mu \in \ll P'
					\rr_G \mbox{ and } \mu \models R \}$,
					\end{enumerate}
				}
			}\\
		where $\mu$ satisfies a filter constraint $R$, denoted $\mu \models R$,
		if one of the following holds:
		\begin{enumerate}
			\item $R$ is $\top$;
			\item $R$ is $x = a$, $x \in dom(\mu)$ and $\mu(x) = a$;
			\item $R$ is $x = y$, $\{x,y\} \subseteq dom(\mu)$ and $\mu(x) =
				\mu(y)$;
			\item $R$ is $bound(x)$ and $x \in dom(\mu)$;
			\item $R$ is a boolean combination of filter constraints evaluating
				to true under the usual interpretation of $\neg,\land$ and
				$\lor$.
		\end{enumerate}
		\end{definition}

		When we don't use the SERVICE or GRAPH operators and assume a graph $G$
		we will implicitly assume a dataset $DS = \{(def,G)\}$ and
		we will denote the evaluation function with $\ll \cdot \rr_G$ instead
		of $\ll \cdot \rr^{DS}_G$.

		We often use the term ``destination'' of a subquery containing the SERVICE or
		GRAPH operator in the top level.
		The destination refers to the URI or variable inbetween the graph
		pattern and the GRAPH or SERVICE operator.

		\begin{definition}[Destination of a SERVICE- or GRAPH-operator]
			Given a pattern $P$ of the form $P = (\mbox{SERVICE } u \ P_1)$ or
			$P	= (\mbox{GRAPH } u \ P_1)$ we call $u$ the
			destination of the pattern $P$.
		\end{definition}

		We will also define the fragment $P_{wdsg}$ which is crucial for proving the
		complexity results. $P_{wdsg}$ is a fragment which allows the arbitrary use of
		the GRAPH and SERVICE operator in any part of the query.
		\begin{definition}[$P_{wdsg}$]
			A pattern $Q \in P_{wdsg}$ if it adheres to the following grammar:
			\begin{align*}
				Q::= &\quad Y \mid (Y \OPT R)  \mid B \\
				Y::= &\quad (Y \AND Y) \mid (\mbox{SERVICE } u \ Y) \mid (\mbox{GRAPH } u \
				Y) \mid  B\\
				R::= &\quad(R \OPT R) \mid (\mbox{GRAPH } u \ R) \mid (\mbox{SERVICE } u \ R) \mid B  \\
				B::= &\quad(u,v,w)
			\end{align*}
			where	$u,v,w \in \U \cup \V$. In the first layer we seperate the pattern
			into an AND-part and an OPT-part. In the AND- and OPT-part we can use
			SERVICE and GRAPH freely.
		\end{definition}

		\section{Thoughts on the Semantics of the SERVICE and GRAPH Operator}
		We were introduced to the GRAPH and
		SERVICE operators by \cite{BuilAranda20131} where Buil-Aranda et al.
		describe the syntax and semantics of the SERVICE operator and how an efficient query
		evaluation system can be implemented. 
		Conversely to the semantic definition of the GRAPH operator in~\ref{def:sparqlsem}, they provided the
		following definition for GRAPH (note the difference in the second case):	

		\scalebox{0.68}{
			\vbox{
				\begin{enumerate}
					\item  $\ll \mbox{GRAPH } u \ P_1 \rr_G^{DS}  = \begin{cases} 
							\ll P_1\rr_{graph(i,DS)}^{DS} &\mbox{if } u \in names(DS)\\	
							\{ \mu_{\emptyset}\} & \mbox{if } u \in \U \backslash names(DS)\\
							\bigg\{\mu \cup [u \rightarrow s ] \mid \exists s \in names(DS), \mu
								\in \ll P_1 \rr^{DS}_{graph(s,DS)} \land [ u \rightarrow s ] \sim \mu
							\bigg\} & \mbox{if } u \in \V
							\end{cases}$\\ for $u \in \U \cup \V$.
					\end{enumerate}
				}
			}

			We tried to propose an algorithm for evaluating CONTAINMENT[$S_1$,$S_2$] in the
			fragment $P_{wdgs}$ and we saw some unintuitive features regarding the semantics of the
			GRAPH operator. Consider the following example:
			\begin{example}
				$DS=\{(def,G_0), (a,G_1) \}$ where $G_0 = \{
				(c,k,g), (a,k,g) \}$, $G_1$ is arbitrary and  $Q = (GRAPH \ x  \
				(GRAPH  \ c \  P_1)) \ AND \ (x,k,g)$.
			\end{example}

			Evaluating the first part of the query, namely $\ll(GRAPH \ x  \ (GRAPH  \
			c \  P_1))\rr^{DS}_{G_0} $ where $P_1$ could be any arbitrary
			pattern yields an interesting result. Because $c \notin	names(DS)$ 
			and thus $\ll (GRAPH \ c  \ P_1) \rr^{DS}_{G_1} = \mu_\emptyset$  we
			obtain $x \rightarrow a$ ($\mu_\emptyset$ is compatible with every
			mapping). This seems unintuitive because it affects the rest of the query.
			If we would evaluate the right side of the $AND$ we would obtain the
			following: $\ll(x,k,g)\rr^{DS}_{G_0} = \{ (x \rightarrow a) , (x \rightarrow c) \}$ considering the
			left part of the $AND$-Operator we get the result of the query: $\{ (x \rightarrow a )
			\}$. Note that even though the graph with the URI $c$ does not exist and we are not able
			to evaluate pattern $P_1$ over it, we are able to somehow receive results using
			the $AND$ operator. This syntax might be fitting for the SERVICE
			operator where we query SPARQL endpoints in the web which might be
			currently unavailable, but for the GRAPH operator where we query
			different graphs within a reachable endpoint
			this semantics is not needed.
			\bigskip

			\noindent Considering containment/equivalence of two queries which just use the GRAPH
			Operator, there are specific instances where it is not obvious if one query is contained in the other query even though it should be intuitively.

			\begin{example}
				$Q_1$ = (GRAPH x ( GRAPH c ( GRAPH y) $P_1$)),\\ 
				$Q_2$ = (GRAPH y ( GRAPH c ( GRAPH x) $ P_1$))\\
				$?X,?Y \notin P_1$.
			\end{example}

			Approaching this example intuitively, one could say that $\ll Q_1
			\rr^{DS}_{G} = \ll Q_2 \rr^{DS}_G$ for all datasets $DS$ and graphs
			$G$ in $DS$ because $x$ and $y$ get bound to every variable in the dataset because neither
			$x$ nor $y$ occur in $P_1$, the triple pattern at the end
			of the query is the same and also the graph queried second is the same. 
			To show the opposite, just  consider the following dataset:
			$DS=\{(def,G), (a,G_1)\}$. Note that the URI $c$ does not occur in the dataset.
			But this means that $\ll Q_1 \rr^{DS}_G = \{x \rightarrow a\}$ and  $\ll Q_2
			\rr^{DS}_G = \{y \rightarrow a\}$.			
			In \cite{BuilAranda20131} the authors
			state that the GRAPH-Operator was introduced formally
			in \cite{perez2009semantics} but we were not able to find the definition
			in it.
			Instead we tried to look into the W3C-recommendation\cite{w3standard} 
			where we found the following definition:

			\begin{algorithm}
				\caption{W3C-recommendation on evaluating the GRAPH operator}
			%\begin{lstlisting}
			if IRI is a graph name in D\\
				\quad eval(D(G), Graph(IRI,P)) = eval(D(D[IRI]), P)\\
			if IRI is not a graph name in D\\
				\quad eval(D(G), Graph(IRI,P)) = the empty multiset\\
				\quad eval(D(G), Graph(var,P)) =\\
			Let R be the empty multiset\\
				\quad foreach IRI i in D\\
					\qquad R := Union(R, Join( eval(D(D[i]), P) , $\Omega$(?var->i) )\\
			the result is R
			%\end{lstlisting}
		\end{algorithm}

			If we look at the second case, we can see that the empty multiset (because the
			definition goes with bag semantics)
			is returned if the URI is not a graph name in the Dataset resulting
			in our original definition of SPARQL semantics. 
			
			However remembering the definition of the SERVICE
			operator~\ref{def:sparqlsem}, we emphasize
			that the second case of the SERVICE-operator (if $c \in \U \backslash
			dom(ep)$ return $\mu_\emptyset$) is intended and correct. It makes
			sure that if a SPARQL endpoint is temporarily unavailable the query
			does not fail.
			
			%Finally I wish to make the remark that bit unintuitive at first that if we have
			%the third case (of both GRAPH and SERVIE), 
			%namely $c \in \V$, the query $P_1$ will not be considered over the
			%default graph. The reason for this can also be found in the definition of the
			%dataset in the standard \cite{w3standard}:
			%``A SPARQL query is executed against an RDF Dataset which represents a collection
			%of graphs. An RDF Dataset comprises one graph, the default graph, which does not
			%have a name, and zero or more named graphs, where each named graph is identified
			%by an IRI. A SPARQL query can match different parts of the query pattern against
			%different graphs as described in section 13.3 Querying the Dataset.''




\chapter{Well-Designed SPARQL}
A first we are going to introduce the fragment of well-designed SPARQL, then we
will analyze the results of~\cite{pichler2014containment}, in three seperate
chapters namely decidable containment, undecidable containment and equivalence.

\section{Introduction to well-designed SPARQL}
Well-designed SPARQL is a good fundamental basis for complexity analysis because
the evaluation problem is only coNP-complete in contrast to general SPARQL where
it is PSPACE-complete~\cite{perez2009semantics}.
%dbpeda query amount 

\begin{definition}[Well-designed SPARQL]
	A graph pattern $P$ built only from AND and OPT is well-designed if there does
	not exist a subpattern $P' = (P_1 \  \mbox{OPT} \ P_2)$ of $P$ and a variable $?X \in
	vars(P_2)$ that occurs in $P$ outside $P'$, but not in $P_1$. A graph pattern $P
	= P_1 \mbox{ UNION } \dots \mbox{ UNION } P_n$ is well-designed if each subpattern $P_i$ is UNION
	free and well-designed.
\end{definition}

In \cite{letelier2013static} it was shown that every well-designed graph pattern
can be transformed into OPT normal form in polynomial time. 
\begin{definition}[OPT normal form]
	A graph pattern containing only the operators AND and OPT is in OPT normal form
	if the OPT operator never occurs in the scope of an AND operator. 
\end{definition}

Graph Patterns in OPT normal form can be displayed in a natural tree representation, 
formalized by so-called well-designed pattern trees. But before defining the
well-designed pattern trees we need some basic notation about pattern trees.
\begin{definition}[Pattern Tree]\label{def:pt}
	A pattern tree (PT) $T$ is a pair $(T,P)$ where  $T= (V,E,r)$ is a rooted,
	unordered tree and $P = (P_n)_{n \in V}$ is a labelling of the nodes in $V$,
	s.t. $P_n$ is a non-empty set of triple patterns for every $n \in V$.  
\end{definition}

In Example~\ref{ex:pt} we can see how a pattern in OPT normal form is
transformed into a pattern tree.
\begin{example}[~\cite{pichler2014containment}]\label{ex:pt}
	Consider the graph pattern
	\begin{align*}
		P = \big(\big( (x,name,y) \AND (x,email,z) \big) \OPT (x,web,x') \big)
		\OPT \big((x,phone,x'') \OPT (x,fax,x''')\big)
	\end{align*} and the corresponding pattern tree:\\
	\begin{tikzpicture}[sibling distance=10em]
			\node {\{$(x,name,y),(x, email, z)$\}}
			child { node {$\{(x,web,x')\}$} }
			child { node {$\{(x,phone,x'')\}$}
		child { node {$\{(x,fax,x''')\}$}}};
		\end{tikzpicture}
\end{example}

We will now define further details of a pattern tree and after that
proceed with defining well-designed pattern trees.
\begin{definition}[Components of a pattern tree~\cite{pichler2014containment}]
Let $T = ((V,E,r), P)$ be a pattern tree.

We call the pattern tree $T' = ((V',E',r'), (P_n)_{n \in V'})$ a subtree of $T$
if $(V',E',r')$ is a subtree of $T$. 

An extension of $\hat{T}'$ of a subtree
$T'$ of $T$ is a subtree $\hat{T}'$ of $T$, s.t. $T'$ is in turn a subtree of
$\hat{T}'$. A subtree or extension is proper if some node of the bigger tree is
missing in the smaller tree.

Given $T$, we write $V(T)$ to denote the set $V$ of vertices.

We denote with $pat(T)$ the set $\bigcup_{n\in V(T)} P_n$ and write $vars(T)$ as an
abbreviation for $vars(pat(T))$.

Given a node $n \in V(T)$, we define $branch(n) = n^1, \dots, n^k$ with $n^1 = r$
and $n^k = n$ as the unique sequence of nodes from the root $r$ to $n$.

For nodes $n,\hat{n} \in V(T)$,s.t. $\hat{n}$ is the parent of $n$, let
$newvars(n) = vars(n) \backslash vars(branch(\hat{n}))$. 

A node $n$ is a child
of a PT $T$ if $n\notin V(T)$ and $n$ is the child of some node $n' \in V(T)$.
\end{definition}
After thoroughly describing special details of pattern trees, we are now ready
to look at the definition of well-designed pattern trees(wdPTs).
\begin{definition}[Well-designed pattern tree
	(wdPT)~\cite{pichler2014containment}]
	A well-designed pattern tree (wdPT) is a pattern tree $T = (T,P)$ where for
	every variable $x \in V(T)$, the nodes $\{n \in V(T) \mid x \in vars(n)\}$
	induce a connected subgraph of $T$.
\end{definition}
We can observe that the pattern tree in Example~\ref{ex:pt} is well-designed.
There are two important properties of wdPTs that need to be mentioned. 
\begin{proposition}
	\begin{enumerate}	
		\item Every wdPT $T$ can be transformed efficiently into NR normal form, if $newvars(n) \neq
			\emptyset$ for every $n \in V(T)$~\cite{letelier2012static}. From now on we
			assume w.l.o.g. that all wdPTs dealth with in this paper are in NR normal form.
		\item For every variable $x \in vars (T)$, there is a unique node $n \in
			V(T)$, s.t. $x \in newvars(n)$ and all other nodes $n' \in V(T)$
			with $x \in vars(n')$ are descendants of
			$n$.~\cite{pichler2014containment}
	\end{enumerate}
\end{proposition}


We denote the results of evaluating a wdPT $T$ over some RDF graph by $\ll T
\rr_G$. The set $\ll T \rr_G$ of solutions can be defined via a translation to
graph patterns. We will do something similar in our alternate Definition of
wdPT\ref{def:wdpt}. Lemma~\ref{lemma:altsemwdpt} allows us to use a direct
characterization in terms of maximal subtrees of a wdPT $T$ assuming it is in NR
normal form.
\begin{lemma}[Semantics of pattern trees~\cite{letelier2012static}]\label{lemma:altsemwdpt}
	Let $T$ be a wdPT in NR normal form and $G$ an RDF graph. then $\mu \in
	\ll T\rr_G$ iff there exists a subtree $T'$ of $T$, s.t. 
	\begin{enumerate}
		\item  $dom(\mu) = vars(T')$, and
		\item $T'$ is the maximal subtree of $T$, s.t. $\mu \sqsubset \ll
			pat(T') \rr_G$.
	\end{enumerate}
\end{lemma}

Projection is interpretet as a top level operator on top of a graph pattern.
\begin{definition}[Projection of a mapping]
For a mapping $\mu$ and a set $X$ of variables we denote $\mu_{|X}$ as the
projectino of $\mu$ to the variables in $X$, call it $\mu'$. $dom(\mu'):=X \cap
dom(\mu)$ and $\mu'(x) := \mu(x)$ for all $x \in dom(\mu')$.
\end{definition}

We are now ready to define projected well-designed trees:
A projected well-designed tree (pwdPT) is a pair $(T,X)$ where $T$ is a wdPT and
$X$ a set of variables.
\begin{definition}[Result of pwdPTs \cite{pichler2014containment}]
Let $G$ be a RDF graph. 
Then $\ll (T,X)\rr_G = \{ \mu_{|X} \mid \mu \in \ll T \rr_G\}$.
\end{definition}

In a projected wdPT there are free variables and existential variables.
\begin{definition}[Free and existentail variables
	~\cite{pichler2014containment}]
	Let $(T,X)$ be a pwdPT. Then the free variables of $(T,X)$ are defined as
	$fvars(T)= vars(T) \cap X$ and
	the existential variables of $(T,X)$ are defined as $vars(T) \backslash
	fvars(T)$.
	Analogously, we write $fvars(n)$ and $evars(n)$, respecitvely, for nodes
	$n\in V(T)$.
\end{definition}

In $\cite{letelier2013static}$ it was shown that a 
similar characterization of solution as in Lemma~\ref{lemma:altsemwdpt} exists.
A pwdPT $(T,X)$ is in NR normal form if $T$ is. W.l.o.g., we assume that
existential variables when we look at more than one pwdPT are always renamed apart.

\begin{lemma}[\cite{letelier2013static}]
	Let $(T,X)$ be a pwdPT in $NR$ normal form, $G$ an RDF graph and $\mu$ a
	mapping with $dom(\mu) \subseteq X$. Then $\mu \in \ll ( T,X)\rr_G$ iff
	there exists a subtree $T'$ of $T$, s.t. 
	\begin{enumerate}
		\item $dom(\mu) = fvars(T')$ and
		\item there exists a mapping $\lambda: evars(T') \mapsto dom(G)$, s.t.
			$\mu \cup \lambda \in \ll T \rr_G$.

	\end{enumerate}
\end{lemma}

The last outgrowth of pwdPTs we will consider are unions of well-designed
pattern trees. Unions of well-design patterns correspond to the usage of the
UNION operator in a graph pattern consisting over several well-designed graph
patterns connected by UNION, i.e. $P = P_1 \UNION \dots \UNION P_k$. They can be
considered as a set $\{T_1, \dots, T_k\}$ of wdPTs. Such a set is called
well-designed pattern forest (wdPF) $F$. We can analogously define projected
well-designed pattern forests (pwdPFs) as a tuple $(F',X)$ where $F'=\{(T_1,X),\dots,(T_k,X)\}$.

\begin{definition}[Result of wdPFs and pwdPF \cite{pichler2014containment}]
Let $G$ be a RDF graph. 
Then $\ll F \rr_G = \bigcup_{T\in F}\ll T \rr_G$ and $\ll(F',X)\rr_G =
\bigcup_{(T,X) \in F'} \ll (T,X) \rr_G$. 
\end{definition}


A subforest of a forest $F$ is a set of subtrees of $F$.
We can extend $pat(\cdot)$ to wdPFs: Let $F$ be a wdPF then $pat(F) =
\cup_{T \in F}  pat(T)$ and analogously $vars(F)$.

\bigskip \noindent
To capture the UNION operator the definition of well designed pattern trees
need to be modified.  

Pichler and Skritek studied query containment and query equivalence of well
designed SPARQL in~\cite{pichler2014containment}. As we want to analyse the
complexity of the SERVICE-operator regarding containment and equivalence it 
is crucial to understand the work that was previously done regarding other
SPARQL operators.
Research in query containment and query equivalence is very important in the
context of static query analysis and optimization: Optimization of queries can
be done by replacing a query with a new query which posesses better computational properties. 
Replacing queries
preserving the meaning of the original meaning can only be done by checking if
the replaced query is equivalent to the original query. Also, later on in
Chapter~\ref{chapter:serviceeval}, when we analyse
boundedness of the destination variable of the service operator, we will see
that equivalence plays a role whether a query is feasible to be
evaluated. 
Pichler and Skritek did a fine grained analysis of well-designed SPARQL by
dividing it into several fragments. The fragments are distinguished by
the operators used. All fragments contain triple patterns and the AND-operator. This
fragment corresponds to conjunctive queries which was proven in~\cite{letelier2013static} 
and is denoted with $\{\emptyset\}$. Adding the UNION-operator yields the fragment 
denoted by $\{\cup\}$. Adding projection yields the fragment denoted by $\{\pi\}$ 
and adding both the UNION-operator and projection yields the fragment denoted by $\{\pi,\cup\}$.

In order to define the problems \textbf{CONTAINMENT}, \textbf{EQUIVALENCE} and
\textbf{SUBSUMPTION} which we will investigate on the fragments that we
mentioned above, we need to define what containment, equivalence and subsumption mean
in the SPARQL language.
\begin{definition}[Containment]
\medskip\noindent $P_1$ is contained in $P_2$ $(P_1 \subseteq P_2)$ 
if $\ll P_1 \rr_G \subseteq
\ll P_2 \rr_G$ for every RDF Graph $G$.
\end{definition}
If a graph pattern $P_1$ is contained by a graph pattern $P_2$, all
solutions of $P_1$ are solutions of $P_2$ regardless of the graph
they are evaluated over.

\begin{definition}[Equivalence]
Given two graph patterns $P_1$ and $P_2$ they are equivalent (written as $P_1
\equiv P_2$) if ($\ll P_1 \rr_{G}  = \ll P_2 \rr_{G} $ for every RDF Graph $G$).
\end{definition}
 Equivalence between two graph patterns $P_1,P_2$ means that regardless of the graph
 they get evaluated over, they always have the same solutions.

\begin{definition}[Subsumption]
\medskip\noindent $P_1$ is subsumed by $P_2$ (denoted $P_1 \sqsubseteq P_2$ if for every $\mu \in \ll P_1 \rr_G$ there exists a $\mu' \in
\ll P_2 \rr_G$, s.t. $\mu \sqsubseteq \mu'$. 
\end{definition}
If a graph pattern $P_1$ is subsumed by the graph pattern $P_2$ all the
solutions of $P_1$ can be extended to a solution of $P_2$. This means that if we
add additional mappings to a solution of $P_1$ we can ``build'' a solution of
$P_2$. 


\noindent After reading the definitions of containment, equivalence and
subsumption the following complexity theoretical problems arise:\\
\begin{framed}\noindent CONTAINMENT[$S_1,S_2$]\\
	INPUT: Graph pattern $P_1$ from wd-SPARQL$[S_1]$,
		 graph pattern $P_2$ from wd-SPARQL$[S_2]$\\
	QUESTION: Does $P_1 \subseteq P_2$ hold?
\end{framed}
\begin{framed}\noindent EQUIVALENCE[$S_1,S_2$]\\
	INPUT: Graph pattern $P_1$ from wd-SPARQL$[S_1]$,
		 graph pattern $P_2$ from wd-SPARQL$[S_2]$\\
	QUESTION: Does $P_1 \equiv P_2$ hold?
\end{framed}
\begin{framed}\noindent SUBSUMPTION[$S_1,S_2$]\\
	INPUT: Graph pattern $P_1$ from wd-SPARQL$[S_1]$,
		 graph pattern $P_2$ from wd-SPARQL$[S_2]$\\
	QUESTION: Does $P_1 \sqsubseteq P_2$ hold?
\end{framed}
$S_1$ and $S_2$ always denote a fragment of well-designed SPARQL for example
$\{\cup,\pi\}$. The main results of~\cite{pichler2014containment} are the
solutions to CONTAINMENT[$S_1,S_2$] and EQUIVALENCE[$S_1,S_2$] where $S_1,S_2$
are any of the well designed sparql SPARQL fragments mentioned before.
The results are best represented by the two
tables~\ref{conttable} and~\ref{equivtable}.

\begin{table}
	\begin{minipage}[b]{0.5\hsize}
		\begin{tabular}{|l | l | l | l | l|}
			\hline
			$\downarrow S_1 \backslash S_2 \rightarrow$ & $\{\emptyset\}$  &
			$\{\cup\}$& $\{\pi \}$ & $\{\cup,\pi \}$ \\
			\hline
			$\{\emptyset\}$           & NP-c.	& $\Pi^P_2-c.$  & undec. & undec. \\
			$\{\pi \}$       & NP-c.	& $\Pi^P_2-c.$  & undec. & undec. \\
			$\{\cup \}$      & NP-c.	& $\Pi^P_2-c.$  & undec. & undec. \\
			$\{\cup,\pi \}$ & NP-c.	& $\Pi^P_2-c.$  & undec. & undec. \\
			\hline
		\end{tabular}
		\caption{Containment[$S_1,S_2$]~\cite{pichler2014containment}}
		\label{conttable}
	\end{minipage}
\end{table}

\bigskip
\begin{table}
	\begin{minipage}[b]{0.5\hsize}
\begin{tabular}{|l |  l | l | l | l|}
	\hline
	$\downarrow S_1 \backslash S_2 \rightarrow$ & $\{\emptyset\}$ &
	$\{\cup\}$& $\{\pi \}$ & $\{\cup,\pi \}$ \\
	\hline
	$\{\emptyset\}$			&  NP-c.				& -				 & -         & - \\
	$\{\pi \}$		&  $\Pi^P_2-c.$ 		& $\Pi^P_2-c.$   & -		 & - \\
	$\{\cup \}$		&  $\Pi^P_2-c.$ 		& $\Pi^P_2-h.$   & undec.    & - \\
	$\{\cup,\pi \}$&  $\Pi^P_2-c.$ 		& undec.         & undec.    & undec. \\
\hline
\end{tabular}
\captionof{table}{Equivalence[$S_1,S_2$]~\cite{pichler2014containment}}
\label{equivtable}
	\end{minipage}
\end{table}


\section{Decidable Containment}\label{section:decidablecontainment}

First the decidable cases of Table~\ref{conttable} are proven, i.e., $S_1$ is an arbitrary subsetset of
$\{\cup,\pi\}$ and $S_2$ doesn't contain $\pi$. 
In order to minimize the number of proofs a well known strategy is applied: When membership of a
general case is shown, the more specific case is implied and when showing
hardness for a more specific case, the more general cases are implied.  
Showing the following results fill up the first two columns of the
Containment[$S_1,S_2$] table:
\begin{itemize}
	\item NP-membership of \textbf{CONTAINMENT[$\{\cup,\pi\},\emptyset$]}
	\item $\Pi^P_2$-membership of \textbf{CONTAINMENT[$\{\cup,\pi\},\{\cup\}$]}
	\item $\Pi^P_2$-hardness of \textbf{CONTAINMENT[$\emptyset,\{\cup\}$]}
\end{itemize}
NP-hardness of \textbf{CONTAINMENT[$\emptyset,\emptyset$]} follows immediately
from the NP-hardness of \textbf{EQUIVALENCE[$\emptyset,\emptyset$]} which was shown
in~\cite{letelier2013static}.

For the NP-membership of \textbf{CONTAINMENT[$\{\cup,\pi \}, \emptyset$]}, 
first \textbf{CONTAINMENT[$\{\pi\} , \emptyset$]} is shown and extended to 
\textbf{CONTAINMENT[$\{\cup,\pi \}, \emptyset$]}. We notice that we deal with
both pwdPTs and wdPTs as $S_1$ contains projection. Theorem~\ref{projwd} provides a
necessary and sufficient criterion to decide $(T_1,X) \subseteq T_2$.

\begin{theorem}\label{projwd}\cite{pichler2014containment}
	Let $(T_1,X)$ be a pwdPT and let $T_2$ be a wdPT. Then $(T_1,X) \subseteq
	T_2$ iff for every subtree $T_1'$ of $T_1$,
	\begin{enumerate}
		\item either there exists a child node $n$ of $T_1'$ and a homomorphism
			$h:pat(n) \rightarrow pat(T_1')$ with $h(x) = x$ for all $x \in
			vars(n) \cap vars(T_1')$,
		\item or there exists a subtree $T_2'$ of $T_2$, s.t.
			\begin{enumerate}
				\item $fvars(T_1') = vars(T_2')$
				\item $pat(T_2') \subseteq pat(T_1')$, and
				\item for all extensions $\hat{T_2'}$ of $T'_2$ there exists an
					extension $\hat{T_1'}$ of $T_1'$ and a homomorphism 
					$h: pat(\hat{T_1'}) \mapsto pat(T_1') \cup pat(\hat{T_2'})$
					with $h(x) = x$ for all $x \in vars (T_1')$.
			\end{enumerate}
	\end{enumerate}
\end{theorem}
\begin{proofidea}
Let $G$ be an arbitrary RDF graph. Let $T_1'$ be an arbitrary subtree
of $T_1$. Let $\sigma$ be a mapping s.t. $dom(\sigma) = vars(T_1')$ and $\sigma
\in \ll T_1 \rr_G$. We need to show that $\sigma_{|X} \in \ll T_2 \rr_G$.

The first case of the theorem captures the case where we have a subtree which is
not maximal. For this subtree we could take any $\sigma
\in \ll T_1' \rr_G$ and extend it with the variables in $n$ because  
$h:pat(n) \rightarrow pat(T_1')$ with $h(x) = x$ for all $x \in
vars(n) \cap vars(T_1')$ holds. But such a mapping $\sigma$ can not be in $\ll
T_1 \rr_G$. Thus the subset relation is obviously fulfilled.

The second case captures solutions $\sigma$ with $dom(\sigma) = vars(T'_1)$.
It remains to check if $\mu = \sigma_{|X}$ is in $\ll T_2\rr_G$.
The first two conditions make sure that $\mu$ also maps $T'_2$ into $G$. To make
it a solution for $T_2$ it remains to show that there is no extension $\mu'$ of $\mu$
so that $\mu' \in \ll T_2\rr_G$. Property (2c) makes sure that if $\mu'$ exists,
we can find an extension $\sigma'$ such that it binds more variables than
$\sigma$. But then $\sigma$ is no solution.
\end{proofidea}

To get a better picture of the procedure consider
Example~\ref{ex:membershipcuppiempty}.
\begin{example}[\cite{pichler2014containment}]\label{ex:membershipcuppiempty}
	Assume a $pwdPT(T_1,X)$ with $X = \{x_1, x_2, x_3\}$ and wdPT $T_2$ as
	shown below.

	\bigskip\noindent
	\begin{tikzpicture}[sibling distance=15em,
			every node/.style = {shape=rectangle, rounded corners,
				draw, align=center,
	top color=white, bottom color=blue!20}]]
	\node {$r_1$: $\{x_1,b,x_1),(y_1,c,y_1),(y_1,d,y_1)\}$}
	child { node {$n_1:$ $\{(x_1,e,x_1), (x_2,f,x_2)\}$}}
	child { node {$n'_1:$ $\{(x_3,c,x_3), (x_3,d,x_3)\}$}};
\end{tikzpicture}	

	\bigskip\noindent
	\begin{tikzpicture}[sibling distance=15em,
			every node/.style = {shape=rectangle, rounded corners,
				draw, align=center,
	top color=white, bottom color=blue!20}]]
	\node {$r_2$: $\{x_1,b,x_1),(x_3,c,x_3),(x_3,d,x_3)\}$}
	child { node {$n_2:$ $\{(x_2,f,x_2)\}$}};
\end{tikzpicture}	

\bigskip\noindent
We will now use the criteria in Theorem~\ref{projwd} to check if $(T_1,X) \subseteq T_2$ holds.
Consider all the subtrees of $T_1$: $\{r_1\}, \{r_1,n_1\}, \{r_1,n'_1\}, \{r_1,n_1,n'_1\}$ and check if either of the two properties in Theorem~\ref{projwd} hold.
Consider the subtree $\{r_1, n'_1\}$ and property (2):
We can easily see that the properties (2a) and (2b) are satisfied by the subtree of $T_2$ that contains only 
$r_2$. It remains to check (2c) and this is where the problem occurs:
The required homomorphism from $pat(n_1)$ into $pat(r_1) \cup pat(n'_1) \cup
pat(r_2) \cup pat(n_2)$.
We can also provide the following counterexample:
$G = \{ (g_{x1},b,g_{x1}), (h_{y1},c,h_{y1}),(h_{y1},d,h_{y1}), (j_{x2},f,j_{x2})\}$. Clearly
$\mu \cup \lambda \in \ll T_1 \rr_g$ for $\mu (x_1) = g_{x1}, \mu (x_3) =
h_{y1}$ and $\lambda(y_1) = h_{y1}$ and thus $\mu \in \ll(T_1,X)\rr_G$. However
$\mu \not\in \ll T_2 \rr_G$ because $\mu'$ an extension of $\mu$ can be created
with $\mu'(x_2)=j_{x2}$ and $\mu'(pat(n_2)) \subseteq G$ and thus $(T_1, x)
\not\subseteq T_2$.
We could change $pat(n_2)$ to get $(T_1, x)
\subseteq T_2$: Let $pat(n_2) = \{(x_2,f,x_2,x_1,e,x_1)\}$ and observe that a
homomorphism $h$ for property (2c) of Theorem~\ref{projwd} exists if we define
$h$ as the identity. In fact, then $(T_1,X) \subseteq T_2$.
\end{example}

It is easy to see that Theorem~\ref{projwd} can be transformed into a
$\Pi^P_2$-algorithm: All subtrees $T_1'$ of $T_1$ are tested if there
exists a subtree $T_2'$ of $T_2$ which produces the same mappings as $T_1'$.
On the other hand it is far from obvious, 
that the problem is not $\Pi^P_2$ hard: One can in fact
get rid of one source of complexity and push the complexity of the
\textbf{CONTAINMENT[$\{\pi\},\emptyset$]} problem down to
NP-completeness.
The crucial idea is, that we don't need to look at all the subtrees $T_1'$ of
$T_1$ but polynomially many. The subtrees of interest can be described by
defining the closure of a variable.

It is easy to test if $vars(T_2) = fvars(T_1)$ by traversing the trees once, so
we assume w.l.o.g. that this property holds. Also, we know that $vars(T_2) =
fvars(T_1)$ must hold for $(T_1,X) \subseteq T_2$ to be true as this is an
immediate consequence from Theorem~\ref{projwd} (just consider $T_1' = T_1$).

\begin{definition}[Closure $(C_1(x),C_2(x))$ of a
	variable~\cite{pichler2014containment}]
	Let $(T_1,X)$ be a pwdPT and let $T_2$ be a wdPT with
	$vars(T_2) = fvars(T_1)$. Consider $x \in fvars(T_1)$. The closure
	of $x$ in $(T_1,X)$ and $T_2$ is the pair $(C_1(x),C_2(x))$ where
	$C_i(x)$ (for $i \in \{1,2\}$) is a subtree of $T_i$ such that the
	following conditions are met:
	\begin{enumerate}
		\item $branch(\mbox{new-node}_{T_1}(x)) \subseteq V(C_1(x))$,
		\item $r_2 \in V(C_2(x))$,
		\item $fvars(C_1(x)) = vars(C_2(x))$, and
		\item $C_1(x)$ and $C_2(x)$ are minimal with regard to properties 1-3.
	\end{enumerate}
	Minimality in (4) means that for all subtrees $D_1$ of $T_1$ and $D_2$ of
	$T_2$, if $D_1$ and $D_2$ satisfy conditions 1-3, then $V(C_1) \subseteq
	V(D_1)$ and $V(C_2) \subseteq V(D_2)$ meaning the nodes for the subtrees
	$V(C_1)$ and $V(C_2)$ are minimized.
\end{definition}

Because we assumed $vars(T_2) = fvars(T_1)$ we can easily see that the closure
always exists. 

\begin{proposition}[~\cite{pichler2014containment}]
	Let $(T_1,X)$ be a pwdPT and let $T_2$ be a wdPT with $fvars(T_1) =
	vars(T_2)$. Then the closure $(C_1(x),C_2(x))$ exists and can be
	efficiently computed.
\end{proposition}
\begin{proofidea}
	The algorithm chooses a not yet chosen variable $x \in fvars(T_1)$ and
	initializes the trees $C_1(x)$ and $C_2(x)$ by setting $C_1(x) =
	\mbox{branch}(\mbox{new-node}_{T_1})$ and $C_2(x) = V(C_2(x)) = \{r_2\}$ where $r_2$ is
	the root of $T_2$. It remains to fulfill the condition $fvars(C_1(x)) =
	vars(C_2(x))$ in such a way that $C_1(X)$ and $C_2(X)$ are minimal with
	regard to the number of vertices. We either have one of the following cases:
	\begin{enumerate}
		\item $fvars(C_1(x)) = vars(C_2(x))$: This means we are done and have
			successfully computed the closure $(C_1(x), C_2(x))$
		\item $fvars(C_1(x)) \supset vars(C_2(x))$: This means we miss the
			variables\\ $fvars(C_1(x)) \backslash vars(C_2(x))$. By
			iteratively adding $\mbox{branch}(\mbox{new-node}_{T_2}(y))$ to $C_2(x)$ for all
			variables\\ $y \in fvars(C_1(x)) \backslash vars(C_2(x))$ all
			the missing variables are now in $C_2(x)$.
		\item $fvars(C_1(x)) \subset vars(C_2(x))$: This means we miss the
			variables\\ $vars(C_2(x)) \backslash fvars(C_1(x))$. By iteratively
			adding $\mbox{branch}(\mbox{new-node}_{T_1}(y))$ to $C_1(x)$ for all variables
			$y \in vars(C_2(x)) \backslash fvars(C_1(x))$, all the missing
			variables are now in $C_1(x)$.
	\end{enumerate}
	By the assumed condition $fvars(T_1) = vars(T_2)$, the procedure will
	eventually reach a fixpoint~\cite{pichler2014containment}.
\end{proofidea}

Using the closure of the variable is inspired by the following idea:
Assume $\mu$ is a solution mapping of $(T_1,X)$ and $y$ is part of $\mu$, i.e.,
a free variable. Now we need to show that $\mu$ is also part of $T_2$, i.e.,
$\mu$ must bind all the variables that occur in $T_2$ on the branch from the
root to the node $n$ where $y$ is introduced. If again an additional variable
$z$ is introduced in this path in tree of $T_2$, $\mu$ must bind all the free
variables in $(T_1,X)$ along the branch from the root to the first occurrence of
$z$. Using the idea of the closure allows us to formulate an alternative characterization of
$(T_1,X) \subseteq T_2$. The improvement is that we only need to check
polynomially many closures and not exponentially many subtrees of $(T_1,X)$.

\begin{theorem}[\cite{pichler2014containment}]\label{projwd2}
	Let $(T_1,X)$ be a pwdPT and let $T_2$ be a wdPT. Then $(T_1,X) \subseteq
	T_2$ if and only if $fvars(T_1) = vars(T_2)$ and for every $x \in
	fvars(T_1)$
	\begin{enumerate}
		\item $pat(C_2(x)) \subseteq pat(C_1(x))$
		\item for every $n \in V(C_1(x)) \ \mbox{branch} (\mbox{new-node}_{T_1}(x))$, there
			exists a homomorphism $h_1: pat(n) \mapsto pat(\mbox{branch}(\hat{n})) \cup
			pat(\mbox{branch}(\mbox{new-node}_{T_1}(x)))$ (where $\hat{n}$ is the parent node
			of n in $T_1$) with $h_1(x) = x$ for all $x \in vars(n) \cap
			(vars(branch(\cap{n})) \cup vars(\mbox{branch}(\mbox{new-node}_{T_1}(x))))$, and
		\item for every child node $m$ of $C_2(x)$, and for every variable $y
			\in newvars(m)$, the following property holds: let $n \in
			\mbox{branch}(\mbox{new-node}_{T_1}(y))$. Then there exists a homomorphism
			$h_2:pat(n) \mapsto pat(C_1(x)) \cup pat(m) \cup
			pat(\mbox{branch}(\hat{n}))$ (where $\hat{n}$ is the parent node of $n$)
			with $h_2(x) = x$ for all $x \in vars(n) \cap (vars(C_1(x)) \cup
			vars(\mbox{branch}(\hat{n})))$.
	\end{enumerate}
\end{theorem}
\begin{proofidea}
	The first property is the most obvious one: every mapping $\sigma$ which is
	a solution for $C_1(x)$ assuming $G$ as our arbitrary graph, results in $\mu =
	\sigma_{|X}$ also being a solution for $C_2(x)$.
	The second condition makes sure that when $x$ is in the domain of the
	solution mapping $\sigma$ all of $C_1(x)$ was used when retrieving the
	mapping $\sigma$ from $G$.
	The last condition is similar to the condition (2c) in the
	Theorem\ref{projwd}: When $\mu = \sigma_{|X}$ with $dom(\mu) =
	vars(C_2(x))$ is not a solution for $T_2$ because $C_2(x)$ could have been
	extended with some child of $C_2(x)$ it must be that it is possible to
	extend $\sigma$ to some child of $C_1(x)$.
\end{proofidea}

In the actual proof of Theorem~\ref{projwd2}, Theorem~\ref{projwd} is used to show
that the conditions of Theorem~\ref{projwd2} are also sufficient.

\begin{theorem}[\cite{pichler2014containment}]\label{ccuppiempty}
	CONTAINMENT$[\{\cup,\pi\},\emptyset]$ is in NP.
\end{theorem}
\begin{proof}
	Similarly to Theorem~\ref{projwd} we can construct a procedure using Theorem~\ref{projwd2} 
	which is in NP.
	Let $(F,X)$ be a well designed pattern forest (pwdPF) and let $T$ be a
	$wdPT$. Consider now every pwdPT $F'$ in the forest $(F,X)$ and
	use our procedure to decide $(F',X) \subseteq T$. Put more formally we get $(F,X) \subseteq T$ iff.
	$(T_i,X) \supseteq T$ for every $(T_i,X) \in (F,X)$.
\end{proof}

The next problem that we tackle is \textbf{CONTAINMENT$[\{\cup,\pi\},\{\cup\}]$}: First
a necessary and sufficient condition for containment is given and then turned
into an algorithm. For this formulation the definition of a renamed proper
extension of a wdPF is important.
\begin{definition}[\cite{pichler2014containment}]
	Let $F = \{ T_i \mid \mbox{with } 1\leq i \leq k\}$ be a wdPF, $F'$ a
	subforest of $F$. For every $T_i \in F$, an injective renaming function
	$\rho_i$ with $dom(\rho_i) = vars(T_i)$, s.t. 
	\begin{enumerate}
		\item $\rho_i(x) = x$ for all $x \in vars(F')$,
		\item $\rho(x) \neq \rho(y)$ for every $x \in vars(T_i) \backslash
			vars(F'),i\neq j \in \{ 1, \dots, k\}$ and $y \in dom(\rho_j)$.
			Finally let $\hat{F}$ be the wdPF $\{ \rho_i(T_i) \mid 1 \leq i \leq
			k\}$. 
	\end{enumerate}
	Then a renamed proper extension of $F'$ is a subforest of
	$\hat{F}$, call it $\hat{F}'$ that has $F'$ as a proper subforest, i.e.,
	$\hat{F}'$ is not equal to $F'$.
\end{definition}

\begin{theorem}[\cite{pichler2014containment}]\label{projwd3}
	Let $(T_1,X)$ be a pwdPT and let $F_2$ be a wdPF. Then $(T_1,X) \subseteq
	F_2$ iff for every subtree $T'_1$ of $(T_1,X)$:
	\begin{enumerate}
		\item either there exists a child node $n$ of $T'_1$ s.t. there is a
			homomorphism $h: path(n) \mapsto pat(T'_1)$ with $h(x) = x$ for
			all $x \in varS(n) \cap vars(T_1')$
		\item or there exists a subtree $T'_2$ of $F_2$ with $vars(T'_2) =
			fvars(T_1')$ and $pat(T'_2) \supseteq pat(T'_1)$ s.t. every
			renamed proper extension $F'_2$ of $\{T'_2\}$ in $F_2$ satisfies one
			of the following properties:
			\begin{enumerate}
				\item \label{firstcaseprojwd3}there exists a proper renamed extension $\hat{F}'_2$ of
					$F'_2$ (i.e. a bigger extension than $F'_2$ nodewise),
					and a homomorphism $h_a: pat(\hat{F}'_2) \mapsto
					pat(F'_2)$ with $h(x) = x$ for all $x \in
					vars(\hat{F}'_2)\cap (vars(F'_2) \cup vars(T'_1))$, or
				\item there exists an extension $\hat{T}'_1$ of $T'_1$ and a
					homomorphism $h_b: pat(\hat{T}'_1) \mapsto pat(F_2') \cup
					pat(T'_1)$ with $h(x) =x$ for all $x \in vars(T_1')$, or
				\item case (2a) does not apply and there
					exists a tree $T \in F'_2$ with $vars(T)= fvars(T'_1)$.
			\end{enumerate}
	\end{enumerate}
\end{theorem}
\begin{proofidea}
 The conditions presented in Theorem~\ref{projwd3} are similar to those presented in
 Theorem~$\ref{projwd}$: First we choose an arbitrary subtree $T'_1$ of $(T_1,X)$ and
 inspect the mappings $\sigma$ that $T'_1$ induces assuming an arbitrary Graph
 $G$: 
 \begin{itemize}
	 \item Either $\sigma$ can be extended to some child node of $T_1'$ but then
 again $\sigma$ is not a solution of $(T_1,X)$ over $G$
	 \item  or $\sigma_{|X} \in \ll F_2\rr_G$.
\end{itemize}
Condition (1) takes care of subtrees for which $\sigma$ could be extended
to some additional child node of the subtree. This would mean that the subtree
is not a valid solution for $(T_1,X)$.
Condition (2) extends condition (2) from Theorem\ref{projwd}:
It is easy to see that we fulfill condition (2a) and condition (2b)
from Theorem~\ref{projwd} similarly by saying that the subtree $T'_2$ of $F_2$
with $vars(T'_2) = fvars(T'_1)$ and $pat(T'_2) \subseteq pat(T_1')$.  
The condition(2c) of Theorem~\ref{projwd} was used to make sure
that any extension $\mu'$ of $\mu$ for which $\mu' \in \ll T_2\rr_G$ holds would
result in $\sigma$ being no solution of $T_1$ over $G$. To extend
condition (2c) of Theorem~\ref{projwd} the notion of proper renamed extension
comes into use. This is due to the fact that we now have a pattern forest and
$\mu$ might still be a solution of another tree in $F_2$. This means that all
subtrees in $F_2$ with $vars(T'_2) = fvars(T'_1)$ and $pat(T'_2) \subseteq
pat(T'_1)$ must be eligible to be extended to show that $\mu$ is indeed not a
solution of $F_2$ but extensions of $\mu$ are solutions.
Condition $(2a)$ forces a certain maximality condition onto the proper renamed
extensions of $T'_2$, i.e., we cant have more nodes in the proper renamed
extension $F'_2$ without having a pattern mismatch in the proper renamed
extension $F'_2$ and
the extension $\hat{F}'_2$ of it. A pattern mismatch means that the homomorphism $h_a:
pat(\hat{F}'_2) \mapsto pat(F'_2)$ does not exist.
Condition $(2c)$ makes sure that the proper renamed extension $F'_2$
extends all the relevent subtrees of $F_2$, i.e. a relevant subtree is a tree $T
\in F'_2$ where $vars(T)  =fvars(T'_1)$.
%!!anm: that is not already included into
%the subforest $F'_2$.???!!
Condition (2b) checks for the existence of a homomorphism $h_b$ that maps
$pat(\hat{T'_1})$ into the patterns of the renamed proper extension $pat(F_2)
\cup pat(T_1')$.

\end{proofidea}

\begin{theorem}[\cite{pichler2014containment}]
	CONTAINMENT$[\{\cup,\pi\},\{\cup\}]$ is in $\Pi^P_2$.
\end{theorem}
\begin{proof}
	The characterization in Theorem~\ref{projwd3} yields a more or less
	straightforward $\Sigma^P_2$-algorithm for testing $(T_1,X) \not\subseteq
	F_2$: first guess $T'$ and the proper renamed extension $\hat{F}'_2$, and
	then use a coNP-oracle that there does not exist a child node $n$ and
	homomorphism $h$ as described by property $(2b)$.
	To now show \textbf{CONTAINMENT[$\{\cup,\pi\},\{\cup\}$]} we extend the algorithm
	wdPF in the following way: $(F_1,X) \subseteq F_2$ iff. $(T_i,X) \subseteq
	F_2$ for every $(T_i,X) \in (F_1,X)$.
\end{proof}

The hardness proof of \textbf{CONTAINMENT[$\emptyset,\{\cup\}$]} is done by a reduction from
the well known $\Pi^P_2$-complete problem 3-QSAT$_{\forall,2}$. !!citation.

\begin{framed}\noindent 3-QSAT$_{\forall,2}$\\
	INPUT:A formula $\phi= \forall \vec{x}\exists \vec{y} \psi$,\\
	where $\psi$ is a Boolean formula in CNF over the variables
	$\vec{x}\cup\vec{y}$.
	
	QUESTION: Can every assignment $I$ on the variables in $\vec{x}$ be extended
	to an assignment $J$ on $\vec{y}$, s.t. $J \models \psi$?
\end{framed}


\begin{theorem}[\cite{pichler2014containment}]\label{cemptycup}
	CONTAINMENT$[\emptyset,\{\cup\}]$ is $\Pi^P_2$-hard.
\end{theorem}
\begin{proofidea}
	Assume an arbitrary instance of $3-QSAT_{\forall,2}$ and construct an
	instance of \textbf{CONTAINMENT$[\emptyset,\{\cup\}]$} as follows: The wdPT $T_1$
	consists of the root and two child nodes: $n_i, n'_i$ for every variables
	$x_i \in \vec{x}$. In this way we are able to model the assignment $I$ of
	the 3-QSAT$_{\forall,2}$ problem in form of subtrees of $T_1$. 
	Additionally we have a child node $n_0$ containing the
	variables in $\vec{y}$ and an encoding of the formula $\psi$. 
	
	It remains to deal with the ``unintended'' subtrees of $T_1$ where given an
	$i$ either both $n_i$ and $n'_i$, or neither $n_i$ and $n'_i$ are in the subtree.
	This is done by adding certain wdPTs to $F_2$ which take care of the two
	problems. The last wdPT added to the forest $F_2$ contains the triples
	encoding the formula $\psi$ in its root plus the child nodes $n_i,n'_i$.
	This wdPT produces the solutions of all ``intended'' subtrees of $T'_1$ if
	and only if every assignment $I$ on the variables in $\vec{x}$ can be
	extended to an assignment $J$ on $\vec{y}$, s.t. $J \models \psi$.
\end{proofidea}

The section dealing with decidable containment closes with settling the
complexity of SUBSUMPTION$[S_1,S_2]$ problem for every $S_1, S_2 \subseteq \{
\cup, \pi \}$. In prior work \cite{letelier2012static} it was shown, that the simple case
$S_1=S_2=\emptyset$ is $\Pi_2^P-complete$. 
Later on in~\cite{letelier2013static} the $\Pi^P_2$-membership 
was extended to the case where $S_1 = S_2 = \{\pi\}$ holds.
To establish the $\Pi^P_2$-completeness to arbitrary $S_1,S_2 \subseteq \{ \cup,
\pi \}$, it obviously suffices to show the $\Pi_2^P$-membership for the most general case.

\begin{theorem}[\cite{pichler2014containment}]\label{scuppicuppi}
	\textbf{SUBSUMPTION[$\{\cup,\pi\}, \{ \cup, \pi \}$]} is in $\Pi^P_2$.
\end{theorem}
\begin{proofidea}
	Let $(F_1,X)$ and $(F_2,X)$ be two pwdPFs.
	The main proofidea is that the following critera for subsumption was found:
	$(F_1,X) \sqsubseteq (F_2,X)$ iff for every subtree $T'_1$ of $F_1$, there
	exists a subtree $T'_2$ of $F_2$, s.t.
	\begin{enumerate}
		\item $fvars(T'_1) \subseteq fvars(T'_2)$ and
		\item there exists a homomorphism $h:pat(T'_2) \mapsto pat(T'_1)$
			with $h(x)= x$ for all $x \in fvars(T_1')$.
	\end{enumerate}
	Consider now the following procedure:
	For all subtrees $T'_1$ of $F_1$ check that there exists a subtree $T'_2$ of
	$F_2$ together with a homomorphism of the desired property. This procedure
	can be executed in $\Pi_2^P$ because we have need to check all subtrees
	for a property which would be in co-NP, but then the property is a
	homomorphism check which is in NP. So we have a co-NP$^{\mbox{NP}}$ runtime.
\end{proofidea}

In~\cite{letelier2012static} the authors noticed an interesting feature of
subsumption in SPARQL: In the fragment of wd-SPARQL$[\emptyset]$ subsumption is
able to characterize equivalence. Assume graph patterns $P_1,P_2 \in$
wd-SPARQL[$\emptyset$]. Then $P_1 \equiv P_2$ iff. $P_1 \sqsubseteq P_2$ and 
$P_2 \sqsubseteq P_1$. In~\cite{letelier2013static} the authors were able to
find a counterexample for wd-SPARQL[$\pi$]. 
Now in~\cite{pichler2014containment} the result is strengtened:
Assuming one pattern $P_1$ in wd-SPARQL[$\emptyset$] and one pattern $P_2$ in either
wd-SPARQL[$\{\pi\}$] or wd-SPARQL[$\{\cup\}$] one can show that $P_1 \not\equiv
P_2$ but $P_1 \sqsubseteq P_2$ and $P_2 \sqsubseteq P_1$.

\begin{proposition}[\cite{pichler2014containment}]
	There exist pairs $P_1,P_2$ of graph patterns $P_1$ from
	wd-SPARQL[$\emptyset$] and $P_2$ from either wd-SPARQL[$\{\pi\}$]
	wd-SPARQL[$\{\cup\}$], s.t. $P_1 \sqsubseteq P_2$ and $P_2 \sqsubseteq P_1$
	hold but $P_1 \not\equiv P_2$.
\end{proposition}
\begin{proofidea}
	It is an easy observation that we need two counterexamples to prove the
	proposition. 
	\begin{enumerate}	
		\item At first we consider the case where  $P_1 \in$
			wd-SPARQL[$\emptyset$] and  $P_2 \in$ wd-SPARQL[$\{\cup\}$]. Let
			$P_1 = (t_1 \OPT t_2)$ and $P_2 = ((t_1) \UNION (t_1 \AND t_2))$. The patterns $t_1$ and $t_2$
			are assumed to be distinct.
			$P_1 \sqsubseteq P_2$ can be easily seen because $P_2$ more or less models
			the semantics of $OPT$. $P_2 \sqsubseteq P_1$ Also holds because again the
			semantics of opt are projected into $P_2$. But $P_1 \sqsubseteq P_2$ does
			obviously not hold since $P_2$ always has mappings which are solely created by
			$t_1$.
		\item 
			Now for the second counterexample where  $P_1 \in$
			wd-SPARQL[$\emptyset$] and  $P_2 \in$ wd-SPARQL[$\{\pi\}$]. 
			$P_1 = (x_1,a,x_2) \OPT ((x_3,a,x_2) \AND (x3,a,x_3))$\\
			and
			$P_2 = ((x_1,a,x_2) \AND (y_1,a,Y_2)) \OPT ((x_3,a,x_2) \AND
			(x_3,a,x_3)\\ \AND (y_3,a,y_2) \AND (y_3,a,y_3))$ and $X =
			\{x_1,x_2,x_3\}$.
			It remains to show that $P_1 \sqsubseteq (P_2,X)$ and $(P_2,X)
			\sqsubseteq P_1$ but $P_1 \not\equiv (P_2,X)$.
		    Towards this goal we observe that the triple patterns in $P_1$ are
			contained in $P_2$. Also, $P_2$ contains triple patterns with
			existential variables. We can easily see that there is a
			homomorphism mapping the triple patterns containing the existential
			variables to into the
			patterns of $P_1$. Thus $P_1 \sqsubseteq (P_2,X)$ holds. Also
			$(P_2,X) \sqsubseteq P_1$ holds, but in general $(P_2,X) \not\subseteq
			P_1$ since one can provide a graph $G$ and an appropriate
			instantiation of the existential variables $y_1, y_2$ in the root
			of $P_2$ that block the extension of mapping to the child node.
			Consider the following RDF graph $G = \{ a(1,1), a(2,3)\}$. Then
			$\mu = \{x_1 \mapsto 1, x_2 \mapsto 1\} \in \ll (T_2,X)\rr_G$,
			because of the mapping $\lambda = \mu \cup \{ y_1 \mapsto 2,
			y_2\mapsto 3 \} \in \ll T_2 \rr_G$ which cannot be extended to the
			child node of $r_2$ However $\mu \notin \ll T_1\rr_G$ since $\mu$
			can be extended to the child node of $r_1$ by adding $[x_3 \mapsto
		1]$.
			
	\end{enumerate}
\end{proofidea}

\section{Undecidable Containment}
We can see in table~\ref{conttable} that
\textbf{CONTAINMENT$[S_1,S_2]$} is undecidable, when $\pi \in S_2$. Again we don't need to show
the undecidability for every entry in the table, as we can just show it for the
most specific entry, i.e.,
\textbf{CONTAINMENT[$\emptyset,\{\pi\}$]}. We do this proof by
reducing from the conjunctive query answering problem under integrity 
constraints in form of tuple generating dependencies (abbr. tgds)~\cite{JOHNSON1984167,cali2008taming}. This problem
is well known to be undecidable~\cite{JOHNSON1984167,cali2008taming}. We are not going to reduce the original problem but a modified version of it to our problem. 
Three changes need to be made, to make the problem suitable for us:
\begin{enumerate}
	\item  The undecidability results for the problem refer to arbitrary
		databases, which would include infinite databases. Our RDF graphs
		however are a finite set of triples.
	\item The problem allows predicates of arbitrary length but our RDF graphs
		contains only triples. 
	\item Finally, for the problem reduction in the end, it turns out to be
		convenient to restrict the problem from a set of tgds to a single tgd.
\end{enumerate}

Before doing the first step, it is crucial to define tuple generating dependencies 
and the undecidable problem, namely \textbf{CQ-UNDER-TGDs}.

\begin{definition}[tuple generating dependency]
	Let $\phi(\vec{x})$ and $\psi(\vec{x},\vec{y})$ be conjunctive queries.
	Also, let all variables $\vec{x}$ occur in $\phi(\vec{x})$.
	A tuple generating dependency (tgd) is a first-order formula of the form 
	$\forall \vec{x} (\phi(\vec{x}) \rightarrow \exists \vec{y})
	\psi(\vec{x},\vec{y})$.
\end{definition}
To simplify the notation of a tgd, the $\forall$-quantifiers are omitted.
Let $I$ be a database instance and $\tau$ be a tgd. Then we define $I \models
\tau$ %holds,
%like the usual satisfaction relation for first-order formulas. Another way to
%define satisfaction would be  
using homomorphisms. This is feasible, because a tgd is an implication:
For every homomorphism $h: \phi(\vec{x}) \mapsto I$ (mapping constants to
themselves), which is responsible for the antecedent of the implication, there
must be an extension $h'$ of $h$, for which $h':\psi(\vec{x},\vec{y}) \mapsto I$
holds. It is then natural to define satisfaction for a set of tgds: Let $\Sigma$
be a set of tgds. Then $I \models \Sigma$ iff. $I \models \tau$ for every $\tau
\in \Sigma$.
Let $Q$ be conjunctive query. For a set $\Sigma$ of tgds, a database instance
$I$ and a BCQ $Q$, we say that $I,\Sigma \models Q$ holds, if for every
(possible infinite) database instance $M$, s.t. $M \models \Sigma$ and $I
\subseteq M$, we also have $M \models Q$. We write $I, \Sigma \models_f Q$ if
only finite models $M$ are allowed.

Consider now the two problems:
\begin{framed}\noindent \textbf{CQ-UNDER-TGDs}\\
	INPUT: set $\Sigma$ of tgds, database instance $I$ and a CQ $Q$.\\
	QUESTION: does $\Sigma, I\models Q$ hold? 
\end{framed}
\begin{framed}\noindent \textbf{FINITE-CQ-UNDER-TGDs}\\
	INPUT: set $\Sigma$ of tgds, database instance $I$ and a CQ $Q$.\\
	QUESTION: does $\Sigma, I\models_f Q$ hold? 
\end{framed}
We proceed with step one procedure:
By~\cite{cali2008taming} \textbf{CQ-UNDER-TGDS} is undecidable. The proof
in~\cite{cali2008taming} will be changed so that
\textbf{FINITE-CQ-UNDER-TGDs} remains
undecidable even though $|\Sigma| = 1$.
%Then a reduction from this restricted version of \textbf{FINITE-CQ-UNDER-TGD} to
%\textbf{CONTAINMENT[$\emptyset,\{\pi\}$]} will be produced to establish the sought-after
%result.
Examining the undecidability proof of \textbf{BCQ-UNDER-TGDs} in~\cite{cali2008taming}, 
a reduction from the HALTING problem to \textbf{BCQ-UNDER-TGDs} is given: The
initial configuration of the Turing machine is encoded into the instance $I$ and
several $tgds$ are used to describe the transitions of the $TM$. The query $Q$
descrbes the halting condition.
It is then shown that the Turing Machine halts iff $\Sigma,I \models Q$ holds.
When the machine doesn't halt, one can construct a counter-model $M$ for
$\Sigma,I \models Q$. For this model $M \models \Sigma$ and $I \subseteq M$ but
$M \not \models Q$. This construction is defined by the straightforwards
encoding of the infinite run of the $TM$. 
\begin{proposition}~\cite{pichler2014containment}
	The same reduction cannot work for \textbf{FINITE-CQ-UNDER-TGDs}.
\end{proposition}

\begin{proof}
Proof by contradiction.	
Assume the same reduction works for \textbf{FINITE-CQ-UNDER-TGDs}.
Testing if $I,\Sigma \not\models_f Q$ holds) (the co-problem of
\textbf{FINITE-CQ-UNDER-TGDs}) is semi-decidable:
In a loop enumerate all finite models $M$ and then in the loop check, if $M
\models I$, $M \models \Sigma$ and $M \not\models_f Q$ holds. If we then use the
assumption that the same reduction works, we could reduce from
\textbf{co-CQ-UNDER-TGDs}
to \textbf{co-FINITe-CQ-UNDER-TGDs}. Which would mean \textbf{co-CQ-UNDER-TGDs} is semidecidable.
But asking if $I,\Sigma \models Q$, i.e. \textbf{co-CQ-UNDER-TGDs} is semi-decidable because it is the same as asking for unsatisfiability of the set of FO formulas $I,\Sigma,\not Q$ due to
the semi-decidability of first-order logic. But then the established
undecidability of the halting problem would not halt and we have a
contradiction.
\end{proof}
Even though the proof cannot be used to prove \textbf{FINITE-BCQ-UNDER-TGDs} undecidable
the following theorem was established:

\begin{theorem}\label{fund}~\cite{pichler2014containment}
	\textbf{FINITE-BCQ-UNDER-TGDs} is undecidable.
\end{theorem}
\begin{proofidea}
	The main idea is that \textbf{co-HALTING} is reduced to
	\textbf{FINITE-CQ-UNDER-TGDs}. 
	The initial configuration of the TM is encoded in the instance $I$ and the
	transitions of the TM are encoded by the tgds. 
	The atoms $state(x,q), cursor(x,p)$ and $contains(x,y,s)$ are used to
	represent a configuration of the turing machine. Using the mentioned atoms one can
	express that at some time instant $x$, the TM is in state $q$, the cursor is
	in position $p$ and the tape content of tape cell $y$ is $s$.
	The successor relation $next(x,x')$ is defined that can be applied to time
	instants and tape positions.
	Now back to the adaptions from the original transformation:
	\begin{enumerate}
		\item A relation $smaller(\cdot,\cdot)$ is introduced and
		used to encode the transitive closure of $next(\cdot,\cdot)$.
	\item The query $Q$, remember, this was prior used to encode the halting
			condition, now asks for $smaller(x,x)$, i.e., if there exists some
		``loop'' in the time instants.  
	\end{enumerate}
	$I,\Sigma \models_f Q$ holds iff the TM does not halt can be shown.
	Assuming the TM halts, a simple countermodel $M$ in form of the natural encoding
	of the halting run of the TM can be found.
	Suppose that the TM does not halt. Then every model $M$ of $I,\Sigma$
	contains an encoding of the infinite number of steps in the non-halting run
	of the TM. Now we use our assumption that $M$ is finite and every step is
	identified by some time instant. Thus at least one symbold $a$ is used to
	encode more than one time instant (which results in the loop). Thus
	$smaller(a,a) \in M$. If $smaller(x,x)$ occurs we have a (non-halting) run of the TM. 
	Since each step(state, cursor position and cell content) is identified by
	some time instant and $M$ is finite. 
\end{proofidea}

For completing step two and three of our procedure, we need to strengthen the
undecidability result from~\ref{fund} to atoms with arity two and to restrict
the set of tgds in \textbf{FINITE-CQ-UNDER-TGDs} to a singleton. 
Notice that atoms of arity two are just a different representation of triples. 
$p(s,o) \sim (s,p,o)$.

\begin{theorem}\label{fundplus}~\cite{pichler2014containment}
	\textbf{FINITE-BCQ-UNDER-TGDs} is undecidable, even if the arity of every relation
	symbol is at most two and even if $\Sigma$ consits of a single tgd.
\end{theorem}

\begin{proofidea}
To construct a single tgd $\tau$ from $\Sigma$ all antecedents of the thfd in
$\Sigma$ are combined into one antecedent in $\tau$. The variables of the
various antecedents are renamed. The same is done for the consequent of $\tau$.
The implication is additionally modified: Switches are introduced such that for
every tgd $\tau_i \in \Sigma$. If the $i-th$ switch is turned-on, every switch
$j\neq i$, may be turned off which means that $\tau_j$ is trivially satisfied.
This switch idea models the various implications in only implication.

To only use binary atoms, every atom of arity $k>2$ is replaced by $k$ binary
atoms such that a chain of equivalences hold: for any such binary atom in the
tgd or query, there exists a homomorphism into an instance $I$ iff. the
homomorphism can be extended to map all $k$ atoms into $I$ iff. this
homomorphism is also a homomorphism in the original non-binary case.
\end{proofidea}

Having the strenghtened version of \textbf{FINITE-CQ-UNDER-TGDs} we can now prove
\textbf{CONTAINMENT[$\emptyset,\{\pi\}$]} undecidable.

\begin{theorem}\label{cemptypi}~\cite{pichler2014containment}
	\textbf{CONTAINMENT[$\emptyset,\{\pi\}$]} is undecidable.
\end{theorem}
\begin{proofidea}
	Assume an arbitrary instance of \textbf{FINITE-CQ-UNDER-TGDs} containing only a
	single tgd. We construct our instance of
	\textbf{CONTAINMENT[$\emptyset,\{\pi\}]$} in the following way:
	Let $T_1$ be a wdPT and $T_2$ be a pwdPT $(T_2,X)$ each consisting of a root
	node, with one child node. Both root nodes contains the antecedent of the
	single tgd $\tau$ and the instance $I$.
	The root $r_2$ of $T_2$ contains in addition another copy of the antecedent
	of $\tau$, such that the variables in the antecedent are realized by
	existential variables in $evars(r_2)$.
	The consequent of the tgd is contained in the child nodes $n_1,n_2$ of $r_1$ and
	$r_2$. The child node $n_1$ in $T_1$ contains the query. The child node
	$n_2$ in $T_2$ contains in addition another copy of the consequent of $\tau$
	realized by existential variables in $evars(n_2)$.
	There are auxiliary graph patterns in $r_1$ and $n_1$ which deal with the
	lack of projection. 
	The construction ensures that $T_1 \sqsubseteq(T_2,X)$ holds. 
	Hence the only reason for $T_1 \not\subseteq (T_2,X)$ is that for some RDF
	graph $G$, we have the following situation: Some solution $\mu \in \ll
	T_1\rr_G$ sends the root into $G$ but cannot be extended to $n_1$, while in
	$(T_2,X)$ every extension of $\mu$ can be further extended to the existential 
	variables in the root
	to send also the child node $n_2$ into $G$.
	The following three facts are deduced:
	\begin{enumerate}
		\item $Q$ is not satisfied by $G$: indeed, $n_2$ consists of triples from
			$n_1$ plus the triples encoding the CQ $Q$. Since $n_2$ can be
			mapped into $G$ by an extension of $\mu$ this is also true for all
			triple patterns in $n_1$ excpets for those encoding $Q$.
		\item $G$ satisfies $\tau$: indeed, recall that $T_2$ uses existential
			variables to encode a copy of the antecedent of $tau$ in the root
			and a copy of the consequent of $tau$ in $n_2$ respectively. We are
			assuming that every mapping on $vars(r_2)$ that maps the root into
			$G$ can be extended to the existential variables in $n_2$ s.t. $n_2$
			is mapped into $G$. Hence, $G$ satisfies $\tau$ by the homomorphism
			criterion.
		\item $I$ must be contained in $G$, since we are assuming that $\mu$
			sends the root of both, $T_1$ and $T_2$ into $G$.
	\end{enumerate}
	We can see that $G$ provides a countermodel for $I,\tau \models_f Q$.
\end{proofidea}

\section{Equivalence}
When looking at the equivalence table~\ref{equivtable}, it is not easy to
distinguish the decidable cases from the undecdiable ones: Even though CONTAINMENT$[S_1,S_2]$ becomes
undecidable iff $\pi \in S_2$, EQUIVALENCE$[\{\pi,\cup\},\emptyset]$
is decidable. To keep the number of proofs to an absolute minimum, the
fact that membership results propagate to the more special cases and the hardness
results to the more general cases is made use of.
The following results are proven:
\begin{itemize}
	\item $\Pi^P_2$-membership of \textbf{EQUIVALENCE[$\{\cup,\pi\},\emptyset$]}
	\item $\Pi^P_2$-hardness of \textbf{EQUIVALENCE[$\{\cup\},\emptyset$]}
	\item $\Pi^P_2$-hardness of \textbf{EQUIVALENCE[$\{\pi\},\emptyset$]}
	\item Undecidability of \textbf{EQUIVALENCE[$\{\pi,\cup \},\{\cup \}$]}
	\item Undecidability of \textbf{EQUIVALENCE[$\{\pi\},\{\pi\}$]}
\end{itemize}

After the completion of the above proofs we can conclude all the complexity results in the cells of table~\ref{equivtable}
except two:
\begin{enumerate}
	\item \textbf{EQUIVALENCE[$\emptyset,\emptyset$]}: This result was shown in
\cite{letelier2012static}. 
\item \textbf{EQUIVALENCE[$\cup,\cup$]} follows immediately from the $\Pi^P_2$-membership
	of the \textbf{CONTAINMENT[$\{\cup\}, \{\cup\}$]} problem and the $\Pi^P_2$-hardness of
	\textbf{EQUIVALENCE[$\{\cup\}, \emptyset$]} to be shown.
\end{enumerate}
Completeness for \textbf{EQUIVALENCE[$\pi,\cup$]} is not established. The hardness result
carries over from the proof of \textbf{EQUIVALENCE[$\{\pi\},\emptyset$]} and
\textbf{EQUIVALENCE[$\{\cup\},\emptyset$]}.

We begin with a proof for $\Pi^P_2$-membership of
\textbf{EQUIVALENCE[$\{\cup,\pi\},\emptyset$]}. 

\begin{theorem}~\cite{pichler2014containment}\label{equivcuppiempty}
Let $T$ be a wdPT and $(F,X)$ be a pwdPF.
Then $T \equiv (F,X)$ iff.
\begin{enumerate}
\item $T \sqsubseteq (F,X)$ and
\item $(F,X) \subseteq T$.
\end{enumerate}
\end{theorem}
\begin{proof}
It is obvious that both properties are necessary for equivalence because 
$T \subseteq (F,X)$ implies $T \sqsubseteq (F,X)$ and if  $T \subseteq (F,X)$
and
$T \supseteq (F,X)$ are assumed then $T \equiv (F,X)$ holds by definition of equivalence.

It thus remains to show that  under assumption of 
$(F,X) \subseteq T$, $T \sqsubseteq (F,X)$ iff. $T \subseteq (F,X)$  holds. 
The ``only if'' direction is trivial as mentioned before.
We now sketch the proof of the if direction:
Assume  $(F,X) \subseteq T$ and $T \sqsubseteq (F,X)$.
Now proceed with a proof by contradiction: Assume $T \subseteq (F,X)$ doesn't
hold. Thus there exists a graph $G$, where some solution $\mu$ of $T$ is not a
solution of $F(,X)$ over $G$. But there we can find some extension $\mu'$ of
$\mu$ which is a solution of $(F,X)$. But then $\mu'$ must include mappings
$\mu$ didn't, and is thus a proper extension of $\mu$. But then again by
condition (2), $\mu'$ is a also a solution of $T$. But this cannot be true
because a mapping and its proper extension are both solutions to a wdPT. 
\end{proof}
We can easily see that the characterization in Theorem~\ref{equivcuppiempty} can be
transformed into an algorithm which yields the membership proof for
\textbf{EQUIVALENCE[$\emptyset,\{\cup,\pi\}$]}.
\begin{theorem}~\cite{pichler2014containment}
	\textbf{EQUIVALENCE[$\emptyset,\{\cup,\pi\}$]} is in $\Pi_2^P$.
\end{theorem}
\begin{proof}
By Theorem\ref{ccuppiempty} deciding the second property is $NP-complete$ and
by Theorem\ref{scuppicuppi} deciding the first property is $\Pi_2^P$-complete
rendering the complexity of the algorithm $\Pi_2^P$-complete.
\end{proof}

Following up we have the hardness result of\\
\textbf{EQUIVALENCE[$\emptyset,\{\cup\}$]} and
\textbf{EQUIVALENCE[$\emptyset,\{\pi\}$]}.
\begin{theorem}~\cite{pichler2014containment}
	\textbf{EQUIVALENCE[$\emptyset,\{\cup\}$]} is $\Pi_2^P$-hard
\end{theorem}
\begin{proofidea}
The same construction as in Theorem~\ref{cemptycup} can be used to prove the
desired result: The same reduction from \textbf{3-QSAT$_{\forall,2}$} is used.
Remembering that in this construction a $wdPT T_1$ and a wdPF $F_2$ such that
$\phi$ is valid iff.  $T_1 \subseteq F_2$ holds is constructed.
One can
not only show $T_1 \subseteq F_2$ but also $T_1 \supseteq F_2$ (iff. $phi$ is
valid of course). This
argumentation yields the desired result.
\end{proofidea}

\begin{theorem}~\cite{pichler2014containment}
	\textbf{EQUIVALENCE[$\emptyset,\{\pi\}$]} is $\Pi_2^P$-hard
\end{theorem}
\begin{proofidea}
	A reducion from \textbf{3-QSAT$_{forall,2}$} to
	\textbf{EQUIVALENCE[$\emptyset,\{\pi\}$]}
is needed to obtain the desired result.
\end{proofidea}

The following two theorems are proven by adapting the reduction from
\textbf{FINITE-BCQ-UNDER-TGDs} to \textbf{CONTAINMENT[$\emptyset,\{\pi\}$]} in the proof of 
Theorem~\ref{cemptypi}.
\begin{theorem}~\cite{pichler2014containment}
	\textbf{EQUIVALENCE[$\{\cup,\pi\},\{\cup\}$]} is undecidable.
\end{theorem}
\begin{theorem}~\cite{pichler2014containment}
	\textbf{EQUIVALENCE[$\{\pi\},\{\pi\}$]} is undecidable.
\end{theorem}


\chapter{The SERVICE-operator in Practice}\label{chapter:serviceeval}
The first part of this chapter is a summary of ~\cite[p. 4-7]{BuilAranda20131}. 
In the second part of this chapter we discuss the difference of the notions we
introduced in the first part.
In order to get a
deeper understanding of the SERVICE operator it is mandatory to understand which
problems occur when evaluating the SERVICE operator.
A direct implementation of the SERVICE operator based of the semantics is
infeasible in practice. Given $(\mbox{SERVICE }  x \
P_1)$, if $x$ is not restricted to a finite set, we would have to evaluate $P_1$ over every 
possible SPARQL endpoint in $dom(ep)$. This is obviously impossible. To ensure that $P_1$
only gets evaluated over a finite set of URIs, $x$ needs to be limited to exactly
those. In the W3C standard only indications are provided on how to evaluate the
service operator~\cite{w3standardservice} when the location is a variable and
not an URI. In~\cite{BuilAranda20131} the authors deal with this issue by
providing a notion of boundedness. In order to demonstrate how one could
evaluate a service operator using a variable to evaluate a query over more than
one endpoint the following example is given:
\begin{example}[\cite{BuilAranda20131}]
	Let $G$ be an RDF graph that uses triples of the form\\ $(a, service\_address,b)$
	with the intention to express that $b$ is a SPARQL endpoint URI with name $a$.
	Then we consider the following query $P$ over the graph $G$ in the dataset $DS$:
	\begin{align*}
		P=((x, service\_address, y) \AND (\mbox{SERVICE } y \ (z_n,email,z_e)))
	\end{align*}
	It is easy to see that $P$ is used to compute the list of names and email
	addresses that can be retrieved from the SPARQL endpoints stored in the RDF
	graph $G$ through the $service\_address$ triple. 
	The whole point of this example is to point out that there is a simple practical
	way to evaluate $P$ over $G$ that is also feasible:
	By evaluating $\ll (x,service\_address,y) \rr^{DS}_G$ first and then for every
	mapping $\mu$ in this set we further evaluate $\ll (\mbox{SERVICE } a \ (z_n, email, z_e)
	\rr^{DS}_G$, where $a = \mu(y)$. 
\end{example}
Throughout the chapter we will provide four different definitions on how the
destinations of a SERVICE-operator can be bounded within a graph pattern if the
destination is a variable. Those definitions were first introduced
in~\cite{BuilAranda20131}.
\begin{enumerate}
	\item Boundedness: Boundedness is a na{\"i}ve semantic approach to the
		problem. After formally defining the property, we will show that
		deciding whether a variable is bounded in a graph pattern
		is undecidable and thus not feasible for practical use.

	\item Strong boundedness: Strong boundedness is a syntactical approach to
		the problem. We will, by defining the property provide a recursive
		procedure on how to decide which variables in a graph pattern are
		strongly bounded. It will also be shown that if a variable is strongly
		bounded, it is bounded aswell. 
		Although this procedure would be feasible for practical
		use complexity-wise, it is not able to decide whether a graph pattern
		can be evaluated in practice. We will provide an example in form of a
		graph pattern which is feasible to be evaluated in practice but not
		strongly bounded.

	\item Service-boundedness: Service-boundedness would be the optimal solution for
		deciding which graph patterns can be evaluated in practice. If a pattern
		is service-bounded, it can be evaluated in practice. The problem however
		is, that the definition builds on the definition of boundedness which is
		undecidable. Therefore service-boundedness is not feasible for practical
		use.
	\item Service-safeness: The definition of service-safeness builds on the
		definition of strong boundedness. 
		service-safeness is easy to decide and we will show that if a pattern is
		service-safe, it is also service-bounded. This solution should be used in
		practice for the original problem.
\end{enumerate}

In the last section we will discuss the difference of boundedness and strong
boundedness (and thus service-boundedness and service-safeness because the
latter definitions build on former definitions). 

\section{The four different ways to bind the destination of a SERVICE-operator}

To describe boundedness we need three definitions namely the domain of a graph,
a dataset and a graph pattern.

\begin{definition}[Domain of a graph, a dataset and a graph pattern,\cite{BuilAranda20131}]
	The domain of a graph $G$ denoted $dom(G)$ is defined as $dom(G) = \bigcup\limits_{(u,v,w) \in G}
	vars(u,v,w)$. The domain of a dataset $DS$, denoted $dom(DS)$ is defined as
	$dom(DS) = \bigcup\limits_{G \in names(DS)} dom(G)$.
	The domain of a graph pattern $P$ is denoted $dom(P)$ and refers to the set of URIs that are
	mentioned in $P$.
\end{definition}

Boundedness of a variable $x$ in a graph pattern $P$ makes sure that the
variable is always in the domain of every solution $\mu$ and the image $\mu(x)$ is either in the 
domain of the dataset $P$ gets evaluated over, it is a graph name or it is in the domain of the pattern.

\begin{definition}[Boundedness,\cite{BuilAranda20131}]
	Let $P$ be a graph pattern and $x \in var(P)$. Then $x$ is bounded in $P$ if the
	following condition holds:
	For every dataset $DS$, every graph $G$ in $DS$ 
	and every $\mu \in \ll P \rr^{DS}_G:$\\
	\[ x \in dom(\mu) \mbox{ and } \mu(x) \in (dom(DS) \cup names(DS) \cup dom(P)). \]
\end{definition}

Resulting from this definition a very naive way to ensure that
a graph pattern $P$ can be evaluated in practice seems to arise:
Assuming we want to evaluate a subpattern $(\mbox{SERVICE } x \ P_1)$ of $P$
we require $x$ to be bounded in $P$. We can then define the problem for deciding if a variable
is bounded in a graph pattern $P$:

\begin{framed}\noindent \textbf{BOUND IN PATTERN}\\
	\textbf{INPUT:} A graph pattern $P$ and a variable $x \in var(P)$.\\
	\textbf{QUESTION:} Is $x$ bounded in $P$?
\end{framed}
Unfortunately \textbf{BOUND IN PATTERN} is undecidable which can be shown by
reducing from \textbf{SPARQL SAT} to \textbf{BOUND IN PATTERN}.
\begin{framed}\noindent \textbf{SPARQL SAT}\\
	\textbf{INPUT:} A graph pattern $P$.\\
	\textbf{QUESTION:} Does a dataset $DS$ and a graph $G$ in $DS$ exist such
	that $\ll P \rr^{DS}_G$?
\end{framed}
It is a well known result that $\textbf{SPARQL SAT}$ is undecidable~\cite{angles2008expressive}.

\begin{theorem}[\cite{BuilAranda20131}]\label{boundundec}
	\textbf{BOUND IN PATTERN} is undecidable.
\end{theorem}
\begin{proof}
	By providing a reduction from the \textbf{SPARQL SAT} problem to the
	\textbf{VARIABLE BOUND IN PATTERN} Problem we will be able to prove the theorem.
	Let $P$ be a graph pattern, i.e., an arbitrary instance of \textbf{SPARQL SAT} and
	$x,y,z$ variables not mentioned in $P$. Then define the graph pattern $Q$,
	i.e. the instance of \textbf{VARIABLE BOUND IN GRAPH} pattern as follows: $Q =
	((x,y,z) \UNION  P)$ and choose $x$ to be the potentially bounded variable.
	It remains to show that $x$ is bounded in $Q$  if and only if $P$ is not
	satisfiable.\\
	$(\Rightarrow)$:\\
	Assume now that the variable $x$ is bounded in $Q$, i.e., for every RDF graph $G$
	in $DS$ and every $\mu \in \ll Q \rr^{DS}_G:$ $x \in dom(\mu)$ and $\mu(x) \in
	(dom(DS) \cup names(DS) \cup dom(P))$. Let $DS$ be an arbitrary dataset and let
	$G$ be an arbitrary graph in $DS$. Distinguish the following two cases:
	\begin{enumerate}
		\item $\ll Q \rr^{DS}_G = \emptyset$. Because of $\ll Q \rr^{DS}_G =
			\emptyset$, we can instantly see that $P$ is unsatisfiable.
		\item $\ll Q \rr^{DS}_G \neq \emptyset$.\\
			Let $\mu \in \ll Q \rr^{DS}_G$ be arbitrary.
			We can instantly see that $x \in dom(\mu)$ must hold. 
			Because by construction $P$ doesn't contain $x$, 
			$\mu \notin \ll P \rr^{DS}_G$. Thus $P$ is unsatisfiable.
	\end{enumerate}
	\noindent$(\Leftarrow)$:\\
	Assume now that $P$ is not satisfiable. Let $DS$ be an arbitrary dataset and let
	$G$ be an arbitrary graph in $DS$. Because of our initial assumption $\ll P
	\rr^{DS}_G = \emptyset$. We will now further distinguish between two cases:
	\begin{enumerate}
		\item $\ll Q \rr^{DS}_G = \emptyset$. Then $x$ is trivially bounded in $Q$.

		\item $\ll Q \rr^{DS}_G \neq \emptyset$.\\
			Let $\mu \in \ll Q \rr^{DS}_G$ be arbitrary.
			We can instantly see by construction of $Q$ that $x \in
			dom(\mu)$ must hold. By SPARQL semantics $\mu(x) \in dom(DS)$. Thus $x$
			is bounded in $Q$.\qedhere
	\end{enumerate}
\end{proof} 

As deciding boundedness for a variable is undecidable the notion of
\emph{strong boundedness} is introduced, which is a syntactic condition and
efficiently verifiable. 

\begin{definition}[Strong
	Boundedness~\cite{BuilAranda20131}]\label{def:strongboundedness}
	Let $P$ be a graph pattern. Then the set of strongly bounded variables in $P$,
	denoted by $SB(P)$, is recursively defined as follows.

	\begin{itemize}
		\item if $P =t$, where $t$ is a triple pattern, then $SB(P) = vars(t)$;
		\item if $P = (P_1 \ AND \ P_2)$, then $SB(P) = SB(P_1) \cup SB(P_2)$ 
		\item if $P = (P_1  \ UNION \ P_2)$, then $SB(P) = SB(P_1) \cap SB(P_2)$ 
		\item if $P = (P_1 \ OPT \ P_2)$, then $SB(P) = SB(P_1)$ 
		\item if $P = (GRAPH \ u \ P_1)$, with $u \in U\cup V$, 
			then\\
			\begin{align*}
				SB(P) = 
				\begin{cases}
					SB(P_1) & \mbox{$u \in U$} \\
					SB(P_1) \cup \{u\} &\mbox{$u \in V$} 
				\end{cases}
			\end{align*}

		\item if $P = (SERVICE \ u \ P_1)$, with $u \in U \cup V$, then $SB(P) = \emptyset$.
	\end{itemize}
\end{definition}

It is a simple observation that this recursive definition collects a set of
variables that are guaranteed to be bounded in $P$. The following proposition
documents this observation.

\begin{proposition}[\cite{BuilAranda20131}]\label{sbinb}
	For every graph pattern $P$ and a variable $x \in var(P)$, if $x \in SB(P)$,
	then $x$ is bounded in $P$.
\end{proposition}
The proof is a very straight forward induction and can be found
in~\cite[Appendix A]{BuilAranda20131}.

We notice that this proposition is not an if and only if statement and hence we may be
able to provide a pattern $P$ where variables $x \in vars(P)$ exist, which are
bounded but not
strongly bounded. Furthermore another example is provided which makes things more
complicated. In Example~\ref{ex:boundedbad} we provide a graph pattern $P$ where a variable in the
destination of a SERVICE-operator occurs which is neither bounded nor strongly
bounded. We will then provide a plan on how to evaluate $P$ and thus show that
the definition of neither boundedness nor strongly boundedness is sufficient for
practical usage. 

\begin{example}\label{ex:boundedbad}
	Consider the following graph pattern:
	\begin{align*}
		P_1 = [(x,service\_description,z) \UNION ((x,service\_address, y) \AND
			\\\
		(\mbox{SERVICE } y \ (x_n,email,x_e)))]
	\end{align*}
	The variables $x$ and $z$ store the name of a SPARQL endpoint and a description of its
	functionalities through the $service\_description$ triple. The variables $x$ and $y$ store the
	name of a SPARQL endpoint and the URI where it is located through the
	service\_address triple. The problem is, that variable $y$ is neither
	bounded nor
	strongly bounded in $P_1$. However we can still easily evaluate the pattern by
	assuming a dataset $DS$ and an RDF graph $G$ in $DS$:\\
	Compute $\ll (x, service\_description,z)\rr^{DS}_G$, then compute \\
	$\ll (x, service\_addres,y) \rr^{DS}_G$ and finally for every
	$\mu \in \ll (x, service\_addres,y) \rr^{DS}_G$, compute $\ll (\mbox{SERVICE }
	a \ (x_n,email,x_e))\rr^{DS}_G$ with $a = \mu(y)$. We can easily see that $y$ is
	bounded and strongly bounded in the subpattern \\ $((x,service\_address, y) \AND
	(\mbox{SERVICE } y \ (x_n,email,x_e)))$ of $P_1$ and thus the evaluation is possible.
\end{example}

\noindent To describe a condition that ensures all SPARQL queries containing the SERVICE
operator can be evaluated in practice, the definition of Service-Boundedness is
introduced. The definition of service-boundedness uses a parse tree to make sure
that our evaluation takes bounded subpatterns into account.

\begin{example}\label{ex:parsetree}[\cite{BuilAranda20131}]
	Parse tree $\T(Q)$ for the graph pattern \\ $Q = ((y,a,z) \UNION ((x,b,c) \AND (\mbox{SERVICE } x \ (y,a,z))))$.

	\begin{tikzpicture}[sibling distance=15em,
			every node/.style = {shape=rectangle, rounded corners,
	draw, align=center, top color=white}]]
	\node {$u_1$: $((y,a,z) \UNION ((x,b,c) \AND (\mbox{SERVICE } x \ (y,a,z))))$}
	child { node {$u_2$: $(y,a,z)$} }
	child { node {$u_3$: $((x,b,c) \AND (\mbox{SERVICE } x \ (y,a,z)))$}
		child { node {$u_4$: $(x,b,c)$}}
		child { node {$u_5$: $(\mbox{SERVICE } x \ (y,a,z))$}
	child	{ node {$u_6$: $(y,a,z)$}} }};
\end{tikzpicture}

\end{example}

\begin{definition}[Parse Tree, \cite{BuilAranda20131}]
	A parse tree of a graph pattern $P$, $\T(P)$ is a tree where each
	node is a sub-pattern of $P$. Each node has an identifier.
	In the parse tree the child relation is used to store the structure of the
	sub-patterns of the graph pattern. The root of the parse tree contains the pattern $P$. Then, 
	in the child(ren) of $P$ the pattern is split up into the respective sub-pattern(s). This is done recursively 	until a node contains only a triple. 
	%Assume we have a SPARQL pattern $P$. 
	%	We will construct the parse tree $(V,E,r,\lambda)$, where $V$ is the set of vertices, 
	%	$E$ the set of edges, $r$ the root and $\lambda$ a labelling function as follows:


\end{definition}
In Example~\ref{ex:parsetree} a parse tree can be found

Using the definition of a Parse Tree, Service-Boundedness can be defined.

\begin{definition}[\cite{BuilAranda20131}]
	A graph pattern $P$ is service-bounded if for every node $u$ of $\T(P)$ with
	label $(SERVICE \ x\  P_1)$, it holds that
	\begin{enumerate}
		\item there exists a node $v$ of $\T(P)$ with label $P_2$ such that $v$
			is an ancestor of $u$ in $\T(P)$ and $x$ is bounded in $P_2$,
		\item $P_1$ is service-bounded.
	\end{enumerate}
\end{definition}

Corresponding to this definition we will introduce the \textbf{SERVICE BOUND}
problem:
\begin{framed}\noindent \textbf{SERVICE BOUND}\\
	\textbf{INPUT:} A graph pattern $P$.\\
	\textbf{QUESTION:} Is $P$ service-bounded?
\end{framed}

Intuitively one can already imagine that deciding the \textbf{SERVICE BOUND}
problem is undecidable because it uses boundedness in it.

\begin{theorem}[\cite{BuilAranda20131}]
	\textbf{SERVICE BOUND} is undecidable.
\end{theorem}
\begin{proof}
	By providing a reduction from the \textbf{SPARQL SAT} problem to the  
	\textbf{SERVICE BOUND} problem we will be able to show undecidability.
	Let $P$ be a graph pattern, i.e., an arbitrary instance of \textbf{SPARQL SAT} and
	$x,y,z,x',y',z'$ variables not mentioned in $P$.
	Also assume that $P$ does not mention the operator SERVICE which is not required
	to make the SPARQL satisfiability problem undecidable.
	Then define the graph pattern $Q$, i.e., the instance of \textbf{SERVICE BOUND} as: 
	\begin{align*}
		Q = (((x,y,z) \UNION  P) \AND (\mbox{SERVICE } x \ (x', y', z'))).
	\end{align*}

	\noindent It remains to show that $Q$ is service-bounded if and only if $P$ is not
	satisfiable.

	\bigskip\noindent
	$(\Leftarrow)$ \quad If $P$ is not satisfiable, then $Q$ is equivalent to the
	pattern: 
	\begin{align*}
		Q' = ((x,y,z)) \AND (\mbox{SERVICE} \ x \ (x', y', z')).
	\end{align*} 
	But $Q'$ is service bounded because $x$ is bounded in $Q'$: Assume an
	arbitrary dataset $DS$, let $G$ be in $DS$. Then 
	for any $\mu \in \ll Q' \rr^{DS}_G$, $x \in dom(\mu)$ and for sure $\mu(x) \in
	(dom(DS) \cup names(DS) \cup dom(P))$ by semantics of SPARQL and the fact that
	$x$ occurs in the triple pattern $(x,y,z)$.


	\bigskip\noindent
	$(\Rightarrow)$\quad Assume that $P$ is satisfiable. Then we know that variable
	$x \not\in vars(P)$ and thus we have that $x$ is not a bounded variable in $Q$,
	and thus because $x$ occurs in a service operator, $Q$ is not service
	bounded.
\end{proof}

As the problem of undecidability prevails with the definition of
service-boundedness, we can again use the syntactic condition, i.e., strong boundedness to define
service-safeness which is the same as service-boundedness but uses strong
boundedness instead of boundedness. We know from
Definition~\ref{def:strongboundedness} that evaluating if a variable is strongly
bounded in a graph pattern can be done efficiently.

\begin{definition}[Service Safeness~\cite{BuilAranda20131}]
	A graph pattern $P$ is service-safe if for every node $u$ of $\T(P)$ with
	label $(SERVICE \ x \ P_1)$ it holds that:
	\begin{enumerate}
		\item there exists a node $v$ of $\T(P)$ with label $P_2$ such that $v$
			is an ancestor of $u$ in $\T(P)$ and $x \in SB(P_2)$.
		\item $P_1$ is service safe.
	\end{enumerate}
\end{definition}

As corollary to Proposition 1, the following proposition is obtained:
\begin{proposition}[\cite{BuilAranda20131}]
	If a graph pattern $P$ is service-safe, then $P$ is service-bounded.
\end{proposition}

\section{Boundedness and strong boundedness}
An interesting question is stirred up through introducing the notions of
boundedness and strong boundedness: What is the difference between the two
notions? 
Because boundedness is not equivalent to strong boundedness, there
must be patterns which are bounded but not strongly bounded. The following problem could arise:
Assume a pattern is bounded but not strongly bounded. Then it is feasible to be evaluated, 
but the proposed algorithm returns that it is not.

\begin{example}\label{bbutnotsbound}
	\begin{align*}
		P = ((x,a, b) \OPT (x, a, y)).
	\end{align*}
	We can see that $y$ is bounded in $P$ because for every RDF graph $G$ in $DS$ and
	every $\mu \in \ll P \rr^{DS}_G$ we have that $y \in dom(\mu)$ and
	$\mu(y) \in (dom(DS))$: Assume we have a $\mu \in \ll P \rr^{DS}_G$. Then $x
	\in dom(\mu)$ by semantics of OPT. Assume w.l.o.g. $\mu(x) = c$. Then
	$(c,a, b) \in G$. But then the mapping $\{ x \mapsto c , y\mapsto b \}
	\in \ll P \rr^{DS}_G$.
	The set of strongly bounded variables, i.e., $SB(P) = \{ x \}$ by
	Definition~\ref{def:strongboundedness} contains only $x$.
\end{example}
Example~\ref{bbutnotsbound} shows that there are patterns which are bounded but not strongly bounded. 
One could easily resolve the problem of Example~\ref{bbutnotsbound} and make $y$
strongly bounded again: 
It is easy to see that $((x,a, b) \OPT (x, a, y)) \equiv ((x,a, b) \AND (x, a,
y))$ holds. Thus we replace the graph pattern $P$ with the graph pattern $Q = ((x,a, b) \AND (x, a, y))$. 
While preserving the meaning of $P$, $Q$ also makes sure that $y$ is now strongly bound
in $Q$, i.e., $SB(Q) = \{ x, y\}$. 

Example~\ref{bbutnotsbound} also elicits a new question: Could deciding boundedness be
feasible in fragments of SPARQL where the problem \textbf{EQUIVALENCE} is
decidable? An idea for a decision procedure of \textbf{BOUND IN PATTERN} is illustrated
in Algorithm~\ref{bsbalgorithm}. This procedure would not contradict the results
of Theorem~\ref{boundundec} because in Theorem~\ref{boundundec} general SPARQL
is considered and it is a well known result that \textbf{EQUIVALENCE} is
undecidable in general SPARQL.

\begin{algorithm}
	\caption{BSB}\label{bsbalgorithm}
	INPUT: pattern $P$ and $x \in \mathit{vars}(P)$\\
	If $x \in SB(P)$ return true;\\
	For all subpatterns $(P_1 \OPT P_2)$ of $P$:\\
	If $(P_1 \OPT P_2) \equiv (P_1 \AND P_2)$ holds, replace $(P_1 \OPT
	P_2)$ with $(P_1 \AND P_2)$ in $P$.\\
	Call the resulting graph pattern
	$Q$.\\
	If $(x \in \mathit{SB}(Q))$ return true;\\
	else return false;
\end{algorithm}

%\begin{framed}\noindent \textbf{EQUIVOPT}\\
%	\textbf{INPUT:} A graph pattern $P$ and a variable $x \in \mathit{vars}(P)$
%	which is not strongly bound in $P$.\\
%	\textbf{QUESTION:} Is there a graph pattern $Q$, where $P \equiv Q$ 
%	and $Q$ replaced all subpatterns $(P_1 \OPT P_2)$ of $P$
%	with $(P_1 \AND P_2)$ in $Q$ where $(P_1 \OPT P_2) \equiv (P_1 \AND P_2)$ holds? 
%\end{framed}

%\begin{theorem}\label{ezprop}
%	Let $P$ be a graph pattern in a SPARQL fragment $\mathbf{P}$ and $x \in
%	\mathit{vars}(P)$. The problem of deciding whether $x$ is bounded in $P$ can
%	be reduced to the problem whether $((P_1 \OPT P_2) \equiv (P_1 \AND P_2))$
%	for all subpatterns $(P_1 \OPT P_2)$ of $P$. 
%\end{theorem}
%\begin{proof}
%	Consider the following decision procedure:
%	\begin{algorithm}
%		\caption{BSB}
%		INPUT: pattern $P$ and $x \in \mathit{vars}(P)$\\
%		If $x \in SB(P)$ return true;\\
%		For all subpatterns $(P_1 \OPT P_2)$ of $P$:\\
%		If $(P_1 \OPT P_2) \equiv (P_1 \AND P_2)$ holds, replace $(P_1 \OPT
%		P_2)$ with $(P_1 \AND P_2)$ in $P$.\\
%		Call the resulting graph pattern
%		$Q$.\\
%		If $(x \in \mathit{SB}(Q))$ return true;\\
%		else return false;
%	\end{algorithm}
%\end{proof}
%
%Notice that the decision procedure $BSB$ uses a subprocedure to decide $(P_1
%\OPT P_2) \equiv (P_1 \AND P_2)$.
%Let $(P,x)$ be an arbitrary instance of \textbf{BOUNDED IN PATTERN}.
%It remains to show that $\textit{BSB}(P,x)$ returns true iff $x$ is bounded in
%$P$. 
%If $BSB$ returns true from 1. We know from Proposition~\ref{sbinb} that $x$ is
%bounded in $P$ and we are done.
%
%Assume now that $\textit{BSB}(P,x)$ returned true from line 6.
%By construction of BSB and the fact that $x \notin SB(P)$ we know that there was
%a subpattern $(P_1 \OPT P_2)$ in $Q$ where $(P_1 \OPT P_2) \equiv (P_1 \AND
%P_2)$ was true and $x \in \vars(P_2)$. It remains to show that $x$ is bounded in
%$P$.
%Let $\mathit{DS}$ be an arbitrary dataset and $G$ a graph in $\mathit{DS}$.
%Case distinction:
%\begin{enumerate}
%	\item $\ll P \rr^{\DS}_{G}$ is empty: then $x$ is trivially bounded in $P$.
%	\item Let $\mu \in \ll P \rr^{\DS}_G$:
%		We can use the fact that $x \in SB(Q)$ and $x \notin SB(P)$. Thus there
%		must have been at least one subpattern $(P_1 \OPT P_2)$ where $(P_1 \OPT
%		P_2) \equiv (P_1 \AND P_2)$ was true. By the fact that $x \in SB(Q)$ and
%		Proposition~\ref{sbinb} we know that $x$ is bounded in $Q$.
%		Because we only substitute the subpatterns of $P$ with equivalent
%		subpatterns,i.e. $(P_1 \OPT P_2)$ with $(P_1 \AND P_2)$, to receive $Q$
%		we know that $Q \equiv P$. Thus $x \in dom(\mu)$ and $\mu(x) \in
%		(dom(DS) \cup names(DS) \cup dom(P))$ holds and thus $x$ is bounded in
%		$P$.
%\end{enumerate}
%Assume that $x \notin SB(Q)$(line 7).
%We need to prove that $x$ is not bounded in $P$ and because again $Q \equiv
%P$, it suffices to show that $x$ is not bounded in $Q$. \\
%%Proof by contradiction:\\
%%Assume $x$ was bounded in $Q$. By assumption we know that for every dataset
%%$DS$, every graph $G$ in $DS$ and every $\mu \in \ll Q\rr^{DS}_G$:
%%$x \in dom(\mu)$ and $\mu(x) \in(dom(DS) \cup names(DS) \cup dom(Q))$.
%%Because $x \notin SB(Q)$ we know that there must have been a subpattern in $Q$,
%%call it $Q'$, where $x \in vars (Q')$, but $x \notin SB(Q')$ because $x \in
%%\vars(Q)$. 
%We also know that there are no subpatterns $(P_1 \OPT P_2)$ in $Q$
%where $(P_1 \OPT P_2) \equiv (P_1 \AND P_2)$ holds.
%We will show the following statement by induction on the structure of $Q$: 
%If $x \notin SB(Q)$ then $x$ is not bounded in $Q$. Obviously $x$ is not
%bounded in $Q$ if it holds because we assumed $x \notin SB(Q)$ and $x \in
%\vars(Q)$.
%\begin{itemize}
%	\item BC: $Q = t$, we have a case distinction: 
%		\begin{enumerate}
%			\item If $x \in \vars(t)$ the implication holds as $SB(Q) = \vars(t)$ and
%
%			\item$x \notin vars(t)$:
%				If $x \notin vars(t)$ the implication holds:
%				Let $\DS$ be a dataset and $G$ be a graph in $DS$ where $\ll t
%				\rr^{\DS}_G \neq \emptyset$ holds. Let $\mu \in \ll t
%				\rr^{\DS}_G \neq \emptyset$. By SPARQL semantics $x \notin dom(\mu)$.
%		\end{enumerate}
%
%	\item IS: $Q = (Q_1 \AND Q_2)$. 
%		Again case distinction:	
%		\begin{enumerate}
%			\item If $x \in vars(Q)$:
%				By definition of $SB$, we have that $SB(Q) = SB(Q_1) \cup SB(Q_2)$.
%				By induction hypothesis we have that if $x \notin SB(Q_i)$,
%				$i = 1,2$ then $x$ is not bounded in $Q_i$.
%				Assume now that $x \notin SB(Q)$. We know by
%				semantics of $SB$ that thus $x \notin SB(Q_i)$ for $i=1,2$.
%				Thus $x$ is not bounded in $Q_i$ for $i=1,2$. Thus there must
%				have been datasets $DS_i$ and a graphs $G_i$ such that 
%				there is a $\mu_i \in \ll Q_i \rr^{DS_i}_{G_i}$:
%				where either $x \notin dom(\mu_i)$ or $\mu_i(x) \notin (dom(DS_i) \cup
%				names(DS_i) \cup dom(Q_i))$, $i=1,2$.
%				Case distinction:
%				\begin{enumerate} 
%					\item If $x \notin dom(\mu_i)$ for $i=1,2$.	
%
%				\end{enumerate}
%			\item $x \notin vars(Q)$: 
%				Let $\DS$ be a dataset and $G$ be a graph in $DS$ where $\ll Q
%				\rr^{\DS}_G \neq \emptyset$ holds. Let $\mu \in \ll Q
%				\rr^{\DS}_G \neq \emptyset$. By SPARQL semantics $x \notin dom(\mu)$.
%		\end{enumerate}
%	\item IS: $Q = (Q_1 \UNION Q_2)$.
%		Again case distinction:	
%		\begin{enumerate}
%			\item If $x \in vars(Q)$:
%				By definition of $SB$, we have that $SB(Q) = SB(Q_1) \cap SB(Q_2)$.
%				By induction hypothesis we have that if $x \notin SB(Q_i)$,
%				$i = 1,2$ then $x$ is not bounded in $Q_i$.
%				Assume now that $x \notin SB(Q)$. We know by
%				semantics of $SB$ that thus $x \notin SB(Q_i)$ for $i=1$ or $i=2$ or both.
%				Thus $x$ is not bounded in $Q_i$ for $i=1$, $i=2$ or both. Thus there must
%				have been datasets $DS_i$ and a graphs $G_i$ such that 
%				there is a $\mu_i \in \ll Q_i \rr^{DS_i}_{G_i}$:
%				where either $x \notin dom(\mu_i)$ or $\mu_i(x) \notin (dom(DS_i) \cup
%				names(DS_i) \cup dom(Q_i))$, $i=1$ or $i=2$ or both.
%				W.l.o.g. assume $x$ is not bounded in $Q_1$. Then take
%				the counterexample,i.e. $DS_1$ and $G_1$ and consider $\ll Q
%				\rr^{DS_1}_{G_1}$. Because $x$ was not bounded in $\ll Q_1
%				\rr^{DS_1}_{G_1}$ we had a mapping $\mu_1$ where the definition
%				of boundedness was not fulfilled. By semantics of
%				$\mbox{UNION}$ we have that $\mu_1 \in \ll Q \rr^{DS_1}_{G_1}$. Thus $x$ is
%				not bounded in $Q$.
%			\item $x \notin vars(Q)$: 
%				Let $\DS$ be a dataset and $G$ be a graph in $DS$ where $\ll Q
%				\rr^{\DS}_G \neq \emptyset$ holds. Let $\mu \in \ll Q
%				\rr^{\DS}_G \neq \emptyset$. By SPARQL semantics $x \notin dom(\mu)$.
%		\end{enumerate}
%	\item IS: $Q = (Q_1 \OPT Q_2)$.
%		Again case distinction:	
%		\begin{enumerate}
%			\item If $x \in vars(Q)$:
%				By definition of $SB$, we have that $SB(Q) = SB(Q_1)$.
%				By induction hypothesis we have that if $x \notin SB(Q_i)$,
%				$i = 1,2$ then $x$ is not bounded in $Q_i$.
%				Assume now that $x \notin SB(Q)$. We know by
%				semantics of $SB$ that thus $x \notin SB(Q_1)$.
%				Thus $x$ is not bounded in $Q_1$. Thus there must
%				have been a dataset $DS_1$ and a graph $G_1$ such that 
%				there is a $\mu_1 \in \ll Q_1 \rr^{DS_1}_{G_1}$:
%				where either $x \notin dom(\mu_1)$ or $\mu_1(x) \notin (dom(DS_1) \cup
%				names(DS_1) \cup dom(Q_1))$.
%				Take the counterexample, i.e. $DS_1$ and $G_1$ and consider $\ll Q
%				\rr^{DS_1}_{G_1}$. Because $x$ was not bounded in $\ll Q_1
%				\rr^{DS_1}_{G_1}$ we had a mapping $\mu_1$ where the definition
%				of boundedness was not fulfilled. 
%				Case distinction:
%				\begin{enumerate}
%					\item If $\mu_1(x) \notin (dom(DS_1) \cup names(DS_1) \cup dom(Q_1))$
%						caused the problem then by semantics of
%						$\mbox{OPT}$ we have that $\mu_1 \cup A \in \ll Q
%						\rr^{DS_1}_{G_1}$. $A$ cannot change the value $\mu_1(x)$ had. Thus $x$ is
%						not bounded in $Q$.
%					\item If $x \notin dom(\mu_1)$
%							
%				\end{enumerate}
%			\item $x \notin vars(Q)$: 
%				Let $\DS$ be a dataset and $G$ be a graph in $DS$ where $\ll Q
%				\rr^{\DS}_G \neq \emptyset$ holds. Let $\mu \in \ll Q
%				\rr^{\DS}_G \neq \emptyset$. By SPARQL semantics $x \notin dom(\mu)$.
%		\end{enumerate}
%\end{itemize}


%\begin{theorem}\label{ezprop}
%	The problem of verifying, given a graph pattern $P$ in a SPARQL fragment
%	$\mathbf{P}$ and a variable $x \in
%	var(P)$, whether $x$ is bounded in $P$ is as hard as deciding $(P_1 \AND 
%	P_2)
%	\equiv (P_1 \OPT P_2)$ %for every subpattern $(P_1 \OPT  P_2)$ of $P$% 
%	in the SPARQL fragment $\mathbf{P}$.
%\end{theorem}
%\begin{proof}
%	It is very easy to decide strong boundedness and Proposition~\ref{sbinb}
%	shows that if a variable $x$ is in $SB(P)$ then $x$ is also bounded in $P$.
%	We thus modify our pattern $P$ in such a way, that $SB(P)$ collects all the
%	variables which are bounded, even those which are only bounded but not strongly
%	bounded.
%	Assume $x$ is bounded in $P$ and $x \notin SB(P)$.
%	There must thus be a subpattern $Q$ where $x \in vars(Q)$ but $x \notin
%	SB(Q)$. We will go through every step of the recursive approach of
%	Definition~\ref{def:strongboundedness} to find the subpattern where $x \in
%	vars(Q)$, $x \notin	SB(Q)$ but $x$ is bounded in $Q$.
%	\begin{itemize}
%		\item If $Q$ is a triple $t$ then $SB(Q) = var(t)$ and by assumption
%			that $x \in vars(Q)$ we get that $x \in SB(Q)$. 
%
%		\item If $Q$ is $(P_1 \AND P_2)$ then $SB(Q) = SB(P_1) \cup SB(P_2)$. By
%			assumption we know that $x \in vars(Q)$. By semantics of $\cup$ we
%			get that $x \in SB(Q)$. 
%			
%		\item If Q is $(P_1 \UNION  P_2)$ then $SB(P) = SB(P_1) \cap SB(P_2)$.
%			By assumption we have that $x \in vars(P_1 \UNION P_2)$.
%			Assume $x \notin SB(Q)$. 
%			We have a case distinction:
%			\begin{enumerate}
%				\item $x \in SB(P_1)$ and $x \in SB(P_2)$.
%				 We instantly see by semantics of $\cap$ that $x \in SB(Q)$. 
%
%				\item Assume w.l.o.g. that $x \in SB(P_1)$ but $x \notin SB(P_2)$. 
%					This contradicts the fact that $x$ is bounded in $P$:
%					We could construct a dataset $DS$ with a graph $G$ in it s.t. there
%					is a solution $\mu \in \ll P_2
%					\rr^{DS}_G$. By assumption we would then have that $x \notin dom(\mu)$.
%			\end{enumerate}
%
%		\item If $Q$ is $(P_1 \OPT  P_2)$ we have by definition of SB that $SB(P) = SB(P_1)$
%			So a variable $x$ gets lost if it is in $SB(P_2)$, but not in
%			$SB(P_1)$. When we have the case that $x$ is bound in $Q$ but not
%			strongly bounded in $Q$ we must thus have $(P_1 \AND  P_2) \equiv
%			(P_1 \OPT  P_2)$ for every Dataset, Graph and $\mu \in \ll Q
%			\rr^{DS}_G$.
%
%		\item If $Q$ is $(\mbox{GRAPH } u \ P_1)$. We have a case distinction:
%			\begin{enumerate}
%				\item If $u \in \V$ then it is obvious that if $x \in vars(Q)$ then 
%					$x \in SB(Q)$ holds because $vars(Q) = SB(Q)$.
%				\item If $u \in \U$ no variables in $vars(Q)$ are bounded: 
%					Assume a dataset $DS$ with a graph $G$ in it. Assume
%					further that $u \notin names(DS)$. Then $\ll Q \rr^{DS}_{G}
%					= \emptyset$. Obviously if $x \in vars(Q)$, $x$ is not bound
%					in $Q$.
%			\end{enumerate}
%
%		\item If $Q = (\mbox{SERVICE }  u \ P_1)$ We have a case distinction:
%			\begin{enumerate}
%				\item If $u \in \V$. We again have two cases. 
%					\begin{enumerate}
%						\item $u = x$. 
%							Assume a dataset $DS$ with a graph $G$ in it.	
%							Assume $\mu \in \ll Q\rr^{DS}_G$ arbitrary.
%							Then $x$ is not bound in $P$ because $\mu(x) \notin (dom(DS) \cup
%							names(DS) \cup dom(P))$ could hold because there
%							might be an URI referring to an endpoint which is
%							not in $(dom(DS) \cup names(DS) \cup dom(Q))$. This
%							obervation implies that $x$ is not bound in $Q$.
%						\item $x \in SB(P_1)$ But then we could possibly have that
%							$\mu(x) \notin (dom(DS) \cup names(DS) \cup dom(P))$
%							because we evluate $P_1$ not over $DS$ but a
%							possibly different dataset.
%					\end{enumerate}
%
%				\item If $u \in \U$ no variables in $vars(Q)$ are bounded because we
%					don't know whether $u \in dom(ep)$, SERVICE returns the empty
%					mapping $\mu_\emptyset$ in this case by SPARQL semantics. Thus if $x \in vars(Q)$,
%					$x \notin SB(Q)$ but $x$ is not bounded in $P$ either
%					because $x \notin \mu_\emptyset$.
%			\end{enumerate}
%	\end{itemize}
%
%	To modify our pattern $P$ in such a way, that $SB(P)$ collects all the
%	variables which are bound we would need to replace all occurrences of  $(P_1
%	\OPT P_2)$ with $(P_1 \AND P_2$ iff $(P_1 \AND  P_2) \equiv	(P_1 \OPT  P_2)$
%	holds.
%\end{proof}

%Note that deciding $(P_1 \AND P_2) \equiv
%(P_1 \OPT  P_2)$ is undecidable for general SPARQL patterns and is the reason
%why deciding \textbf{BOUND IN PATTERN} is undecidable. Furthermore it is the key to explain
%the difference between strongly boundedness and boundedness.

%\begin{corollary}
%	The problem of verifying, given a graph pattern $P$ in a fragment
%	$\mathbf{P}$ , whether $P$ is
%	service-bound is as hard as deciding $(P_1 \AND \ P_2) \equiv (P_1 \OPT
%	P_2)$ for every subpattern $(P_1 \OPT P_2)$ of $P$ in the fragment
%	$\mathbf{P}$.
%\end{corollary}
%
%\begin{proof}
%	It is obvious that checking the boundedness in the ancestor of a service
%	node causes the undecidability. Thus we use the result in Theorem~\ref{ezprop} to
%	show that deciding whether a variable $x$ is bound in this ancestor is as
%	hard as deciding $(P_1 \AND P_2)
%	\equiv (P_1 \OPT  P_2)$ for every subpattern $(P_1 \OPT P_2)$ of $P$, where
%	$P$ is the label of the ancestor node.
%\end{proof}
%\begin{corollary}
%	The problem of verifying, given a well-designed graph pattern $P$, whether $P$ is
%	service-bound is NP-hard.
%\end{corollary}
%\begin{proof}
%	Hier sind auch meine results wichtig weil $P_1$ und $P_2$ wiederum SERVICE
%	und GRAPH beinhalten koennten. Sonst ist natuerlich
%	$\cite{pichler2014containment}$ zu zitieren.
%\end{proof}




\chapter{Complexity of well-designed SPARQL with GRAPH and SERVICE}
We need a different but equivalent definitions of WDPTs in this section. The
main structural difference to Definition~\ref{def:pt} is that the nodes in the
tree are labelled with a set
of relational atoms over a schema instead of a set of triples. Also the semantic
evaluation of WDPTs uses the notion of conjunctive queries.

\begin{definition}[WDPTs~\cite{barcelo2015efficient}]\label{def:wdpt}
	A WDPT over a relational schema $\sigma$ is a tuple $(T, \lambda, \overline{x})$
	such that the following holds:
	\begin{enumerate}
		\item $T$ is tree rooted in a distinguished node $r$, the root and $\lambda$
			maps each node $t$ in $T$ to a set of relational atoms over $\sigma$.
		\item For every variable $y$ that appears in $T$, the set of nodes of $T$ where
			$y$ is mentioned is connected.
		\item We have that $\overline{x}$ is a tuple of distinct variables occurring in
			$T$. They are the free variables of the WDPT.
	\end{enumerate}
\end{definition}

\begin{definition}
	A WDPT $(T,\lambda, \overline{x})$ is called projection-free if $\overline{x}$
	contains all variables mentioned in $T$.
\end{definition}

\begin{definition}\label{wdptq}
	Assume $p = (T,\lambda,\overline{x})$ is a WDPT over $\sigma$. We write $r$ to
	denote the root of $T$. Given a subtree $T'$ of $T$ rooted in $r$ we define
	$q_{T'}$ to be the CQ $Y \leftarrow R_1(\overline{v_1}), \dots,
	R_m(\overline{v}_m)$, where the $R_i(\overline{v_i})$'s are the reational atoms
	that label the nodes of $T'$, i.e., 
	\begin{align*}
		\{ R_1(\overline{v}_1), \dots, R_m(\overline{v_m}) \} = \bigcup_{t \in T'} \lambda(t) 
	\end{align*} and $\overline{y}$ are all the variables that are mentioned in
	$T'$. 
\end{definition}

The main idea of this equivalent but different semantics of a wdPT is to look at each subtree $T'$ of
$T$ rooted in $r$. As mentioned above, each of them describes a pattern, i.e.,
the conjunctive query CQ $q_T'$. A mapping $h$ satisfies $(T,\lambda)$ over
a database $D$ if $h$ satisfies the pattern defined by a subtree $T'$ and there
is no subtree $T''$ which is bigger than $T'$ and $h$ can be extended to satisfy
$T''$.
\begin{definition}[Semantics of wdPTs~\cite{barcelo2015efficient}]
	Let $p=(T,\lambda,\overline{x})$ be a WDPT and $D$ a database over $\sigma$.

	\begin{enumerate}
		\item A homomorphism from $p$ to $D$ is a partial mapping $h: X \rightarrow U$,
			where $X$ is an infinite set of a variables and $U$ an infinite set of
			constants, for which it is the case that there is a subtree $T'$ of $T$ rooted
			in $r$ such that $h \in q_{T'}(D)$.
		\item The homomorphism $h$ is maximal if there is no homomorphism $h'$ from $p$
			to $D$ such that $h \sqsubset h'$.
	\end{enumerate}
	The evaluation of WDPT $p = (T,\lambda,\overline{x})$ over $D$ denoted $p(D)$,
	corresponds to all mappings of the form $h_{\overline{x}}$, such that $h$ is a
	maximal homomorphism from $p$ to $D$.
\end{definition}

It is important to notice that WDPTs properly extend CQs.
Given a CQ $q(x)$ of the form $X \leftarrow R_1(v_1),\dots,R_m(v_m)$
it is easy to see that $q(x)$ is equivalent to the WDPT $p = (T,\lambda,x)$,
where $T$ consists of a single node $r$ and $\lambda(r) =
\{R_1(v_1),\dots,R_m(v_m)\}$. In other words, $q(D) = p(D)$ for each database
$D$. Further on we will not distinguish between a CQ and the single node WDPT
that represents it. WDPTs can on the other hand represent interesting properties
that cannot be expressed as CQs, which we will discuss in the following example:
\begin{example}
	$Q =  \bigg(((x,recorded\_by,y) \ AND \ (x,published, ``after\_2010'')) \ \\ OPT \ (x,
	NME\_rating,z)\bigg) \ OPT \ (y, formed\_in, z') $

	\noindent Consider now the following RDF database $D$ consisting of the triples:
	\begin{verbatim}
	(``title_1'' recorded_by, ``studio_1''),
	(``title_1'' published, ``after_2010''),
	(``title_2'' recorded_by, ``studio_1''),
	(``title_2'' published, ``after_2010''),
	(``title_2'' NME_rating, ``2'')
	(``studio_1'' formed_in, ``miami''),
	\end{verbatim}
	Evaluating the query $Q$ over $D$ results in two mappings $\mu_1$ and $\mu_2$:
	$\mu_1 = \{ x \rightarrow ``title\_1'', y \rightarrow ``studio_1'',z'
	\rightarrow ``miami'' \}$ 
	$\mu_2 = \{ x \rightarrow ``title\_2'', y \rightarrow ``studio_1'',z\rightarrow
	``2'', z' \rightarrow ``miami'' \}$ 
\end{example}

To capture the UNION operator the definition of well designed pattern trees
need to be modified.  
\begin{definition}[Unions of wdPTs or well-designed pattern Forests (wdPF)]
	A Union of WDPTs is an expression $\phi$ of the form $\bigcup_{1\leq i \leq n} p_i$, 
	where each $p_i$ is a WDPT over $\sigma$.
	We denote $\varphi(D)$ as the evaluation of $\phi$ over database $D$.
	It corresponds to the set $\bigcup_{1\leq i \leq n}p_i(D)$.
	Unions of WDPTs are also called Well-designed pattern
	forests(WDPF).
\end{definition}



First we are going to show an easy example of how to translate a graph pattern
only using AND and triple patterns to a conjunctive query. Then we propose a
polynomial time translation from a graph pattern $P \in P_{wdgs}$ to $Q \in P_{wd}$. For this
translation we need to construct a special database and a wdpt depending on the
original query. After we established the translation we prove the equivalence of $P$ and $Q$ in
Theorem~\ref{biglemma}. The last section deals with the problem
\textbf{EVAL}($P_{wdgs}$.
%, \textbf{CONTAINMENT}($P_{wdgs}$),
%\textbf{EQUIVALENCE}($P_{wdgs}$) and \textbf{SUBSUMPTION}($P_{wdgs}$).

\section{Translations to well designed pattern forests}
It is an easy observation that if we restrict SPARQL to the AND operator, we
simply get conjunctive queries without existentially quantified variables.
We will show this in Example~\ref{exconjq}. 
\begin{example}\label{exconjq}
	Consider the following SPARQL default Graph $G$ in a dataset $DS$ with the query $Q$ in $SPARQL[\land]$
	\begin{align*}
		G &=\{ (a,a,a), (b,c,c), (b,c,a)  \}\\
		Q &= (x,y,a) \ AND \ (z,c,c) \ AND \ (x,c,y)
	\end{align*}
	Let $T$ be a three-ary relation, $D = \{ T \}$ be the following database and
	$CQ$ be the following 
	conjunctive query: 
	\begin{align*}
		T &= \{ (a,a,a), (b,c,c), (b,c,a)\}\\
		CQ &= ans(x,y,z) \leftarrow T(x,y,a), T(z,c,c), T(x,c,y)\\
	\end{align*}
	It is easy to see that the mappings in $CQ(D)$ are the same as in $\ll Q
	\rr_G^{DS}$.
\end{example}
Similar to Example~\ref{exconjq} we are going to transform the dataset and query 
not into conjunctive queries but an extension of them: well designed pattern
trees. We 
will define a polynomial time translation from a graph pattern $P \in P_{wdgs}$
to a well designed pattern tree. This enables us to use well known algorithms 
for which the computational complexity is known. 

\subsection{Creating the database}
We first describe the function $data$ which transforms a dataset into the
database.

Consider the function $data: DS \mapsto D$, where $DS$ is a
dataset and $D$ is a database.
$data$ then is defined as follows:
Let $DS$ be an arbitrary dataset and $u_{DS}$ the URI such that $ep(u_{DS}) = DS$.
\begin{align*}
	DS=\{(def,G),(u_1,G_1),\dots,(u_n,G_n)\}
\end{align*}
The output of $data$ is the database $D = \{ T,LOC\}$ where $T$ is a 5-ary relation containing all the triples of a graph, the corresponding graph URI and the dataset URI. The binary relation $LOC$
captures all graph URIs and their corresponding dataset URI. 
For our dataset this would mean, assuming 
\begin{align*}
	G &= \{(x_1,y_1,z_1), \dots, (x_a,y_a,z_a)\},\\ 
	G_1 &= \{(x_{11}, y_{11},z_{11}), \dots, (x_{1b},y_{1b},z_{1b}) \},\dots,\\ 
	G_n &= \{(x_{n1},y_{n1},z_{n1}),\dots,(x_{nc},y_{nc},z_{nc})\}
\end{align*} that we construct our output database $D$ as follows:
\begin{align*}
	T &= \{ (u_{DS},def,x_1,y_1,z_1), (u_{DS},def,x_a,y_a,z_a), \dots, (u_{DS},u_1,x_{11}, y_{11},z_{11}),
	\dots,\\& (u_{DS},u_1,x_{1b},y_{1b},z_{1b} ), \dots,
(u_{DS},u_n,x_{n1},y_{n1},z_{n1}), \dots (u_{DS},u_n,x_{nc},y_{nc},z_{nc})\}&. \\
	LOC &= \{ (u_{DS},def),(u_{DS},u_1),\dots (u_{DS},u_n) \}&.
\end{align*}

\subsection{Transforming the pattern to a wdpt}
We proceed in defining the function \textit{trans} which will in polynomial time 
transform a graph pattern in $P_{wdgs}$ to a well designed pattern tree with the same meaning without the
SERVICE and GRAPH operators. Lemma~\ref{smalllemma} concludes that the outputpattern and the
inputpattern are equivalent.

\bigskip\noindent
The transformation function $trans: P \times \U\cup\{\V\} \times \U \cup \{def \}\cup\{\V\}
\mapsto Q$ takes  three parameters as input: 
$P$ is a graph pattern in OPT normal form which allows the usage of AND, OPT, GRAPH, SERVICE and
UNION, $\U\cup\V$ is the infinite
set of uris with the infinite set of variables and $\U\cup\{def\}\cup \V$ is the infinite set of
uris containing the $def$ identifier conjoined with the infinite set of
variables. The output $Q$ is a well designed pattern tree. 

\bigskip
\noindent
Assume the input $(P,ds,g)$ and let each of the parameters be
arbitrary. 
\begin{enumerate}
	\item If $P$ is a triple pattern $(u,v,w)$,  
		\begin{align*}
		trans(P,ds,g) = vars(ds,g,u,v,w) \leftarrow T(ds,g,u,v,w).\\	
		\end{align*}

	\item If $P$ is $(P_1  \AND  P_2)$, let
		\begin{align*}
			&trans(P_1,ds,g)	= O_1 \leftarrow q_1,\\
			&trans(P_2,ds,g)	= O_2 \leftarrow q_2\\
			&trans(P,ds,g)		= O_1\cup O_2 \leftarrow q_1, q_2.
		\end{align*}
	\item If $P$ is $(P_1  \OPT  P_2)$, let\\
		\begin{align*}
			&trans(P_1,ds,g) =  (T_1, \lambda_1,x_1) \mbox{ and }\\
			&trans(P_2,ds,g) = (T_2, \lambda_2, x_2).
		\end{align*}
		$trans(P,ds,g) = (T,\lambda,x)$ for which $T = T_1 \cup T_2 \cup (r_1,
		r_2)$ where $r_1,r_2$ are the roots of $T_1,T_2$ respectively,
		$\lambda = \lambda_1 \cup \lambda_2$ and $x = x_1 \cup x_2$.

	\item If $P$ is $(\mbox{GRAPH} \ u \ P_1)$, let\\
		$(T_1,\lambda,x_1) = trans(P_1,ds,u)$.	
		Assuming $r_1$ is the root of $T_1$,
		and $\lambda(r_1) = q_1$ we define \[ \lambda'(x) =\begin{dcases*} 
				q_1, LOC(u,ds),LOC(g,ds)& if $x = r_1$\\
				\lambda(x) & otherwise	\\
			\end{dcases*}
		\] and $trans(P,ds,g) = (T_1,\lambda',x_1)$.

	\item $P$ is of the form $(\mbox{SERVICE} \ u \ P_1)$. 
		Case distinction:
		\begin{enumerate}
			\item If $u \in \U$ and $u \notin dom(ep)$:
				$trans(P,ds,g) =  \{\}\leftarrow$.
			\item Otherwise let $(T_1,\lambda,x_1) = trans(P_1,u,g)$.
				Assuming $r_1$ is the root of $T_1$,
				and $\lambda(r_1) = q_1$ we define \[ \lambda'(x) =\begin{dcases*} 
				q_1, LOC(def,u),LOC(g,ds)& if $x = r_1$\\
				\lambda(x) & otherwise	\\
			\end{dcases*}
		\]  and $trans(P,ds,g) = (T_1,\lambda',x_1)$.
		\end{enumerate}

%	\item If $P_1$ and $P_2$ are graph patterns and $P$ is $(P_1 \UNION  P_2)$,
%		then $P_1$ must correspond to a well formed pattern forest $Q_1 =
%		trans(P_1,ds,g)$ and $P_2$ to a
%		well designed pattern forest $Q_2 = trans(P_2,g,ds)$. 
%		Define $Q = Q_1 \cup Q_2$.  
\end{enumerate}
Observe that the function is well-defined since only patterns $P_{wdgs}$ are
considered: This allows us to argue over conjunctive queries for the case $P$ is $P_1
\AND P_2$ as the AND-operator may not occur in a scope of an OPT-operator in the fragment
$P_{wdgs}$. \\
Towards our goal to show that for all datasets $DS$ identified by URI
$ds$ and graph patterns $P$ the following property holds: 
Let $Q = trans(P,def,ds)$. Then $Q(D) = \ll P
\rr^{DS}_{graph(def,DS)}$, where $D=\bigcup\limits_{c \in dom(ep)} data(ep(c))$, we prove Lemma~\ref{smalllemma} first.

\begin{lemma}\label{smalllemma}
	Let $P$ be a graph pattern in OPT normal form containing the SERVICE and GRAPH operator
	and $D = \bigcup\limits_{c \in dom(ep)} data(ep(c))$.
	\begin{align*}
		Q(D) = \begin{dcases*}
			\ll P \rr^{ep(ds)}_{graph(g,ep(ds))} 
			& if $g \in \U,ds \in \U$\\
			\bigcup\limits_{ds' \in dom(ep)} \{ \mu \cup [ds \mapsto ds'] \mid\\ \mu \in
			\ll P\rr^{ep(ds')}_{graph(g,ep(ds))}, \mu \sim 	[ds\mapsto ds'] \} 
			&if $g \in \U,ds \in \V$\\
			\bigcup\limits_{g' \in names(ep(ds))}\{ \mu \cup [g \mapsto g']
				\mid\\ \mu \in
			\ll P\rr ^{ep(ds)}_{graph(g',ep(ds))} , \mu \sim [g\mapsto g'] \} 
			& if $g \in \V,ds \in \U$ \\
			\bigcup\limits_{ds' \in dom(ep), g' \in names(ep(ds'))} \big\{ \mu \cup
				\{[ds\mapsto ds'],[g \mapsto g']\} \mid\\ \mu \in
				\ll P\rr^{ep(ds')}_{graph(g',ep(ds'))}, 
				\mu \sim
			\{[ds\mapsto ds'], [g \mapsto g']\}\big\} 
			& if $g \in \V,ds \in \V$\\
		\end{dcases*}
	\end{align*}
%	\begin{eumerate}
%		\ite Let $ds \in \U$, $g \in \V \cup \U \cup \{def\}$ and $Q = trans(P,ds,g).$ Then 	
%		\ite if $g \in \U:$
%			begin{align*}
%			\ll P \rr^{ep(ds)}_{graph(g,ep(ds))} = Q(D)  
%			end{align*}

%		\item if $g \in \V$:\\
%		\begin{align*}
%				\bigcup\limits_{g' \in names(ep(ds))}\{ \mu \cup [g \mapsto g'] \mid \mu \in
%					\ll P\rr ^{ep(ds)}_{graph(g',ep(ds))} , \mu \sim
%				[g\mapsto g'] \}\\ = Q(D)
%			\end{align*}
%		\item Let $ds\in \V$, $g\in \V\cup \U\cup \{def\}$ and $Q = trans(P,ds,g).$ Then 	
%		\item if $g \in \U$:\\
%			\begin{align*}
%				\bigcup\limits_{ds' \in dom(ep)} \{ \mu \cup [ds \mapsto ds'] \mid \mu \in
%					\ll P\rr^{ep(ds')}_{graph(g,ep(ds))}, \mu \sim
%				[ds\mapsto ds'] \}\\  = Q(D)
%			\end{align*}
%
%		\item if $g \in \V$: \\
%			\begin{align*}
%				\bigcup\limits_{ds' \in dom(ep), g' \in names(ep(ds'))} \big\{ \mu \cup
%					\{[ds\mapsto ds'],[g \mapsto g']\} \mid\\ \mu \in
%					\ll P\rr^{ep(ds')}_{graph(g',ep(ds'))}, 
%					\mu \sim
%				\{[ds\mapsto ds'], [g \mapsto g']\}\big\} = Q(D)  
%			\end{align*}
%	\end{enumerate}
\end{lemma}


\begin{proof}
	We proceed to prove the statement using structural induction on $P$.
	\noindent 
	\begin{enumerate}
		\item  For the base case we assume that $P=(u,v,w)$ is a triple pattern:\\
			By construction we have that $Q: vars(ds,g,u,v,w) \leftarrow
			T(ds,g,u,v,w)$.\\
			$\subseteq:$\\

			Let $ds,g \in \U$. Let $DS = ep(ds)$ and $G = graph(g,DS)$.
			Let $\mu \in \ll P \rr^{DS}_{G}$ be arbitrary. 
			Then $dom(\mu) = vars(P)$ and $\mu(P) \in G$ by SPARQL semantics. 
			Because we have $\mu(P) \in G$ and our database $D$ contains by
			construction $T(ds,g,\mu(x),\mu(y),\mu(z))$ we have that $\mu \in Q(D)$.

			\bigskip\noindent
			Let $ds \in \V$ and $g \in \U$. 
			Because $ds$ is a variable we have to show that 
			$\bigcup\limits_{ds' \in dom(ep)} \{ \mu \cup [ds \mapsto ds'] \mid \mu \in \ll P
			\rr^{ep(ds')}_{graph(g,ep(ds'))},\mu \sim [ds \mapsto ds']\} \subseteq Q(D)$. 
			Let $ds' \in dom(ep)$ be arbitrary, let $\mu \in \ll P
			\rr^{ep(ds')}_{graph(g,ep(ds'))}$ where $\mu \sim [ds \mapsto ds']$.
			Because we have $\mu(P) \in graph(g,ep(ds'))$ by SPARQL semantics and our database
			$D$ contains by construction\\ $T(ds',g,\mu(x),\mu(y),\mu(z))$ we have that
			$\mu \cup [ds \rightarrow ds'] \in Q(D)$. 

			\bigskip\noindent
			Let $ds \in \U$ and $g \in \V$. Let $DS = ep(ds)$. 
			Because $g$ is a variable we have to show that 
			$\bigcup\limits_{g' \in names(ep(ds))} \{ \mu \cup [g \mapsto g'] \mid \mu \in \ll P
			\rr^{DS}_{graph(g',DS)},\mu \sim [g \mapsto g']\} \subseteq Q(D)$.
			Let $g' \in names(DS)$ be arbitrary, let $\mu \in \ll P
			\rr^{DS}_{graph(g',DS)}$ where $\mu \sim [g \mapsto g']$.
			Because we have $\mu(P) \in graph(g',DS)$ by SPARQL semantics and our database
			$D$ contains by construction\\ 
			$T(ds,g',\mu(x),\mu(y),\mu(z))$ we have that $\mu \cup [g
			\rightarrow g'] \in Q(D)$. 

			\bigskip\noindent
			Let $ds,g \in \V$.
			Because $ds$ and $g$ are variables we have to show that \\
			$\bigcup\limits_{g' \in names(ds'), ds'\in dom(ep)} \{ \mu \cup \{[ds\mapsto ds'] [g
				\mapsto g']\} \mid \mu \in \ll P
				\rr^{ep(ds')}_{graph(g',ep(ds'))},\mu 
			\sim \{[ds \mapsto ds'][g \mapsto g']\}\} \subseteq Q(D)$. 
			Let $ds'\in dom(ep), g' \in names(ep(ds'))$ be arbitrary, let $\mu \in \ll P
			\rr^{ep(ds')}_{graph(g',ep(ds'))}$ where $\mu \sim \{[ds\mapsto ds'][g \mapsto
			g'] \}$.
			Because we have $\mu(P) \in graph(g',ep(ds'))$ by SPARQL semantics and our database
			$D$ contains by construction\\ $T(ds',g',\mu(x),\mu(y),\mu(z))$ we have that
			$\mu \cup \{[g \rightarrow g'][ds \rightarrow ds'] \}\in Q(D)$. 
			This can be done because $\mu 
			\sim \{[ds \rightarrow ds'][g \rightarrow g']\}$ was assumed. 

			\bigskip\noindent$\supseteq:$\\
			Let $ds,g \in \U$. Let $DS = ep(ds)$ and $G = graph(g,DS)$.
			Let $\mu \in Q(D)$ be arbitrary.
			We thus have a mapping $\mu$ from the variables in $vars(P)$ to constants s.t.
			$T(ds,g,\mu(u),\mu(v),\mu(w)) \in D$. 
			But by construction of $D$ this means that
			$(\mu(u),\mu(v),\mu(w)) \in G$ 
			and obviously $dom(\mu) = vars(P)$ thus $\mu \in \ll P
			\rr^{DS}_{G}$. 

			\bigskip\noindent
			Let $ds \in \V$, $g \in \U$.
			Because $ds$ is a variable we have to show that 
			$\bigcup\limits_{ds' \in dom(ep)} \{ \mu \cup [ds \mapsto ds'] \mid \mu \in \ll P
			\rr^{ep(ds')}_{graph(g,ep(ds'))},\mu \sim [ds \mapsto ds']\} \supseteq Q(D)$.
			Let $\sigma \in Q(D)$ be arbitrary.
			We thus have a mapping $\sigma$ from the 
			variables in $vars(ds,g,u,v,w)$ to constants s.t.\\
			$T(\sigma(ds),g,\sigma(u),\sigma(v),\sigma(w)))
			\in D$. Because $\sigma \in Q(D)$ and $ds \in \V$, we have
			$\sigma(ds) = ds'$, s.t. by construction $ds' \in dom(ep)$. 
			Let $\mu = \sigma\backslash[ds\mapsto ds']$. 
			By construction this means that
			$(\mu(u),\mu(v),\mu(w)) \in graph(g,ep(ds'))$ and obviously
			$dom(\mu) = vars(P)$. Thus $\mu \in \ll
			P\rr^{ep(ds')}_{graph(g,ep(ds'))}$ holds. 
			It remains to show that
			$\mu \sim [ds \mapsto ds']$
			but this is obvious because $\sigma$ contains $[ds \mapsto ds']$ and $\mu = \sigma
			\backslash [ds \mapsto ds']$. 

			\bigskip\noindent
			Let $ds \in \U$, $g \in \V$. Let $DS = ep(ds)$.
			Because $g$ is a variable we have to show that 
			$\bigcup\limits_{g' \in names(ep(a))} \{ \mu \cup [g \mapsto g'] \mid \mu \in \ll P
			\rr^{DS}_{graph(g,DS)},\mu \sim [g \mapsto g']\} \supseteq Q(D)$.
			Let $\sigma \in Q(D)$ be arbitrary.
			We thus have a mapping $\sigma$ from the variables in
			$vars(ds,g,u,v,w)$ to constants s.t.\\
			$T(ds,\sigma(g),\sigma(u),\sigma(v),\sigma(w))) \in D$.
			Because $\sigma \in Q(D)$ and $g \in \V$, we have
			$\sigma(g) = g'$, s.t. by construction $g' \in names(ds)$. Let $\mu =
			\sigma\backslash[g\mapsto g']$ be a mapping. By construction 
			this means that	$(\mu(u),\mu(v),\mu(w) \in graph(g',DS)$ 
			and obviously $dom(\mu) = vars(P)$. Thus $\mu \in \ll P
			\rr^{DS}_{graph(g',DS)}$ holds. It remains to show that $\mu \sim [g \mapsto g']$
			but this is obvious because $\sigma$ contains $[g \mapsto g']$ and $\mu = \sigma
			\backslash [g \mapsto g']$.

			\bigskip\noindent
			Let $ds \in \V$, $g \in \V$.
			Because $ds,g$ are variables we have to show that \\
			$\bigcup\limits_{ds' \in dom(ep),g' \in names(ep(ds'))} 
			\{ \mu \cup \{[g\mapsto g'][ds \mapsto ds'] \} \\ 
				\mid \mu \in \ll P
			\rr^{ep(ds')}_{graph(g',ep(ds'))},
			\mu \sim \{[ds \mapsto ds'], [g \mapsto g']\}\} \supseteq Q(D)$.
			Let $\sigma \in Q(D)$ be arbitrary.
			We thus have a mapping $\sigma$ from the variables in $vars(ds,g,u,v,w)$ to constants s.t.
			$T(\sigma(ds),\sigma(g),\sigma(u),\sigma(v),\sigma(w)) \in D$.
			Because $\sigma \in Q(D)$ and $g \in \V, ds \in \V$, we have
			$\sigma(g) = g', \sigma(ds) = ds'$, s.t. by construction $g' \in
			names(ds)$ and $ds \in dom(ep)$. Let $\mu =
			\sigma\backslash\{[g\mapsto g'],
			[ds\mapsto ds']\}$. By construction
			this means that 	$\mu(u),\mu(v),\mu(w) \in graph(g',ep(ds'))$ 
			and obviously $dom(\mu) = vars(P)$. Thus $\mu \in \ll P
			\rr^{ep(ds')}_{graph(g',ep(ds'))}$ holds. It remains to show that $\mu \sim \{[ds \mapsto
			ds'],[g \mapsto g']\}$
			but this is obvious because $\sigma \supseteq \{[ds \mapsto d'],[g \mapsto
			g']\}$  and $\mu = \sigma \backslash \{[ds \mapsto ds'],[g \mapsto
			g']\}$.


			%%%%%AND%%%%
		\item Consider the case where $P$ is a graph pattern $(P_1 \AND P_2)$. \\
			By construction we have $Q = O_1 \cup O_2 \leftarrow q_1, q_2$. \\
			$\subseteq:$\\
			$ds,g \in \U$: Let $DS = ep(ds)$ and $G = graph(g,DS)$.
			Let $\mu \in \ll P \rr^{DS}_{G}$ be arbitrary. 
			By induction hypothesis we know that
			$\ll P_1 \rr^{DS}_{G} = Q_1(D)$  and
			$\ll P_2 \rr^{DS}_{G} = Q_2(D)$ hold.
			By semantics of SPARQL and $\mu \in \ll P \rr^{DS}_{G}$ we know that 
			$\mu \in \{\mu_1 \cup \mu_2 \mid \mu_1 \in \ll P_1 \rr^{DS}_{G}, \mu_2 \in \ll
			P_2 \rr^{DS}_{G} \mbox{ and }  \mu_1 \sim \mu_2\}$. 
			That means that there is a $\mu_1 \in \ll P_1 \rr_{G}^{DS}$ 
			and a $\mu_2 \in \ll P_2 \rr^{DS}_{G}$ such that $\mu = \mu_1 \cup \mu_2$.
			By induction hypothesis $\mu_1 \in Q_1(D)$ and $\mu_2 \in Q_2(D)$ hold.
			Because $\mu_1 \sim \mu_2$ holds we know that $\mu \in Q(D)$ holds. 

			\bigskip\noindent
			$ds \in \V$, $g \in \U$:
			Let $ds'$ be an arbitrary URI in $dom(ep)$.
			Let $\mu \in \ll P \rr^{ep(ds)}_{graph(g,ep(ds))}$ be arbitrary. 
			By induction hypothesis we know that
			$\bigcup\limits_{ds' \in dom(ep)} \{\mu_i \cup [ds \mapsto ds' ] \mid 
				\mu_i \in \ll P_i \rr^{ep(ds')}_{graph(g,ep(ds'))}, \mu \sim [ds
			\mapsto ds']\}  = Q_i(D)$
			for $i = 1,2$.
			By semantics of SPARQL and 
			$\mu \in \ll P \rr^{ep(ds')}_{graph(g,ep(ds'))}$ we know that 
			$\mu \in \{\mu_1 \cup \mu_2 \mid \mu_1 \in \ll P_1 \rr^{ep(ds')}_{graph(g,ep(ds')}, 
				\mu_2 \in \ll P_2 \rr^{ep(ds')}_{graph(g,ds')}\\ 
			\mbox{ and }  \mu_1 \sim \mu_2\}$. 
			That means that there is a $\mu_1 \in \ll P_1 \rr_{graph(g,ep(ds'))}^{ep(ds')}$ 
			and a $\mu_2 \in \ll P_2 \rr^{ep(ds')}_{graph(g,ep(ds'))}$ such that $\mu = \mu_1 \cup \mu_2$.
			Let  $\mu_1'= \mu_1\cup[ds\mapsto ds'] $ and $\mu_2' = \mu_2\cup[ds\mapsto
			d']$.By induction hypothesis we have a $\mu_1' \in Q_1(D)$ and a $\mu_2' \in
			Q_2(D)$.
			Because $\mu_1 \sim \mu_2$ and $\mu_1 \sim [ds\mapsto ds']$ and $\mu_2 \sim [ds
			\mapsto ds']$ we have $\mu \cup [ds\mapsto ds'] \in Q(D)$.

			\bigskip\noindent
			$ds \in \U$, $g \in \V$: Let $DS = ep(ds)$.
			Let $g'$ be an arbitrary URI in $names(ep(a))$.
			Let $\mu \in \ll P \rr^{ep(ds)}_{graph(g',ds)}$ be arbitrary. 
			By induction hypothesis we know that
			$\bigcup\limits_{g' \in names(DS)}\{\mu_i \cup [g \mapsto g' ] \mid 
				\mu_i \in \ll P_i \rr^{DS}_{graph(g',DS)}, \mu \sim [g
			\mapsto g']\} = Q_i(D)$ for  $i = 1,2$.
			By semantics of SPARQL and $\mu \in \ll P \rr^{DS}_{graph(g',DS)}$ we know that 
			$\mu \in \{\mu_1 \cup \mu_2 \mid \mu_1 \in \ll P_1 \rr^{DS}_{graph(g',DS)}, \mu_2 \in \ll
			P_2 \rr^{DS}_{graph(g',DS)}\\ \mbox{ and }  \mu_1 \sim \mu_2\}$. 
			That means that there is a $\mu_1 \in \ll P_1 \rr_{graph(g',DS)}^{DS}$ 
			and a $\mu_2 \in \ll P_2 \rr^{DS}_{graph(g',DS)}$ such that $\mu = \mu_1 \cup \mu_2$.
			 Let $\mu_1' =  \mu_1\cup[g\mapsto g']$ and $\mu_2' = \mu_2\cup[g\mapsto g']$.
			 By induction hypothesis we have a $\mu_1' \in Q_1(D)$ and a $\mu_2' \in
			Q_2(D)$.
			Because $\mu_1 \sim \mu_2$ and $\mu_1 \sim [g\mapsto g']$ and $\mu_2 \sim [g
			\mapsto g']$ we have $\mu \cup [g\mapsto g'] \in Q(D)$.

			\bigskip\noindent
			$ds,g \in \V$:
			Let $ds'$ be an arbitrary URI in $dom(ep)$.
			Let $g'$ be an arbitrary URI in $names(ep(d))$.
			Let $\mu \in \ll P \rr^{ep(ds')}_{graph(g',ep(ds'))}$ be arbitrary. 
			By induction hypothesis we know that
			$\bigcup\limits\limits_{ds'\in dom(ep),g' \in names(ep(ds'))}\{\mu_i \cup \{[g \mapsto g'
				],[ds \mapsto ds']\} \mid 
				\mu_i \in \ll P_i \rr^{ep(ds')}_{graph(g',ep(ds'))}, \mu
			\sim \{[g \mapsto g'], [ds\mapsto ds']\}\} \subseteq Q_i(D)$
			for $i = 1,2$.
			By semantics of SPARQL and $\mu \in \ll P \rr^{ep(ds')}_{graph(g',ep(ds'))}$ we know that 
			$\mu \in \{\mu_1 \cup \mu_2 \mid \mu_1 \in \ll P_1
				\rr^{ep(ds')}_{graph(g',ep(ds'))}, \mu_2 \in \ll
			P_2 \rr^{ep(ds')}_{graph(g',ep(ds'))}\\ \mbox{ and }  \mu_1 \sim \mu_2\}$. 
			That means that there is a $\mu_1 \in \ll P_1 \rr_{graph(g',ep(ds'))}^{ep(ds')}$ 
			and a $\mu_2 \in \ll P_2 \rr^{ep(ds')}_{graph(g',ep(ds')}$ such that 
			$\mu = \mu_1 \cup \mu_2$.
			 Let $\mu_1' = \mu_1\cup\{[ds\mapsto ds'],[g\mapsto g']\}$ and
			$\mu_2' = \mu_2\cup\{[g\mapsto g'],[ds\mapsto ds']\}$.
			By induction hypothesis we have a $\mu_1' \in Q_1(D)$ and a $\mu_2' \in
			Q_2(D)$.
			Because $\mu_1 \sim \mu_2$ and $\mu_1 \sim \{[ds \mapsto ds'],[g\mapsto g']\}$ and
			$\mu_2 \sim\{[ds \mapsto ds'], [g
			\mapsto g']\}$ we have $\mu \cup \{[g\mapsto g'], [ds \mapsto ds']\} \in Q(D)$.

			\bigskip
			\noindent$\supseteq:$\\
			%$trans(P_1,ds,g)=Q_1= O_1 \leftarrow q_1$ and $trans(P_2,ds,g)=Q_2=
			%O_2 \leftarrow q_2$. By construction: $Q: O_1 \cup O_2 \leftarrow
			%q_1,q_2$.
			$ds,g \in \U$.
			Let $\mu \in Q(D)$ be arbitrary. Let $DS = ep(ds)$ and $G = graph(g,ep(ds))$.
			By induction hypothesis we know that $\ll P_1 \rr^{DS}_{G} = Q_1(D)$ and
			$\ll P_2 \rr^{DS}_{G} = Q_2(D)$ hold. 
			Thus there must be some $\mu_1 \in Q_1(D)$, $\mu_2 \in Q_2(D)$ 
			where $\mu= \mu_1 \cup \mu_2$ holds by construction of $Q$.
			But thus $\mu_1 \in \ll P_1 \rr^{DS}_{G}$ and $\mu_2 \in \ll
			P_2\rr^{DS}_{G}$ by induction hypothesis. 
			By semantics of AND we know that $\mu \in \ll P \rr^{DS}_{G}$.

			\bigskip\noindent
			$ds \in V, g \in \U$.
			We want to show \\
			$\bigcup\limits_{ds' \in dom(ep)} \{ \mu \cup
				[ds \mapsto ds'] \mid \mu \in \ll P\rr^{ep(ds')}_{graph(g,ep(ds'))}, \mu \sim
			[ds\mapsto ds'] \}  \supseteq Q(D)$.
			Let $\mu \in Q(D)$ be arbitrary.
			By induction hypothesis we know that  
			$\bigcup\limits_{ds' \in dom(ep)} \{ \mu \cup [ds \mapsto ds'] \mid \mu \in
				\ll P_i\rr^{ep(ds)}_{graph(g',ep(ds))}, \mu \sim
			[ds\mapsto ds'] \}  = Q_i(D) $ for $i\in \{1,2\}$.
			Thus there must be some $\mu_1 \in Q_1(D)$, $\mu_2 \in Q_2(D)$ 
			where $\mu= \mu_1 \cup \mu_2$ holds by construction of $Q$.
			By induction hypothesis we know that $\mu_i(ds) = ds'$ for some $ds'
			\in dom(ep)$ for $i \in \{1,2\}$.
			Again by i.h. $\mu_i\backslash\{[ds\mapsto ds'] \} \in \ll P_i
			\rr^{ep(ds)}_{graph(g,ep(ds'))}$ for $i \in \{1,2\}$.
			By semantics of AND we know that $\mu \backslash \{[ds\mapsto ds'] \} \in \ll P
			\rr^{ep(ds')}_{graph(g,ep(ds'))}$ and by induction hypothesis we know that we have 
			$\mu \sim \{[ds\mapsto ds'] \}$.

			\bigskip\noindent
			$ds \in \U, g \in \V$. Let $DS = ep(ds)$.
			We want to show 
			$\bigcup\limits_{g' \in names(DS)} \{ \mu \cup \{[g
				\mapsto g']\} \mid \mu \in
				\ll P\rr^{DS}_{graph(g',DS)}, \mu \sim
			\{[g \mapsto g']\} \}  \supseteq Q(D) $.
			Let $\mu \in Q(D)$ be arbitrary.
			By induction hypothesis we know that  
			$\bigcup\limits_{g' \in names(DS)} \{ \mu \cup \{[g
				\mapsto g']\} \mid \mu \in
				\ll P_i\rr^{DS}_{graph(g',DS)}, \mu \sim
			\{[g \mapsto g']\} \}  = Q_i(D) $ for $i=1,2$.
			Thus there must be some $\mu_1 \in Q_1(D)$, $\mu_2 \in Q_2(D)$ 
			where $\mu= \mu_1 \cup \mu_2$ construction of $Q$.
			By induction hypothesis we know that $\mu_i(g) = g'$ for some $g'
			\in names(ds)$ for $i \in \{1,2\}$.
			Again by i.h. $\mu_i\backslash\{ [g \mapsto g'] \} \in \ll P_i \rr^{DS}_{graph(g',DS)}$
			$i \in \{1,2\}$ by induction hypothesis. 
			By semantics of AND  we know that $\mu \backslash \{[g \mapsto g'] \} \in \ll P
			\rr^{DS}_{graph(g',DS)}$ and by induction hypothesis we know that we have 
			$\mu \sim \{[g \mapsto g'] \}$.

			\bigskip\noindent
			$ds,g \in \V$.
			We want to show 
			$\bigcup\limits_{ds \in dom(ep),g' \in names(ep(ds'))} \{ \mu \cup \{[ds \mapsto ds'],[g
				\mapsto g']\} \mid \mu \in
				\ll P \rr^{ep(ds')}_{graph(g', ep(ds'))}, \mu \sim
			\{[ds \mapsto ds'], [g \mapsto g']\} \}  \supseteq Q(D) $.
			Let $\mu \in Q(D)$ be arbitrary.
			By induction hypothesis we know that  
			$\bigcup\limits_{ds' \in dom(ep), g' \in names(ep(ds'))} 
			\{ \mu \cup \{[ds \mapsto ds'], [g	\mapsto g']\} \mid\\
				\mu \in	\ll P_i\rr^{ep(ds')}_{graph(g', ep(ds))}, \mu \sim
			\{[ds\mapsto ds'],[g \mapsto g']\} \}  = Q_i(D) $ for $i=1,2$.
			Thus there must be some $\mu_1 \in Q_1(D)$, $\mu_2 \in Q_2(D)$ 
			where $\mu= \mu_1 \cup \mu_2$ holds by semantics of $\land$ and construction of $Q$.
			By induction hypothesis we know that $\mu_i(ds) = ds'$ for some
			$ds' \in dom(ep)$ and $\mu_i(g) = g'$ for some $g' \in names(ds')$ for $i \in \{1,2\}$.
			By induction hypothesis $\mu_i\backslash\{[ds\mapsto ds'], [g \mapsto g'] \} \in \ll P_i
			\rr^{ep(ds')}_{graph(g',ep(ds'))}$ for $i \in \{ 1,2 \}$.
			By semantics of $AND$  we know that $\mu \backslash \{[ds\mapsto ds'], [g
			\mapsto g'] \} \in \ll P
			\rr^{ep(ds')}_{graph(g,ep(ds'))}$ and by induction hypothesis we know that we have 
			$\mu \sim \{[ds\mapsto ds'], [g \mapsto g'] \}$.


			%%%OPT

		\item Consider the case where $P$ is a graph pattern $(P_1 \OPT	P_2)$. \\
			By construction we have that $ Q_1 = trans(P_1,ds,g) =
			(T_1,\lambda_1,x_1)$,\\ $Q_2 = trans(P_2,ds,g) = (T_2,\lambda_2,x_2)$ 
			and $Q = trans(P,ds,g) = (T,\lambda,x)$ for which $T = T_1 \cup T_2
			\cup (r_1,r_2)$ where $r_1,r_2$ are the roots of $T_1,T_2$
			respectively, $\lambda = \lambda_1 \cup \lambda_2$ and $x = x_1 \cup
			x_2$.

			\bigskip\noindent
			$\subseteq:$\\
			Let $ds,g \in \U$:
			Let $DS = ep(ds)$ and $G = graph(g,DS)$.
			Let $\mu \in \ll P \rr^{DS}_{G}$ be arbitrary. 
			By semantics of OPT we have that $\mu \in \ll P_1 \ AND \ P_2
			\rr^{DS}_{G}$ or $\mu \in \big\{ \mu_1 \in \ll P_1 \rr^{DS}_{G} \mid \forall \mu_2 \in \ll P_2
				\rr^{DS}_{G}:
			\mu_1 \not\sim \mu_2 \big\}$. 
			We thus proceed by case distinction:
			\begin{enumerate}
				\item Assume $\mu \in \ll P_1 \ AND \ P_2 \rr^{DS}_{G}$. By induction
					hypothesis and semantics of AND, we thus have some $\mu_1
					\in Q_1(D)$ and some $\mu_2 \in Q_2(D)$ so that $\mu = \mu_1
					\cup \mu_2$. Because $\mu_i \in Q_i(D)$ we have by semantics
					of wdpts (recall Definition~\ref{wdptq}) that $\mu_i \in Q_{i,T'_i}$ so that $T'_i \subseteq
					T_i$ for $i\in\{1,2\}$.
					Let $T' = T'_1 \cup T'_2$.
					Because we have $\mu = \mu_1 \cup \mu_2$ we know that $\mu
					\in Q_T'$.
					It remains to show that this homomorphism is maximal: assume there was a
					bigger subtree $\hat{T}$ which would allow a mapping $\mu'
					\sqsupset \mu$. Then the
					node which was put additionally to $T$ must either be in
					$T_1$ or $T_2$.
					But then either $\mu_1$ or $\mu_2$ were not maximal defying the assumption
					that $\mu_1 \in Q_1$ and $\mu_2 \in Q_2$.
					Thus $\mu \in Q(D)$. 

				\item Assume $\mu \in \big\{\mu_1 \in \ll P_1 \rr^{DS}_{G} \mid 
					\forall \mu_2 \in \ll P_2 \rr^{DS}_{G}: \mu_1 \not\sim \mu_2
				\big\}$. \\
					Because of our assumption $\mu = \mu_1$ for some $\mu_1 \in \ll P_1
					\rr^{DS}_{G}$. We know that $\mu_1 \in Q_1(D)$ by induction
					hypothesis. Thus $\mu_1 \in Q_{1,T'_1}$ for some $T'_1
					\subset T_1$ by
					semantics of wdpts. $\mu \in Q_{T'_1}(D)$ follows.
					It remains to show that there is no $\mu' \sqsupset \mu$.
					Because of our assumption we know there is no mapping
					$\mu_2 \in Q_2$ which would be compatible with $\mu$. Thus $\mu$ could
					only be enlarged by a bigger mapping in $Q_1(D)$ but this defies the
					assumption that $\mu_1 \in Q_1(D)$.
			\end{enumerate}

			\bigskip\noindent
			Let $ds \in \V$ and $g \in \U$:
			We want to show  
			$\bigcup\limits_{ds' \in dom(ep)} \{ \mu \cup [ds \mapsto ds'] \mid \mu \in
				\ll P\rr^{ep(ds')}_{graph(g,ep(ds'))}, \mu \sim
			[ds\mapsto ds'] \}  \subseteq Q(D) $.
			Let $ds'$ be an arbitrary URI in $dom(ep)$. Let $\mu \in \ll P
			\rr^{ep(ds')}_{graph(g,ds')}$ be arbitrary.
			By semantics of OPT we have that $\mu \in \ll P_1 \mbox{ AND }  P_2
			\rr^{ep(ds')}_{graph(g,ds')}$ or 
			$\mu \in \big\{ \mu_1 \in \ll P_1 \rr^{ep(ds')}_{graph(g,ds')} \mid \forall \mu_2 \in \ll P_2
			\rr^{ep(ds')}_{graph(g,ds')}: \mu_1 \not\sim \mu_2 \big\}$. 
			We thus proceed by case distinction:
			\begin{enumerate}
				\item Assume $\mu \in \ll P_1 \AND  P_2 \rr^{ep(ds')}_{graph(g,ds')}$. 
					By induction hypothesis we have 
					$\bigcup\limits_{ds' \in dom(ep)} \{ \mu \cup [ds \mapsto ds'] \mid \mu \in
						\ll P_i\rr^{ep(ds')}_{graph(g,ep(ds'))}, \mu_i \sim
					[ds\mapsto ds'] \}  \subseteq Q_i(D) $ for $i \in \{1,2\}$.
					By semantics of SPARQL and $\mu \in \ll P \rr^{ep(ds')}_{graph(g,ds')}$ we
					know that $\mu \in \{\mu_1 \cup \mu_2 \mid \mu_1 \in \ll P_1
						\rr^{ep(ds')}_{graph(g,ds')}, \mu_2 \in \ll P_2
						\rr^{ep(ds')}_{graph(g,ds')}
					\mbox{ and } \mu_1 \sim \mu_2 \}$. Thus there is a $\mu_1 \in \ll P_1
					\rr^{ep(ds')}_{graph(g,ds')}$ and a $\mu_2 \in \ll P_2
					\rr^{ep(ds')}_{graph(g,ds')}$ such that $\mu = \mu_1 \cup \mu_2$.
					Let $\mu'_i = \mu_i \cup [ds \mapsto ds']$ for $i \in
					\{1,2\}$.  Because $\mu_i \in Q_i(D)$  we have a
					$T'_i \subseteq T_i$ such that $\mu'_i \in Q_{i,T'_i}(D)$ for $i\in \{1,2\}$.
					Let $T' = T'_1 \cup T'_2$.
					Because $\mu_1 \sim \mu_2$ and $\mu_1 \sim [ds \mapsto ds']$ 
					and $\mu_2 \sim [ds \mapsto ds']$ we have $\mu \cup [ds
					\mapsto ds'] \in Q_T'(D)$.
					It remains to show that this homomorphism $\mu\cup [ds \mapsto ds']$ is maximal: 
					assume there was a bigger subtree $\hat{T}$ which would allow a mapping 
					$\mu' \sqsupset (\mu\cup [ds \mapsto ds'])$. Then the
					node which was put additionally to $T$ must either be in $T_1$ or $T_2$.
					But then either $\mu_1'$ or $\mu_2'$ were not maximal defying the assumption
					that $\mu_1' \in Q_1(D)$ and $\mu_2' \in Q_2(D)$.
					Thus $\mu \cup [ds \mapsto ds']\in Q(D)$. 

				\item Assume $\mu \in \big\{ \mu_1 \in  \ll P_1 \rr^{ep(ds')}_{graph(g,ds')} \mid 
						\forall \mu_2 \in \ll P_2 \rr^{ep(ds')}_{graph(g,ds')}: 
					\mu_1 \not\sim \mu_2 \big\}$. 
					By induction hypothesis we have \\
					$\bigcup\limits_{ds' \in dom(ep)} \{ \mu_1 \cup	[ds	\mapsto ds'] 
						\mid \mu_1 \in \ll	P_1\rr^{ep(ds')}_{graph(g,ep(ds'))}, \mu \sim
					[ds\mapsto ds'] \}  = Q_1(D)$.
					Because $\mu_1 \in \ll P_1
					\rr^{ep(ds')}_{graph(g,ds')}$ 
					we have $\mu\cup [ds \mapsto ds'] \in Q_{T'_1}(D)$, where
					$T'_1 \subseteq T_1$ by wdpt semantics.
					It remains to show that there
					is no $\mu' \sqsupset \mu$. But because we have $\{\mu_1 \not\sim \mu_2 \mid
					\forall \mu_2 \in \ll P_2 \rr^{ep(ds')}_{graph(g,ds')} \}$, 
					we know there is no mapping
					of $\mu_2$ of $Q_2$ which would be compatible with $\mu$. Thus $\mu$ could
					only be a bigger mapping $\mu' \in Q_1(D)$ but this defies the
					assumption that $\mu_1 \in Q_1(D)$ and thus we are done.
			\end{enumerate}

			\bigskip\noindent
			Let $ds\in \U$ and $g \in \V$: Let $DS = ep(ds)$.
			We want to show 
			$\bigcup\limits_{g' \in names(DS)}\{ \mu \cup [g \mapsto g'] \mid \mu \in
				\ll P\rr^{DS}_{graph(g',DS)} , \mu \sim
			[g\mapsto g'] \} \subseteq Q(D) $
			Let $g'$ be an arbitrary graph in $names(DS)$. Let $\mu \in \ll P
			\rr^{DS}_{graph(g',DS)}$ be arbitrary.
			By semantics of OPT we have that $\mu \in \ll P_1 \AND  P_2
			\rr^{DS}_{graph(g',DS)}$ or 
			$\mu \in \big\{ \mu_1 \in  \ll P_1 \rr^{DS}_{graph(g',DS)} \mid \forall \mu_2 \in \ll P_2
				\rr^{DS}_{graph(g',DS)}:
			\mu_1 \not\sim \mu_2 \big\}$. 
			We proceed by case distinction:
			\begin{enumerate}
				\item Assume $\mu \in \ll P_1 \ AND \ P_2 \rr^{DS}_{graph(g',DS)}$. 
					By induction hypothesis we have 
					$\bigcup\limits_{g' \in names(DS)} \{ \mu \cup [g \mapsto g'] \mid \mu \in
						\ll P_i\rr^{DS}_{graph(g',DS)}, \mu_i \sim
					[g\mapsto g'] \}  = Q_i(D) $ and $i = 1,2$.
					By semantics of SPARQL and $\mu \in \ll P
					\rr^{DS}_{graph(g',DS)}$ we
					know that $\mu \in \{\mu_1 \cup \mu_2 \mid \mu_1 \in \ll P_1
						\rr^{DS}_{graph(g',DS)}, \mu_2 \in \ll P_2
						\rr^{DS}_{graph(g',DS)}
					\mbox{ and } \mu_1 \sim \mu_2 \}$.
					Thus there is a $\mu_1 \in \ll P_1
					\rr^{ep(ds)}_{graph(g',ds)}$ and a $\mu_2 \in \ll P_2
					\rr^{ep(ds)}_{graph(g',ds)}$ such that $\mu = \mu_1 \cup \mu_2$.
					Let $\mu'_i = \mu_i \cup [g \mapsto g']$ for $i \in
					\{1,2\}$.  Because $\mu_i \in Q_i(D)$  we have a
					$T'_i \subseteq T_i$ such that $\mu'_i \in Q_{i,T'_i}(D)$ for $i\in \{1,2\}$.
					Let $T' = T'_1 \cup T'_2$.
					Because $\mu_1 \sim \mu_2$ and $\mu_1 \sim [g \mapsto g']$ 
					and $\mu_2 \sim [g \mapsto g']$ we have $\mu \cup [g
					\mapsto g'] \in Q_T'(D)$.
					It remains to show that this homomorphism $\mu\cup [g
					\mapsto g']$ is maximal: 
					assume there was a bigger subtree $\hat{T}$ which would allow a mapping 
					$\mu' \sqsupset (\mu\cup [g \mapsto g'])$. Then the
					node which was put additionally to $T$ must either be in $T_1$ or $T_2$.
					But then either $\mu_1'$ or $\mu_2'$ were not maximal defying the assumption
					that $\mu_1' \in Q_1(D)$ and $\mu_2' \in Q_2(D)$.
					Thus $\mu \cup [g \mapsto g']\in Q(D)$. 

				\item Assume $\mu \in \big\{ \mu_1 \in  \ll P_1 \rr^{DS}_{graph(g',DS)} \mid 
					\forall \mu_2 \in \ll P_2 \rr^{DS}_{graph(g',DS)}: \mu_1 \not\sim \mu_2 \big\}$. 
					By induction hypothesis we have \\
					$\bigcup\limits_{g' \in names(DS)} 
					\{ \mu_1 \cup [g \mapsto g'] \mid \mu_1 \in
						\ll P_1\rr^{DS}_{graph(g',DS)}, \mu \sim
					[g\mapsto g'] \}  = Q_1(D)$.
					Because $\mu_1 \in \ll P_1
					\rr^{ep(ds)}_{graph(g',ds)}$ 
					we have $\mu\cup [g \mapsto g'] \in Q_{T'_1}(D)$, where
					$T'_1 \subseteq T_1$ by wdpt semantics.
					It remains to show that there
					is no $\mu' \sqsupset \mu$. But because we have $\{\mu_1 \not\sim \mu_2 \mid
					\forall \mu_2 \in \ll P_2 \rr^{ep(ds)}_{graph(g',ds)} \}$, 
					we know there is no mapping
					of $\mu_2$ of $Q_2$ which would be compatible with $\mu$. Thus $\mu$ could
					only be a bigger mapping $\mu' \in Q_1(D)$ but this defies the
					assumption that $\mu_1 \in Q_1(D)$ and thus we are done.
			\end{enumerate}

			\bigskip\noindent
			Let $ds,g \in \V$:
			We want to show  $\bigcup\limits_{ds' \in dom(ep), g' \in names(ep(d))} \{ \mu \cup
			\{[ds \mapsto ds'],[g \mapsto g']\} \mid \mu \in
			\ll P\rr^{ep(ds')}_{graph(g',ep(ds'))}, 
			\mu \sim \{[ds\mapsto ds'], [g \mapsto g']\}\} \supseteq Q(D) $. 
			Let $ds'$ be an arbitrary URI in dom(ep), and $g'$ be an arbitrary
			URI in $names(ep(d))$.
			Let $\mu \in \ll P \rr^{ep(ds')}_{graph(g',ep(ds'))}$ be arbitrary.
			By semantics of OPT we have that $\mu \in \ll P_1 \AND P_2
			\rr^{ep(ds')}_{graph(g',ep(ds'))}$ or \\
			$\mu \in \big\{ \mu_1 \in  \ll P_1 \rr^{ep(ds')}_{graph(g',ep(ds'))} \mid \forall \mu_2 \in \ll P_2
				\rr^{ep(ds')}_{graph(g',ep(ds'))}:
			\mu_1 \not\sim \mu_2 \big\}$. 
			We thus proceed by case distinction:
			\begin{enumerate}
				\item Assume $\mu \in \ll P_1 \AND P_2 \rr^{ep(ds')}_{graph(g',ep(ds'))}$. 
					By induction hypothesis we thus have \\
					$\bigcup\limits_{ds'\in dom(ep),g' \in names(ep(d))} \{ \mu \cup \{[ds \mapsto
						ds'][g \mapsto g']\} \mid \mu \in
						\ll P_i\rr^{ep(ds')}_{graph(g',ep(ds'))}, \mu_i \sim
					\{[g\mapsto g'], [ds \mapsto ds']\} \}  = Q_i(D) $ and $i = 1,2$.
					By semantics of SPARQL and $\mu \in \ll P
					\rr^{ep(ds')}_{graph(g',ep(ds'))}$ we
					know that $\mu \in \{\mu_1 \cup \mu_2 \mid \mu_1 \in \ll P_1
						\rr^{ep(ds')}_{graph(g',ep(ds'))}, \mu_2 \in \ll P_2
						\rr^{ep(ds')}_{graph(g',ep(ds'))}
					\mbox{ and } \mu_1 \sim \mu_2 \}$. Thus there is a $\mu_1 \in \ll P_1
					\rr^{ep(ds')}_{graph(g',ep(ds'))}$ and a $\mu_2 \in \ll P_2
					\rr^{ep(ds')}_{graph(g',ep(ds'))}$ such that $\mu = \mu_1 \cup \mu_2$.
					Let $\mu'_i = \mu_i \cup \{ [ds \mapsto ds'][g \mapsto g']\}$ for $i \in
					\{1,2\}$.  Because $\mu_i \in Q_i(D)$  we have a
					$T'_i \subseteq T_i$ such that $\mu'_i \in Q_{i,T'_i}(D)$ for $i\in \{1,2\}$.
					Let $T' = T'_1 \cup T'_2$.
					Because $\mu_1 \sim \mu_2$ and $\mu_i \sim \{[g \mapsto
					g'],[ds \mapsto ds'] \}$ for
					$i\in\{1,2\}$, we have $\mu \cup [g	\mapsto g'] \in Q_T'(D)$.
					It remains to show that the mapping $\mu\cup \{[ds\mapsto
					ds'], [g\mapsto g'] \}$ is maximal: 
					assume there was a bigger subtree $\hat{T}$ which would allow a mapping 
					$\mu' \sqsupset (\mu\cup \{[ds \mapsto ds'],[g \mapsto g']\})$. Then the
					node which was put additionally to $T$ must either be in $T_1$ or $T_2$.
					But then either $\mu_1'$ or $\mu_2'$ were not maximal defying the assumption
					that $\mu_1' \in Q_1(D)$ and $\mu_2' \in Q_2(D)$.
					Thus $\mu \cup \{[g \mapsto g'],[ds \mapsto ds']\} \in Q(D)$. 


	\item Assume $\mu \in \big\{ \mu_1 \in  \ll P_1 \rr^{ep(ds')}_{graph(g',ep(ds'))} \mid 
			\forall \mu_2 \in \ll P_2 \rr^{ep(ds')}_{graph(g',ep(ds'))}:
		\mu_1 \not\sim \mu_2 \big\}$. By induction hypothesis we have 
		$\bigcup\limits_{ds' \in dom(ep), g' \in names(ep(ds'))} \{ \mu_1 \cup \{[ds' \mapsto
			ds] [g \mapsto g'] \} \mid \mu_1 \in
			\ll P_1\rr^{ep(ds')}_{graph(g',ep(ds'))}, \mu \sim
			\{[ds \mapsto ds'],[g\mapsto g']\}  = Q_1(D)$.
			Because $\mu_1 \in \ll P_1
			\rr^{ep(ds)}_{graph(g',ds)}$ 
			we have $\mu\cup \{[ds \mapsto ds'],[g \mapsto g'] \}\in Q_{T'_1}(D)$, where
			$T'_1 \subseteq T_1$ by wdpt semantics.
			It remains to show that there
			is no $\mu' \sqsupset \mu$. But because we have $\{\mu_1 \not\sim \mu_2 \mid
			\forall \mu_2 \in \ll P_2 \rr^{ep(ds')}_{graph(g',ds')} \}$, 
			we know there is no mapping
			of $\mu_2$ of $Q_2$ which would be compatible with $\mu$. Thus $\mu$ could
			only be a bigger mapping $\mu' \in Q_1(D)$ but this defies the
			assumption that $\mu_1 \in Q_1(D)$ and thus we are done.
	\end{enumerate}

	\bigskip\noindent$\supseteq:$\\
	Because of the construction of $Q$, a solution, call it $\mu$ must either
	adhere $\mu \in Q_{T'}(D)$ for $T' \subseteq T$. We will further distinguish two
	cases:
	\begin{enumerate}
		\item $\mu \in Q_{T'}(D)$ for some $T' \subseteq T_1$ 
		\item  $\mu \in	Q_{T'}(D)$ for some $T' = T'_1 \cup T'_2$ where $T'_1 \subseteq T_1$ and
	$T'_2 \subseteq T_2$.
	\end{enumerate}

	\bigskip\noindent
	Let $ds,g \in \U$. Let $DS= ep(ds)$ and $G = graph(g,DS)$. 
	Let $\mu \in Q(D)$ be arbitrary.
	Case distinction:
	\begin{enumerate}
		\item $\mu \in Q_{T'_1}(D)$: Thus $\mu \in Q_1(D)$ and by i.h. $\mu \in \ll P_1 \rr^{DS}_{G}$ 
			and then also $\{\mu \not\sim \mu_2 \mid \forall \mu_2 \in
			\ll P_2 \rr^{DS}_{G} \}$ by assumption. Thus $\mu \in \ll P
			\rr_{G}^{DS}$.
		\item If $\mu \in Q_{T'}(D)$ we then have $\mu_{|vars(Q_1)} \in Q_1(D)$  (restricted to
			the variables in $Q_1$) and $\mu_{|vars(Q_2)} \in Q_2(D)$(restricted to the variables
			in $Q_2$). Thus by i.h. and semantics of AND we have $\mu \in \ll P_1 \ AND \ P_2 \rr^{DS}_{G}$
			and $\mu \in \ll P \rr^{DS}_{G}$.
	\end{enumerate}

	%The case where we \sigma \in Q_1(D) maximal only because of value of ds is
	%taken care of:
	% because $\mu_1$ and $\mu_2$ need to both be compatible with assignment of
	% the variable ds$
	\bigskip\noindent
	Let $ds \in \V, g \in \U$.
	We want to prove  
	$\bigcup\limits_{ds' \in dom(ep)} \{ \mu \cup [ds \mapsto ds'] \mid\\ \mu
	\in \ll P \rr^{ep(ds')}_{graph(g,ep(ds'))}, \mu \sim [ds\mapsto ds'] \}
	\supseteq Q(D) $.
	Let $\sigma \in Q(D)$ be arbitrary.
	Because $\sigma \in Q(D)$ and $ds \in V$ we have that $\sigma(ds) =
	ds'$ for some $ds' \in dom(ep)$.
	Case distinction:
	\begin{enumerate}
		\item $\sigma \in Q_{T'_1}(D)$: Because the assumption implies $\sigma
			\in Q_1(D)$ we can then use the induction hypothesis  
			$\bigcup\limits_{ds' \in dom(ep)} \{ \mu \cup [ds \mapsto ds'] \mid \mu
			\in \ll P_1 \rr^{ep(ds')}_{graph(g,ep(ds'))}, \mu \sim [ds\mapsto ds'] \}  =
			Q_1(D) $. Let $\mu = \sigma \backslash [ds \mapsto ds']$. 
			Thus $\mu \in \ll P_1 \rr^{ep(ds')}_{graph(g,ep(ds'))}$ 
			and we also have $\{\mu \not\sim \mu_2 \mid \forall \mu_2 \in
			\ll P_2 \rr^{ep(ds')}_{graph(g,ep(ds'))} \}$ because of our
			assumption. Thus $\mu \in \ll P
			\rr_{graph(g,ep(ds')}^{ep(ds')}$ and $\mu \sim [ds \mapsto ds']$ by
			induction hypothesis.
		\item If $\sigma \in Q_{T'}(D)$  we can use the induction hypothesis twice:\\
			$\bigcup\limits_{ds' \in dom(ep)} \{ \mu_i \cup [ds \mapsto ds'] \mid \mu_i
			\in \ll P_i \rr^{ep(ds')}_{graph(g,ep(ds'))}, \mu_i \sim [ds\mapsto ds'] \}  =
			Q_i(D)$ for $i=1,2$.\\
			Let $\mu = \sigma \backslash [ds \mapsto ds']$.
			Because $\sigma \in Q_{T'}(D)$ we have $\mu = \mu_1 \cup \mu_2$ for
			some $\mu_{1|vars(Q_1)}\cup[ds \mapsto ds'] \in Q_1(D)$ (restricted to
			the variables in $Q_1$) and $\mu_{2|vars(Q_2)}\cup [ds \mapsto ds'] \in Q_2(D)$ (restricted to
			the variables in $Q_2$).
			Thus $\mu \in \ll P_1 \AND P_2 \rr^{ep(ds')}_{graph(ds',g)}$
			and $\mu \in \ll P \rr^{ep(ds')}_{graph(g,ds')}$. Also $\mu \sim [ds \mapsto
			ds']$ holds by induction hypothesis.
	\end{enumerate}

	\bigskip\noindent
	Let $ds \in \U, g \in \V$. Let $DS = ep(ds)$.
	We want to prove  
	$\bigcup\limits_{g' \in names(DS)} \{ \mu \cup [g \mapsto g'] \mid \mu
	\in \ll P \rr^{DS}_{graph(g,DS)}, \mu \sim [b\mapsto g] \}  \supseteq Q(D) $.
	Let $\sigma \in Q(D)$ be arbitrary.
	Because $\sigma \in Q_{D}$ and $g \in V$ we have that $\sigma(g) =
	g'$ for some $g' \in names(DS)$.
	Case distinction:
	\begin{enumerate}
		\item $\sigma \in Q_{T_1}(D)$:  Because the assumption implies $\sigma
			\in Q_1(D)$	we can then use the induction hypothesis  
			$\bigcup\limits_{g \in names(DS)} \{ \mu \cup [g \mapsto g'] \mid \mu
			\in \ll P_1 \rr^{DS}_{graph(g',DS)}, \mu \sim [g\mapsto g'] \}  =
			Q_1(D) $. Let $\mu = \sigma \backslash [g \mapsto g']$. 
			Thus $\mu \in \ll P_1 \rr^{DS}_{graph(g',DS)}$ 
			and we also have $\{\mu \not\sim \mu_2 \mid \forall \mu_2 \in
			\ll P_2 \rr^{DS}_{graph(g',DS)} \}$ because of our assumption.
			Thus $\mu \in \ll P \rr_{graph(g',DS)}^{DS}$ and $\mu \sim [g
			\mapsto g']$ by induction hypothesis.
		\item If $\sigma \in Q_{T'}(D)$  we can use the induction hypothesis twice:\\
			$\bigcup\limits_{g \in names(DS)} \{ \mu_i \cup [g \mapsto g'] \mid \mu_i
			\in \ll P_i \rr^{ep(ds)}_{graph(g',ep(ds))}, \mu_i \sim [g\mapsto g'] \}  =
			Q_i(D)$ for $i=1,2$.\\
			Let $\mu = \sigma \backslash [g \mapsto g']$.
			Because $\sigma \in Q_{T'}(D)$ we have $\mu = \mu_1 \cup \mu_2$ for
			some $\mu_{1|vars(Q_1)}\cup[g \mapsto g'] \in Q_1(D)$ (restricted to
			the variables in $Q_1$) and $\mu_{2|vars(Q_2)}\cup [g \mapsto g'] \in Q_2(D)$ (restricted to
			the variables in $Q_2$).
			Thus $\mu \in \ll P_1 \AND P_2 \rr^{ep(ds)}_{graph(ds,g')}$
			and $\mu \in \ll P \rr^{ep(ds)}_{graph(g',ds)}$. Also $\mu \sim [g \mapsto
			g']$ holds by induction	hypothesis.
	\end{enumerate}

	\bigskip\noindent
	Let $ds,g \in \V$ We want to prove  
	$\bigcup\limits_{g' \in names(ep(ds')), ds' \in dom(ep)} 
	\{ \mu \cup \{[ds \mapsto ds'] [g \mapsto g']\} \mid \mu
		\in \ll P \rr^{ep(ds')}_{graph(g',ep(ds'))}, \mu \sim \{[ds \mapsto ds'] [g
	\mapsto g']\} \}  \supseteq Q(D) $.
	Let $\sigma \in Q(D)$ be arbitrary.
	Because $\sigma \in Q_{D}$ and $ds,g \in V$ we have that $\sigma(g) =
	g'$ and $\sigma(ds) = ds'$ for some $g' \in names(DS)$ and $ds' \in dom(ep)$.

	Case distinction:
	\begin{enumerate}
		\item $\sigma \in Q_{T_1}(D)$: 
			we can then use the induction hypothesis\\  
			$\bigcup\limits_{ds' \in dom(ep), g' \in names(ep(d))} \{ \mu \cup \{[ds
				\mapsto ds'] [g	\mapsto g']\} \mid \mu
				\in \ll P_1 \rr^{ep(ds')}_{graph(g',ep(ds'))}, \mu \sim \{[ds
			\mapsto ds'] [g \mapsto g']\} \}  =	Q_1(D) $. 

			Thus $\mu \in \ll P_1 \rr^{ep(ds')}_{graph(g',ep(ds'))}$ 
			and we also have\\ $\{\mu \not\sim \mu_2 \mid \forall \mu_2 \in
			\ll P_2 \rr^{ep(ds')}_{graph(g',ep(ds'))} \}$ because of our assumption.
			Thus $\mu \in \ll P \rr_{graph(g',ep(ds'))}^{ep(ds')}$ and $\mu \sim
			\{[ds' \mapsto ds],[g'\mapsto g']\}$ by induction hypothesis.

		\item If $\sigma \in Q_{T}(D)$  we can use the induction hypothesis twice:\\
			$\bigcup\limits_{ds' \in dom(ep),g' \in names(ep(ds'))} \{ \mu_i \cup\{[ds
					\mapsto ds'] [g \mapsto g'\}  \mid \\ \mu_i
					\in \ll P_i \rr^{ep(ds')}_{graph(g',ep(ds'))}, \mu_i \sim \{[ds
				\mapsto ds'] [g \mapsto g']\}\}  = Q_i(D)$ for $i=1,2$.\\
				Let $\mu = \sigma \backslash \{[ds \mapsto ds'] [g \mapsto g']\}$.
				Because $\sigma \in Q_{T'}(D)$ we have $\mu = \mu_1 \cup \mu_2$ for
				some $\mu_{1|vars(Q_1)}\cup\{[ds \mapsto ds'] [g \mapsto g']\}$(restricted to
			the variables in $Q_1$) and 
			$\mu_{1|vars(Q_1)}\{[ds \mapsto ds'] [g \mapsto g']\}$(restricted to
			the variables in $Q_2$).
				Thus $\mu \in \ll P_1 \ AND \ P_2 \rr^{ep(ds')}_{graph(g',ep(ds'))}$
				and $\mu \in \ll P \rr^{ep(ds')}_{graph(g,ep(ds'))}$. Also $\mu \sim
				\{[ds \mapsto ds'] [g \mapsto g']\}$ holds by induction
				hypothesis.
		\end{enumerate}

	%%% GRAPH %%%%%%
	\item Consider the case where $P$ is a graph pattern $(\mbox{GRAPH} \ u \ P_1)$. \\
		Our outputquery is constructed as follows: Let $Q_1 = trans(P_1,u,g)$. 
		Assuming $r_1$ is the root of $T_1$ and $\lambda(r_1) = q_1$ we define
		\[ \lambda'(x) =\begin{dcases*} 
				q_1, LOC(u,ds),LOC(g,ds)& if $x = r_1$\\
				\lambda(x) & otherwise	\\
			\end{dcases*}
		\] and $trans(P,ds,g) = (T_1,\lambda',x_1)$.
		
		$\subseteq:$\\
		$ds,g \in \U:$ \\
		Let $DS = ep(ds)$ and $G = graph(g,DS)$.
		Let $\mu \in \ll P \rr^{DS}_{G}$ be arbitrary.
		Proceed by case distinction:
		\begin{enumerate}
			\item $u \in names(DS)$: We have $\mu \in \ll P_1
				\rr^{DS}_{graph(u,DS)}$ by SPARQL semantics and assumption. By i.h. we have 
				$Q_1(D) =  \ll P_1\rr^{DS}_{graph(u,DS)}$ and thus $\mu \in
				Q_1(D)$. By
				construction of our query $Q$ we see that $\mu \in Q(D)$.

			\item $u \in \U \backslash names(DS)$:
				Then $\ll P_1 \rr^{DS}_{graph(u,DS)} = \{\}$ by SPARQL semantics. But then
				$Q(D) = \{\}$ because we added $LOC(u,ds)$ to the root of our
				query Q.
			\item $u \in V$:
				Then $\mu \in S_1$ where\\ $S_1 =  \bigg\{\mu_1 \cup [ u \rightarrow s ] \mid
					s \in names(DS), \mu_1 \in \ll P_1
					\rr^{DS}_{graph(s,DS)}, [ u \rightarrow s ] \sim
				\mu_1 \bigg\}$. 
				By induction hypothesis we know that for $Q_1
				= trans(P_1,ds,u)$ we have
				$\bigcup\limits_{g'\in names(DS)} \{ \mu_1 \cup [u\mapsto g'] \mid
					\mu_1 \in \ll P_1 \rr^{DS}_{graph(g',DS)}, \mu_1 \sim [u
				\mapsto g']\} = Q_1(D)$. Because $\mu \in S_1$ there
				is an $s \in names(DS)$ such  that there is a $\mu_1 \in \ll P_1
				\rr^{DS}_{graph(s,DS)}$, $\mu = \mu_1 \cup [u \rightarrow
				s]$ and  $\mu_1 \sim [u \mapsto s]$. By 
				induction hypothesis we get that $\mu \in Q_1(D)$.
				Looking at the construction of $Q$ we get $\mu \in Q(D)$.
		\end{enumerate}

		\bigskip\noindent
		$ds \in \V, g \in \U:$ \\
		We need to show 
		$\bigcup\limits_{ds' \in dom(ep)} \{ \mu \cup [ds \mapsto ds'] \mid \mu \in
			\ll P\rr^{ep(ds')}_{graph(g,ep(ds'))}, \mu \sim
		[ds\mapsto ds'] \}  \subseteq Q(D)$.
		Let $ds' \in dom(ep)$ be arbitrary.
		Let $\mu \in \ll P \rr^{ep(ds')}_{graph(g,ds')}$ where 
		$\mu \sim [ds \mapsto ds']$ holds be arbitrary.
		Proceed by case distinction:
		\begin{enumerate}
			\item $u \in names(ep(ds')):$ We have $\mu \in \ll P_1
				\rr^{ep(ds')}_{graph(u,ep(ds'))}$ by SPARQL semantics. The
				induction hypothesis $\bigcup\limits_{ds' \in dom(ep)} \{ \mu \cup [ds \mapsto ds'] \mid \mu \in
					\ll P_1 \rr^{ep(ds')}_{graph(u,ep(ds'))}, \mu \sim
				[ds\mapsto ds'] \}  = Q_1(D)$ yields $\mu\cup [ds\mapsto ds'] \in Q_1(D)$. 
				By construction of our query $Q$ we see that $\mu \cup [ds\mapsto ds'] \in Q(D)$.
			\item $u \in \U \backslash names(ep(d))$:
				Then $\ll P_1 \rr^{ep(ds')}_{graph(u,ep(ds'))} = \{\}$. But then
				$Q(D) = \{\}$ because we added $LOC(u,ds)$ to the root of our
				query Q.
			\item $u \in V$:
				then $\mu \in S_1$ where $S_1 =  \bigg\{\mu_1 \cup [ u \rightarrow s ] \mid
					s \in names(ep(ds')),\\ \mu_1 \in \ll P_1
					\rr^{ep(ds')}_{graph(s,ep(ds'))} \land [ u \rightarrow s ] \sim
				\mu_1 \bigg\}$. 
				By induction hypothesis we know that we receive a wdpt $Q_1
				= trans(P_1,ds,u)$ for which \\
				$\bigcup\limits_{g'\in names(ep(ds')),ds' \in dom(ep)} \{ \mu_1 \cup
					\{[u\mapsto g'],[ds \mapsto ds']\} \mid\\
					\mu_1 \in \ll P_1 \rr^{ep(ds')}_{graph(g',ep(ds'))}, \mu_1 \sim
					\{[u \mapsto g'], [ds \mapsto ds']\} = Q_1(D)$ holds. 
					Because $\mu \in S_1$ there	is an $s \in names(ep(ds'))$ such
					that there is a $\mu_1 \in \ll P_1
					\rr^{ep(ds')}_{graph(s,ep(ds'))}$,
					$\mu = \mu_1 \cup \{[u \rightarrow
					s]\}$. We get that $\mu \in Q_1(D)$ by induction hypothesis.
					Also we have that $\mu \cup [ds \rightarrow ds'] \in Q(D)$
					because we conjunctively added $LOC(g,ds)$ to the root of $Q$ and we
					assumed $\mu \sim [ds \sim ds']$ .
			\end{enumerate}

			\bigskip\noindent
			$ds \in \U, g \in \V:$ \\ Let $DS = ep(ds)$.
			We need to show\\
			$\bigcup\limits_{g' \in names(DS)} \{ \mu \cup [g\mapsto g'] \mid \mu \in
				\ll P \rr^{DS}_{graph(g',DS)}, \mu \sim
			[g\mapsto g'] \}  \subseteq Q(D) $.
			Let  $g' \in names(ep(ds))$ be arbitrary.
			Let $\mu \in \ll P \rr^{DS}_{graph(g',DS)}$ where $\mu \sim [g
			\mapsto g']$ holds be arbitrary.
			Proceed by case distinction:
			\begin{enumerate}
				\item $u \in names(DS)$:
					By SPARQL semantics we have that $\mu \in \ll P_1
					\rr^{DS}_{graph(u,DS)}$.
					We know that we receive a wdpt $Q_1	= trans(P_1,ds,u)$ 
					for which by i.h.
					$\ll P_1 \rr^{DS}_{graph(u,DS)} = Q_1(D)$ holds. Thus $\mu
					\in Q_1(D)$.
					Because of the construction of our query and especially the
					conjunct $LOC(g,ds)$ in the root of $Q$ we have that 
					$\mu \cup [g\mapsto g'] \in Q_1(D)$.

				\item $u \in \U \backslash names(ep(ds))$:
					then $\ll P_1 \rr^{DS}_{graph(u,DS)} = \{\}$. But then
					$Q(D) = \{\}$ because we added $LOC(u,ds)$ to the root of
					our query Q.
				\item $u \in V$:
					then $\mu \in S_1$ where $S_1 =  \bigg\{\mu_1 \cup [ u \rightarrow s ] \mid
						s \in names(DS), \mu_1 \in \ll P_1
						\rr^{DS}_{graph(s,DS)} \land [ u \rightarrow s ] \sim
					\mu_1 \bigg\}$. 
					By induction hypothesis we know that we receive a wdpt $Q_1
					= trans(P_1,ds,u)$ for which\\ $\bigcup\limits_{g'\in names(DS)} 
					\{ \mu_1 \cup \{[u\mapsto g']\} \mid 
						\mu_1 \in \ll P_1 \rr^{DS}_{graph(g',DS)}, 
					\mu_1 \sim \{[u \mapsto g']\}\} = Q_1(D)$ holds. Because $\mu \in S_1$ there
					is an $s \in names(DS)$ such  that there is a $\mu_1 \in \ll P_1
					\rr^{DS}_{graph(s,DS)}$, 
					$\mu = \mu_1 \cup \{[u \rightarrow s]\}$. 
					We get that $\mu \in Q_1(D)$ by induction hypothesis. 
					We thus get that $\mu\cup[g\mapsto g'] \in Q(D)$ because we
					conjunctively added $LOC(g,ds)$ to root of $Q$ and we
					asssumed $\mu \sim [g\mapsto g']$.
			\end{enumerate}

			\bigskip\noindent
			$ds,g \in \V:$ \\
			We need to show 
			$\bigcup\limits_{g' \in names(ep(ds')), ds' \in dom(ep)} \{ \mu \cup
				\{[g \mapsto g'],[ds \mapsto ds']\} \mid \mu \in
				\ll P\rr^{ep(ds')}_{graph(g',ep(ds'))}, \mu \sim
			\{[ds \mapsto ds'][g\mapsto g']\} \}  \subseteq Q(D) $.
			Let  $ds' \in dom(ep)$ and $g' \in names(ep(ds'))$ be arbitrary.
			Let $\mu \in \ll P \rr^{ep(ds')}_{graph(g',ep(ds'))}$ where $\mu
			\sim\{[ds\mapsto ds'], [g\mapsto g'] \}$ holds be arbitrary.
			Proceed by case distinction:
			\begin{enumerate}
				\item $u \in names(ep(ds))$:\\
					By SPARQL semantics we have $\mu \in \ll P_1
					\rr^{DS}_graph(u,DS)$.
					We know that we receive a wdpt $Q_1	= trans(P_1,ds',u)$ 
					for which by i.h.
					$\bigcup\limits_{ds' \in dom(ep)} \{ \mu \cup [ds \mapsto ds'] \mid \mu \in
						\ll P_1\rr^{ep(ds')}_{graph(u,ep(ds'))}, \mu \sim
					[ds\mapsto ds'] \}  = Q_1(D) $
					holds. Thus $\mu \cup [ds \mapsto ds'] \in Q_1(D)$.
					By construction of our query and especially the conjunct in the
					root of $Q$, i.e., $LOC(g,ds)$ we have that $\mu \cup
					\{[ds\mapsto ds'],[g \mapsto g']\} \in Q(D)$.

				\item $u \in \U \backslash names(ep(ds'))$
					then $\ll P_1 \rr^{ep(ds')}_{graph(u,ep(ds'))} = \{\}$. But then
					$Q(D) = \{\}$ because we added $LOC(u,ds)$ to the root of
					our query. 

				\item $u \in V$:
					then $\mu \in S_1$ where $S_1 =  \bigg\{\mu_1 \cup [ u \rightarrow s ]
						\mid s \in names(ep(ds')),\\ \mu_1 \in \ll P_1
						\rr^{ep(ds')}_{graph(s,ep(ds'))} \land [ u \rightarrow s ] \sim
					\mu_1 \bigg\}$. 
					By induction hypothesis we know that we receive a wdpt $Q_1
					= trans(P_1,ds',u)$ for which \\
					$\bigcup\limits_{ds' \in dom(ep), g' \in names(ep(ds'))} \{ \mu
						\cup \{[ds \mapsto ds'],[u \mapsto g']\} \mid \\ \mu \in
						\ll P\rr^{ep(ds')}_{graph(g',ep(ds'))}, 
					\mu \sim \{[ds\mapsto ds'], [g \mapsto g']\}\} = Q_1(D) $
					holds. Because $\mu \in S_1$ there
					is an $s \in names(ep(ds'))$ such  that there is a $\mu_1 \in \ll P_1
					\rr^{ep(ds')}_{graph(s,ep(ds'))}$, $\mu = \mu_1 \cup \{[u \rightarrow
					s]\}$ and  $\mu_1 \sim \{[u \mapsto s]\}$ by induction hypothesis. 
					We thus get that $\mu\cup \{[g\mapsto g'],[ds \mapsto ds']\} \in Q(D)$ because we
					conjunctively added $LOC(g,ds)$ to $Q$ and we assumed $\mu
					\sim \{[g\mapsto g'],[ds \mapsto ds']\}$.
			\end{enumerate}

			\bigskip\noindent$\supseteq:$\\
			$ds,g \in \U$\\
			Let $\mu \in Q(D)$ be arbitrary.
			\begin{enumerate}
				\item $u$ is a constant: Because $\mu \in Q(D)$ a part of $\mu$
					must also satisfy $Q_1(D)$ by construction of $Q$. %then $Q_1 = trans(P_1,ds,u)$ and 
					We have by induction hypothesis that $Q_1(D) = \ll P_1
					\rr^{ep(ds)}_{u}$. And thus by SPARQL semantics 
					$\mu \in \ll P\rr^{ep(ds)}_{g}$ holds.
				\item $u$ is a variable: By induction hypothesis
					$\bigcup\limits_{g' \in names(ep(ds))} \{ \mu_1 \cup [u\mapsto
						g'] \mid \mu_1 \in \ll P_1
						\rr^{ep(ds)}_{graph(g',ep(ds))}, \mu_1 \sim [u
					\mapsto g']\} = Q_1(D)$ holds. 
					By the fact that $\mu \in Q(D)$ and both $g$ and $ds$ are
					URIs we know that $\mu \in Q_1(D)$ by construction of $Q$. 
					By induction hypothesis we have $\mu = \mu_1 \cup [u \mapsto g']$ for some $g' \in
					names(ep(ds))$. We know that $\mu_1 \in \ll P_1
					\rr^{ep(ds)}_{graph(g',ep(ds))}$  and $\mu_1 \sim
					[u\mapsto g']$ by induction hypothesis.
					But this means by semantics of the GRAPH operator that $\mu \in \ll P
					\rr^{ep(ds)}_{graph(g,ep(ds))}$.
			\end{enumerate}

			\bigskip\noindent
			$ds \in \V,g \in \U$\\
			We want to show $\bigcup\limits_{ds' \in dom(ep)} \{ \mu \cup [ds
				\mapsto ds'] 
				\mid \mu \in \ll P\rr^{ep(ds')}_{graph(g,ep(ds'))}, \mu \sim
			[ds\mapsto ds'] \}  \supseteq Q(D)$.

			Let $\sigma \in Q(D)$ be arbitrary:
			\begin{enumerate}
				\item $u$ is a constant:
				   %Because $\sigma \in Q(D)$ a part of $\sigma$ must also satisfy
				   %$Q_1(D)$ by construction of $Q$.	
					We have by induction hypothesis that 
					$\bigcup\limits_{ds' \in dom(ep)} \{ \mu \cup [ds \mapsto ds'] \mid \mu \in
						\ll P_1\rr^{ep(ds')}_{graph(u,ep(ds'))}, \mu \sim
					[ds\mapsto ds'] \}  \supseteq Q_1(D)$. A part of
					$\sigma$, call it $\mu$ must fulfill $Q_1$ because of the
					construction of $Q$. Thus $\mu \in \ll P_1
					\rr^{ep(ds')}_{graph(u,ep(ds'))}$  and by SPARQL semantics $\mu \in \ll P_1
					\rr^{ep(ds')}_{graph(g,ep(ds'))}$ hold. 
					Looking at the construction of $Q$ we see that 
					$\sigma = \mu \cup [ds \mapsto ds']$ for some $ds' \in
					dom(ep)$ thus $\mu \sim [ds \mapsto ds']$ .

				\item $u$ is a variable: By induction hypothesis
					$\bigcup\limits_{ds' \in dom(ep), g' \in names(ep(ds'))} \{
						\mu_1 \cup \{[ds \mapsto ds'],[u \mapsto g']\} \mid \mu_1 \in
						\ll P_1\rr^{ep(ds)}_{graph(g',ep(ds))}, 
						\mu_1 \sim
					\{[ds \mapsto ds'], [u \mapsto g']\}\} = Q_1(D) $  
					holds.  Some submapping of $\sigma$ must fulfill $Q_1(D)$ by
					construction of $Q$ and $\sigma \in Q(D)$. Call it $\mu$.
					%Also $\sigma(ds) = ds'$ for some $ds' \in dom(ep)$ by
					%construction of $Q$ and $D$.
					By the fact that $\mu \in Q_1(D)$ we thus know by induction
					hypothesis that
					$\mu = \mu_1 \cup \{[u \mapsto g'],[ds \mapsto ds']\}$
					for some $g' \in names(ep(ds'))$ 
					and some $ds' \in dom(ep)$. Also, we know that $\mu_1 \in \ll P_1
					\rr^{ep(ds')}_{graph(g',ep(ds'))}$ and $\mu_1 \sim
					[u\mapsto g']$.
					But this means by semantics of the GRAPH operator that
					$(\mu_1\cup[u\mapsto ds]) \in \ll P
					\rr^{ep(ds')}_{graph(g,ep(ds'))}$.
					Because $\mu \in Q_1(D)$ and $\mu = \mu_1
					\cup \{ [ds \mapsto ds'],
					[u \mapsto g'] \}$ implies
					$(\mu_1\cup\{u \mapsto g'\}) \sim   [ds
					\mapsto ds']$ we are done.  
			\end{enumerate}

			\bigskip\noindent
			$ds \in \U,g \in \V$: Let $DS = ep(ds)$.
			Let $\sigma(g) = f$.  By construction of $Q$ and $D$ we
			get that $f \in names(ep(ds))$.

			We want to show 
			$\bigcup\limits_{f \in names(DS)}\{ \mu \cup [g \mapsto f] \mid \mu \in
				\ll P\rr^{DS}_{graph(f,DS)} , \mu \sim
			[g\mapsto f] \} \supseteq Q(D)$.
			Let $\sigma \in Q(D)$ be arbitrary.
			\begin{enumerate}
				\item 					By induction hypothesis we know that 
					$Q_1(D) = \ll P_1\rr^{DS}_{u}$. Looking at our mapping
					$\sigma$ we know that there is a part of $\sigma$, call it $\mu$
					for which $\mu \in Q_1(D)$ and thus $\mu \in \ll P_1
					\rr^{DS}_{u}$ and  by SPARQL semantics $\mu \in \ll P_1
					\rr^{DS}_{f}$ hold. 
					We have $\mu \sim [g \mapsto f]$ for some 
					$f \in names(DS)$ because of the
					conjunct $LOC(g,ds)$ in the root of $Q$ and $\sigma = \mu \cup [g \mapsto f]$.
				\item If $u$ is a variable then	consider $Q_1 = trans(P_1,ds,
					u)$. By induction hypothesis
					$\bigcup\limits_{g'\in names(DS)} \{ \mu_1 \cup [u\mapsto g'] \mid
						\mu_1 \in \ll P_1 \rr^{DS}_{graph(g',DS)}, \mu_1 \sim [u
					\mapsto g']\} = Q_1(D)$ holds. 
					Some part of $\sigma$ must satisfy $Q_1$ by the construction of
					$Q$ and $\sigma \in Q(D)$, call it $\mu$.
					We first show that $\mu \in \ll P\rr^{DS}_{graph(f,DS)}$:
					This means, we need to show that $\mu \in \ll P_1
					\rr^{DS}_{graph(s,DS}$ for $s \in names(DS)$ and $[u \mapsto s]
					\sim \mu$.
					By our i.h. and $\mu \in Q_1(D)$ we know that
					$\mu = \mu_1 \cup \{[u \mapsto s] \}$, for some $s \in
					names(DS)$. We know again by i.h. that $\mu_1 \in \ll P_1
					\rr^{DS}_{graph(s,DS)}$ and $\mu_1 \sim [u\mapsto s']$.
					But this means by semantics of the GRAPH operator that $\mu_1
					\cup [u \mapsto s] \in \ll P
					\rr^{DS}_{graph(f,DS)}$ It remains to show that $\mu \sim
					[g \mapsto f]$. Looking at the construction of the query we know
					that $LOC(g,ds)$ must be fulfilled by $\sigma$. 
					Thus $\mu \sim [g\mapsto f]$
					because $\mu$ is a part of $\sigma$.
			\end{enumerate}

			\bigskip\noindent
			$ds,g \in \V$:\\
			We want to show 
			$\bigcup\limits_{ds' \in dom(ep), f \in names(ep(ds'))} \{ \mu \cup
				\{[ds \mapsto ds'],[g \mapsto f]\} \mid \mu \in
				\ll P\rr^{ep(ds')}_{graph(g',ep(ds'))}, 
				\mu \sim
			\{[ds\mapsto ds'], [g \mapsto f]\}\} \supseteq Q(D) $
			Let $\sigma \in Q(D)$ be arbitrary. Let $\sigma(ds) = ds'$ and
			$\sigma(g) = f$. By the construction of $Q$ and $D$, $ds' \in
			dom(ep)$ and $f \in names(ep(ds'))$.
			\begin{enumerate}
				\item $u$ is a constant: \\
					By induction hypothesis we know that 
					$\bigcup\limits_{ds' \in dom(ep)} \{ \mu \cup [ds \mapsto ds'] \mid \mu \in
						\ll P_1 \rr^{ep(ds')}_{graph(g,ep(ds'))}, \mu \sim
					[ds\mapsto ds'] \}  = Q_1(D) $. Looking at our mapping
					$\sigma$ we know that there is a part of $\sigma$, call it $\mu$
					for which $\mu \in Q_1(D)$ and thus
					$\mu \backslash [ds\mapsto ds'] \in \ll P_1
					\rr^{ep(ds')}_{u}$. By semantics of graph we have 
					$\mu \backslash [ds\mapsto ds'] \in \ll P_1
					\rr^{ep(ds')}_{f}$.
					We have $\mu \backslash [ds \mapsto ds'] \sim [g \mapsto f]$ 
					and $\mu \sim [ds \mapsto ds']$ because  $\mu \backslash [ds \mapsto
					ds']\subseteq \sigma$, $\sigma \in Q(D)$ and 
					because of the conjunct $LOC(g,ds)$ in the root of $Q$.
				\item $u$ is a variable:
					By induction hypothesis
					$\bigcup\limits_{ds' \in dom(ep), g' \in names(ep(ds'))} \{ \mu
						\cup \{[ds \mapsto ds'],[u \mapsto g']\} \mid \mu \in
						\ll P_1\rr^{ep(ds')}_{graph(g',ep(ds'))}, 
						\mu \sim
					\{[ds\mapsto ds'], [u \mapsto g']\}\} = Q_1(D) $  
					holds. 
					Some part of $\sigma$ must satisfy $Q_1$ by the construction of
					$Q$ and $\sigma \in Q(D)$, call it $\mu$. Let $\mu(u) = g'$
					for some $g' \in names(ds')$. Because $\mu \in
					Q_1(D)$ we can use the induction hypothesis:
					This means $\mu\backslash\{[ds \mapsto ds'],[u
					\mapsto g']\} \in \ll
					P_1\rr^{ep(ds')}_{graph(g',ep(ds'))}$. Obviously $\mu
					\backslash\{[ds \mapsto ds']\} \sim [u \mapsto g']\}$.
					From the semantics of the GRAPH operator we get 
					$\mu\backslash\{[ds \mapsto ds']\} \in \ll
					P_1\rr^{ep(ds')}_{graph(f,ep(ds'))}$.
					It remains to show that $\mu \sim
					[g \mapsto f]$ and $\mu \sim [ds \mapsto ds']$. Looking at
					the construction of the query we know
					that $LOC(g,ds)$ must be fulfilled by $\sigma$. 
					Thus $\mu \sim [g\mapsto f]$
					because $\mu$ is a part of $\sigma$ the same arguments
					hold for $\mu \sim [ds \mapsto ds']$.
			\end{enumerate}

			%%% SERVICE %%%%%
		\item Consider the case where $P$ is a graph 
			pattern of the form $(\mbox{SERVICE} \ u \ P_1)$.
			$trans$ checks if $u \in \U$ and $u \notin dom(ep)$. If this is the
			case $Q: \{\} \rightarrow$. Otherwise we let $Q = trans(P_1,u,g)$
			and add $LOC(u,ds)$ and $LOC(g,ds)$ to the root of $Q$.\\
			$\subseteq:$\\
			Let $ds,g \in U$. Let $DS = ep(ds)$ and $G = ep(g,DS)$.
			Let $\mu \in \ll P \rr^{DS}_{G}$ be arbitrary.
			\begin{enumerate}
				\item  $u \in dom(ep)$:
					that means that $\mu \in \ll P_1
					\rr^{ep(u)}_{graph(def,ep(u))}$
					but by i.h. we know that 
					$\ll P_1 \rr^{ep(u)}_{graph(def,ep(u))}$ =
					$Q_1(D)$.
					By construction of $Q$ we have $\mu \in Q(D)$.
				\item $u \in \U\backslash dom(ep)$:
					Thus $\mu$ = $\mu_\emptyset$ which is the empty
					mapping. This is the same mapping our query
					$\{\} \leftarrow$ returns and we are
					done.
				\item $u \in \V$:
					Thus $\mu \in \{ \mu_1 \cup [u \rightarrow s ] \mid
						s \in dom(ep), \mu_1 \in \ll P_1
						\rr^{ep(s)}_{graph(def,ep(s))} \land
					[u \rightarrow s] \sim \mu_1 \}$ by semantics. 
					Assume $\mu(u) = s$.
					By induction hypothesis we know that 
					$\bigcup\limits_{ds' \in dom(ep)} \{ \mu_1 \cup [u
						\mapsto ds'] \mid \mu_1 \in
						\ll P_1\rr^{ep(ds')}_{graph(def,ep(ds'))},
					\mu_1 \sim [u\mapsto ds'] \}  =
					Q_1(D) $. By semantics of SPARQL $\mu = \mu_1 \cup
					[u \mapsto s]$ 
					for $\mu_1 \in \ll P_1 \rr^{ep(s)}_{graph(def,ep(s))}$. 
					By induction hypothesis we can conclude
					$\mu \in Q_1(D)$.
					By construction of $Q$ we have have $\mu = (\mu_1 \cup [u
					\mapsto s])	\in Q(D)$. 
			\end{enumerate}

			\bigskip\noindent
			Let $ds \in \V,g \in \U$.
			We want to show\\ 
			$\bigcup\limits_{ds' \in dom(ep)} \{ \mu \cup [ds \mapsto ds'] \mid \mu \in
			\ll P\rr^{ep(ds')}_{graph(g,ep(ds'))}, \mu \sim  [ds\mapsto ds'] \}  
			\subseteq Q(D)$.
			Let $ds' \in dom(ep)$ be arbitrary.
			Let $\mu \in \ll P \rr^{ep(ds')}_{graph(g,ds')}$ so that $\mu \sim
			[ds\mapsto ds']$.
			\begin{enumerate}
				\item  $u \in dom(ep)$:
					$\mu \in \ll P_1\rr^{ep(u)}_{graph(def,ep(u))}$ by semantics.
					By i.h. we know that 
					$\ll P_1 \rr^{ep(u)}_{graph(def,ep(u))}$ =
					$Q_1(D)$. By construction of $q$ and especially the conjunct
					$LOC(ds,g)$ in the root, we get $\mu\cup[ds \mapsto ds'] \in Q$.
				\item $u \in I\backslash dom(ep)$:
					By semantics $\mu$ = $\mu_\emptyset$ which is the empty
					mapping. This is the same mapping our query
					$\{\} \leftarrow$ returns and we are
					done.
				\item $u \in \V$:\\
					By semantics $\mu \in \{ \mu_1 \cup [u \rightarrow s ] \mid
						s \in dom(ep), \mu_1 \in \ll P_1
						\rr^{ep(s)}_{graph(def,ep(s))} \land
					[u \rightarrow s] \sim \mu_1 \}$. Assume $\mu(u) = s$.
					By induction hypothesis we know that 
					$\bigcup\limits_{ds' \in dom(ep)} \{ \mu_1 \cup [u
						\mapsto ds'] \mid \mu_1 \in
						\ll P_1\rr^{ep(ds')}_{graph(def,ep(ds'))},
					\mu_1 \sim [u\mapsto ds'] \}  =
					Q_1(D) $. By semantics of SPARQL $\mu = \mu_1 \cup
					[u \mapsto s]$ for $\mu_1 \in \ll P_1 \rr^{ep(s)}_{graph(def,ep(s))}$.
					By induction hypothesis we can conclude
					$\mu \in Q_1(D)$.
					By construction of $Q$ and especially the conjunct
					$LOC(ds,g)$ in the root, we have 
					$\mu \cup [ds\mapsto ds'] = (\mu_1 \cup [u
					\mapsto s] \cup [ds \mapsto ds']) \in Q(D)$. 
			\end{enumerate}

			\bigskip\noindent
			Let $ds \in \U,g \in \V$.
			We want to show  $\bigcup\limits_{g' \in names(ep(ds))}\{ \mu \cup [g
				\mapsto g'] \mid \mu \in \ll P\rr^{ep(ds)}_{graph(g',ep(ds))},
			\mu \sim	[g\mapsto g'] \} \subseteq Q(D)$.
			Let $g \in names(ep(ds))$ be arbitrary.
			Let $\mu \in \ll P \rr^{ep(ds)}_{graph(g',ep(ds))}$ so that $\mu \sim
			[g\mapsto g']$.

			\begin{enumerate}
				\item  $u \in dom(ep)$:
					By semantics $\mu \in \ll P_1
					\rr^{ep(u)}_{graph(def,ep(u))}$
					and by i.h. we know that $\ll P_1 \rr^{ep(u)}_{graph(def,ep(u))}$ =	$Q_1(D)$.
					By construction of $Q$ and especially the conjunct
					$LOC(ds,g)$ in the root of $Q$ we have that $\mu\cup[g \mapsto g'] \in Q$.
				\item $u \in \U\backslash dom(ep)$:
					By semantics $\mu$ = $\mu_\emptyset$ which is the empty
					mapping. This is the same mapping our query
					$\{\} \leftarrow$ returns and we are
					done.
				\item $u \in \V$:
					By semantics $\mu \in \{ \mu_1 \cup [u \rightarrow s ] \mid
						s \in dom(ep), \mu_1 \in \ll P_1
						\rr^{ep(s)}_{graph(def,ep(s))} \land
					[u \rightarrow s] \sim \mu_1 \}$. Assume $\mu(u) = s$.
					By induction hypothesis we know that 
					$\bigcup\limits_{ds' \in dom(ep)} \{ \mu_1 \cup [u
						\mapsto ds'] \mid \mu_1 \in
						\ll P_1\rr^{ep(ds')}_{graph(def,ep(ds'))},
					\mu_1 \sim [u\mapsto ds'] \} = Q_1(D) $.
					By semantics of SPARQL $\mu = \mu_1 \cup
					[u \mapsto s]$ for  $\mu_1 \in \ll
					P_1\rr^{ep(s)}_{graph(def,ep(s))}$. 
					By induction hypothesis we can conclude
					$\mu \in Q_1(D)$.
					By construction of $Q$ and especially the conjunct
					$LOC(ds,g)$ we have $(\mu \cup [g \mapsto g']) = (\mu_1 \cup [u \mapsto s] 
					\cup [g	\mapsto g']) \in Q(D)$. 
			\end{enumerate}

			\bigskip\noindent
			Let $ds,g \in \V$.
			We want to show  	
			$\bigcup\limits_{ds' \in dom(ep), g' \in names(ep(ds'))} \{ \mu \cup
				\{[ds \mapsto ds'],[g \mapsto g']\} \mid \mu \in
				\ll P\rr^{ep(ds')}_{graph(g',ep(ds'))}, 
				\mu \sim
			\{[ds \mapsto ds'], [g \mapsto g']\}\} = Q(D)$.
			Let $ds' \in dom(ep)$ and $g' \in nameS(ep(ds'))$ be arbitrary.
			Let $\mu \in \ll P \rr^{ep(ds')}_{graph(g,g')}$ so that $\mu \sim
			[ds\mapsto ds']$.
			\begin{enumerate}
				\item  $u \in dom(ep)$:
					By semantics $\mu \in \ll P_1
					\rr^{ep(u)}_{graph(def,ep(u))}$
					but by i.h. we know that 
					$\ll P_1 \rr^{ep(u)}_{graph(def,ep(u))}$ =
					$Q_1(D)$.
					By construction of $Q$ and especially the conjunct
					$LOC(ds,g$ in the root of $Q$ we
					have that $\mu\cup \{[ds \mapsto ds'][g \mapsto g']\} \in Q$.
				\item $u \in \U\backslash dom(ep)$:
					By semantics $\mu$ = $\mu_\emptyset$ which is the empty
					mapping. This is the same mapping our query
					$\{\} \leftarrow$ returns and we are
					done.
				\item $u \in \V$:
					
					By semantics $\mu \in \{ \mu_1 \cup [u \rightarrow s ] \mid
						s \in dom(ep), \mu_1 \in \ll P_1
						\rr^{ep(s)}_{graph(def,ep(s))} \land
					[u \rightarrow s] \sim \mu_1 \}$. Assume $\mu(u) = s$. We know that $s \in dom(ep)$.
					By induction hypothesis we know that 
					$\bigcup\limits_{ds' \in dom(ep)} \{ \mu_1 \cup [u
						\mapsto ds'] \mid \mu_1 \in
						\ll P_1\rr^{ep(ds')}_{graph(def,ep(ds'))},
					\mu_1 \sim [u\mapsto ds'] \}  = Q_1(D) $. 
					By semantics of SPARQL $\mu = \mu_1 \cup
					[u \mapsto s]$ for $\mu_1 \in \ll P_1\rr^{ep(s)}_{graph(def,ep(s))}$. 
					By induction hypothesis we can conclude
					$\mu \in Q_1(D)$.
					By construction of $Q$ and especially its conjunct in the
					root $LOC(ds,g)$ we have 
					$(\mu \cup \{[ds \mapsto ds'][g \mapsto g']\}) =
					\mu_1 \cup \{[ds \mapsto ds'][g \mapsto g'], [u
					\mapsto s]\} \in Q(D)$. 
			\end{enumerate}



			\bigskip\noindent$\supseteq:$\\
			Let $ds,g \in \U$ and $\mu \in Q(D)$ be arbitrary.
			\begin{enumerate}
				\item Assume $u$ is an URI. 
					By i.h. we know that 
					$\ll P_1 \rr^{ep(u)}_{graph(def,ep(u))}$ =
					$Q_1(D)$ and thus by construction of the query $Q$ and the
					fact that $\mu \in Q(D)$, $\mu \in \ll P
					\rr^{ep(ds)}_{graph(g,ep(ds))}$
				\item Assume $u$ is a variable. By induction hypothesis we
					know that  $\bigcup\limits_{ds' \in dom(ep)} \{ \mu_1 \cup [u
						\mapsto ds'] \mid \mu_1 \in
						\ll P_1\rr^{ep(ds')}_{graph(def,ep(ds'))},
					\mu_1 \sim [u\mapsto ds'] \}  = Q_1(D) $.
					%As $ Q = Q_1(D)  \land LOC(ds,g) \land  LOC(u,g)$ and the fact that
					%$\mu \in Q(D)$ we can instantly see that $\mu \in \ll P
					%\rr^{ep(ds)}_{graph(g,ep(ds))}$:
					Assume w.l.o.g. $\mu(u) = s$.
					$\mu_1 = \mu \backslash [u \mapsto s]$. By induction hypothesis we know
					that $\mu_1 \in \ll P_1 \rr^{ep(s)}_{graph(def,ep(s))}$. By
					construction of $Q$ especially the conjunct $LOC(u,g)$, $s \in dom(ep)$ 
					and by our induction hypothesis $\mu_1 \sim	[u \mapsto s]$ holds.
					By semantics $\mu \in \ll P \rr^{ds}_{graph(g,ep(ds))}$ follows.
			\end{enumerate}

			\bigskip\noindent
			Let $ds \in \V, g \in \U$ and $\sigma \in Q(D)$ be arbitrary.
			We need to show  $\bigcup\limits_{ds' \in dom(ep)} \{ \mu \cup [ds
				\mapsto ds'] \mid \mu \in
				\ll P\rr^{ep(ds')}_{graph(g,ep(ds'))}, \mu \sim
			[ds \mapsto ds'] \}  \supseteq Q(D) $.
			Assume that $\sigma(ds) = ds'$, because of the conjunct $LOC(ds,g)$
			we have that $ds' \in dom(ep)$.
			\begin{enumerate}
				\item Assume $u$ is an URI. 
					By i.h. we know that 
					$\ll P_1 \rr^{ep(u)}_{graph(def,ep(u))}$ =
					$Q_1(D)$. By the fact that $\sigma \in Q(D)$ and the
					construction of $Q$ we can deduce that for a part of $\sigma$,
					call it $\mu$, $\mu \in Q_1(D)$ holds. By i.h. we get
					$\mu \in  \ll P_1 \rr^{ep(u)}_{graph(def,ep(u))}$.
					From this we can deduce $\mu \in  \ll P_1
					\rr^{ep(ds')}_{graph(g,ep(ds'))}$. Because $[ds \mapsto
					ds'] \in \sigma$ and $\mu \subseteq \sigma$ we have $\mu
					\sim [ ds \mapsto ds']$.

					%and  because $Q = Q_1 \land LOC(ds,g) \land
					%LOC(u,g)$ 
					%we have $\sigma = \mu \cup [ ds \mapsto ds']$ for some $\mu \in Q_1 (D)$.
					%It is by construction of our query thus obvious that
					%$\mu \sim [ds \mapsto ds']$ 
					%and $\mu \in \ll P \rr^{ep(ds')}_{graph(g,ep(ds')}$ 
					%for some $ds'\in dom(ep)$ hold.
				\item Assume $u$ is a variable. By induction hypothesis we
					know that  $\bigcup\limits_{ds' \in dom(ep)} \{ \mu_1 \cup [u
						\mapsto ds'] \mid \mu_1 \in
						\ll P_1\rr^{ep(ds')}_{graph(def,ep(ds'))},
					\mu_1 \sim [u\mapsto ds'] \}  = Q_1(D)$.
					Let $\mu = \sigma \backslash [ds \mapsto ds']$. 
					We can see that $\mu \in \ll P
					\rr^{ep(ds')}_{graph(g,ep(ds'))}$ for some $ds' \in dom(ep)$:
					Assume w.l.o.g. $\mu(u) = s$.
					$\mu_1 = \mu \backslash [u \mapsto s]$. By induction hypothesis we know
					that $\mu_1 \in \ll P_1 \rr^{ep(s)}_{graph(def,ep(s))}$. By our
					construction $s \in dom(ep)$ and $\mu_1 \sim
					[u \mapsto s]$ holds. It remains to show that $\mu \sim
					[ds \mapsto ds']$
					which follows from the construction of the query $Q$, i.e., the conjunct
					$LOC(ds,g)$ in the root of $Q$ and the fact that $\sigma \in Q(D)$.
			\end{enumerate}

			\bigskip\noindent
			Let $ds \in \U, g \in \V$ and $\sigma \in Q(D)$ be arbitrary.
			We need to show 
			$\bigcup\limits_{g' \in names(ep(ds))}\{ \mu \cup [g \mapsto g'] \mid \mu \in
				\ll P\rr^{ep(ds)}_{graph(g',ep(ds))} , \mu \sim
			[g\mapsto g'] \} \supseteq Q(D) $. 
			Assume that $\sigma(g) = g'$, because of the conjunct $LOC(ds,g)$
			we have that $g' \in names(ep(ds))$.
			\begin{enumerate}
				\item Assume $u$ is an URI. 
					By i.h. we know that 
					$\ll P_1 \rr^{ep(u)}_{graph(def,ep(u))}$ =
					$Q_1(D)$. By the fact that $\sigma \in Q(D)$ and the
					construction of $Q$ we can deduce that for a part of $\sigma$,
					call it $\mu$, $\mu \in Q_1(D)$ holds. By i.h. we get
					$\mu \in  \ll P_1 \rr^{ep(u)}_{graph(def,ep(u))}$.
					From this we can deduce $\mu \in  \ll P_1
					\rr^{ep(ds')}_{graph(g,ep(ds'))}$. Because $[g \mapsto
					g'] \in \sigma$ and $\mu \subseteq \sigma$ we have $\mu
					\sim [ g \mapsto g']$.
					%By i.h. we know that 
					%$\ll P_1 \rr^{ep(u)}_{graph(def,ep(u)}$ =
					%$Q_1(D)$ and  because $Q = Q_1 \land LOC(ds,g) \land
					%LOC(u,g)$ we have $\sigma = \mu \cup [ ds \mapsto ds']$ 
					%for some $\mu \in Q_1 (D)$.
					%It is by construction of our query obvious that $\mu
					%\sim [g\mapsto g']$  
					%and $\mu \in \ll P \rr^{ep(ds)}_{graph(g',ep(ds))}$ 
					%for some $g' \in names(ep(ds))$ hold.

				\item Assume $u$ is a variable. By induction hypothesis we
					know that  $\bigcup\limits_{ds' \in dom(ep)} \{ \mu_1 \cup [u
						\mapsto ds'] \mid \mu_1 \in
						\ll P_1\rr^{ep(ds')}_{graph(def,ep(ds'))},
					\mu_1 \sim [u\mapsto ds'] \}  = Q_1(D) $.
					%We have by construction $ Q = Q_1(D)  \land LOC(ds,g) \land  LOC(u,g)$
					%and by assumption 
					%$\sigma \in Q(D)$. 
					%Let $\sigma(g) = g'$ by the fact that
					%$\sigma \in Q(D)$ and $Q =Q_1 \land LOC(ds,g) \land
					%LOC(u,g)$ we have that $g \in names(ds))$. 
					Let $\mu = \sigma \backslash [g
					\mapsto g']$.
					We can see that $\mu \in \ll P
					\rr^{ep(ds)}_{graph(g',ep(ds))}$:
					Assume w.l.o.g. $\mu(u) = s$.
					$\mu_1 = \mu \backslash [u \mapsto s]$. By induction hypothesis we know
					that $\mu_1 \in \ll P_1 \rr^{ep(s)}_{graph(def,ep(s))}$. By our
					construction $s \in dom(ep)$ and by our induction hypothesis $\mu_1 \sim
					[u \mapsto s]$ holds. It remains to show that $\mu \sim
					[g \mapsto g']$
					which follows from the construction of the query $Q$, i.e., the conjunct
					$LOC(ds,g)$ and the fact that $\sigma \in Q(D)$.
			\end{enumerate}

			\bigskip\noindent
			Let $ds,g \in \V$ and $\sigma \in Q(D)$ be arbitrary.
			We need to show  \\
			$\bigcup\limits_{ds' \in dom(ep), g' \in names(ep(d))} \{ \mu \cup
				\{[ds \mapsto ds'],[g \mapsto g']\} \mid\\ \mu \in
				\ll P\rr^{ep(ds')}_{graph(g',ep(ds'))}, 
				\mu \sim
			\{[ds\mapsto ds'], [g \mapsto g']\}\} \supseteq Q(D) $. 	
			Assume that $\sigma(ds) = ds'$ and $\sigma(g) = g'$, because of the conjunct $LOC(ds,g)$
			we have that $ds' \in dom(ep)$ and $g' \in names(ep(ds'))$.
			\begin{enumerate}
				\item Assume $u$ is an URI. 
					By i.h. we know that 
					$\ll P_1 \rr^{ep(u)}_{graph(def,ep(u))}$ =
					$Q_1(D)$. By the fact that $\sigma \in Q(D)$ and the
					construction of $Q$ we can deduce that for a part of $\sigma$,
					call it $\mu$, $\mu \in Q_1(D)$ holds. By i.h. we get
					$\mu \in  \ll P_1 \rr^{ep(u)}_{graph(def,ep(u))}$.
					From this we can deduce $\mu \in  \ll P_1
					\rr^{ep(ds')}_{graph(g,ep(ds'))}$. Because $\{[ds \mapsto
					ds'],[g \mapsto g']\} \in \sigma$ and $\mu \subseteq \sigma$ we have $\mu
					\sim \{[ ds \mapsto ds'], [g \mapsto g']\}$.
					%By i.h. we know that 
					%$\ll P_1 \rr^{ep(u)}_{graph(def,ep(u)}$ =
					%$Q_1(D)$ and  because $Q = Q_1 \land LOC(ds,g) \land
					%LOC(u,g)$ 
					%we have $\sigma = \mu \cup [ ds \mapsto ds']$ for some $\mu \in Q_1 (D)$.
					%It is by construction of our query obvious that $\mu
					%\sim [g\mapsto g']$  
					%and $\mu \in \ll P \rr^{ep(ds')}_{graph(g',ep(ds'))}$ 
					%for some $ds' \in dom(ep)$ and $g' \in names(ep(ds'))$ hold.

				\item Assume $u$ is a variable. By induction hypothesis we
					know that  $\bigcup\limits_{ds' \in dom(ep)} \{ \mu_1 \cup [u
						\mapsto ds'] \mid \mu_1 \in
						\ll P_1\rr^{ep(ds')}_{graph(def,ep(ds'))},
					\mu_1 \sim [u\mapsto ds'] \}  = Q_1(D) $.
				%	We have by construction $ Q = Q_1(D)  \land LOC(ds,g) \land  LOC(u,g)$
				%	and by assumption 
				%	$\sigma \in Q(D)$.
				%	
				%	Let $\sigma(g) = g'$ and $\sigma(ds) = ds'$ by the fact that
				%	$\sigma \in Q(D)$ and $Q =Q_1 \land LOC(ds,g) \land
				%	LOC(u,g)$ we have that $ds' \in dom(ep)$ and $g \in names(ds'))$. 
					Let $\mu = \sigma \backslash \{ [g
				\mapsto g'][ds \mapsto ds']\}$.
					We can see that $\mu \in \ll P
					\rr^{ep(ds')}_{graph(g',ep(ds'))}$ for some $ds' \in
					dom(ep)$ and $g' \in names(ep(ds'))$:
					Assume w.l.o.g. $\mu(u) = s$.
					$\mu_1 = \mu \backslash [u \mapsto s]$. By induction hypothesis we know
					that $\mu_1 \in \ll P_1 \rr^{ep(s)}_{graph(def,ep(s))}$. By our
					construction $s \in dom(ep)$ and by our induction hypothesis $\mu_1 \sim
					[u \mapsto s]$ holds. 
					It remains to show that $\mu \sim \{[ds \mapsto ds'] [g
					\mapsto g'] \}$
					which follows from the construction of the query $Q$, i.e., the conjunct
					$LOC(ds,g)$ and the fact that $\sigma \in Q(D)$.
			\end{enumerate}

			%%% UNION%%%%%
	%	\item Consider the case where $P$ is a graph pattern $(P_1 \ UNION \ P_2)$. 
	%		Let $ds,g \in \U$ because UNION is top level.\\
	%		$\subseteq:$\\
	%		Let $\mu \in \ll P \rr^{ep(ds)}_{graph(g,ep(ds))}$ be arbitrary:
	%		By semantics of SPARQL $\mu \in \ll P_1
	%		\rr^{ep(ds)}_{graph(g,ep(ds))}$ or 
	%		$\mu \in \ll P_2 \rr^{ep(ds)}_{graph(g,ep(ds))}$. By i.h. 
	%		$\ll P_i \rr^{ep(ds)}_{graph(g,ep(ds))} = Q_i(D)$
	%		where $i = 1,2$. 
	%		Assume $\mu_1 = \mu$ then we have by induction hypothesis
	%		$\mu_1 = \mu \in Q(D)$.
	%		Analogously for $\mu_2 = \mu$.

	%		\noindent$\supseteq:$\\
	%		Let $\mu \in Q(D)$ be arbitrary:
	%		Because $Q = \{Q_1,Q_2\}$ is a union of conjunctive queries we have
	%		that $\mu \in Q_1(D)$ or $\mu \in Q_2(D)$.
	%		Assume $\mu \in Q_1(D)$.
	%		Because  $ Q_1(D) = \ll P_1 \rr_{ep(ds)}^{graph(g,ds)}$ 
	%		we have that $\mu \in \ll P \rr^{ep(ds)}_{graph(g,ds)}$.
	\end{enumerate}
\end{proof}

%\noindent Let $tr$ be the following function $tr: P \times DS \times \mu \mapsto Q \times D \times h$
%where $P$ is a SPARQL graph pattern, 
%$DS$ is a SPARQL dataset and $\mu$ is a mapping.  $Q$ is a
%well-designed pattern tree, $D$ is a database and $h$ is a mapping 
%(homomorphism from $Q$ to $D$). Our aim is to define $trans$ in 
%such way, that the following equivalence holds: 
%$\mu \in \ll P \rr^{DS}_{G} \Leftrightarrow h \in Q(D)$.
The next step is to show that our well designed pattern tree that we received
from our $trans$ function is a pattern tree, as it was defined by Definition~\ref{def:wdpt}.

\begin{lemma}
	Let $ds \in dom(ep)$, $g \in names(ep(ds))$ and $P \in P_{wdgs}$ be a SPARQL pattern.
	Then $F = trans(P,ds,g)$ is a well-designed pattern tree. 
\end{lemma}

\begin{proof}
	Let $Q=(T,\lambda,x) \in F$ be arbitrary. We prove each property seperately:
	\begin{enumerate}
		\item $T$ is rooted in a distinguished node $r$, the root and maps
			each node $t$ in  $T$ to a set of relational atoms over $\sigma$:
			This is easy to see as we only use $LOC$ and $T$ in our
			construction. Also $T$ always has a distinguished node $r$ as root
			because the only step of $trans$ that changes the structure of $T$
			is if a subppattern of the form $(P_1 \OPT P_2)$ occurs: 
			$Q_1 = (T_1,\lambda_1,x_1)  = trans(P_1,ds,g)$,
			$Q_2 = (T_2,\lambda_2,x_2)  = trans(P_2,ds,g)$. 
			It connects the roots of $T_1$ and $T_2$, call them $r_1,r_2$ with
			an edge $(r_1,r_2)$ making $r_1$ the new root.  

		\item For every variable $x$ that appears
			in $T$, the tree of the wdPT, the set of nodes of $T$ where $x$ is mentioned is connected.
			It is important to remember that we assumed that our inputpattern $P
			\in P_{wdgs}$ and is well-designed.
			The induction will be over the structure of the subpatterns $\hat{P}$ of $P$.
			\begin{enumerate}
			
				\item Basecase: If $\hat{P}$ is a triple pattern we only return one node, so there can't be any
					violation of the property.

				\item Induction Step: If $\hat{P}$ is $(P_1 \AND  P_2)$ there can 
					also never be a violation of the
					property because we assumed that $P$ is in $P_{wdgs}$ and thus OPT never
					occurs in the scope of AND. We thus also only return a single node making a
					violation impossible.

				\item Induction Step: If $\hat{P}$ is $(P_1 \OPT  P_2)$:
					By induction hypothesis $Q_1 = trans(P_1,ds,g)$ and $Q_2 =
					trans(P_2,ds,g)$ fulfill the property and $P \in P_{wdgs}$ 
					and is a well designed SPARQL pattern by assumption. 

					Assume that making $T_2$ a child of the root of $T_1$ 
					results in a wdPT $Q$ which does not fulfill the well designedness property.
					There must thus be a variable $x$ in $T$ (the roots of $T_1$
					and $T_2$ get connected) which appears in two different subgraphs of $T$. This
					two subgraphs can only be situated in $T_1$ and $T_2$ respectively by induction
					hypothesis.	Because these subgraphs are not connected 
					by our assumption we proceed by	case distinction and assume 
					\begin{enumerate}
						\item $T_1$ contains a subgraph containing $x$ but the root $r_1$ does not
							contain the variable $x$. There must be a subpattern of $P_1$ by our
							construction $(P' \OPT P'')$ where $P''$ contains $x$ but $P'$ doesn't. 
							As $(P' \OPT P'')$ is part of $P_1$ 
							and we have $(P_1 \OPT P_2)$ and $P_2$ contains $x$ we have a
							contradiction to the assumption that $P$ is a well designed SPARQL
							patterns.
						\item $T_2$ contains a subgraph containing $x$ but the root $r_2$ does not
							contain the variable $x$. This can be shown
							analogously. %Analogously to the previous case.
					\end{enumerate}


					%If we then make the wdPT $Q_2$ a child of the wdPT $Q_1$ we know 
					%that $Q$ fulfills the property. 

				\item Induction Step: If $\hat{P}$ is $(\mbox{GRAPH }  u \ P_1)$:
					We know that $trans(P_1,ds,u) = (T, \lambda,x)$ 
					is part of the endresult $Q$ and by induction hypothesis,
					we know that $(T, \lambda,x)$ fulfills the property. Let
					$r_1$ be the root of $T$. $trans(\cdot)$ adds the conjunct
					$LOC(u,ds)$ and $LOC(g,ds)$ to $r_1$. W.l.o.g. assume $u$,$g$
					and $ds$ are variables. For every of those three variables it
					remains to prove that the property holds.
					Let $x \in \{u,g,ds\}$ and assume that our resulting wdPT $Q$ is
					not fulfilling the property. There must thus be a node $n_1$ in $Q$ which
					doesn't contain $x$ and a node $n_2$ for which $n_1$ is a parent 
					which again contains $x$ creating the conflict.
					By the definition of the function $trans(\cdot)$ we know that
					there must have been a subpattern $(P' OPT P'')$ and both $n_1,n_2$ must
					have been created by this subpattern. But this again means, that
					$P'$ did not contain $x$ and $P''$ did contain $x$. 
					%As the root
					%of $Q$, i.e. $r_1$, cannot be $n_1$, we have that there must be
					%another OPT subpattern making $r_1$ an ancestor of $n_1$.(not
					%needed)
					Depending on whether $x = u$, $x=g$ or $x=ds$ we have a
					contradiction:
					\begin{enumerate}
						\item Let $x=u$. Because we assumed $\hat{P}=
							(\mbox{GRAPH } u \ P_1)$ we know that
							$P$ is not well-designed because $P_1$ contains the
							subpattern $(P' \OPT P'')$.
						\item Let $x = g$. This means that we are inside a graph
							pattern $(\mbox{GRAPH } g \ P^\sim)$ and thus $P$ is
							not well designed because $P_1$ contains the
							subpattern $(P' \OPT P'')$.
						\item Let $x = ds$. This means that we are inside a
							SERVICE	pattern\\ $(\mbox{SERVICE } g \ P^\sim)$ and thus $P$ is
							not well designed because $P_1$ contains the
							subpattern $(P' \OPT P'')$.
					\end{enumerate}



				\item Induction Step: If $\hat{P}$ is $(\mbox{SERVICE } u \ P_1)$:
						The proof is analogously to case where $\hat{P}$ is
						$(\mbox{GRAPH} u \ P_1)$.
				%	We know that $Q = trans(P_1,ds,u)$ and by induction hypothesis,
				%	we know that $Q = (T, \lambda,x)$ fulfills the property. Let
				%	$r_1$ be the root of $T$. $trans(\cdot)$ adds the conjunct
				%	$LOC(def,u)$ and $LOC(g,ds)$ to $r_1$. W.l.o.g. assume $u$,$g$
				%	and $ds$ are variables. For every of those three variables it
				%	remains to prove that the property holds.
				%	Let $x \in \{u,g,ds\}$ and assume that our resulting wdPT $Q$ is
				%	not fulfilling the property. There must thus be a node $n_1$ in $Q$ which
				%	doesn't contain $x$ and a node $n_2$ for which $n_1$ is a parent 
				%	which again contains $x$ creating the conflict.
				%	By the definition of the function $trans(\cdot)$ we know that
				%	there must have been a subpattern $(P' \OPT P'')$ and both $n_1,n_2$ must
				%	have been created by this subpattern. But this again means, that
				%	$P'$ did not contain $x$ and $P''$ did contain $x$. 
				%	%As the root
				%	%of $Q$, i.e. $r_1$, cannot be $n_1$, we have that there must be
				%	%another OPT subpattern making $r_1$ an ancestor of $n_1$.(not
				%	%needed)
				%	Depending on whether $x = u$, $x=g$ or $x=ds$ we have a
				%	contradiction:
				%	\begin{enumerate}
				%		\item Let $x=u$. 
				%			Because we assumed $\hat{P}= (\mbox{SERVICE } u \ P_1)$ we know that
				%			$P$ is not well designed because $P_1$ contains the
				%			subpattern $(P' \OPT P'')$.
				%		\item Let $x = g$. This means that we are inside a graph
				%			pattern $(\mbox{GRAPH } g \ P^\sim)$ and thus $P$ is
				%			not well designed because $P_1$ contains the
				%			subpattern $(P' \OPT P'')$.
				%		\item Let $x = ds$. This means that we are inside a
				%			SERVICE
				%			pattern $(\mbox{SERVICE } g \ P^\sim)$ and thus $P$ is
				%			not well designed because $P_1$ contains the
				%			subpattern $(P' \OPT P'')$.
				%	\end{enumerate}

			%	\item Induction Step: If $P$ is $(P_1 \UNION  P_2)$:
			%		We get a forest and because the property held for $Q_1$ and $Q_2$ by i.h. we are
			%		done.
			\end{enumerate}
		\item The last property is that $x$ is a tuple of distinct variables
			occurring in $T$. As we don't use projection and use set operations
			to merge our free variables in the OPT case, this is an obivous
			observation.
	\end{enumerate}
\end{proof}


We will now write our final result: When we have a graph pattern $P$ in
$P_{wdgs}$ with a dataset $DS$ and a function $ep$ we want to transform it into a
wdPT $Q$ and database $D$ with our function $trans$ so that
$\ll P \rr^{DS}_def = Q(D)$.
\begin{theorem}\label{biglemma}
    Let $P$ be a graph pattern in $P_{wdgs}$, $DS$ a dataset and $G$ a graph in $DS$.
    Let $DS = ep(ds)$  and $G = graph(g,DS)$. Let $Q = trans(P,ds,g)$ be the
    wdPT and $D = \bigcup\limits_{c \in dom(ep)} data(c)$. Then $\ll P \rr^{DS}_G = Q(D)$.
\end{theorem}
\begin{proof}
	The database $D$ is the same database that is created in
	Lemma~\ref{smalllemma}.
	Use Lemma~\ref{smalllemma}: 
	$DS = ep(ds)$  and $G = graph(g,DS)$ hold and thus $ds,g \in \U$:
	$\ll P \rr^{DS}_G = Q(D)$ follows.
\end{proof}

%For the static analysis we dont have $DS$ as input, so we want to
%generalize the theorem in the following way:
%\begin{theorem}\label{biglemma}
%	Let $P$ be a graph pattern in $P_{wdgs}$, 
%	$DS$ be an arbitrary dataset in $img(ep)$.
%	Let $x \in \V$. Then $Q = trans(P,x,def)$ be the wdPT and 
%	$D = \bigcup\limits_{c \in dom(ep)} data(c)$. 
%	Then $\ll P \rr^{DS}_G = Q(D)$.
%\end{theorem}

\section{The complexity of EVAL($\l$) where $\l \in \{P,\u,\s\}$ }

\begin{corollary}
	The problems $EVAL(P_{wdgs})$ and $EVAL(\u_{wdgs})$ are coNP-complete and
	$EVAL(\s_{wdgs})$ is $\Pi^P_2$-complete.
\end{corollary}
\begin{proof}
The problem EVAL is defined in the following way:
\begin{framed}\noindent \textbf{EVAL($\mathcal{L}$)}\\
	\textbf{INPUT:} Dataset $DS$, graph pattern $P \in  \mathcal{L}$ and a mapping $\mu$.\\
	\textbf{QUESTION:} Is $\mu$ in $\ll P \rr^{DS}_{def}$.
\end{framed}
Assume $ds = ep^{-1}(DS)$. %and $G = graph(def,DS)$.
For the membership we propose the following procedure:
Use the transformation function on the input graph pattern $P$ to obtain a wdPF
$Q$, and the data function to obtain the database $D$. More formally:
$Q = trans(P,ds,def)$ and $D = \bigcup\limits_{c\in dom(ep)} data(ep(c))$. This transformation is obviously
possible in polynomial time if we assume $dom(ep)$ to be finite. We know by Theorem~\ref{biglemma} that 
$\ll P \rr^{DS}_{def} = Q(D)$.
For the projection-free case we can use the results in 
\cite{perez2009semantics}, i.e., that the evaluation problem for well-designed
pattern trees without projection is coNP-complete and conclude a coNP runtime to
check if $\mu \in Q(D)$.
The hardness of the problem follows immediately from the hardness of
\textbf{EVAL($P_{wd}$)}.
In case of projection we use the results in \cite{letelier2013static}
that the evaluation of $\mu \in Q(D)$ is $\Pi^P_2$-complete.
Again hardness of the problem follows immediately from the hardness of 
\textbf{EVAL($\s_{wd}$)}.
\end{proof}

%\section{Static Analysis}
%\begin{theorem}
%	The problems \textbf{EQUIVALENCE}($\l$), \textbf{CONTAINMENT}($\l$)  
%	are $NP$-complete for $\l = P_{wdgs}$ and $\Pi^p_2$-complete $\l =
%	\u_{wdgs}$ when we assume the function $ep(\cdot)$ as additional input.
%\end{theorem}
%\begin{proof}
%	Hardness for each of these problems follows immediately from the fact that
%	that they are already NP-complete for the language
%	$P_{wd}$~\cite{letelier2012static}.
%	%and $\Pi^{P}_2$ for the language $\U_{wd}$~\cite{pichler2014containment}.\\
%	For the membership, we propose the following algorithm:
%	Let $P_1,P_2$ be two arbitrary patterns where $P_1,P_2 \in P_{wdgs}$ or
%	$P_1,P_2 \in \u_{wdgs}$ holds.
%	Let $x \in \V$ and use our polynomial time transformation function twice to obtain two wdpts: 
%	$Q_i = trans(P_i,x,def)$ for $i\in \{1,2\}$. 
%	Using the NP-algorithm proposed in~\cite{letelier2012static} for $\l =
%	P_{wdgs}$ we can decide whether $Q_1 \subseteq Q_2$:
%%	or the
%%	$\Pi^P_2$ algorithm proposed in~\cite{pichler2014containment} when $\l =
%%	\u_{wdgs}$.
%	\begin{enumerate}
%		\item $Q_1 \subseteq Q_2$:
%			Let $ds \in dom(ep)$ be arbitrary, i.e., 
%			an arbitrary URI of a dataset and $DS = ep(ds)$. 
%			Let $D =\bigcup\limits_{c\in dom(ep)} data(ep(c))$. The function data is also
%			obviously in polynomial time if $dom(ep)$ is assumed to be finite.
%			Let $\mu$ be an arbitrary mapping in $\ll P_1 \rr^{DS}_{def}$. Because
%			of Theorem~\ref{biglemma} we can conclude that $\mu \in Q_1(D)$.
%			Because of our assumption we get that $\mu \in Q_2(D)$. Because
%			again $Q_2(D) = \ll P_2 \rr^{DS}_{def}$ we get that $\mu \in \ll P_2
%			\rr^{DS}_{def}$.
%
%		\item $Q_1 \not\subseteq Q_2$: 
%			Let $ds \in dom(ep)$ be arbitrary, i.e., an arbitrary URI of a dataset 
%			and $DS = ep(ds)$.
%			Let $D =\bigcup\limits_{c\in dom(ep)} data(ep(c))$. The function data is also
%			obviously in polynomial time if $dom(ep)$ is assumed to be finite.
%			By assumption there is a $\mu \in
%			Q_1(D)$ such that $\mu \notin Q_2(D)$. Because
%			of Theorem~\ref{biglemma} and $\mu \in Q_1(D)$ we can conclude that $\mu \in \ll P_1
%			\rr^{DS}_{def}$.
%			Because of our assumption we get that $\mu \notin Q_2(D)$. Because
%			again $Q_2(D) = \ll P_2 \rr^{DS}_{def}$ we get that $\mu \notin \ll P_2
%			\rr^{DS}_{def}$.
%	\end{enumerate}
%\end{proof}
%%der proof geht nicht weil wenn man oben die variable reinhaut kriegt man kein
%%nicht \ll P \rr^DS_G = Q(D) sondern was anderes.
%To generalize this result which means removing the $ep(\cdot)$ function from the
%input, we observed that a problem occurred in the step where we transformed the
%SERVICE function into a wdpt. 
%In detail, this is the case when we have a query of the form $(\mbox{SERVICE }
%a \
%P_1)$, where $a \in \U \backslash dom(ep)$. In this case our transformation
%function $trans(\cdot)$ directly accesses the function $ep(\cdot)$ and checks if $a \in
%ep(\cdot)$ and produces a result directly depending on the outcome of this
%function. When transforming the GRAPH operator into a wdPT we don't run into the
%same problem. Assume we want to transform $(\mbox{GRAPH} \ a \ P_1)$ with $trans(\cdot)$
%and our current dataset is $ds$.
%Then $trans(\cdot)$ produces the same output regardless of the outcome of $a \in
%dom(names(ds))$.
%\begin{theorem}
%	The problem \textbf{CONTAINMENT}($P_{wdgs}$)  
%	is NP-hard and in $\Sigma^P_2$.
%\end{theorem}
%\begin{proof}
%	NP-Hardness for the problem follows immediately from the fact that
%	that they are already NP-complete for the language
%	$P_{wd}$~\cite{letelier2012static}.
%	%and $\Pi^{P}_2$ for the language $\U_{wd}$~\cite{pichler2014containment}.\\
%
%	For the $\Sigma^P_2$ membership  of \textbf{CONTAINMENT}($P_{wdgs}$), we propose the following algorithm:
%	Let $P_1,P_2$ be two arbitrary patterns where $P_1,P_2 \in P_{wdgs}$ holds.
%	Scan the queries $P_1,P_2$ for every occurrence of $(\mbox{SERVICE} \ a \ P_1)$ where $a \in
%	\U$. Put each destination URI of a SERVICE subpattern in $P_i$ into the set $O_i$ for $i\in
%	\{1,2\}$.
%	Let $O = O_1 \cup O_2$ so we have all the distinct occurences of destination
%	URIs in both queries. Assume $|O| = n$. 
%	Notice that we have exactly $2^n$ possibilities to construct $dom(ep)$ that would
%	influence our translation function $trans(\cdot)$ because this is the
%	number of subsets of the set $O$.
%	Let this family of domains be described as $Doms$.
%	Let $x \in \V$ and guess a domain $D_i\in Doms$. Then let 
%	$Q_i = trans(P_i,x,def,D_i)$ for $i\in \{1,2\}$. 
%	Using the NP-algorithm proposed in~\cite{letelier2012static} 
%	we can decide whether $Q_1 \subseteq Q_2$:
%	\begin{enumerate}
%		\item $Q_1 \subseteq Q_2$:
%			Let $ep$ have the domain $D_i$ and assume it else to be arbitrary.
%			Let $ds \in dom(ep)$ be arbitrary, i.e., an arbitrary
%			URI of a dataset and $DS = ep(ds)$. 
%			Let $D =\bigcup\limits_{c\in dom(ep)} data(ep(c))$. The function data is also
%			obviously in polynomial time if $dom(ep)$ is assumed to be finite. 
%			Let $\mu$ be an arbitrary mapping in $\ll P_1 \rr^{DS}_{def}$. Because
%			of Theorem~\ref{biglemma} we can conclude that $\mu \in Q_1(D)$.
%			Because of our assumption we get that $\mu \in Q_2(D)$. Because
%			again $Q_2(D) = \ll P_2 \rr^{DS}_{def}$ we get that $\mu \in \ll P_2
%			\rr^{DS}_{def}$.
%		\item $Q_1 \not\subseteq Q_2$: 
%			Let $ep$ have the domain $D_i$ and assume it else to be arbitrary.
%			Let $ds \in dom(ep)$ be arbitrary, i.e., an arbitrary URI of a dataset 
%			and $DS = ep(ds)$.
%			Let $D =\bigcup\limits_{c\in dom(ep)} data(ep(c))$. The function data is also
%			obviously in polynomial time if $dom(ep)$ is assumed to be finite.
%			By assumption there is a $\mu \in
%			Q_1(D)$ such that $\mu \notin Q_2(D)$. Because
%			of Theorem~\ref{biglemma} and $\mu \in Q_1(D)$ we can conclude that $\mu \in \ll P_1
%			\rr^{DS}_{def}$.
%			Because of our assumption we get that $\mu \notin Q_2(D)$. Because
%			again $Q_2(D) = \ll P_2 \rr^{DS}_{def}$ we get that $\mu \notin \ll P_2
%			\rr^{DS}_{def}$.
%	\end{enumerate}
%\end{proof}
%Through Lemma~\ref{biglemma} we get that SPARQL evaluation with the GRAPH and
%the SERVICE operator are the
%same as without the $GRAPH$ and $SERVICE$ patterns.
%Thus the complexity for the problems CONTAINMENT, EQUIVALENCE and SUBSUMPTION do not 
%differ from the results in \cite{pichler2014containment} using $GRAPH$ and
%$SERVICE$.


\chapter{Beyond well-designed SPARQL}
Although well-designed SPARQL is a fragment that covers a lot of practical SPARQL
queries, (50\% of the queries over DBpedia that use the
OPT-operator~\cite{Picalausa})
many practical queries are not well designed and need to be analyzed. 
Kaminski and Kostylev found another
interesting fragment of SPARQL called weakly well-designed fragment~\cite{kaminski_bwd}. This
fragment captures 99\% of the queries over DBpedia that use the OPT-operator. 
There are mainly two use cases of queries that are not well-designed but used in
practice.
\begin{enumerate}
	\item The first practical use of non well-designed patterns are the so called
		preference patterns:
		Consider the following example:
		\begin{example}[Preference Pattern~\cite{kaminski_bwd}]\label{prefpattern}
			\begin{align*}
				P_1 =	\SELECT x, y  \WHERE
				((\mathit{x,type,person})\\ \OPT
				(\mathit{x,name,y}))\\
				\OPT (\mathit{x,v\_card:name,y})
			\end{align*}
		\end{example}
		It is obvious that the pattern $P$ in Example~\ref{prefpattern} 
		is not well-designed because variable
		$y$ does occur in two unrelated OPT parts of $P$. The
		intuitive meaning might seem unclear at the first moment, 
		but looking at the semantics of OPT sheds light on it: If the first OPT-operator
		does not bind the variable $y$ through the triple $\mathit{(x,
		name, y)}$ 
		and only then, $y$ is bound through the triple $(x, v\_card:name, y)$.
		This could be used when we have two relations, in this case
		$\mathit{name}$
		and $v\_card:name$, which connect a name to an identifier but we prefer
		the relation $\mathit{name}$ over $v\_card:name$.

	\item The second practical use of non well-designed patterns are top level
		FILTER expressions. For this usage consider the following query:
		\begin{example}[Top level
			Filter~\cite{kaminski_bwd}]\label{toplevelfilter}
			\begin{align*}
				P_2 = \SELECT x,y \WHERE ((\mathit{x,type,person})\\
				\OPT (\mathit{x,
				name,y}))\\ 
				\FILTER(\neg bound(y) \lor \neg(y = Ana)).
			\end{align*}
		\end{example}
		The query $P_2$ is not well-designed because the FILTER constraint
		mentions the variable $y$, which occurs only in the optional
		part. The practical intention of the query is to filter for people
		where the name is not 'Ana' or do only have an id which is bound by
		variable $x$. 
\end{enumerate}

To capture the two use-cases mentioned in
Example~\ref{prefpattern} and Example~\ref{toplevelfilter}, well-designed SPARQL is extended to weakly well-designed SPARQL.
This new fragment subsumes well-designed queries and it was shown in~\cite{kaminski_bwd} 
that it has the same complexity 
of query evaluation as well-designed queries. 
Also, $99\%$ of the practical queries over DBPedia containing OPT are captured
by the weakly-well designed fragment of SPARQL and thus proven to be efficient to evaluate.
To define the fragment of weakly well-designed pattern we need to define what it
means for a subpattern to be dominated by another subpattern:
\begin{definition}[~\cite{kaminski_bwd}]
	Given a graph pattern $P$, an occurrence $i_1$ in $P$ dominates another
	occurrence $i_2$ if there exists a OPT-pattern such
	that $i_1$ is inside the left argument of the OPT-pattern and $i_2$ is
	inside the right argument of the OPT-pattern.
\end{definition}

Now we can proceed to define weakly well-designed patterns.
\begin{definition}[Weakly well-designed patterns~\cite{kaminski_bwd}]
	A pattern $P$ is weakly well-designed (wwd-pattern) if each occurrence $i$
	of an OPT-subpattern $(P_1\OPT P_2)$ variables in $vars(P_2) \backslash
	vars(P_1)$ appear outside $i$ only in
	\begin{itemize}
		\item subpatterns whose occurrences are dominated by $i$, and
		\item constraints of top-level occurrences of FILTER-patterns.
	\end{itemize}
\end{definition}
Checking if a pattern is wwd is very easy computationally:
\begin{proposition}[\cite{kaminski_bwd}]
	Checking whether a pattern $P$ belongs to the fragment $P_{wwd}$ can be done
	in time $O(|P|^2)$, where $|P|$ is the length of the string representation
	of $P$.
\end{proposition}
\begin{proofidea}
	In a simple recursive procedure, the top-level occurrences of filters are
	removed.  Then in yet another recursive procedure  the first condition of
	weakly well-designed patterns is checked. 
\end{proofidea}

\section{OPT-FILTER-Normal Form and Constraint Pattern Trees}
When used wd-patterns we can convert them to the so-called
OPT-normal form. In the OPT-normal form, all AND- and FILTER- subpatterns are
OPT-free and most importantly the pattern can then be naturally represented as a
tree. The resulting tree can be used to visualize how to evaluate and optimize
the original pattern~\cite{letelier2013static, pichler2014containment}. The tree
notation can be generalised to wwd-patterns.

\begin{definition}[OPT-FILTER-normal form~\cite{kaminski_bwd}]\label{ofnf}
	A pattern $P$ is in OPT-FILTER-normal form (or OF-normal form)
	if the following grammar can be tested positively:
	\begin{align*}
		&P ::=  F \mid (P \FILTER R) \mid (P \OPT S), \qquad S::= F\mid(S \OPT S),\\ 
	 &F ::=(B \FILTER R)
	\end{align*}
	where $B$ is a set of triple patterns and $R$ is a filter
	constraint.
\end{definition}
As we can see from Definition~\ref{ofnf}, each triple pattern has a FILTER expression. 
This is no restriction because one can easily insert a dummy FILTER expression 
by letting $R = \top$. The sets of triple patterns with FILTER expressions form the bottom layer.
On top of the bottom layer there is a combination of OPT and FILTER.
The layers cause that each occurrence of a FILTER-pattern in
the top layer is top-level. The normal form is AND-free: all conjunctions are
expressed via a set of triple patterns.

\begin{definition}[Constraint pattern tree
	(CPT)~\cite{kaminski_bwd}]\label{defcpt}
	A constraint pattern tree (CPT) $T(P)$ of a pattern $P$ in OF-normal form is
	the directed ordered labelled rooted tree, which can be recursively
	constructed as follows:
	\begin{enumerate}	
		\item if $B$ is a set of triple patterns then $T(B \FILTER R)$ is a single node
			$v$ labelled by the pair $(B,R)$;
		\item if $P'$ is not a set of triple patterns then $T(P' \FILTER R)$ is obtained
			by adding a special node labelled by $R$ as the last child of the
			root of $T(P')$;
		\item $T(P_1 \OPT P_2)$ is the tree obtained from $T(P_1)$ and $T(P_2)$
			by adding the root of $T(P_2)$ as the last child of the root of
			$T(P_1)$.
	\end{enumerate}
\end{definition}
Looking at Definition~\ref{defcpt} one can see the similarity of CPTs and
patterns in OF-normal form: A CPT displays the semantic structure of OPT and FILTER nesting.
The next step is to prove that every wwd-pattern can be converted to OF-normal
form and can be represented by a CPT, analogously to wd-patterns, which can be
transformed into  OPT-normal form and thus pattern trees. 
Towards this goal, the following equivalence is needed:
\begin{proposition}[\cite{kaminski_bwd}]
	Let $P_1,P_2,P_3$ be patterns and $R$ a filter constraint such that
	$vars(P_2) \cap vars(P_3) \subseteq vars(P_1)$ and $vars(P_2) \cap vars(R)
	\subseteq vars(P_1)$. Then the following equivalences hold:
	\begin{align*}
		(P_1 \OPT P_2) \AND P_3 \equiv (P_1 \AND P_3) \OPT P_2,\\
		(P_1 \OPT P_2) \FILTER R \equiv (P_1 \FILTER R) \OPT P_2.
	\end{align*}
\end{proposition}
Using the two equivalences we can achieve our goal stated in
Proposition~\ref{wwdtoofnf}.

\begin{proposition}[\cite{kaminski_bwd}]\label{wwdtoofnf}
	Each wwd-pattern $P$ is equivalent to a wwd-pattern in $OF$-normal form of
	size $O(|P|)$.
\end{proposition}

Let $\prec$ be a relation which contains the topological sorting of the nodes in
$T(P)$ computed by a depth first search traversal. $v \prec u$ holds if $v$ is visited before
$u$ in the search process. Assuming such a relation $\prec$, Proposition~\ref{charwdcpt} 
provides a condition to decide whether a pattern is weakly well-designed looking
at its CPT.

\begin{proposition}[\cite{kaminski_bwd}]\label{charwdcpt}
	A pattern $P$ in OF-normal form is weakly well-designed iff. for each edge
	$(v,u)$ in its CPT $T(P)$ every variable $x \in vars(u) \backslash vars(v)$
	occurs only in nodes $w$ such that $v \prec w$. The pattern is well-designed
	iff for every variable $x$ in $P$ the set of all nodes $v$ in $T(P)$ with
	$x \in vars(v)$ is connected.
\end{proposition}

In~\cite{kaminski_bwd} a unique property, which applies only for
wwd-patterns was found: Each wwd-pattern is semantically equivalent to a pattern whose corresponding
CPT has depth one.
\begin{definition}[\cite{kaminski_bwd}]
	A pattern $P$ is in depth-one normal form if it has the structure
	\begin{align*}
		(\cdots((B \ op_1 \ S_1) \ op_2 \ S_2) \cdots)\ op_n \ S_n,
	\end{align*}
	where $B$ is a set of triple patterns and each $op_i S_i, 1 \leq i \leq n$ is either
	$\OPT (B_i \FILTER R_i)$ with $B_i$ a basic pattern and $R_i$ a filter
	constraint, or just FILTER $R_i$.
\end{definition}

Towards our goal to show that every wwd-pattern can be brought to the depth-one
normal form the following equivalence can be used.
\begin{proposition}[\cite{kaminski_bwd}]\label{equivdep1}
	For patterns $P_1, P_2,P_3$ with $vars(P_1) \cap vars(P_3) \subseteq
	vars(P_2)$ it holds that 
	\begin{align*}
		P_1 \OPT (P_2 \OPT P_3) \equiv (P_1 \OPT P_2) \OPT (P_2 \AND P_3).
	\end{align*}
\end{proposition}

\bigskip\noindent OPT operators in pattern can be nested in two different ways:
\begin{enumerate}
	\item $(P_1 \OPT (P_2 \OPT P_3))$ (OPT-R)
	\item $((P_1 \OPT P_2) \OPT P_3)$ (OPT-L)
\end{enumerate}
When we then use CPTs to display this patterns using triple patterns we get the following result:

\begin{figure}(1) \begin{tikzpicture}[sibling distance=10em,
		every node/.style = {shape=rectangle, rounded corners,
			draw, align=center,
	top color=white}]
\node {$B_1$}
child { node {$B_2$} 
child { node {$B_3$} }};
\end{tikzpicture}
\hspace{5cm}
(2) \begin{tikzpicture}[sibling distance=10em,
		every node/.style = {shape=rectangle, rounded corners,
			draw, align=center,
top color=white}]
\node {$B_1$}
child { node {$B_2$} }
child { node {$B_3$} };
\end{tikzpicture}
\caption{(1) is the CPT of $(B_1 \OPT (B_2 \OPT B_3))$ and 
(2) the CPT of $((B_1 \OPT B_2) \OPT B_3)$}
\end{figure}\label{optrl}


Using the equivalence from left to right keeps the pattern weak well-designed and
transforms a weakly well-designed OPT nesting of type (OPT-R) to a nesting type
(OPT-L). Looking at the Figure~\ref{optrl} we can see that this step reduces the depth by one.
\begin{corollary}[\cite{kaminski_bwd}]
	Every wwd-pattern is equivalent to a wwd-pattern in depth-one normal form.
\end{corollary}
The regular structure of the depth-one normal form might prove attractive in
practice but using the equivalence results in an exponential blowup in the size
of the pattern. This can be seen in Proposition~\ref{equivdep1}: $P_2$ gets
copied twice in every application of the equivalence.

\section{Evaluation of wwd-Patterns}
The next step is to look at the complexity of the evaluation problem for
wwd-patterns and the extensions with union and projection. The goal is to show
that in all three cases complexity doesn't change in comparison to wd-patterns.
Consider first the formal evaluation problem for a given SPARQL fragment $\mathcal{L}$:
\begin{framed}\noindent \textbf{EVAL($\mathcal{L}$)}\\
	\textbf{INPUT:} Graph $G$, query $Q \in  \mathcal{L}$ and a mapping $\mu$.\\
	\textbf{QUESTION:} Is $\mu$ in $\ll Q \rr_G$.
\end{framed}
The fragment of general SPARQL graph patterns is denoted with $\u$ and it is a
well known result that \textbf{EVAL($\u$)} is PSPACE-complete~\cite{perez2009semantics}.
For the fragment of general queries extended with projection (i.e., $\s$)
it was shown that \textbf{EVAL($\s$)} is PSPACE-complete~\cite{letelier2013static}.


Similar to Algorithm~\ref{conpevalwd}, an algorithm for wwd-patterns exists. 
We first define the potential partial solutions. 
We know from the previous section that we can transform any wwd-pattern $P$ to a
pattern in OF-normal form. This pattern $P'$ in OF-normal corresponds to some
CPT. An $r$-subtree of $T(P')$ is a subtree containing the root of $T(P')$ and
all its special children. Every $r$-subtree obviously corresponds to some
wwd-pattern which is obtained by dropping the rightmost arguments of some
OPT-subpatterns.

\begin{definition}[\cite{kaminski_bwd}]
	Let $P$ be a wwd-pattern and $P'$ the corresponding pattern in OF-normal
	form.
	A mapping $\mu$ is a potential partial solution (or $pp$-solution for short)
	to a wwd-pattern $P$ over a graph $G$ if there is an r-subtree $T(P')$ of
	$T(P)$ such that $dom(\mu) = vars(P')$, $\mu(pat(P')) \subseteq G$ and $\mu
	\models R$ for the constraint $R$ of any ordinary node in $T(P')$.
\end{definition}

It could happen that several $r$-subtrees correspond to the same mapping $\mu$
for a pattern $P$ over $G$. We then take the union of all the nodes in exactly
those $r$-subtrees as they are a subtree as well. (They are all connected by the root). 
From this observation we can see that there exists a unique maximal r-subtree
corresponding to a mapping $\mu$ which we will from now on denote with
$T(P_\mu)$, as this subtree corresponds to a wwd-pattern $P_\mu$. The
big difference to partial solutions for wd-patterns is that not every pp-solution can be
extended to a real solution. A real solution may not only extend the domain of a pp-solution
with previously undefined variables but it might also extend $T(P_\mu)$ to a
child that is smaller in the order $\prec$ than some other node which was
already in $T(P_\mu)$. But this means that variables bindings might get
overridden. The next difference are non-well-designed top-level filters.
$pp$-solutions ignore top-level filters as it would be too restrictive as real
solutions do not satisfy them either. 

\begin{example}[\cite{kaminski_bwd}]
	Consider the graph $G = \{(1,a,1), (3,a,3)\}$ and the wwd-pattern:
	\begin{align*}
		P = (((x,a,1) \OPT (y,a,2)) \FILTER \neg bound(y)) \OPT (y,a,3).
	\end{align*}
	%	Consider now $T(P')$:\\
	%
	%	\begin{tikzpicture}[sibling distance=10em,
	%	 every node/.style = {shape=rectangle, rounded corners,
	%    draw, align=center,
	%    top color=white, bottom color=blue!20}]]
	%	\node {$(x,a,1),\top$}
	%	child { node {$(y,a,2),\top$} }
	%	child { node {$?(y,a,3),\top$}}
	%	child { node {$\neg bound(y)$}};
	%
	%	\end{tikzpicture}\\
	Obviously $\neg bound(y)$ is a top level filter and the whole pattern is
	not well designed as $y$ occurs in the top level filter.
	Also, the mapping $\mu = \{?X \mapsto 1, ?Y \mapsto 3\}$ is a solution in
	$\ll P \rr_G$, but we can see, that $\mu \not\models \neg bound(y)$.
\end{example}

The following characterisation takes care of the two difficulties and describes
solutions.
\begin{definition}[\cite{kaminski_bwd}]
	Given a wwd-pattern $P$, a node $v \in T(P)$, a graph $G$ and a
	$pp$-solution $\mu$ of $P$ over $G$, let $\mu_{|v}$ be the projection
	$\mu_{|X}$ of $\mu$ to the set $X$ of all variables appearing in nodes $u$
	of $T(P_\mu)$ such that $u \prec v$.
\end{definition}
$\mu_{|v}$ is in other words the mapping $T(P_\mu)$ until the node $v$ could be
considered in the order $\prec$ would traverse the tree $T(P)$.
\begin{lemma}[\cite{kaminski_bwd}]\label{lemsolwwd}
	A mapping $\mu$ is a solution to a wwd-pattern $P$ over a graph $G$ iff. 
	\begin{enumerate}
		\item $\mu$ is a pp-solution to $P$ over $G$.
		\item for any child $v$ of $T(P_\mu)$ labelled with $(B,R)$ there is no
			$\mu'$ such that $\mu_{|v} \sqsubset \mu'$, $\mu' \models R$, and
			$\mu'(B) \subseteq G$;
		\item $\mu_{|s} \models R$ for any special node $s$ in $T(P)$ labelled
			with $R$.
	\end{enumerate}
\end{lemma}
Intuitively we can conclude the following from the lemma: (1) is obvious and as
mentioned before, each solution needs to be a pp-solution of $P$ over $G$.
(2) makes sure that every node $v$ which is not added to $T(P_\mu)$ is not addable
to the tree ``until'' that node $v$ ``appears'' (i.e. in order of $\prec$) first. 
If it is addable it would mean,
that one could use the mapping bound up to node $v$ (looking at $\prec$, this is
$\mu_{|v}$) to create a new mapping
$\mu'$ which uses the node $v$. There are two ways of mapping $\mu'$
differently to the original mapping $\mu$: Either $dom(\mu)$ is extended, or
some variables are assigned differently because $\prec$ visits some node first
which contains the variable in a different triple resulting in a different
assignment.
(3) makes sure that the mapping fulfills the top-level filter at exactly the
time point where the traversal reaches the corresponding special node $s$, which
is denoted by $\mu_{|s}$.

\begin{theorem}[\cite{kaminski_bwd}]
	\textbf{EVAL($P_{wwd}$)} is coNP-complete.
\end{theorem}
\begin{proofidea}
	We can easily transform the characterisation in
	Lemma~\ref{lemsolwwd} to an	algorithm which is in coNP: 
	Compute the maximal tree for $\mu$, $T(P_\mu)$. This is doable in polynomial
	time. Now it remains to check that this tree is not extendible to linearly
	many of its children. As we then have to do a homomorphism check against
	$G$, this is in coNP. The checks for top-level filters are polynomial again.
\end{proofidea}

This result can easily be carried over to top level unions of wdd-patterns. The fragment
is called $\u_{wwd}$. The fragment where also projection is done over unions of
wdd-patterns is calles $\s_{wdd}$.

\begin{corollary}[\cite{kaminski_bwd}]
	\textbf{EVAL($\u_{wwd}$)} is coNP-complete and \textbf{EVAL($\s_{wwd}$)} is
	$\Sigma^P_2$-complete.
\end{corollary}
\begin{proofidea}
	The coNP-algorithm for $\u_{wwd}$  applies the
	algorithm for \textbf{EVAL($P_{wwd}$)} for every part of the top level UNION. And
	if the mapping $\mu$ is the solution for any of the patterns we return
	true. Hardness follows by the coNP-completeness of \textbf{EVAL($P_{wwd}$)}.
	It is a well known results that \textbf{EVAL($\s_{wd}$)} is
	$\Sigma^P_2$-complete~\cite{letelier2013static},
	thus $\Sigma^P_2$-hardness of \textbf{EVAL($\s_{wdd}$)} follows. 
	For the $\Sigma^P_2$-membership of \textbf{EVAL($\s_{wdd}$)} apply the following algorithm:
	Guess the values of the existential variables and then call a coNP-oracle
	for $\u_{wwd}$ on the resulting mapping. This yields a $\Sigma^P_2$ algorithm.
\end{proofidea}

\section{Expressivity of wwd-Patterns and their Extensions}
In \cite{kaminski_bwd} Kaminski et al. established that one has a greater set
of tools available if one uses weakly-well designed SPARQL instead of
well-designed SPARQL. Even though this fact may seem to be of obvious nature it
must be proven mathematically. This characteristic of a SPARQL fragment (or
language) is called expressive power.

\begin{definition}[Expressive power of languages~\cite{kaminski_bwd}]
	A language $\l_1$ has the same expressive power as language $\l_2$, denoted by
	$\l_1 \sim \l_2$, if for every query $Q_1 \in \l_1$ there is a query $Q_2 \in
	\l_2$ such that $Q_1 \equiv Q_2$ and for every query $Q_2 \in \l_2$ there is a
	query $Q_1 \in \l_1$ such that $Q_1 \equiv Q_2$ holds. 
	We write $\l_2 < \l_1$ if we can only find a query $Q_1 \in \l_1$ for every
	query $Q_2 \in \l_2$ but not vice versa.
\end{definition}
The goal for this section is to show the following results: 
\begin{enumerate}
	\item $P_{wd} < P_{wwd} <P$
	\item $\u_{wd} < \u_{wwd} < \u$
	\item $\s_{wwd} \sim \s$
\end{enumerate}
To prove the first item on the list we need the definition of ``weakly
monotone''. This characteristic was already used by Arenas and Perez in~\cite{arenas2011querying}
to prove that $P_{wd} < P$. They showed that unlike $P$ the fragment $P_{wd}$ is
weakly monotone and hence $P_{wd} < P$.
\begin{definition}[~\cite{kaminski_bwd}]
	A query $Q$ is weakly monotone if $\ll Q \rr_{G_1} \sqsubseteq \ll Q
	\rr_{G_2}$ for any two graphs $G_1$ and $G_2$ with $G_1 \subseteq G_2$.
	A fragment $\l$ is called weakly monotone if it contains only weakly
	monotone queries.
\end{definition}

Now we are ready to establish our desired result.

\begin{theorem}[\cite{kaminski_bwd}]
	It holds that $P_{wd} < P_{wwd}$.
\end{theorem}
\begin{proof}
	We show the desired result by observing that $P_{wwd}$ is not weakly
	monotone.
	Consider the pattern
	\begin{align*}
		Q=((\mathit{x,type,person)} \OPT \mathit{(x,name,y)})\\
		\OPT \mathit{(x,v\_card:name,y)}
	\end{align*}
	and the two graphs
	\begin{align*}
		G_1 =& \{\mathit{(P1,type,person)} (P1, v\_card:name,Anastasia) \}
		\mbox{ and } \\
		G_2=& \{\mathit{(P1,type,person)}, (P1, v\_card:name,Anastasia),\\
			   &\mathit{(P1, name, Ana)} \}.
	\end{align*}
	Clearly $G_1 \subseteq G_2$. Let $\mu_1 = \{x \mapsto
	P1, y \mapsto Anastasia\}$ and let $\mu_2 = \{x \mapsto
	P1, y \mapsto Ana\}$. It is also clear considering that $Q$ is a priority
	pattern that $\mu_1 \in \ll Q \rr_{G_1}$ and  $\mu_2 \in \ll Q \rr_{G_2}$. But
	we cannot extend $\mu_1$ to $\mu_2$. Thus $\mu_1 \not\sqsubseteq \mu_2$ and
	thus $P_{wd} < P_{wwd}$ follows from the fact that $P_{wd}$ is weakly
	monotone and the fact that $P_{wd} \subseteq P_{wwd}$.
\end{proof}

\begin{definition}[\cite{kaminski_bwd}]
	A query $Q$ is non-reducing if for any two graphs $G_1,G_2$ such that $G_1
	\subseteq G_2$ and any mapping $\mu_1 \in \ll Q \rr_{G_1}$, there is no
	$\mu_2 \in \ll Q \rr_{G_2}$ such that $\mu_2 \sqsubset \mu_1$ (i.e.$\mu_2
	\sqsubset \mu_1$ and $\mu_2 \neq \mu_1$).
	A fragment $\l$ is non-reducing if it contains only non-reducing queries.
\end{definition}

If a query is non-reducing it can happen that previously bound variable will get
unbound. It is easy to see that $wwd$-patterns are non-reducing as we can only
replace prior bindings but not erase them.

\begin{theorem}[\cite{kaminski_bwd}]
	It holds that $P_{wwd} < P$.
\end{theorem}
\begin{proof}
	Let 
	\begin{align*}
		P = (x, a, 1) \OPT ((y,a,2) \OPT(x,a,3)),
	\end{align*}
	\begin{align*}
		G_1 = \{(1,a,1),(2,a,2)\} \mbox{ and } G_2 = G_1 \cup \{(3,a,3)\}.
	\end{align*}
	Then $\mu_1 = \{x \mapsto 1, y \mapsto 2\}$ is the only mapping in $\ll
	P\rr_{G_1}$ while $\mu_2 = \{x \mapsto 1\}$ is the only mapping in $\ll P
	\rr_{G_2}$. Hence $\ll P \rr_{G_2} \sqsubset \ll P \rr_{G_1}$ even though
	$G_1 \subseteq G_2$. As we already mentioned $wwd$-patterns are non-reducing
	but this pattern is not non-reducing. Thus the desired result follows.
\end{proof}

The next step is to inspect the fragments $\u_{wd}$, $\u_{wwd}$ and $\u$.
It is easy to see that $\u_{wd}$ inherits weak monotonicity from
$P_{wd}$. Thus we can see that $\u_{wd} < \u_{wwd}$.
Unfortunately we cannot use non-reducibility as it doesn't propagate to unions.
Therefore a new condition is defined, namely extension-witnessing.

\begin{definition}
	A query $Q$ is extension-witnessing if for any two graphs $G_1 \subseteq
	G_2$ and mapping $\mu \in \ll Q \rr_{G_2}$ such that $\mu \not \in \ll Q
	\rr_{G_1}$ there is a triple $t$ in $Q$ such that $vars(t) \subseteq
	dom(\mu)$ and $\mu(t) \in G_2 \backslash G_1$. A fragment is
	extension-witnessing if all its queries are extension-witnessing.
\end{definition}

With this new condition, we can show that unions of wwd-patterns are
e-witnessing. 

\begin{theorem}[\cite{kaminski_bwd}]
	It holds that $\u_{wd} < \u_{wwd} < \u$.
\end{theorem}
\begin{proof}
	Again consider the counterexample from before:	
	\begin{align*}
		P = (x, a, 1) \OPT ((y,a,2) \OPT(x,a,3)),
	\end{align*}
	\begin{align*}
		G_1 = \{(1,a,1),(2,a,2)\} \mbox{ and } G_2 = G_1 \cup \{(3,a,3)\}.
	\end{align*}
	Then $\mu_1 = \{x \mapsto 1, y \mapsto 2\}$ is the only mapping in $\ll
	P\rr_{G_1}$ while $\mu_2 = \{x \mapsto 1\}$ is the only mapping in $\ll P
	\rr_{G_2}$.
	This is clearly not e-witnessing.
\end{proof}
Queries over unions of wwd-patterns are as expressive as full SPARQL.
\begin{theorem}[\cite{kaminski_bwd}]
	It holds that $\s_{wwd} \sim \s.$
\end{theorem}
\begin{proofidea}
The Tr-normal form can be described by the following grammar:
\begin{align*}
	S::= T \FILTER R  \qquad T::= B \mid T \OPT B
\end{align*}
$B$ is a set of triple patterns and $R$ is a filter condition.
A pattern $P$ is called Tr-normal if $P$ is a union of patterns $S$ described
in the grammar. Every Tr-normal pattern is contained in $\u_{wwd}$.
Let $Q$ be a graph pattern of the form\\ $\mbox{SELECT } X \mbox{ WHERE} \UNION U$. Where
$U=\{P_1,\dots,P_n\}$ is a set of UNION-free patterns and UNION $U$ stands for
$(\dots (P_1 \UNION P_2) \UNION \dots )  \UNION P_n$.
$Tr(Q)$ describes the graph pattern $\mbox{SELECT } X \mbox{ WHERE } \UNION \{Tr_X(P) \mid
P \in U\}$. A UNION-normal pattern is a graph pattern of the form 
$(P_1 \UNION P_2 \UNION P_3 \\ \UNION \cdots \UNION P_n)$, where each $P_i \ (1\leq i
\leq n)$ is UNION-free.

Assume a pattern $P$ from general SPARQL which is UNION-free. The authors
in~\cite{kaminski_bwd} developed a recursive procedure called $Tr_S(\cdot)$
which transforms a graph pattern into a graph pattern in a special form.
The following facts could be shown for $Tr_S(\cdot)$:
$P|_S \equiv Tr_S(P)|_S$ for any finite set $S$ which means that $Tr_s(\cdot)$
preserves the meaning of projection. $Tr(P)$ has been
shown to have a result of the form $\mbox{UNION} U$ where every pattern $P' \in U$ is weakly well-designed.
Additionally it has been shown that $Tr(P)$ is Tr-normal and hence in $\u_{wwd}$.
For any UNION-normal pattern $Q$, it follows immediately that $Q \equiv Tr(Q)$. And as
every query is equivalent to a UNION-normal pattern we are
done~\cite{perez2009semantics}.






\end{proofidea}

\section{Static Analysis of wwd-Patterns}

The following results technically imply some of the results in
Section~\ref{section:decidablecontainment} but the results in the Section~\ref{section:decidablecontainment}
were also of algorithmical interest because one could derive an ``out of the box''
procedure to decide the problems. The proof in this section is technically
correct but doesn't go into detail about the witnessing mapping that would allow
for such procedure.

\begin{theorem}[\cite{kaminski_bwd}]
	The problems \textbf{EQUIVALENCE($\l$)}, \textbf{CONTAINMENT($\l$)} and
	\textbf{SUBSUMPTION($\l$)} are $\Pi_2^P$-complete 
	for any $\l \in \{ P_{wwd}, U_{wwd} \}$.
\end{theorem}
\begin{proofidea}
	The membership result for \textbf{CONTAINMENT($\U_{wwd}$)} follows from the following
	counterexample property: 
	if $P \not\subseteq P'$ for some $P,P' \in \U_{wwd}$, then
	there is a witnessing mapping of size $O(|P| + |P'|)$.
	Consider the following $\Pi^P_2$-algorithm:
	Guess a mapping $\mu$ and a graph $G$ of linear size and check that $\mu
	\notin \ll P' \rr_G$ and then call a coNP oracle for checking if $\mu
	\in \ll P \rr_G$.
	Using this result we can show that \textbf{EQUIVALENCE($\u_{wwd}$)}  
	and \textbf{SUBSUMPTION($\u_{wwd}$)} are also in $\Pi^P_2$.
	The hardness proofs for \textbf{SUBSUMPTION($P_{wwd}$)} and\\
	\textbf{EQUIVALENCE($P_{wwd}$)} are by reduction from
	$3-QSAT_{\forall,2}$.\\
	\textbf{CONTAINMENT($P_{wwd}$)} is $\Pi^p_2$-hard by the results from~\cite{pichler2014containment}.
\end{proofidea}


\backmatter

% Use an optional list of figures.
\listoffigures % Starred version, i.e., \listoffigures*, removes the toc entry.

% Use an optional list of tables.
\listoftables % Starred version, i.e., \listoftables*, removes the toc entry.

% Use an optional list of alogrithms.
\listofalgorithms
\addcontentsline{toc}{chapter}{List of Algorithms}

% Add an index.
\printindex

% Add a glossary.
\printglossaries

% Add a bibliography.
\bibliographystyle{alpha}
\bibliography{intro}

\end{document}
